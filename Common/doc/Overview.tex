\documentclass{article}

\parindent 0pt
\parskip 5pt

\begin{document}

\title{Common infrastructure for the CASL language family}

\author{C. Maeder}

\maketitle

\section{Preliminaries}

see the documentation for HetCATS

\section{Commonalities for all CASL extensions}

The directory \texttt{Common} contains basic utilities and data types that are
needed by most other directories of \texttt{HetCATS}. The basic types are
simple ids (= tokens), mixfix- and compound- ids, annotations and global
annotations. For these types, parsing, pretty printing and latex printing is
provided for.

\subsection{Compiling a test program}

\texttt{cd HetCATS/Common} and call \texttt{./ghc-call} (a script that contains
all necessary ghc options and parameters). An alternative is to call
\texttt{gmake capa} in the top directory.

This creates a binary called \texttt{annos} that can parse (and pretty
print) various ids and annotations. 

Basic CASL specifications must be parsed by \texttt{CASL/capa}.

Note: if module \texttt{/AS\_Annotation.hs} can not be found, then
it was not ``drifted'' (i.e., derived from \texttt{AS\_Annotation.der.hs}). 

\section{Testing}

A test call might be: 

\texttt{./annos < test/Annotations.casl}

Test cases are given as \texttt{*.casl} or \texttt{*.annos} files in the
\texttt{test} subdirectory. Calling \texttt{./runcheck.sh ../annos} performs
many tests and compares the results with corresponding \texttt{*.output}
files.

\texttt{WrongMixIds.casl} contains wrong CASL ids and should produce
an error message for every line. 

If a \texttt{diff}
with the expected output fails, then also the number of
``\texttt{error}'' occurrences (in the produced output) is counted. For a
successful comparison a ``\texttt{passed}'' is emitted.

Calling ``\texttt{./runcheck.sh ../annos set}'' will update the
\texttt{*.output} files (and a subsequent \texttt{./runcheck.sh ../annos}
should always pass).

Alternatively, all checks can be preformed by \texttt{make check}. Output
files will be overwritten by \texttt{make output}. With \texttt{cvs up} or
\texttt{cvs diff} changes can be compared to checked-in versions of the output
files.

The script \texttt{test/checkFunctions.sh} is reused in various other
test directories.  

\section{Source files in \texttt{HetCATS/Common}}

\begin{description}
\item[AS\_Annotation.der.hs] data types for annotations
\item[AnnoState.hs] collecting annotations when parsing
\item[Anno\_Parser.hs] actual parsing of annotations
\item[CaslLanguage.hs] (unused) parser for CASL mixfix ids based on
  \texttt{Parsec.Token} and \texttt{Parsec.Language}
\item[GlobalAnnotations.hs] data type for global annotations (a
  \texttt{PrecedenceGraph} is based on a relation \texttt{Common.Lib.Rel})
\item[GlobalAnnotationsFunctions.hs] conversion to global annotations
\item[GraphUtils.hs] no longer used?
\item[Id.hs] data types for \texttt{Token} and \texttt{Id} 
  (reuses \texttt{Parsec.Pos})
\item[Keywords.hs] most keywords as named identifiers (to be used
  for parsing and printing and thus ensuring consistent spellings)
\item[LaTeX\_funs.hs] utilities for \LaTeX{} printing
\item[LaTeX\_maps.hs] generated tables for \LaTeX{} printing
\item[Lexer.hs] various scanners and extensions of the Parsec library 
\item[ListBrackets.hs] parser for list brackets in list annotations 
\item[Named.hs] encapsulate named sentences 
\item[PPUtils.hs] pretty printing utilities
\item[PrettyPrint.hs] the class for pretty printing 
\item[Print\_AS\_Annotation.hs] pretty print annotations 
\item[Result.hs] diagnostics and monadic results
\item[RunParsers.hs] reusable test driver
\item[Token.hs] generic parser for ids and some keyword parsers
  (reused in \texttt{HasCASL})
\item[Utils.hs] utility functions (inspired by perl)
\item[annos.hs] main test program for ids and annotations
\item[test\_parser.hs] (unused) test driver for ids and annotations
\end{description}

\section{Other libraries}

The directory \texttt{Common/Lib} contains further source files.

\texttt{Common/Lib/Pretty.lhs} is an adapted copy of\\
\texttt{//research.microsoft.com/\~{}simonpj/downloads/pretty-printer/pretty.html}
that is also included as Haskell (\texttt{text}) library. 

\texttt{Map} and \texttt{Set}
come from \texttt{www.cs.uu.nl/\~{}daan/ddata.html}.

The actually parser combinators come from 
\texttt{www.cs.uu.nl/\~{}daan/parsec.html} that are part of the
Haskell library but also have been slightly extended (by
\texttt{consumeNothing}) and included in the \texttt{Common/Lib}
subdirectory. \texttt{Parsec.hs} simply re-exports parts from
\texttt{Prim.hs}, \texttt{Combinator.hs},
\texttt{Char.hs}, \texttt{Error.hs} and
\texttt{Pos.hs}. 

\end{document}
