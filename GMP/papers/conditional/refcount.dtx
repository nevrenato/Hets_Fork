% \iffalse meta-comment
%
% Copyright (C) 1998, 2000, 2006, 2008 by
%    Heiko Oberdiek <oberdiek@uni-freiburg.de>
%
% This work may be distributed and/or modified under the
% conditions of the LaTeX Project Public License, either
% version 1.3 of this license or (at your option) any later
% version. The latest version of this license is in
%    http://www.latex-project.org/lppl.txt
% and version 1.3 or later is part of all distributions of
% LaTeX version 2005/12/01 or later.
%
% This work has the LPPL maintenance status "maintained".
%
% This Current Maintainer of this work is Heiko Oberdiek.
%
% This work consists of the main source file refcount.dtx
% and the derived files
%    refcount.sty, refcount.pdf, refcount.ins, refcount.drv.
%
% Distribution:
%    CTAN:macros/latex/contrib/oberdiek/refcount.dtx
%    CTAN:macros/latex/contrib/oberdiek/refcount.pdf
%
% Unpacking:
%    (a) If refcount.ins is present:
%           tex refcount.ins
%    (b) Without refcount.ins:
%           tex refcount.dtx
%    (c) If you insist on using LaTeX
%           latex \let\install=y% \iffalse meta-comment
%
% Copyright (C) 1998, 2000, 2006, 2008 by
%    Heiko Oberdiek <oberdiek@uni-freiburg.de>
%
% This work may be distributed and/or modified under the
% conditions of the LaTeX Project Public License, either
% version 1.3 of this license or (at your option) any later
% version. The latest version of this license is in
%    http://www.latex-project.org/lppl.txt
% and version 1.3 or later is part of all distributions of
% LaTeX version 2005/12/01 or later.
%
% This work has the LPPL maintenance status "maintained".
%
% This Current Maintainer of this work is Heiko Oberdiek.
%
% This work consists of the main source file refcount.dtx
% and the derived files
%    refcount.sty, refcount.pdf, refcount.ins, refcount.drv.
%
% Distribution:
%    CTAN:macros/latex/contrib/oberdiek/refcount.dtx
%    CTAN:macros/latex/contrib/oberdiek/refcount.pdf
%
% Unpacking:
%    (a) If refcount.ins is present:
%           tex refcount.ins
%    (b) Without refcount.ins:
%           tex refcount.dtx
%    (c) If you insist on using LaTeX
%           latex \let\install=y% \iffalse meta-comment
%
% Copyright (C) 1998, 2000, 2006, 2008 by
%    Heiko Oberdiek <oberdiek@uni-freiburg.de>
%
% This work may be distributed and/or modified under the
% conditions of the LaTeX Project Public License, either
% version 1.3 of this license or (at your option) any later
% version. The latest version of this license is in
%    http://www.latex-project.org/lppl.txt
% and version 1.3 or later is part of all distributions of
% LaTeX version 2005/12/01 or later.
%
% This work has the LPPL maintenance status "maintained".
%
% This Current Maintainer of this work is Heiko Oberdiek.
%
% This work consists of the main source file refcount.dtx
% and the derived files
%    refcount.sty, refcount.pdf, refcount.ins, refcount.drv.
%
% Distribution:
%    CTAN:macros/latex/contrib/oberdiek/refcount.dtx
%    CTAN:macros/latex/contrib/oberdiek/refcount.pdf
%
% Unpacking:
%    (a) If refcount.ins is present:
%           tex refcount.ins
%    (b) Without refcount.ins:
%           tex refcount.dtx
%    (c) If you insist on using LaTeX
%           latex \let\install=y% \iffalse meta-comment
%
% Copyright (C) 1998, 2000, 2006, 2008 by
%    Heiko Oberdiek <oberdiek@uni-freiburg.de>
%
% This work may be distributed and/or modified under the
% conditions of the LaTeX Project Public License, either
% version 1.3 of this license or (at your option) any later
% version. The latest version of this license is in
%    http://www.latex-project.org/lppl.txt
% and version 1.3 or later is part of all distributions of
% LaTeX version 2005/12/01 or later.
%
% This work has the LPPL maintenance status "maintained".
%
% This Current Maintainer of this work is Heiko Oberdiek.
%
% This work consists of the main source file refcount.dtx
% and the derived files
%    refcount.sty, refcount.pdf, refcount.ins, refcount.drv.
%
% Distribution:
%    CTAN:macros/latex/contrib/oberdiek/refcount.dtx
%    CTAN:macros/latex/contrib/oberdiek/refcount.pdf
%
% Unpacking:
%    (a) If refcount.ins is present:
%           tex refcount.ins
%    (b) Without refcount.ins:
%           tex refcount.dtx
%    (c) If you insist on using LaTeX
%           latex \let\install=y\input{refcount.dtx}
%        (quote the arguments according to the demands of your shell)
%
% Documentation:
%    (a) If refcount.drv is present:
%           latex refcount.drv
%    (b) Without refcount.drv:
%           latex refcount.dtx; ...
%    The class ltxdoc loads the configuration file ltxdoc.cfg
%    if available. Here you can specify further options, e.g.
%    use A4 as paper format:
%       \PassOptionsToClass{a4paper}{article}
%
%    Programm calls to get the documentation (example):
%       pdflatex refcount.dtx
%       makeindex -s gind.ist refcount.idx
%       pdflatex refcount.dtx
%       makeindex -s gind.ist refcount.idx
%       pdflatex refcount.dtx
%
% Installation:
%    TDS:tex/latex/oberdiek/refcount.sty
%    TDS:doc/latex/oberdiek/refcount.pdf
%    TDS:source/latex/oberdiek/refcount.dtx
%
%<*ignore>
\begingroup
  \def\x{LaTeX2e}%
\expandafter\endgroup
\ifcase 0\ifx\install y1\fi\expandafter
         \ifx\csname processbatchFile\endcsname\relax\else1\fi
         \ifx\fmtname\x\else 1\fi\relax
\else\csname fi\endcsname
%</ignore>
%<*install>
\input docstrip.tex
\Msg{************************************************************************}
\Msg{* Installation}
\Msg{* Package: refcount 2008/08/11 v3.1 Data extraction from references (HO)}
\Msg{************************************************************************}

\keepsilent
\askforoverwritefalse

\let\MetaPrefix\relax
\preamble

This is a generated file.

Copyright (C) 1998, 2000, 2006, 2008 by
   Heiko Oberdiek <oberdiek@uni-freiburg.de>

This work may be distributed and/or modified under the
conditions of the LaTeX Project Public License, either
version 1.3 of this license or (at your option) any later
version. The latest version of this license is in
   http://www.latex-project.org/lppl.txt
and version 1.3 or later is part of all distributions of
LaTeX version 2005/12/01 or later.

This work has the LPPL maintenance status "maintained".

This Current Maintainer of this work is Heiko Oberdiek.

This work consists of the main source file refcount.dtx
and the derived files
   refcount.sty, refcount.pdf, refcount.ins, refcount.drv.

\endpreamble
\let\MetaPrefix\DoubleperCent

\generate{%
  \file{refcount.ins}{\from{refcount.dtx}{install}}%
  \file{refcount.drv}{\from{refcount.dtx}{driver}}%
  \usedir{tex/latex/oberdiek}%
  \file{refcount.sty}{\from{refcount.dtx}{package}}%
}

\obeyspaces
\Msg{************************************************************************}
\Msg{*}
\Msg{* To finish the installation you have to move the following}
\Msg{* file into a directory searched by TeX:}
\Msg{*}
\Msg{*     refcount.sty}
\Msg{*}
\Msg{* And install the following script file:}
\Msg{*}
\Msg{*     }
\Msg{*}
\Msg{* To produce the documentation run the file `refcount.drv'}
\Msg{* through LaTeX.}
\Msg{*}
\Msg{* Happy TeXing!}
\Msg{*}
\Msg{************************************************************************}

\endbatchfile
%</install>
%<*ignore>
\fi
%</ignore>
%<*driver>
\NeedsTeXFormat{LaTeX2e}
\ProvidesFile{refcount.drv}%
  [2008/08/11 v3.1 Data extraction from references (HO)]%
\documentclass{ltxdoc}
\usepackage{holtxdoc}[2008/08/11]
\begin{document}
  \DocInput{refcount.dtx}%
\end{document}
%</driver>
% \fi
%
% \CheckSum{198}
%
% \CharacterTable
%  {Upper-case    \A\B\C\D\E\F\G\H\I\J\K\L\M\N\O\P\Q\R\S\T\U\V\W\X\Y\Z
%   Lower-case    \a\b\c\d\e\f\g\h\i\j\k\l\m\n\o\p\q\r\s\t\u\v\w\x\y\z
%   Digits        \0\1\2\3\4\5\6\7\8\9
%   Exclamation   \!     Double quote  \"     Hash (number) \#
%   Dollar        \$     Percent       \%     Ampersand     \&
%   Acute accent  \'     Left paren    \(     Right paren   \)
%   Asterisk      \*     Plus          \+     Comma         \,
%   Minus         \-     Point         \.     Solidus       \/
%   Colon         \:     Semicolon     \;     Less than     \<
%   Equals        \=     Greater than  \>     Question mark \?
%   Commercial at \@     Left bracket  \[     Backslash     \\
%   Right bracket \]     Circumflex    \^     Underscore    \_
%   Grave accent  \`     Left brace    \{     Vertical bar  \|
%   Right brace   \}     Tilde         \~}
%
% \GetFileInfo{refcount.drv}
%
% \title{The \xpackage{refcount} package}
% \date{2008/08/11 v3.1}
% \author{Heiko Oberdiek\\\xemail{oberdiek@uni-freiburg.de}}
%
% \maketitle
%
% \begin{abstract}
% References are not numbers, however they often store numerical
% data such as section or page numbers. \cs{ref} or \cs{pageref}
% cannot be used for counter assignments or calculations because
% they are not expandable, generate warnings, or can even be links,
% The package provides expandable macros to extract the data
% from references. Packages \xpackage{hyperref}, \xpackage{nameref},
% \xpackage{titleref}, and \xpackage{babel} are supported.
% \end{abstract}
%
% \tableofcontents
%
% \section{Usage}
%
% \subsection{Setting counters}
%
% The following commands are similar to \LaTeX's
% \cs{setcounter} and \cs{addtocounter},
% but they extract the number value from a reference:
% \begin{quote}
%   \cs{setcounterref}, \cs{addtocounterref}\\
%   \cs{setcounterpageref}, \cs{addtocounterpageref}
% \end{quote}
% They take two arguments:
% \begin{quote}
%    \cs{...counter...ref} |{|\meta{\LaTeX\ counter}|}|
%    |{|\meta{reference}|}|
% \end{quote}
% An undefined references produces the usual LaTeX warning
% and its value is assumed to be zero.
% Example:
% \begin{quote}
%\begin{verbatim}
%\newcounter{ctrA}
%\newcounter{ctrB}
%\refstepcounter{ctrA}\label{ref:A}
%\setcounterref{ctrB}{ref:A}
%\addtocounterpageref{ctrB}{ref:A}
%\end{verbatim}
% \end{quote}
%
% \subsection{Expandable commands}
%
% These commands that can be used in expandible contexts
% (inside calculations, \cs{edef}, \cs{csname}, \cs{write}, \dots):
% \begin{quote}
%   \cs{getrefnumber}, \cs{getpagerefnumber}
% \end{quote}
% They take one argument, the reference:
% \begin{quote}
%   \cs{get...refnumber} |{|\meta{reference}|}|
% \end{quote}
% The default for undefined references can be changed
% with macro \cs{setrefcountdefault}, for example this
% package calls:
% \begin{quote}
%   \cs{setrefcountdefault}|{0}|
% \end{quote}
%
% Since version 2.0 of this package there is a new
% command:
% \begin{quote}
%   \cs{getrefbykeydefault} |{|\meta{reference}|}|
%   |{|\meta{key}|}| |{|\meta{default}|}|
% \end{quote}
% This generalized version allows the extraction
% of further properties of a reference than the
% two standard ones. Thus the following properties
% are supported, if they are available:
% \begin{quote}
% \begin{tabular}{@{}l|l|l@{}}
%    Key & Description & Package\\
% \hline
%   \meta{empty} & same as \cs{ref} & \LaTeX\\
%   |page| & same as \cs{pageref} & \LaTeX\\
%   |title| & section and caption titles & \xpackage{titleref}\\
%   |name| & section and caption titles & \xpackage{nameref}\\
%   |anchor| & anchor name & \xpackage{hyperref}\\
%   |url| & url/file & \xpackage{hyperref}/\xpackage{xr}
% \end{tabular}
% \end{quote}
%
% \subsection{Undefined references}
%
% Because warnings and assignments cannot be used in
% expandible contexts, undefined references do not
% produce a warning, their values are assumed to be zero.
% Example:
% \begin{quote}
%\begin{verbatim}
%\label{ref:here}% somewhere
%\refused{ref:here}% see below
%\ifodd\getpagerefnumber{ref:here}%
%  reference is on an odd page
%\else
%  reference is on an even page
%\fi
%\end{verbatim}
% \end{quote}
%
% In case of undefined references the user usually want's
% to be informed. Also \LaTeX\ prints a warning at
% the end of the \LaTeX\ run. To notify \LaTeX\ and
% get a normal warning, just use
% \begin{quote}
%   \cs{refused} |{|\meta{reference}|}|
% \end{quote}
% outside the expanding context. Example, see above.
%
% \subsection{Notes}
%
% \begin{itemize}
% \item
%   The method of extracting the number in this
%   package also works in cases, where the
%   reference cannot be used directly, because
%   a package such as \xpackage{hyperref} has added
%   extra stuff (hyper link), so that the reference cannot
%   be used as number any more.
% \item
%   If the reference does not contain a number,
%   assignments to a counter will fail of course.
% \end{itemize}
%
%
% \StopEventually{
% }
%
% \section{Implementation}
%
%    \begin{macrocode}
%<*package>
\NeedsTeXFormat{LaTeX2e}
\ProvidesPackage{refcount}
  [2008/08/11 v3.1 Data extraction from references (HO)]%

\def\setrefcountdefault#1{%
  \def\rc@default{#1}%
}
\setrefcountdefault{0}

% \def\@car#1#2\@nil{#1} % defined in LaTeX kernel
\def\rc@cartwo#1#2#3\@nil{#2}

% generic check without babel support
\long\def\rc@refused#1{%
  \expandafter\ifx\csname r@#1\endcsname\relax
    \protect\G@refundefinedtrue
    \@latex@warning{%
      Reference `#1' on page \thepage\space undefined%
    }%
  \fi
}

% user command, add babel support
\newcommand*{\refused}[1]{%
  \begingroup
    \csname @safe@activestrue\endcsname
    \rc@refused{#1}{}%
  \endgroup
}

% Generic command for "\{set,addto}counter{page,}ref":
% #1: \setcounter, \addtocounter
% #2: \@car (for \ref), \@cartwo (for \pageref)
% #3: LaTeX counter
% #4: reference
\def\rc@set#1#2#3#4{%
  \begingroup
    \csname @safe@activestrue\endcsname
    \rc@refused{#4}%
    \expandafter\rc@@set\csname r@#4\endcsname{#1}{#2}{#3}%
  \endgroup
}
% #1: \r@<...>
% #2: \setcounter, \addtocounter
% #3: \@car (for \ref), \@cartwo (for \pageref)
% #4: LaTeX counter
\def\rc@@set#1#2#3#4{%
  \ifx#1\relax
    #2{#4}{\rc@default}%
  \else
    #2{#4}{%
      \expandafter#3#1\rc@default\rc@default\@nil
    }%
  \fi
}

% user commands:

\newcommand*{\setcounterref}{\rc@set\setcounter\@car}
\newcommand*{\addtocounterref}{\rc@set\addtocounter\@car}
\newcommand*{\setcounterpageref}{\rc@set\setcounter\rc@cartwo}
\newcommand*{\addtocounterpageref}{\rc@set\addtocounter\rc@cartwo}

\newcommand*{\getrefnumber}[1]{%
  \expandafter\ifx\csname r@#1\endcsname\relax
    \rc@default
  \else
    \expandafter\expandafter\expandafter\@car
    \csname r@#1\endcsname\@nil
  \fi
}
\newcommand*{\getpagerefnumber}[1]{%
  \expandafter\ifx\csname r@#1\endcsname\relax
    \rc@default
  \else
    \expandafter\expandafter\expandafter\rc@cartwo
    \csname r@#1\endcsname\rc@default\rc@default\@nil
  \fi
}
\newcommand*{\getrefbykeydefault}[2]{%
  \expandafter\rc@getrefbykeydefault
    \csname r@#1\expandafter\endcsname
    \csname rc@extract@#2\endcsname
}
% #1: \r@<...>
% #2: \rc@extract@<...>
% #3: default
\def\rc@getrefbykeydefault#1#2#3{%
  \ifx#1\relax
    % reference is undefined
    #3%
  \else
    \ifx#2\relax
      % extract method is missing
      #3%
    \else
      \expandafter\rc@generic#1{#3}{#3}{#3}{#3}{#3}\@nil#2{#3}%
    \fi
  \fi
}
% #1: first item in \r@<...>
% #2: remaining items in \r@<...>
% #3: \rc@extract@<...>
% #4: default
\def\rc@generic#1#2\@nil#3#4{%
  #3{#1\TR@TitleReference\@empty{#4}\@nil}{#1}#2\@nil
}
\def\rc@extract@{%
  \expandafter\@car\@gobble
}
\def\rc@extract@page{%
  \expandafter\@car\@gobbletwo
}
\def\rc@extract@name{%
  \expandafter\@car\@gobblefour\@empty
}
\def\rc@extract@anchor{%
  \expandafter\@car\@gobblefour
}
\def\rc@extract@url{%
  \expandafter\expandafter\expandafter\@car\expandafter
      \@gobble\@gobblefour
}
\def\rc@extract@title#1#2\@nil{%
  \rc@@extract@title#1%
}
\def\rc@@extract@title#1\TR@TitleReference#2#3#4\@nil{#3}
%</package>
%    \end{macrocode}
%
% \section{Installation}
%
% \subsection{Download}
%
% \paragraph{Package.} This package is available on
% CTAN\footnote{\url{ftp://ftp.ctan.org/tex-archive/}}:
% \begin{description}
% \item[\CTAN{macros/latex/contrib/oberdiek/refcount.dtx}] The source file.
% \item[\CTAN{macros/latex/contrib/oberdiek/refcount.pdf}] Documentation.
% \end{description}
%
%
% \paragraph{Bundle.} All the packages of the bundle `oberdiek'
% are also available in a TDS compliant ZIP archive. There
% the packages are already unpacked and the documentation files
% are generated. The files and directories obey the TDS standard.
% \begin{description}
% \item[\CTAN{install/macros/latex/contrib/oberdiek.tds.zip}]
% \end{description}
% \emph{TDS} refers to the standard ``A Directory Structure
% for \TeX\ Files'' (\CTAN{tds/tds.pdf}). Directories
% with \xfile{texmf} in their name are usually organized this way.
%
% \subsection{Bundle installation}
%
% \paragraph{Unpacking.} Unpack the \xfile{oberdiek.tds.zip} in the
% TDS tree (also known as \xfile{texmf} tree) of your choice.
% Example (linux):
% \begin{quote}
%   |unzip oberdiek.tds.zip -d ~/texmf|
% \end{quote}
%
% \paragraph{Script installation.}
% Check the directory \xfile{TDS:scripts/oberdiek/} for
% scripts that need further installation steps.
% Package \xpackage{attachfile2} comes with the Perl script
% \xfile{pdfatfi.pl} that should be installed in such a way
% that it can be called as \texttt{pdfatfi}.
% Example (linux):
% \begin{quote}
%   |chmod +x scripts/oberdiek/pdfatfi.pl|\\
%   |cp scripts/oberdiek/pdfatfi.pl /usr/local/bin/|
% \end{quote}
%
% \subsection{Package installation}
%
% \paragraph{Unpacking.} The \xfile{.dtx} file is a self-extracting
% \docstrip\ archive. The files are extracted by running the
% \xfile{.dtx} through \plainTeX:
% \begin{quote}
%   \verb|tex refcount.dtx|
% \end{quote}
%
% \paragraph{TDS.} Now the different files must be moved into
% the different directories in your installation TDS tree
% (also known as \xfile{texmf} tree):
% \begin{quote}
% \def\t{^^A
% \begin{tabular}{@{}>{\ttfamily}l@{ $\rightarrow$ }>{\ttfamily}l@{}}
%   refcount.sty & tex/latex/oberdiek/refcount.sty\\
%   refcount.pdf & doc/latex/oberdiek/refcount.pdf\\
%   refcount.dtx & source/latex/oberdiek/refcount.dtx\\
% \end{tabular}^^A
% }^^A
% \sbox0{\t}^^A
% \ifdim\wd0>\linewidth
%   \begingroup
%     \advance\linewidth by\leftmargin
%     \advance\linewidth by\rightmargin
%   \edef\x{\endgroup
%     \def\noexpand\lw{\the\linewidth}^^A
%   }\x
%   \def\lwbox{^^A
%     \leavevmode
%     \hbox to \linewidth{^^A
%       \kern-\leftmargin\relax
%       \hss
%       \usebox0
%       \hss
%       \kern-\rightmargin\relax
%     }^^A
%   }^^A
%   \ifdim\wd0>\lw
%     \sbox0{\small\t}^^A
%     \ifdim\wd0>\linewidth
%       \ifdim\wd0>\lw
%         \sbox0{\footnotesize\t}^^A
%         \ifdim\wd0>\linewidth
%           \ifdim\wd0>\lw
%             \sbox0{\scriptsize\t}^^A
%             \ifdim\wd0>\linewidth
%               \ifdim\wd0>\lw
%                 \sbox0{\tiny\t}^^A
%                 \ifdim\wd0>\linewidth
%                   \lwbox
%                 \else
%                   \usebox0
%                 \fi
%               \else
%                 \lwbox
%               \fi
%             \else
%               \usebox0
%             \fi
%           \else
%             \lwbox
%           \fi
%         \else
%           \usebox0
%         \fi
%       \else
%         \lwbox
%       \fi
%     \else
%       \usebox0
%     \fi
%   \else
%     \lwbox
%   \fi
% \else
%   \usebox0
% \fi
% \end{quote}
% If you have a \xfile{docstrip.cfg} that configures and enables \docstrip's
% TDS installing feature, then some files can already be in the right
% place, see the documentation of \docstrip.
%
% \subsection{Refresh file name databases}
%
% If your \TeX~distribution
% (\teTeX, \mikTeX, \dots) relies on file name databases, you must refresh
% these. For example, \teTeX\ users run \verb|texhash| or
% \verb|mktexlsr|.
%
% \subsection{Some details for the interested}
%
% \paragraph{Attached source.}
%
% The PDF documentation on CTAN also includes the
% \xfile{.dtx} source file. It can be extracted by
% AcrobatReader 6 or higher. Another option is \textsf{pdftk},
% e.g. unpack the file into the current directory:
% \begin{quote}
%   \verb|pdftk refcount.pdf unpack_files output .|
% \end{quote}
%
% \paragraph{Unpacking with \LaTeX.}
% The \xfile{.dtx} chooses its action depending on the format:
% \begin{description}
% \item[\plainTeX:] Run \docstrip\ and extract the files.
% \item[\LaTeX:] Generate the documentation.
% \end{description}
% If you insist on using \LaTeX\ for \docstrip\ (really,
% \docstrip\ does not need \LaTeX), then inform the autodetect routine
% about your intention:
% \begin{quote}
%   \verb|latex \let\install=y\input{refcount.dtx}|
% \end{quote}
% Do not forget to quote the argument according to the demands
% of your shell.
%
% \paragraph{Generating the documentation.}
% You can use both the \xfile{.dtx} or the \xfile{.drv} to generate
% the documentation. The process can be configured by the
% configuration file \xfile{ltxdoc.cfg}. For instance, put this
% line into this file, if you want to have A4 as paper format:
% \begin{quote}
%   \verb|\PassOptionsToClass{a4paper}{article}|
% \end{quote}
% An example follows how to generate the
% documentation with pdf\LaTeX:
% \begin{quote}
%\begin{verbatim}
%pdflatex refcount.dtx
%makeindex -s gind.ist refcount.idx
%pdflatex refcount.dtx
%makeindex -s gind.ist refcount.idx
%pdflatex refcount.dtx
%\end{verbatim}
% \end{quote}
%
% \begin{History}
%   \begin{Version}{1998/04/08 v1.0}
%   \item
%     First public release, written as answer in the
%     newsgroup \xnewsgroup{comp.text.tex}:
%     \URL{``\link{Re: Adding a \cs{ref} to a counter?}''}^^A
%     {http://groups.google.com/group/comp.text.tex/msg/c3f2a135ef5ee528}
%   \end{Version}
%   \begin{Version}{2000/09/07 v2.0}
%   \item
%     Documentation added.
%   \item
%     LPPL 1.2
%   \item
%     Package rewritten, new commands added.
%   \end{Version}
%   \begin{Version}{2006/02/20 v3.0}
%   \item
%     Support for \xpackage{hyperref} and \xpackage{nameref} improved.
%   \item
%     Support for \xpackage{titleref} and \xpackage{babel}'s shorthands added.
%   \item
%     New: \cs{refused}, \cs{getrefbykeydefault}
%   \end{Version}
%   \begin{Version}{2008/08/11 v3.1}
%   \item
%     Code is not changed.
%   \item
%     URLs updated.
%   \end{Version}
% \end{History}
%
% \PrintIndex
%
% \Finale
\endinput

%        (quote the arguments according to the demands of your shell)
%
% Documentation:
%    (a) If refcount.drv is present:
%           latex refcount.drv
%    (b) Without refcount.drv:
%           latex refcount.dtx; ...
%    The class ltxdoc loads the configuration file ltxdoc.cfg
%    if available. Here you can specify further options, e.g.
%    use A4 as paper format:
%       \PassOptionsToClass{a4paper}{article}
%
%    Programm calls to get the documentation (example):
%       pdflatex refcount.dtx
%       makeindex -s gind.ist refcount.idx
%       pdflatex refcount.dtx
%       makeindex -s gind.ist refcount.idx
%       pdflatex refcount.dtx
%
% Installation:
%    TDS:tex/latex/oberdiek/refcount.sty
%    TDS:doc/latex/oberdiek/refcount.pdf
%    TDS:source/latex/oberdiek/refcount.dtx
%
%<*ignore>
\begingroup
  \def\x{LaTeX2e}%
\expandafter\endgroup
\ifcase 0\ifx\install y1\fi\expandafter
         \ifx\csname processbatchFile\endcsname\relax\else1\fi
         \ifx\fmtname\x\else 1\fi\relax
\else\csname fi\endcsname
%</ignore>
%<*install>
\input docstrip.tex
\Msg{************************************************************************}
\Msg{* Installation}
\Msg{* Package: refcount 2008/08/11 v3.1 Data extraction from references (HO)}
\Msg{************************************************************************}

\keepsilent
\askforoverwritefalse

\let\MetaPrefix\relax
\preamble

This is a generated file.

Copyright (C) 1998, 2000, 2006, 2008 by
   Heiko Oberdiek <oberdiek@uni-freiburg.de>

This work may be distributed and/or modified under the
conditions of the LaTeX Project Public License, either
version 1.3 of this license or (at your option) any later
version. The latest version of this license is in
   http://www.latex-project.org/lppl.txt
and version 1.3 or later is part of all distributions of
LaTeX version 2005/12/01 or later.

This work has the LPPL maintenance status "maintained".

This Current Maintainer of this work is Heiko Oberdiek.

This work consists of the main source file refcount.dtx
and the derived files
   refcount.sty, refcount.pdf, refcount.ins, refcount.drv.

\endpreamble
\let\MetaPrefix\DoubleperCent

\generate{%
  \file{refcount.ins}{\from{refcount.dtx}{install}}%
  \file{refcount.drv}{\from{refcount.dtx}{driver}}%
  \usedir{tex/latex/oberdiek}%
  \file{refcount.sty}{\from{refcount.dtx}{package}}%
}

\obeyspaces
\Msg{************************************************************************}
\Msg{*}
\Msg{* To finish the installation you have to move the following}
\Msg{* file into a directory searched by TeX:}
\Msg{*}
\Msg{*     refcount.sty}
\Msg{*}
\Msg{* And install the following script file:}
\Msg{*}
\Msg{*     }
\Msg{*}
\Msg{* To produce the documentation run the file `refcount.drv'}
\Msg{* through LaTeX.}
\Msg{*}
\Msg{* Happy TeXing!}
\Msg{*}
\Msg{************************************************************************}

\endbatchfile
%</install>
%<*ignore>
\fi
%</ignore>
%<*driver>
\NeedsTeXFormat{LaTeX2e}
\ProvidesFile{refcount.drv}%
  [2008/08/11 v3.1 Data extraction from references (HO)]%
\documentclass{ltxdoc}
\usepackage{holtxdoc}[2008/08/11]
\begin{document}
  \DocInput{refcount.dtx}%
\end{document}
%</driver>
% \fi
%
% \CheckSum{198}
%
% \CharacterTable
%  {Upper-case    \A\B\C\D\E\F\G\H\I\J\K\L\M\N\O\P\Q\R\S\T\U\V\W\X\Y\Z
%   Lower-case    \a\b\c\d\e\f\g\h\i\j\k\l\m\n\o\p\q\r\s\t\u\v\w\x\y\z
%   Digits        \0\1\2\3\4\5\6\7\8\9
%   Exclamation   \!     Double quote  \"     Hash (number) \#
%   Dollar        \$     Percent       \%     Ampersand     \&
%   Acute accent  \'     Left paren    \(     Right paren   \)
%   Asterisk      \*     Plus          \+     Comma         \,
%   Minus         \-     Point         \.     Solidus       \/
%   Colon         \:     Semicolon     \;     Less than     \<
%   Equals        \=     Greater than  \>     Question mark \?
%   Commercial at \@     Left bracket  \[     Backslash     \\
%   Right bracket \]     Circumflex    \^     Underscore    \_
%   Grave accent  \`     Left brace    \{     Vertical bar  \|
%   Right brace   \}     Tilde         \~}
%
% \GetFileInfo{refcount.drv}
%
% \title{The \xpackage{refcount} package}
% \date{2008/08/11 v3.1}
% \author{Heiko Oberdiek\\\xemail{oberdiek@uni-freiburg.de}}
%
% \maketitle
%
% \begin{abstract}
% References are not numbers, however they often store numerical
% data such as section or page numbers. \cs{ref} or \cs{pageref}
% cannot be used for counter assignments or calculations because
% they are not expandable, generate warnings, or can even be links,
% The package provides expandable macros to extract the data
% from references. Packages \xpackage{hyperref}, \xpackage{nameref},
% \xpackage{titleref}, and \xpackage{babel} are supported.
% \end{abstract}
%
% \tableofcontents
%
% \section{Usage}
%
% \subsection{Setting counters}
%
% The following commands are similar to \LaTeX's
% \cs{setcounter} and \cs{addtocounter},
% but they extract the number value from a reference:
% \begin{quote}
%   \cs{setcounterref}, \cs{addtocounterref}\\
%   \cs{setcounterpageref}, \cs{addtocounterpageref}
% \end{quote}
% They take two arguments:
% \begin{quote}
%    \cs{...counter...ref} |{|\meta{\LaTeX\ counter}|}|
%    |{|\meta{reference}|}|
% \end{quote}
% An undefined references produces the usual LaTeX warning
% and its value is assumed to be zero.
% Example:
% \begin{quote}
%\begin{verbatim}
%\newcounter{ctrA}
%\newcounter{ctrB}
%\refstepcounter{ctrA}\label{ref:A}
%\setcounterref{ctrB}{ref:A}
%\addtocounterpageref{ctrB}{ref:A}
%\end{verbatim}
% \end{quote}
%
% \subsection{Expandable commands}
%
% These commands that can be used in expandible contexts
% (inside calculations, \cs{edef}, \cs{csname}, \cs{write}, \dots):
% \begin{quote}
%   \cs{getrefnumber}, \cs{getpagerefnumber}
% \end{quote}
% They take one argument, the reference:
% \begin{quote}
%   \cs{get...refnumber} |{|\meta{reference}|}|
% \end{quote}
% The default for undefined references can be changed
% with macro \cs{setrefcountdefault}, for example this
% package calls:
% \begin{quote}
%   \cs{setrefcountdefault}|{0}|
% \end{quote}
%
% Since version 2.0 of this package there is a new
% command:
% \begin{quote}
%   \cs{getrefbykeydefault} |{|\meta{reference}|}|
%   |{|\meta{key}|}| |{|\meta{default}|}|
% \end{quote}
% This generalized version allows the extraction
% of further properties of a reference than the
% two standard ones. Thus the following properties
% are supported, if they are available:
% \begin{quote}
% \begin{tabular}{@{}l|l|l@{}}
%    Key & Description & Package\\
% \hline
%   \meta{empty} & same as \cs{ref} & \LaTeX\\
%   |page| & same as \cs{pageref} & \LaTeX\\
%   |title| & section and caption titles & \xpackage{titleref}\\
%   |name| & section and caption titles & \xpackage{nameref}\\
%   |anchor| & anchor name & \xpackage{hyperref}\\
%   |url| & url/file & \xpackage{hyperref}/\xpackage{xr}
% \end{tabular}
% \end{quote}
%
% \subsection{Undefined references}
%
% Because warnings and assignments cannot be used in
% expandible contexts, undefined references do not
% produce a warning, their values are assumed to be zero.
% Example:
% \begin{quote}
%\begin{verbatim}
%\label{ref:here}% somewhere
%\refused{ref:here}% see below
%\ifodd\getpagerefnumber{ref:here}%
%  reference is on an odd page
%\else
%  reference is on an even page
%\fi
%\end{verbatim}
% \end{quote}
%
% In case of undefined references the user usually want's
% to be informed. Also \LaTeX\ prints a warning at
% the end of the \LaTeX\ run. To notify \LaTeX\ and
% get a normal warning, just use
% \begin{quote}
%   \cs{refused} |{|\meta{reference}|}|
% \end{quote}
% outside the expanding context. Example, see above.
%
% \subsection{Notes}
%
% \begin{itemize}
% \item
%   The method of extracting the number in this
%   package also works in cases, where the
%   reference cannot be used directly, because
%   a package such as \xpackage{hyperref} has added
%   extra stuff (hyper link), so that the reference cannot
%   be used as number any more.
% \item
%   If the reference does not contain a number,
%   assignments to a counter will fail of course.
% \end{itemize}
%
%
% \StopEventually{
% }
%
% \section{Implementation}
%
%    \begin{macrocode}
%<*package>
\NeedsTeXFormat{LaTeX2e}
\ProvidesPackage{refcount}
  [2008/08/11 v3.1 Data extraction from references (HO)]%

\def\setrefcountdefault#1{%
  \def\rc@default{#1}%
}
\setrefcountdefault{0}

% \def\@car#1#2\@nil{#1} % defined in LaTeX kernel
\def\rc@cartwo#1#2#3\@nil{#2}

% generic check without babel support
\long\def\rc@refused#1{%
  \expandafter\ifx\csname r@#1\endcsname\relax
    \protect\G@refundefinedtrue
    \@latex@warning{%
      Reference `#1' on page \thepage\space undefined%
    }%
  \fi
}

% user command, add babel support
\newcommand*{\refused}[1]{%
  \begingroup
    \csname @safe@activestrue\endcsname
    \rc@refused{#1}{}%
  \endgroup
}

% Generic command for "\{set,addto}counter{page,}ref":
% #1: \setcounter, \addtocounter
% #2: \@car (for \ref), \@cartwo (for \pageref)
% #3: LaTeX counter
% #4: reference
\def\rc@set#1#2#3#4{%
  \begingroup
    \csname @safe@activestrue\endcsname
    \rc@refused{#4}%
    \expandafter\rc@@set\csname r@#4\endcsname{#1}{#2}{#3}%
  \endgroup
}
% #1: \r@<...>
% #2: \setcounter, \addtocounter
% #3: \@car (for \ref), \@cartwo (for \pageref)
% #4: LaTeX counter
\def\rc@@set#1#2#3#4{%
  \ifx#1\relax
    #2{#4}{\rc@default}%
  \else
    #2{#4}{%
      \expandafter#3#1\rc@default\rc@default\@nil
    }%
  \fi
}

% user commands:

\newcommand*{\setcounterref}{\rc@set\setcounter\@car}
\newcommand*{\addtocounterref}{\rc@set\addtocounter\@car}
\newcommand*{\setcounterpageref}{\rc@set\setcounter\rc@cartwo}
\newcommand*{\addtocounterpageref}{\rc@set\addtocounter\rc@cartwo}

\newcommand*{\getrefnumber}[1]{%
  \expandafter\ifx\csname r@#1\endcsname\relax
    \rc@default
  \else
    \expandafter\expandafter\expandafter\@car
    \csname r@#1\endcsname\@nil
  \fi
}
\newcommand*{\getpagerefnumber}[1]{%
  \expandafter\ifx\csname r@#1\endcsname\relax
    \rc@default
  \else
    \expandafter\expandafter\expandafter\rc@cartwo
    \csname r@#1\endcsname\rc@default\rc@default\@nil
  \fi
}
\newcommand*{\getrefbykeydefault}[2]{%
  \expandafter\rc@getrefbykeydefault
    \csname r@#1\expandafter\endcsname
    \csname rc@extract@#2\endcsname
}
% #1: \r@<...>
% #2: \rc@extract@<...>
% #3: default
\def\rc@getrefbykeydefault#1#2#3{%
  \ifx#1\relax
    % reference is undefined
    #3%
  \else
    \ifx#2\relax
      % extract method is missing
      #3%
    \else
      \expandafter\rc@generic#1{#3}{#3}{#3}{#3}{#3}\@nil#2{#3}%
    \fi
  \fi
}
% #1: first item in \r@<...>
% #2: remaining items in \r@<...>
% #3: \rc@extract@<...>
% #4: default
\def\rc@generic#1#2\@nil#3#4{%
  #3{#1\TR@TitleReference\@empty{#4}\@nil}{#1}#2\@nil
}
\def\rc@extract@{%
  \expandafter\@car\@gobble
}
\def\rc@extract@page{%
  \expandafter\@car\@gobbletwo
}
\def\rc@extract@name{%
  \expandafter\@car\@gobblefour\@empty
}
\def\rc@extract@anchor{%
  \expandafter\@car\@gobblefour
}
\def\rc@extract@url{%
  \expandafter\expandafter\expandafter\@car\expandafter
      \@gobble\@gobblefour
}
\def\rc@extract@title#1#2\@nil{%
  \rc@@extract@title#1%
}
\def\rc@@extract@title#1\TR@TitleReference#2#3#4\@nil{#3}
%</package>
%    \end{macrocode}
%
% \section{Installation}
%
% \subsection{Download}
%
% \paragraph{Package.} This package is available on
% CTAN\footnote{\url{ftp://ftp.ctan.org/tex-archive/}}:
% \begin{description}
% \item[\CTAN{macros/latex/contrib/oberdiek/refcount.dtx}] The source file.
% \item[\CTAN{macros/latex/contrib/oberdiek/refcount.pdf}] Documentation.
% \end{description}
%
%
% \paragraph{Bundle.} All the packages of the bundle `oberdiek'
% are also available in a TDS compliant ZIP archive. There
% the packages are already unpacked and the documentation files
% are generated. The files and directories obey the TDS standard.
% \begin{description}
% \item[\CTAN{install/macros/latex/contrib/oberdiek.tds.zip}]
% \end{description}
% \emph{TDS} refers to the standard ``A Directory Structure
% for \TeX\ Files'' (\CTAN{tds/tds.pdf}). Directories
% with \xfile{texmf} in their name are usually organized this way.
%
% \subsection{Bundle installation}
%
% \paragraph{Unpacking.} Unpack the \xfile{oberdiek.tds.zip} in the
% TDS tree (also known as \xfile{texmf} tree) of your choice.
% Example (linux):
% \begin{quote}
%   |unzip oberdiek.tds.zip -d ~/texmf|
% \end{quote}
%
% \paragraph{Script installation.}
% Check the directory \xfile{TDS:scripts/oberdiek/} for
% scripts that need further installation steps.
% Package \xpackage{attachfile2} comes with the Perl script
% \xfile{pdfatfi.pl} that should be installed in such a way
% that it can be called as \texttt{pdfatfi}.
% Example (linux):
% \begin{quote}
%   |chmod +x scripts/oberdiek/pdfatfi.pl|\\
%   |cp scripts/oberdiek/pdfatfi.pl /usr/local/bin/|
% \end{quote}
%
% \subsection{Package installation}
%
% \paragraph{Unpacking.} The \xfile{.dtx} file is a self-extracting
% \docstrip\ archive. The files are extracted by running the
% \xfile{.dtx} through \plainTeX:
% \begin{quote}
%   \verb|tex refcount.dtx|
% \end{quote}
%
% \paragraph{TDS.} Now the different files must be moved into
% the different directories in your installation TDS tree
% (also known as \xfile{texmf} tree):
% \begin{quote}
% \def\t{^^A
% \begin{tabular}{@{}>{\ttfamily}l@{ $\rightarrow$ }>{\ttfamily}l@{}}
%   refcount.sty & tex/latex/oberdiek/refcount.sty\\
%   refcount.pdf & doc/latex/oberdiek/refcount.pdf\\
%   refcount.dtx & source/latex/oberdiek/refcount.dtx\\
% \end{tabular}^^A
% }^^A
% \sbox0{\t}^^A
% \ifdim\wd0>\linewidth
%   \begingroup
%     \advance\linewidth by\leftmargin
%     \advance\linewidth by\rightmargin
%   \edef\x{\endgroup
%     \def\noexpand\lw{\the\linewidth}^^A
%   }\x
%   \def\lwbox{^^A
%     \leavevmode
%     \hbox to \linewidth{^^A
%       \kern-\leftmargin\relax
%       \hss
%       \usebox0
%       \hss
%       \kern-\rightmargin\relax
%     }^^A
%   }^^A
%   \ifdim\wd0>\lw
%     \sbox0{\small\t}^^A
%     \ifdim\wd0>\linewidth
%       \ifdim\wd0>\lw
%         \sbox0{\footnotesize\t}^^A
%         \ifdim\wd0>\linewidth
%           \ifdim\wd0>\lw
%             \sbox0{\scriptsize\t}^^A
%             \ifdim\wd0>\linewidth
%               \ifdim\wd0>\lw
%                 \sbox0{\tiny\t}^^A
%                 \ifdim\wd0>\linewidth
%                   \lwbox
%                 \else
%                   \usebox0
%                 \fi
%               \else
%                 \lwbox
%               \fi
%             \else
%               \usebox0
%             \fi
%           \else
%             \lwbox
%           \fi
%         \else
%           \usebox0
%         \fi
%       \else
%         \lwbox
%       \fi
%     \else
%       \usebox0
%     \fi
%   \else
%     \lwbox
%   \fi
% \else
%   \usebox0
% \fi
% \end{quote}
% If you have a \xfile{docstrip.cfg} that configures and enables \docstrip's
% TDS installing feature, then some files can already be in the right
% place, see the documentation of \docstrip.
%
% \subsection{Refresh file name databases}
%
% If your \TeX~distribution
% (\teTeX, \mikTeX, \dots) relies on file name databases, you must refresh
% these. For example, \teTeX\ users run \verb|texhash| or
% \verb|mktexlsr|.
%
% \subsection{Some details for the interested}
%
% \paragraph{Attached source.}
%
% The PDF documentation on CTAN also includes the
% \xfile{.dtx} source file. It can be extracted by
% AcrobatReader 6 or higher. Another option is \textsf{pdftk},
% e.g. unpack the file into the current directory:
% \begin{quote}
%   \verb|pdftk refcount.pdf unpack_files output .|
% \end{quote}
%
% \paragraph{Unpacking with \LaTeX.}
% The \xfile{.dtx} chooses its action depending on the format:
% \begin{description}
% \item[\plainTeX:] Run \docstrip\ and extract the files.
% \item[\LaTeX:] Generate the documentation.
% \end{description}
% If you insist on using \LaTeX\ for \docstrip\ (really,
% \docstrip\ does not need \LaTeX), then inform the autodetect routine
% about your intention:
% \begin{quote}
%   \verb|latex \let\install=y% \iffalse meta-comment
%
% Copyright (C) 1998, 2000, 2006, 2008 by
%    Heiko Oberdiek <oberdiek@uni-freiburg.de>
%
% This work may be distributed and/or modified under the
% conditions of the LaTeX Project Public License, either
% version 1.3 of this license or (at your option) any later
% version. The latest version of this license is in
%    http://www.latex-project.org/lppl.txt
% and version 1.3 or later is part of all distributions of
% LaTeX version 2005/12/01 or later.
%
% This work has the LPPL maintenance status "maintained".
%
% This Current Maintainer of this work is Heiko Oberdiek.
%
% This work consists of the main source file refcount.dtx
% and the derived files
%    refcount.sty, refcount.pdf, refcount.ins, refcount.drv.
%
% Distribution:
%    CTAN:macros/latex/contrib/oberdiek/refcount.dtx
%    CTAN:macros/latex/contrib/oberdiek/refcount.pdf
%
% Unpacking:
%    (a) If refcount.ins is present:
%           tex refcount.ins
%    (b) Without refcount.ins:
%           tex refcount.dtx
%    (c) If you insist on using LaTeX
%           latex \let\install=y\input{refcount.dtx}
%        (quote the arguments according to the demands of your shell)
%
% Documentation:
%    (a) If refcount.drv is present:
%           latex refcount.drv
%    (b) Without refcount.drv:
%           latex refcount.dtx; ...
%    The class ltxdoc loads the configuration file ltxdoc.cfg
%    if available. Here you can specify further options, e.g.
%    use A4 as paper format:
%       \PassOptionsToClass{a4paper}{article}
%
%    Programm calls to get the documentation (example):
%       pdflatex refcount.dtx
%       makeindex -s gind.ist refcount.idx
%       pdflatex refcount.dtx
%       makeindex -s gind.ist refcount.idx
%       pdflatex refcount.dtx
%
% Installation:
%    TDS:tex/latex/oberdiek/refcount.sty
%    TDS:doc/latex/oberdiek/refcount.pdf
%    TDS:source/latex/oberdiek/refcount.dtx
%
%<*ignore>
\begingroup
  \def\x{LaTeX2e}%
\expandafter\endgroup
\ifcase 0\ifx\install y1\fi\expandafter
         \ifx\csname processbatchFile\endcsname\relax\else1\fi
         \ifx\fmtname\x\else 1\fi\relax
\else\csname fi\endcsname
%</ignore>
%<*install>
\input docstrip.tex
\Msg{************************************************************************}
\Msg{* Installation}
\Msg{* Package: refcount 2008/08/11 v3.1 Data extraction from references (HO)}
\Msg{************************************************************************}

\keepsilent
\askforoverwritefalse

\let\MetaPrefix\relax
\preamble

This is a generated file.

Copyright (C) 1998, 2000, 2006, 2008 by
   Heiko Oberdiek <oberdiek@uni-freiburg.de>

This work may be distributed and/or modified under the
conditions of the LaTeX Project Public License, either
version 1.3 of this license or (at your option) any later
version. The latest version of this license is in
   http://www.latex-project.org/lppl.txt
and version 1.3 or later is part of all distributions of
LaTeX version 2005/12/01 or later.

This work has the LPPL maintenance status "maintained".

This Current Maintainer of this work is Heiko Oberdiek.

This work consists of the main source file refcount.dtx
and the derived files
   refcount.sty, refcount.pdf, refcount.ins, refcount.drv.

\endpreamble
\let\MetaPrefix\DoubleperCent

\generate{%
  \file{refcount.ins}{\from{refcount.dtx}{install}}%
  \file{refcount.drv}{\from{refcount.dtx}{driver}}%
  \usedir{tex/latex/oberdiek}%
  \file{refcount.sty}{\from{refcount.dtx}{package}}%
}

\obeyspaces
\Msg{************************************************************************}
\Msg{*}
\Msg{* To finish the installation you have to move the following}
\Msg{* file into a directory searched by TeX:}
\Msg{*}
\Msg{*     refcount.sty}
\Msg{*}
\Msg{* And install the following script file:}
\Msg{*}
\Msg{*     }
\Msg{*}
\Msg{* To produce the documentation run the file `refcount.drv'}
\Msg{* through LaTeX.}
\Msg{*}
\Msg{* Happy TeXing!}
\Msg{*}
\Msg{************************************************************************}

\endbatchfile
%</install>
%<*ignore>
\fi
%</ignore>
%<*driver>
\NeedsTeXFormat{LaTeX2e}
\ProvidesFile{refcount.drv}%
  [2008/08/11 v3.1 Data extraction from references (HO)]%
\documentclass{ltxdoc}
\usepackage{holtxdoc}[2008/08/11]
\begin{document}
  \DocInput{refcount.dtx}%
\end{document}
%</driver>
% \fi
%
% \CheckSum{198}
%
% \CharacterTable
%  {Upper-case    \A\B\C\D\E\F\G\H\I\J\K\L\M\N\O\P\Q\R\S\T\U\V\W\X\Y\Z
%   Lower-case    \a\b\c\d\e\f\g\h\i\j\k\l\m\n\o\p\q\r\s\t\u\v\w\x\y\z
%   Digits        \0\1\2\3\4\5\6\7\8\9
%   Exclamation   \!     Double quote  \"     Hash (number) \#
%   Dollar        \$     Percent       \%     Ampersand     \&
%   Acute accent  \'     Left paren    \(     Right paren   \)
%   Asterisk      \*     Plus          \+     Comma         \,
%   Minus         \-     Point         \.     Solidus       \/
%   Colon         \:     Semicolon     \;     Less than     \<
%   Equals        \=     Greater than  \>     Question mark \?
%   Commercial at \@     Left bracket  \[     Backslash     \\
%   Right bracket \]     Circumflex    \^     Underscore    \_
%   Grave accent  \`     Left brace    \{     Vertical bar  \|
%   Right brace   \}     Tilde         \~}
%
% \GetFileInfo{refcount.drv}
%
% \title{The \xpackage{refcount} package}
% \date{2008/08/11 v3.1}
% \author{Heiko Oberdiek\\\xemail{oberdiek@uni-freiburg.de}}
%
% \maketitle
%
% \begin{abstract}
% References are not numbers, however they often store numerical
% data such as section or page numbers. \cs{ref} or \cs{pageref}
% cannot be used for counter assignments or calculations because
% they are not expandable, generate warnings, or can even be links,
% The package provides expandable macros to extract the data
% from references. Packages \xpackage{hyperref}, \xpackage{nameref},
% \xpackage{titleref}, and \xpackage{babel} are supported.
% \end{abstract}
%
% \tableofcontents
%
% \section{Usage}
%
% \subsection{Setting counters}
%
% The following commands are similar to \LaTeX's
% \cs{setcounter} and \cs{addtocounter},
% but they extract the number value from a reference:
% \begin{quote}
%   \cs{setcounterref}, \cs{addtocounterref}\\
%   \cs{setcounterpageref}, \cs{addtocounterpageref}
% \end{quote}
% They take two arguments:
% \begin{quote}
%    \cs{...counter...ref} |{|\meta{\LaTeX\ counter}|}|
%    |{|\meta{reference}|}|
% \end{quote}
% An undefined references produces the usual LaTeX warning
% and its value is assumed to be zero.
% Example:
% \begin{quote}
%\begin{verbatim}
%\newcounter{ctrA}
%\newcounter{ctrB}
%\refstepcounter{ctrA}\label{ref:A}
%\setcounterref{ctrB}{ref:A}
%\addtocounterpageref{ctrB}{ref:A}
%\end{verbatim}
% \end{quote}
%
% \subsection{Expandable commands}
%
% These commands that can be used in expandible contexts
% (inside calculations, \cs{edef}, \cs{csname}, \cs{write}, \dots):
% \begin{quote}
%   \cs{getrefnumber}, \cs{getpagerefnumber}
% \end{quote}
% They take one argument, the reference:
% \begin{quote}
%   \cs{get...refnumber} |{|\meta{reference}|}|
% \end{quote}
% The default for undefined references can be changed
% with macro \cs{setrefcountdefault}, for example this
% package calls:
% \begin{quote}
%   \cs{setrefcountdefault}|{0}|
% \end{quote}
%
% Since version 2.0 of this package there is a new
% command:
% \begin{quote}
%   \cs{getrefbykeydefault} |{|\meta{reference}|}|
%   |{|\meta{key}|}| |{|\meta{default}|}|
% \end{quote}
% This generalized version allows the extraction
% of further properties of a reference than the
% two standard ones. Thus the following properties
% are supported, if they are available:
% \begin{quote}
% \begin{tabular}{@{}l|l|l@{}}
%    Key & Description & Package\\
% \hline
%   \meta{empty} & same as \cs{ref} & \LaTeX\\
%   |page| & same as \cs{pageref} & \LaTeX\\
%   |title| & section and caption titles & \xpackage{titleref}\\
%   |name| & section and caption titles & \xpackage{nameref}\\
%   |anchor| & anchor name & \xpackage{hyperref}\\
%   |url| & url/file & \xpackage{hyperref}/\xpackage{xr}
% \end{tabular}
% \end{quote}
%
% \subsection{Undefined references}
%
% Because warnings and assignments cannot be used in
% expandible contexts, undefined references do not
% produce a warning, their values are assumed to be zero.
% Example:
% \begin{quote}
%\begin{verbatim}
%\label{ref:here}% somewhere
%\refused{ref:here}% see below
%\ifodd\getpagerefnumber{ref:here}%
%  reference is on an odd page
%\else
%  reference is on an even page
%\fi
%\end{verbatim}
% \end{quote}
%
% In case of undefined references the user usually want's
% to be informed. Also \LaTeX\ prints a warning at
% the end of the \LaTeX\ run. To notify \LaTeX\ and
% get a normal warning, just use
% \begin{quote}
%   \cs{refused} |{|\meta{reference}|}|
% \end{quote}
% outside the expanding context. Example, see above.
%
% \subsection{Notes}
%
% \begin{itemize}
% \item
%   The method of extracting the number in this
%   package also works in cases, where the
%   reference cannot be used directly, because
%   a package such as \xpackage{hyperref} has added
%   extra stuff (hyper link), so that the reference cannot
%   be used as number any more.
% \item
%   If the reference does not contain a number,
%   assignments to a counter will fail of course.
% \end{itemize}
%
%
% \StopEventually{
% }
%
% \section{Implementation}
%
%    \begin{macrocode}
%<*package>
\NeedsTeXFormat{LaTeX2e}
\ProvidesPackage{refcount}
  [2008/08/11 v3.1 Data extraction from references (HO)]%

\def\setrefcountdefault#1{%
  \def\rc@default{#1}%
}
\setrefcountdefault{0}

% \def\@car#1#2\@nil{#1} % defined in LaTeX kernel
\def\rc@cartwo#1#2#3\@nil{#2}

% generic check without babel support
\long\def\rc@refused#1{%
  \expandafter\ifx\csname r@#1\endcsname\relax
    \protect\G@refundefinedtrue
    \@latex@warning{%
      Reference `#1' on page \thepage\space undefined%
    }%
  \fi
}

% user command, add babel support
\newcommand*{\refused}[1]{%
  \begingroup
    \csname @safe@activestrue\endcsname
    \rc@refused{#1}{}%
  \endgroup
}

% Generic command for "\{set,addto}counter{page,}ref":
% #1: \setcounter, \addtocounter
% #2: \@car (for \ref), \@cartwo (for \pageref)
% #3: LaTeX counter
% #4: reference
\def\rc@set#1#2#3#4{%
  \begingroup
    \csname @safe@activestrue\endcsname
    \rc@refused{#4}%
    \expandafter\rc@@set\csname r@#4\endcsname{#1}{#2}{#3}%
  \endgroup
}
% #1: \r@<...>
% #2: \setcounter, \addtocounter
% #3: \@car (for \ref), \@cartwo (for \pageref)
% #4: LaTeX counter
\def\rc@@set#1#2#3#4{%
  \ifx#1\relax
    #2{#4}{\rc@default}%
  \else
    #2{#4}{%
      \expandafter#3#1\rc@default\rc@default\@nil
    }%
  \fi
}

% user commands:

\newcommand*{\setcounterref}{\rc@set\setcounter\@car}
\newcommand*{\addtocounterref}{\rc@set\addtocounter\@car}
\newcommand*{\setcounterpageref}{\rc@set\setcounter\rc@cartwo}
\newcommand*{\addtocounterpageref}{\rc@set\addtocounter\rc@cartwo}

\newcommand*{\getrefnumber}[1]{%
  \expandafter\ifx\csname r@#1\endcsname\relax
    \rc@default
  \else
    \expandafter\expandafter\expandafter\@car
    \csname r@#1\endcsname\@nil
  \fi
}
\newcommand*{\getpagerefnumber}[1]{%
  \expandafter\ifx\csname r@#1\endcsname\relax
    \rc@default
  \else
    \expandafter\expandafter\expandafter\rc@cartwo
    \csname r@#1\endcsname\rc@default\rc@default\@nil
  \fi
}
\newcommand*{\getrefbykeydefault}[2]{%
  \expandafter\rc@getrefbykeydefault
    \csname r@#1\expandafter\endcsname
    \csname rc@extract@#2\endcsname
}
% #1: \r@<...>
% #2: \rc@extract@<...>
% #3: default
\def\rc@getrefbykeydefault#1#2#3{%
  \ifx#1\relax
    % reference is undefined
    #3%
  \else
    \ifx#2\relax
      % extract method is missing
      #3%
    \else
      \expandafter\rc@generic#1{#3}{#3}{#3}{#3}{#3}\@nil#2{#3}%
    \fi
  \fi
}
% #1: first item in \r@<...>
% #2: remaining items in \r@<...>
% #3: \rc@extract@<...>
% #4: default
\def\rc@generic#1#2\@nil#3#4{%
  #3{#1\TR@TitleReference\@empty{#4}\@nil}{#1}#2\@nil
}
\def\rc@extract@{%
  \expandafter\@car\@gobble
}
\def\rc@extract@page{%
  \expandafter\@car\@gobbletwo
}
\def\rc@extract@name{%
  \expandafter\@car\@gobblefour\@empty
}
\def\rc@extract@anchor{%
  \expandafter\@car\@gobblefour
}
\def\rc@extract@url{%
  \expandafter\expandafter\expandafter\@car\expandafter
      \@gobble\@gobblefour
}
\def\rc@extract@title#1#2\@nil{%
  \rc@@extract@title#1%
}
\def\rc@@extract@title#1\TR@TitleReference#2#3#4\@nil{#3}
%</package>
%    \end{macrocode}
%
% \section{Installation}
%
% \subsection{Download}
%
% \paragraph{Package.} This package is available on
% CTAN\footnote{\url{ftp://ftp.ctan.org/tex-archive/}}:
% \begin{description}
% \item[\CTAN{macros/latex/contrib/oberdiek/refcount.dtx}] The source file.
% \item[\CTAN{macros/latex/contrib/oberdiek/refcount.pdf}] Documentation.
% \end{description}
%
%
% \paragraph{Bundle.} All the packages of the bundle `oberdiek'
% are also available in a TDS compliant ZIP archive. There
% the packages are already unpacked and the documentation files
% are generated. The files and directories obey the TDS standard.
% \begin{description}
% \item[\CTAN{install/macros/latex/contrib/oberdiek.tds.zip}]
% \end{description}
% \emph{TDS} refers to the standard ``A Directory Structure
% for \TeX\ Files'' (\CTAN{tds/tds.pdf}). Directories
% with \xfile{texmf} in their name are usually organized this way.
%
% \subsection{Bundle installation}
%
% \paragraph{Unpacking.} Unpack the \xfile{oberdiek.tds.zip} in the
% TDS tree (also known as \xfile{texmf} tree) of your choice.
% Example (linux):
% \begin{quote}
%   |unzip oberdiek.tds.zip -d ~/texmf|
% \end{quote}
%
% \paragraph{Script installation.}
% Check the directory \xfile{TDS:scripts/oberdiek/} for
% scripts that need further installation steps.
% Package \xpackage{attachfile2} comes with the Perl script
% \xfile{pdfatfi.pl} that should be installed in such a way
% that it can be called as \texttt{pdfatfi}.
% Example (linux):
% \begin{quote}
%   |chmod +x scripts/oberdiek/pdfatfi.pl|\\
%   |cp scripts/oberdiek/pdfatfi.pl /usr/local/bin/|
% \end{quote}
%
% \subsection{Package installation}
%
% \paragraph{Unpacking.} The \xfile{.dtx} file is a self-extracting
% \docstrip\ archive. The files are extracted by running the
% \xfile{.dtx} through \plainTeX:
% \begin{quote}
%   \verb|tex refcount.dtx|
% \end{quote}
%
% \paragraph{TDS.} Now the different files must be moved into
% the different directories in your installation TDS tree
% (also known as \xfile{texmf} tree):
% \begin{quote}
% \def\t{^^A
% \begin{tabular}{@{}>{\ttfamily}l@{ $\rightarrow$ }>{\ttfamily}l@{}}
%   refcount.sty & tex/latex/oberdiek/refcount.sty\\
%   refcount.pdf & doc/latex/oberdiek/refcount.pdf\\
%   refcount.dtx & source/latex/oberdiek/refcount.dtx\\
% \end{tabular}^^A
% }^^A
% \sbox0{\t}^^A
% \ifdim\wd0>\linewidth
%   \begingroup
%     \advance\linewidth by\leftmargin
%     \advance\linewidth by\rightmargin
%   \edef\x{\endgroup
%     \def\noexpand\lw{\the\linewidth}^^A
%   }\x
%   \def\lwbox{^^A
%     \leavevmode
%     \hbox to \linewidth{^^A
%       \kern-\leftmargin\relax
%       \hss
%       \usebox0
%       \hss
%       \kern-\rightmargin\relax
%     }^^A
%   }^^A
%   \ifdim\wd0>\lw
%     \sbox0{\small\t}^^A
%     \ifdim\wd0>\linewidth
%       \ifdim\wd0>\lw
%         \sbox0{\footnotesize\t}^^A
%         \ifdim\wd0>\linewidth
%           \ifdim\wd0>\lw
%             \sbox0{\scriptsize\t}^^A
%             \ifdim\wd0>\linewidth
%               \ifdim\wd0>\lw
%                 \sbox0{\tiny\t}^^A
%                 \ifdim\wd0>\linewidth
%                   \lwbox
%                 \else
%                   \usebox0
%                 \fi
%               \else
%                 \lwbox
%               \fi
%             \else
%               \usebox0
%             \fi
%           \else
%             \lwbox
%           \fi
%         \else
%           \usebox0
%         \fi
%       \else
%         \lwbox
%       \fi
%     \else
%       \usebox0
%     \fi
%   \else
%     \lwbox
%   \fi
% \else
%   \usebox0
% \fi
% \end{quote}
% If you have a \xfile{docstrip.cfg} that configures and enables \docstrip's
% TDS installing feature, then some files can already be in the right
% place, see the documentation of \docstrip.
%
% \subsection{Refresh file name databases}
%
% If your \TeX~distribution
% (\teTeX, \mikTeX, \dots) relies on file name databases, you must refresh
% these. For example, \teTeX\ users run \verb|texhash| or
% \verb|mktexlsr|.
%
% \subsection{Some details for the interested}
%
% \paragraph{Attached source.}
%
% The PDF documentation on CTAN also includes the
% \xfile{.dtx} source file. It can be extracted by
% AcrobatReader 6 or higher. Another option is \textsf{pdftk},
% e.g. unpack the file into the current directory:
% \begin{quote}
%   \verb|pdftk refcount.pdf unpack_files output .|
% \end{quote}
%
% \paragraph{Unpacking with \LaTeX.}
% The \xfile{.dtx} chooses its action depending on the format:
% \begin{description}
% \item[\plainTeX:] Run \docstrip\ and extract the files.
% \item[\LaTeX:] Generate the documentation.
% \end{description}
% If you insist on using \LaTeX\ for \docstrip\ (really,
% \docstrip\ does not need \LaTeX), then inform the autodetect routine
% about your intention:
% \begin{quote}
%   \verb|latex \let\install=y\input{refcount.dtx}|
% \end{quote}
% Do not forget to quote the argument according to the demands
% of your shell.
%
% \paragraph{Generating the documentation.}
% You can use both the \xfile{.dtx} or the \xfile{.drv} to generate
% the documentation. The process can be configured by the
% configuration file \xfile{ltxdoc.cfg}. For instance, put this
% line into this file, if you want to have A4 as paper format:
% \begin{quote}
%   \verb|\PassOptionsToClass{a4paper}{article}|
% \end{quote}
% An example follows how to generate the
% documentation with pdf\LaTeX:
% \begin{quote}
%\begin{verbatim}
%pdflatex refcount.dtx
%makeindex -s gind.ist refcount.idx
%pdflatex refcount.dtx
%makeindex -s gind.ist refcount.idx
%pdflatex refcount.dtx
%\end{verbatim}
% \end{quote}
%
% \begin{History}
%   \begin{Version}{1998/04/08 v1.0}
%   \item
%     First public release, written as answer in the
%     newsgroup \xnewsgroup{comp.text.tex}:
%     \URL{``\link{Re: Adding a \cs{ref} to a counter?}''}^^A
%     {http://groups.google.com/group/comp.text.tex/msg/c3f2a135ef5ee528}
%   \end{Version}
%   \begin{Version}{2000/09/07 v2.0}
%   \item
%     Documentation added.
%   \item
%     LPPL 1.2
%   \item
%     Package rewritten, new commands added.
%   \end{Version}
%   \begin{Version}{2006/02/20 v3.0}
%   \item
%     Support for \xpackage{hyperref} and \xpackage{nameref} improved.
%   \item
%     Support for \xpackage{titleref} and \xpackage{babel}'s shorthands added.
%   \item
%     New: \cs{refused}, \cs{getrefbykeydefault}
%   \end{Version}
%   \begin{Version}{2008/08/11 v3.1}
%   \item
%     Code is not changed.
%   \item
%     URLs updated.
%   \end{Version}
% \end{History}
%
% \PrintIndex
%
% \Finale
\endinput
|
% \end{quote}
% Do not forget to quote the argument according to the demands
% of your shell.
%
% \paragraph{Generating the documentation.}
% You can use both the \xfile{.dtx} or the \xfile{.drv} to generate
% the documentation. The process can be configured by the
% configuration file \xfile{ltxdoc.cfg}. For instance, put this
% line into this file, if you want to have A4 as paper format:
% \begin{quote}
%   \verb|\PassOptionsToClass{a4paper}{article}|
% \end{quote}
% An example follows how to generate the
% documentation with pdf\LaTeX:
% \begin{quote}
%\begin{verbatim}
%pdflatex refcount.dtx
%makeindex -s gind.ist refcount.idx
%pdflatex refcount.dtx
%makeindex -s gind.ist refcount.idx
%pdflatex refcount.dtx
%\end{verbatim}
% \end{quote}
%
% \begin{History}
%   \begin{Version}{1998/04/08 v1.0}
%   \item
%     First public release, written as answer in the
%     newsgroup \xnewsgroup{comp.text.tex}:
%     \URL{``\link{Re: Adding a \cs{ref} to a counter?}''}^^A
%     {http://groups.google.com/group/comp.text.tex/msg/c3f2a135ef5ee528}
%   \end{Version}
%   \begin{Version}{2000/09/07 v2.0}
%   \item
%     Documentation added.
%   \item
%     LPPL 1.2
%   \item
%     Package rewritten, new commands added.
%   \end{Version}
%   \begin{Version}{2006/02/20 v3.0}
%   \item
%     Support for \xpackage{hyperref} and \xpackage{nameref} improved.
%   \item
%     Support for \xpackage{titleref} and \xpackage{babel}'s shorthands added.
%   \item
%     New: \cs{refused}, \cs{getrefbykeydefault}
%   \end{Version}
%   \begin{Version}{2008/08/11 v3.1}
%   \item
%     Code is not changed.
%   \item
%     URLs updated.
%   \end{Version}
% \end{History}
%
% \PrintIndex
%
% \Finale
\endinput

%        (quote the arguments according to the demands of your shell)
%
% Documentation:
%    (a) If refcount.drv is present:
%           latex refcount.drv
%    (b) Without refcount.drv:
%           latex refcount.dtx; ...
%    The class ltxdoc loads the configuration file ltxdoc.cfg
%    if available. Here you can specify further options, e.g.
%    use A4 as paper format:
%       \PassOptionsToClass{a4paper}{article}
%
%    Programm calls to get the documentation (example):
%       pdflatex refcount.dtx
%       makeindex -s gind.ist refcount.idx
%       pdflatex refcount.dtx
%       makeindex -s gind.ist refcount.idx
%       pdflatex refcount.dtx
%
% Installation:
%    TDS:tex/latex/oberdiek/refcount.sty
%    TDS:doc/latex/oberdiek/refcount.pdf
%    TDS:source/latex/oberdiek/refcount.dtx
%
%<*ignore>
\begingroup
  \def\x{LaTeX2e}%
\expandafter\endgroup
\ifcase 0\ifx\install y1\fi\expandafter
         \ifx\csname processbatchFile\endcsname\relax\else1\fi
         \ifx\fmtname\x\else 1\fi\relax
\else\csname fi\endcsname
%</ignore>
%<*install>
\input docstrip.tex
\Msg{************************************************************************}
\Msg{* Installation}
\Msg{* Package: refcount 2008/08/11 v3.1 Data extraction from references (HO)}
\Msg{************************************************************************}

\keepsilent
\askforoverwritefalse

\let\MetaPrefix\relax
\preamble

This is a generated file.

Copyright (C) 1998, 2000, 2006, 2008 by
   Heiko Oberdiek <oberdiek@uni-freiburg.de>

This work may be distributed and/or modified under the
conditions of the LaTeX Project Public License, either
version 1.3 of this license or (at your option) any later
version. The latest version of this license is in
   http://www.latex-project.org/lppl.txt
and version 1.3 or later is part of all distributions of
LaTeX version 2005/12/01 or later.

This work has the LPPL maintenance status "maintained".

This Current Maintainer of this work is Heiko Oberdiek.

This work consists of the main source file refcount.dtx
and the derived files
   refcount.sty, refcount.pdf, refcount.ins, refcount.drv.

\endpreamble
\let\MetaPrefix\DoubleperCent

\generate{%
  \file{refcount.ins}{\from{refcount.dtx}{install}}%
  \file{refcount.drv}{\from{refcount.dtx}{driver}}%
  \usedir{tex/latex/oberdiek}%
  \file{refcount.sty}{\from{refcount.dtx}{package}}%
}

\obeyspaces
\Msg{************************************************************************}
\Msg{*}
\Msg{* To finish the installation you have to move the following}
\Msg{* file into a directory searched by TeX:}
\Msg{*}
\Msg{*     refcount.sty}
\Msg{*}
\Msg{* And install the following script file:}
\Msg{*}
\Msg{*     }
\Msg{*}
\Msg{* To produce the documentation run the file `refcount.drv'}
\Msg{* through LaTeX.}
\Msg{*}
\Msg{* Happy TeXing!}
\Msg{*}
\Msg{************************************************************************}

\endbatchfile
%</install>
%<*ignore>
\fi
%</ignore>
%<*driver>
\NeedsTeXFormat{LaTeX2e}
\ProvidesFile{refcount.drv}%
  [2008/08/11 v3.1 Data extraction from references (HO)]%
\documentclass{ltxdoc}
\usepackage{holtxdoc}[2008/08/11]
\begin{document}
  \DocInput{refcount.dtx}%
\end{document}
%</driver>
% \fi
%
% \CheckSum{198}
%
% \CharacterTable
%  {Upper-case    \A\B\C\D\E\F\G\H\I\J\K\L\M\N\O\P\Q\R\S\T\U\V\W\X\Y\Z
%   Lower-case    \a\b\c\d\e\f\g\h\i\j\k\l\m\n\o\p\q\r\s\t\u\v\w\x\y\z
%   Digits        \0\1\2\3\4\5\6\7\8\9
%   Exclamation   \!     Double quote  \"     Hash (number) \#
%   Dollar        \$     Percent       \%     Ampersand     \&
%   Acute accent  \'     Left paren    \(     Right paren   \)
%   Asterisk      \*     Plus          \+     Comma         \,
%   Minus         \-     Point         \.     Solidus       \/
%   Colon         \:     Semicolon     \;     Less than     \<
%   Equals        \=     Greater than  \>     Question mark \?
%   Commercial at \@     Left bracket  \[     Backslash     \\
%   Right bracket \]     Circumflex    \^     Underscore    \_
%   Grave accent  \`     Left brace    \{     Vertical bar  \|
%   Right brace   \}     Tilde         \~}
%
% \GetFileInfo{refcount.drv}
%
% \title{The \xpackage{refcount} package}
% \date{2008/08/11 v3.1}
% \author{Heiko Oberdiek\\\xemail{oberdiek@uni-freiburg.de}}
%
% \maketitle
%
% \begin{abstract}
% References are not numbers, however they often store numerical
% data such as section or page numbers. \cs{ref} or \cs{pageref}
% cannot be used for counter assignments or calculations because
% they are not expandable, generate warnings, or can even be links,
% The package provides expandable macros to extract the data
% from references. Packages \xpackage{hyperref}, \xpackage{nameref},
% \xpackage{titleref}, and \xpackage{babel} are supported.
% \end{abstract}
%
% \tableofcontents
%
% \section{Usage}
%
% \subsection{Setting counters}
%
% The following commands are similar to \LaTeX's
% \cs{setcounter} and \cs{addtocounter},
% but they extract the number value from a reference:
% \begin{quote}
%   \cs{setcounterref}, \cs{addtocounterref}\\
%   \cs{setcounterpageref}, \cs{addtocounterpageref}
% \end{quote}
% They take two arguments:
% \begin{quote}
%    \cs{...counter...ref} |{|\meta{\LaTeX\ counter}|}|
%    |{|\meta{reference}|}|
% \end{quote}
% An undefined references produces the usual LaTeX warning
% and its value is assumed to be zero.
% Example:
% \begin{quote}
%\begin{verbatim}
%\newcounter{ctrA}
%\newcounter{ctrB}
%\refstepcounter{ctrA}\label{ref:A}
%\setcounterref{ctrB}{ref:A}
%\addtocounterpageref{ctrB}{ref:A}
%\end{verbatim}
% \end{quote}
%
% \subsection{Expandable commands}
%
% These commands that can be used in expandible contexts
% (inside calculations, \cs{edef}, \cs{csname}, \cs{write}, \dots):
% \begin{quote}
%   \cs{getrefnumber}, \cs{getpagerefnumber}
% \end{quote}
% They take one argument, the reference:
% \begin{quote}
%   \cs{get...refnumber} |{|\meta{reference}|}|
% \end{quote}
% The default for undefined references can be changed
% with macro \cs{setrefcountdefault}, for example this
% package calls:
% \begin{quote}
%   \cs{setrefcountdefault}|{0}|
% \end{quote}
%
% Since version 2.0 of this package there is a new
% command:
% \begin{quote}
%   \cs{getrefbykeydefault} |{|\meta{reference}|}|
%   |{|\meta{key}|}| |{|\meta{default}|}|
% \end{quote}
% This generalized version allows the extraction
% of further properties of a reference than the
% two standard ones. Thus the following properties
% are supported, if they are available:
% \begin{quote}
% \begin{tabular}{@{}l|l|l@{}}
%    Key & Description & Package\\
% \hline
%   \meta{empty} & same as \cs{ref} & \LaTeX\\
%   |page| & same as \cs{pageref} & \LaTeX\\
%   |title| & section and caption titles & \xpackage{titleref}\\
%   |name| & section and caption titles & \xpackage{nameref}\\
%   |anchor| & anchor name & \xpackage{hyperref}\\
%   |url| & url/file & \xpackage{hyperref}/\xpackage{xr}
% \end{tabular}
% \end{quote}
%
% \subsection{Undefined references}
%
% Because warnings and assignments cannot be used in
% expandible contexts, undefined references do not
% produce a warning, their values are assumed to be zero.
% Example:
% \begin{quote}
%\begin{verbatim}
%\label{ref:here}% somewhere
%\refused{ref:here}% see below
%\ifodd\getpagerefnumber{ref:here}%
%  reference is on an odd page
%\else
%  reference is on an even page
%\fi
%\end{verbatim}
% \end{quote}
%
% In case of undefined references the user usually want's
% to be informed. Also \LaTeX\ prints a warning at
% the end of the \LaTeX\ run. To notify \LaTeX\ and
% get a normal warning, just use
% \begin{quote}
%   \cs{refused} |{|\meta{reference}|}|
% \end{quote}
% outside the expanding context. Example, see above.
%
% \subsection{Notes}
%
% \begin{itemize}
% \item
%   The method of extracting the number in this
%   package also works in cases, where the
%   reference cannot be used directly, because
%   a package such as \xpackage{hyperref} has added
%   extra stuff (hyper link), so that the reference cannot
%   be used as number any more.
% \item
%   If the reference does not contain a number,
%   assignments to a counter will fail of course.
% \end{itemize}
%
%
% \StopEventually{
% }
%
% \section{Implementation}
%
%    \begin{macrocode}
%<*package>
\NeedsTeXFormat{LaTeX2e}
\ProvidesPackage{refcount}
  [2008/08/11 v3.1 Data extraction from references (HO)]%

\def\setrefcountdefault#1{%
  \def\rc@default{#1}%
}
\setrefcountdefault{0}

% \def\@car#1#2\@nil{#1} % defined in LaTeX kernel
\def\rc@cartwo#1#2#3\@nil{#2}

% generic check without babel support
\long\def\rc@refused#1{%
  \expandafter\ifx\csname r@#1\endcsname\relax
    \protect\G@refundefinedtrue
    \@latex@warning{%
      Reference `#1' on page \thepage\space undefined%
    }%
  \fi
}

% user command, add babel support
\newcommand*{\refused}[1]{%
  \begingroup
    \csname @safe@activestrue\endcsname
    \rc@refused{#1}{}%
  \endgroup
}

% Generic command for "\{set,addto}counter{page,}ref":
% #1: \setcounter, \addtocounter
% #2: \@car (for \ref), \@cartwo (for \pageref)
% #3: LaTeX counter
% #4: reference
\def\rc@set#1#2#3#4{%
  \begingroup
    \csname @safe@activestrue\endcsname
    \rc@refused{#4}%
    \expandafter\rc@@set\csname r@#4\endcsname{#1}{#2}{#3}%
  \endgroup
}
% #1: \r@<...>
% #2: \setcounter, \addtocounter
% #3: \@car (for \ref), \@cartwo (for \pageref)
% #4: LaTeX counter
\def\rc@@set#1#2#3#4{%
  \ifx#1\relax
    #2{#4}{\rc@default}%
  \else
    #2{#4}{%
      \expandafter#3#1\rc@default\rc@default\@nil
    }%
  \fi
}

% user commands:

\newcommand*{\setcounterref}{\rc@set\setcounter\@car}
\newcommand*{\addtocounterref}{\rc@set\addtocounter\@car}
\newcommand*{\setcounterpageref}{\rc@set\setcounter\rc@cartwo}
\newcommand*{\addtocounterpageref}{\rc@set\addtocounter\rc@cartwo}

\newcommand*{\getrefnumber}[1]{%
  \expandafter\ifx\csname r@#1\endcsname\relax
    \rc@default
  \else
    \expandafter\expandafter\expandafter\@car
    \csname r@#1\endcsname\@nil
  \fi
}
\newcommand*{\getpagerefnumber}[1]{%
  \expandafter\ifx\csname r@#1\endcsname\relax
    \rc@default
  \else
    \expandafter\expandafter\expandafter\rc@cartwo
    \csname r@#1\endcsname\rc@default\rc@default\@nil
  \fi
}
\newcommand*{\getrefbykeydefault}[2]{%
  \expandafter\rc@getrefbykeydefault
    \csname r@#1\expandafter\endcsname
    \csname rc@extract@#2\endcsname
}
% #1: \r@<...>
% #2: \rc@extract@<...>
% #3: default
\def\rc@getrefbykeydefault#1#2#3{%
  \ifx#1\relax
    % reference is undefined
    #3%
  \else
    \ifx#2\relax
      % extract method is missing
      #3%
    \else
      \expandafter\rc@generic#1{#3}{#3}{#3}{#3}{#3}\@nil#2{#3}%
    \fi
  \fi
}
% #1: first item in \r@<...>
% #2: remaining items in \r@<...>
% #3: \rc@extract@<...>
% #4: default
\def\rc@generic#1#2\@nil#3#4{%
  #3{#1\TR@TitleReference\@empty{#4}\@nil}{#1}#2\@nil
}
\def\rc@extract@{%
  \expandafter\@car\@gobble
}
\def\rc@extract@page{%
  \expandafter\@car\@gobbletwo
}
\def\rc@extract@name{%
  \expandafter\@car\@gobblefour\@empty
}
\def\rc@extract@anchor{%
  \expandafter\@car\@gobblefour
}
\def\rc@extract@url{%
  \expandafter\expandafter\expandafter\@car\expandafter
      \@gobble\@gobblefour
}
\def\rc@extract@title#1#2\@nil{%
  \rc@@extract@title#1%
}
\def\rc@@extract@title#1\TR@TitleReference#2#3#4\@nil{#3}
%</package>
%    \end{macrocode}
%
% \section{Installation}
%
% \subsection{Download}
%
% \paragraph{Package.} This package is available on
% CTAN\footnote{\url{ftp://ftp.ctan.org/tex-archive/}}:
% \begin{description}
% \item[\CTAN{macros/latex/contrib/oberdiek/refcount.dtx}] The source file.
% \item[\CTAN{macros/latex/contrib/oberdiek/refcount.pdf}] Documentation.
% \end{description}
%
%
% \paragraph{Bundle.} All the packages of the bundle `oberdiek'
% are also available in a TDS compliant ZIP archive. There
% the packages are already unpacked and the documentation files
% are generated. The files and directories obey the TDS standard.
% \begin{description}
% \item[\CTAN{install/macros/latex/contrib/oberdiek.tds.zip}]
% \end{description}
% \emph{TDS} refers to the standard ``A Directory Structure
% for \TeX\ Files'' (\CTAN{tds/tds.pdf}). Directories
% with \xfile{texmf} in their name are usually organized this way.
%
% \subsection{Bundle installation}
%
% \paragraph{Unpacking.} Unpack the \xfile{oberdiek.tds.zip} in the
% TDS tree (also known as \xfile{texmf} tree) of your choice.
% Example (linux):
% \begin{quote}
%   |unzip oberdiek.tds.zip -d ~/texmf|
% \end{quote}
%
% \paragraph{Script installation.}
% Check the directory \xfile{TDS:scripts/oberdiek/} for
% scripts that need further installation steps.
% Package \xpackage{attachfile2} comes with the Perl script
% \xfile{pdfatfi.pl} that should be installed in such a way
% that it can be called as \texttt{pdfatfi}.
% Example (linux):
% \begin{quote}
%   |chmod +x scripts/oberdiek/pdfatfi.pl|\\
%   |cp scripts/oberdiek/pdfatfi.pl /usr/local/bin/|
% \end{quote}
%
% \subsection{Package installation}
%
% \paragraph{Unpacking.} The \xfile{.dtx} file is a self-extracting
% \docstrip\ archive. The files are extracted by running the
% \xfile{.dtx} through \plainTeX:
% \begin{quote}
%   \verb|tex refcount.dtx|
% \end{quote}
%
% \paragraph{TDS.} Now the different files must be moved into
% the different directories in your installation TDS tree
% (also known as \xfile{texmf} tree):
% \begin{quote}
% \def\t{^^A
% \begin{tabular}{@{}>{\ttfamily}l@{ $\rightarrow$ }>{\ttfamily}l@{}}
%   refcount.sty & tex/latex/oberdiek/refcount.sty\\
%   refcount.pdf & doc/latex/oberdiek/refcount.pdf\\
%   refcount.dtx & source/latex/oberdiek/refcount.dtx\\
% \end{tabular}^^A
% }^^A
% \sbox0{\t}^^A
% \ifdim\wd0>\linewidth
%   \begingroup
%     \advance\linewidth by\leftmargin
%     \advance\linewidth by\rightmargin
%   \edef\x{\endgroup
%     \def\noexpand\lw{\the\linewidth}^^A
%   }\x
%   \def\lwbox{^^A
%     \leavevmode
%     \hbox to \linewidth{^^A
%       \kern-\leftmargin\relax
%       \hss
%       \usebox0
%       \hss
%       \kern-\rightmargin\relax
%     }^^A
%   }^^A
%   \ifdim\wd0>\lw
%     \sbox0{\small\t}^^A
%     \ifdim\wd0>\linewidth
%       \ifdim\wd0>\lw
%         \sbox0{\footnotesize\t}^^A
%         \ifdim\wd0>\linewidth
%           \ifdim\wd0>\lw
%             \sbox0{\scriptsize\t}^^A
%             \ifdim\wd0>\linewidth
%               \ifdim\wd0>\lw
%                 \sbox0{\tiny\t}^^A
%                 \ifdim\wd0>\linewidth
%                   \lwbox
%                 \else
%                   \usebox0
%                 \fi
%               \else
%                 \lwbox
%               \fi
%             \else
%               \usebox0
%             \fi
%           \else
%             \lwbox
%           \fi
%         \else
%           \usebox0
%         \fi
%       \else
%         \lwbox
%       \fi
%     \else
%       \usebox0
%     \fi
%   \else
%     \lwbox
%   \fi
% \else
%   \usebox0
% \fi
% \end{quote}
% If you have a \xfile{docstrip.cfg} that configures and enables \docstrip's
% TDS installing feature, then some files can already be in the right
% place, see the documentation of \docstrip.
%
% \subsection{Refresh file name databases}
%
% If your \TeX~distribution
% (\teTeX, \mikTeX, \dots) relies on file name databases, you must refresh
% these. For example, \teTeX\ users run \verb|texhash| or
% \verb|mktexlsr|.
%
% \subsection{Some details for the interested}
%
% \paragraph{Attached source.}
%
% The PDF documentation on CTAN also includes the
% \xfile{.dtx} source file. It can be extracted by
% AcrobatReader 6 or higher. Another option is \textsf{pdftk},
% e.g. unpack the file into the current directory:
% \begin{quote}
%   \verb|pdftk refcount.pdf unpack_files output .|
% \end{quote}
%
% \paragraph{Unpacking with \LaTeX.}
% The \xfile{.dtx} chooses its action depending on the format:
% \begin{description}
% \item[\plainTeX:] Run \docstrip\ and extract the files.
% \item[\LaTeX:] Generate the documentation.
% \end{description}
% If you insist on using \LaTeX\ for \docstrip\ (really,
% \docstrip\ does not need \LaTeX), then inform the autodetect routine
% about your intention:
% \begin{quote}
%   \verb|latex \let\install=y% \iffalse meta-comment
%
% Copyright (C) 1998, 2000, 2006, 2008 by
%    Heiko Oberdiek <oberdiek@uni-freiburg.de>
%
% This work may be distributed and/or modified under the
% conditions of the LaTeX Project Public License, either
% version 1.3 of this license or (at your option) any later
% version. The latest version of this license is in
%    http://www.latex-project.org/lppl.txt
% and version 1.3 or later is part of all distributions of
% LaTeX version 2005/12/01 or later.
%
% This work has the LPPL maintenance status "maintained".
%
% This Current Maintainer of this work is Heiko Oberdiek.
%
% This work consists of the main source file refcount.dtx
% and the derived files
%    refcount.sty, refcount.pdf, refcount.ins, refcount.drv.
%
% Distribution:
%    CTAN:macros/latex/contrib/oberdiek/refcount.dtx
%    CTAN:macros/latex/contrib/oberdiek/refcount.pdf
%
% Unpacking:
%    (a) If refcount.ins is present:
%           tex refcount.ins
%    (b) Without refcount.ins:
%           tex refcount.dtx
%    (c) If you insist on using LaTeX
%           latex \let\install=y% \iffalse meta-comment
%
% Copyright (C) 1998, 2000, 2006, 2008 by
%    Heiko Oberdiek <oberdiek@uni-freiburg.de>
%
% This work may be distributed and/or modified under the
% conditions of the LaTeX Project Public License, either
% version 1.3 of this license or (at your option) any later
% version. The latest version of this license is in
%    http://www.latex-project.org/lppl.txt
% and version 1.3 or later is part of all distributions of
% LaTeX version 2005/12/01 or later.
%
% This work has the LPPL maintenance status "maintained".
%
% This Current Maintainer of this work is Heiko Oberdiek.
%
% This work consists of the main source file refcount.dtx
% and the derived files
%    refcount.sty, refcount.pdf, refcount.ins, refcount.drv.
%
% Distribution:
%    CTAN:macros/latex/contrib/oberdiek/refcount.dtx
%    CTAN:macros/latex/contrib/oberdiek/refcount.pdf
%
% Unpacking:
%    (a) If refcount.ins is present:
%           tex refcount.ins
%    (b) Without refcount.ins:
%           tex refcount.dtx
%    (c) If you insist on using LaTeX
%           latex \let\install=y\input{refcount.dtx}
%        (quote the arguments according to the demands of your shell)
%
% Documentation:
%    (a) If refcount.drv is present:
%           latex refcount.drv
%    (b) Without refcount.drv:
%           latex refcount.dtx; ...
%    The class ltxdoc loads the configuration file ltxdoc.cfg
%    if available. Here you can specify further options, e.g.
%    use A4 as paper format:
%       \PassOptionsToClass{a4paper}{article}
%
%    Programm calls to get the documentation (example):
%       pdflatex refcount.dtx
%       makeindex -s gind.ist refcount.idx
%       pdflatex refcount.dtx
%       makeindex -s gind.ist refcount.idx
%       pdflatex refcount.dtx
%
% Installation:
%    TDS:tex/latex/oberdiek/refcount.sty
%    TDS:doc/latex/oberdiek/refcount.pdf
%    TDS:source/latex/oberdiek/refcount.dtx
%
%<*ignore>
\begingroup
  \def\x{LaTeX2e}%
\expandafter\endgroup
\ifcase 0\ifx\install y1\fi\expandafter
         \ifx\csname processbatchFile\endcsname\relax\else1\fi
         \ifx\fmtname\x\else 1\fi\relax
\else\csname fi\endcsname
%</ignore>
%<*install>
\input docstrip.tex
\Msg{************************************************************************}
\Msg{* Installation}
\Msg{* Package: refcount 2008/08/11 v3.1 Data extraction from references (HO)}
\Msg{************************************************************************}

\keepsilent
\askforoverwritefalse

\let\MetaPrefix\relax
\preamble

This is a generated file.

Copyright (C) 1998, 2000, 2006, 2008 by
   Heiko Oberdiek <oberdiek@uni-freiburg.de>

This work may be distributed and/or modified under the
conditions of the LaTeX Project Public License, either
version 1.3 of this license or (at your option) any later
version. The latest version of this license is in
   http://www.latex-project.org/lppl.txt
and version 1.3 or later is part of all distributions of
LaTeX version 2005/12/01 or later.

This work has the LPPL maintenance status "maintained".

This Current Maintainer of this work is Heiko Oberdiek.

This work consists of the main source file refcount.dtx
and the derived files
   refcount.sty, refcount.pdf, refcount.ins, refcount.drv.

\endpreamble
\let\MetaPrefix\DoubleperCent

\generate{%
  \file{refcount.ins}{\from{refcount.dtx}{install}}%
  \file{refcount.drv}{\from{refcount.dtx}{driver}}%
  \usedir{tex/latex/oberdiek}%
  \file{refcount.sty}{\from{refcount.dtx}{package}}%
}

\obeyspaces
\Msg{************************************************************************}
\Msg{*}
\Msg{* To finish the installation you have to move the following}
\Msg{* file into a directory searched by TeX:}
\Msg{*}
\Msg{*     refcount.sty}
\Msg{*}
\Msg{* And install the following script file:}
\Msg{*}
\Msg{*     }
\Msg{*}
\Msg{* To produce the documentation run the file `refcount.drv'}
\Msg{* through LaTeX.}
\Msg{*}
\Msg{* Happy TeXing!}
\Msg{*}
\Msg{************************************************************************}

\endbatchfile
%</install>
%<*ignore>
\fi
%</ignore>
%<*driver>
\NeedsTeXFormat{LaTeX2e}
\ProvidesFile{refcount.drv}%
  [2008/08/11 v3.1 Data extraction from references (HO)]%
\documentclass{ltxdoc}
\usepackage{holtxdoc}[2008/08/11]
\begin{document}
  \DocInput{refcount.dtx}%
\end{document}
%</driver>
% \fi
%
% \CheckSum{198}
%
% \CharacterTable
%  {Upper-case    \A\B\C\D\E\F\G\H\I\J\K\L\M\N\O\P\Q\R\S\T\U\V\W\X\Y\Z
%   Lower-case    \a\b\c\d\e\f\g\h\i\j\k\l\m\n\o\p\q\r\s\t\u\v\w\x\y\z
%   Digits        \0\1\2\3\4\5\6\7\8\9
%   Exclamation   \!     Double quote  \"     Hash (number) \#
%   Dollar        \$     Percent       \%     Ampersand     \&
%   Acute accent  \'     Left paren    \(     Right paren   \)
%   Asterisk      \*     Plus          \+     Comma         \,
%   Minus         \-     Point         \.     Solidus       \/
%   Colon         \:     Semicolon     \;     Less than     \<
%   Equals        \=     Greater than  \>     Question mark \?
%   Commercial at \@     Left bracket  \[     Backslash     \\
%   Right bracket \]     Circumflex    \^     Underscore    \_
%   Grave accent  \`     Left brace    \{     Vertical bar  \|
%   Right brace   \}     Tilde         \~}
%
% \GetFileInfo{refcount.drv}
%
% \title{The \xpackage{refcount} package}
% \date{2008/08/11 v3.1}
% \author{Heiko Oberdiek\\\xemail{oberdiek@uni-freiburg.de}}
%
% \maketitle
%
% \begin{abstract}
% References are not numbers, however they often store numerical
% data such as section or page numbers. \cs{ref} or \cs{pageref}
% cannot be used for counter assignments or calculations because
% they are not expandable, generate warnings, or can even be links,
% The package provides expandable macros to extract the data
% from references. Packages \xpackage{hyperref}, \xpackage{nameref},
% \xpackage{titleref}, and \xpackage{babel} are supported.
% \end{abstract}
%
% \tableofcontents
%
% \section{Usage}
%
% \subsection{Setting counters}
%
% The following commands are similar to \LaTeX's
% \cs{setcounter} and \cs{addtocounter},
% but they extract the number value from a reference:
% \begin{quote}
%   \cs{setcounterref}, \cs{addtocounterref}\\
%   \cs{setcounterpageref}, \cs{addtocounterpageref}
% \end{quote}
% They take two arguments:
% \begin{quote}
%    \cs{...counter...ref} |{|\meta{\LaTeX\ counter}|}|
%    |{|\meta{reference}|}|
% \end{quote}
% An undefined references produces the usual LaTeX warning
% and its value is assumed to be zero.
% Example:
% \begin{quote}
%\begin{verbatim}
%\newcounter{ctrA}
%\newcounter{ctrB}
%\refstepcounter{ctrA}\label{ref:A}
%\setcounterref{ctrB}{ref:A}
%\addtocounterpageref{ctrB}{ref:A}
%\end{verbatim}
% \end{quote}
%
% \subsection{Expandable commands}
%
% These commands that can be used in expandible contexts
% (inside calculations, \cs{edef}, \cs{csname}, \cs{write}, \dots):
% \begin{quote}
%   \cs{getrefnumber}, \cs{getpagerefnumber}
% \end{quote}
% They take one argument, the reference:
% \begin{quote}
%   \cs{get...refnumber} |{|\meta{reference}|}|
% \end{quote}
% The default for undefined references can be changed
% with macro \cs{setrefcountdefault}, for example this
% package calls:
% \begin{quote}
%   \cs{setrefcountdefault}|{0}|
% \end{quote}
%
% Since version 2.0 of this package there is a new
% command:
% \begin{quote}
%   \cs{getrefbykeydefault} |{|\meta{reference}|}|
%   |{|\meta{key}|}| |{|\meta{default}|}|
% \end{quote}
% This generalized version allows the extraction
% of further properties of a reference than the
% two standard ones. Thus the following properties
% are supported, if they are available:
% \begin{quote}
% \begin{tabular}{@{}l|l|l@{}}
%    Key & Description & Package\\
% \hline
%   \meta{empty} & same as \cs{ref} & \LaTeX\\
%   |page| & same as \cs{pageref} & \LaTeX\\
%   |title| & section and caption titles & \xpackage{titleref}\\
%   |name| & section and caption titles & \xpackage{nameref}\\
%   |anchor| & anchor name & \xpackage{hyperref}\\
%   |url| & url/file & \xpackage{hyperref}/\xpackage{xr}
% \end{tabular}
% \end{quote}
%
% \subsection{Undefined references}
%
% Because warnings and assignments cannot be used in
% expandible contexts, undefined references do not
% produce a warning, their values are assumed to be zero.
% Example:
% \begin{quote}
%\begin{verbatim}
%\label{ref:here}% somewhere
%\refused{ref:here}% see below
%\ifodd\getpagerefnumber{ref:here}%
%  reference is on an odd page
%\else
%  reference is on an even page
%\fi
%\end{verbatim}
% \end{quote}
%
% In case of undefined references the user usually want's
% to be informed. Also \LaTeX\ prints a warning at
% the end of the \LaTeX\ run. To notify \LaTeX\ and
% get a normal warning, just use
% \begin{quote}
%   \cs{refused} |{|\meta{reference}|}|
% \end{quote}
% outside the expanding context. Example, see above.
%
% \subsection{Notes}
%
% \begin{itemize}
% \item
%   The method of extracting the number in this
%   package also works in cases, where the
%   reference cannot be used directly, because
%   a package such as \xpackage{hyperref} has added
%   extra stuff (hyper link), so that the reference cannot
%   be used as number any more.
% \item
%   If the reference does not contain a number,
%   assignments to a counter will fail of course.
% \end{itemize}
%
%
% \StopEventually{
% }
%
% \section{Implementation}
%
%    \begin{macrocode}
%<*package>
\NeedsTeXFormat{LaTeX2e}
\ProvidesPackage{refcount}
  [2008/08/11 v3.1 Data extraction from references (HO)]%

\def\setrefcountdefault#1{%
  \def\rc@default{#1}%
}
\setrefcountdefault{0}

% \def\@car#1#2\@nil{#1} % defined in LaTeX kernel
\def\rc@cartwo#1#2#3\@nil{#2}

% generic check without babel support
\long\def\rc@refused#1{%
  \expandafter\ifx\csname r@#1\endcsname\relax
    \protect\G@refundefinedtrue
    \@latex@warning{%
      Reference `#1' on page \thepage\space undefined%
    }%
  \fi
}

% user command, add babel support
\newcommand*{\refused}[1]{%
  \begingroup
    \csname @safe@activestrue\endcsname
    \rc@refused{#1}{}%
  \endgroup
}

% Generic command for "\{set,addto}counter{page,}ref":
% #1: \setcounter, \addtocounter
% #2: \@car (for \ref), \@cartwo (for \pageref)
% #3: LaTeX counter
% #4: reference
\def\rc@set#1#2#3#4{%
  \begingroup
    \csname @safe@activestrue\endcsname
    \rc@refused{#4}%
    \expandafter\rc@@set\csname r@#4\endcsname{#1}{#2}{#3}%
  \endgroup
}
% #1: \r@<...>
% #2: \setcounter, \addtocounter
% #3: \@car (for \ref), \@cartwo (for \pageref)
% #4: LaTeX counter
\def\rc@@set#1#2#3#4{%
  \ifx#1\relax
    #2{#4}{\rc@default}%
  \else
    #2{#4}{%
      \expandafter#3#1\rc@default\rc@default\@nil
    }%
  \fi
}

% user commands:

\newcommand*{\setcounterref}{\rc@set\setcounter\@car}
\newcommand*{\addtocounterref}{\rc@set\addtocounter\@car}
\newcommand*{\setcounterpageref}{\rc@set\setcounter\rc@cartwo}
\newcommand*{\addtocounterpageref}{\rc@set\addtocounter\rc@cartwo}

\newcommand*{\getrefnumber}[1]{%
  \expandafter\ifx\csname r@#1\endcsname\relax
    \rc@default
  \else
    \expandafter\expandafter\expandafter\@car
    \csname r@#1\endcsname\@nil
  \fi
}
\newcommand*{\getpagerefnumber}[1]{%
  \expandafter\ifx\csname r@#1\endcsname\relax
    \rc@default
  \else
    \expandafter\expandafter\expandafter\rc@cartwo
    \csname r@#1\endcsname\rc@default\rc@default\@nil
  \fi
}
\newcommand*{\getrefbykeydefault}[2]{%
  \expandafter\rc@getrefbykeydefault
    \csname r@#1\expandafter\endcsname
    \csname rc@extract@#2\endcsname
}
% #1: \r@<...>
% #2: \rc@extract@<...>
% #3: default
\def\rc@getrefbykeydefault#1#2#3{%
  \ifx#1\relax
    % reference is undefined
    #3%
  \else
    \ifx#2\relax
      % extract method is missing
      #3%
    \else
      \expandafter\rc@generic#1{#3}{#3}{#3}{#3}{#3}\@nil#2{#3}%
    \fi
  \fi
}
% #1: first item in \r@<...>
% #2: remaining items in \r@<...>
% #3: \rc@extract@<...>
% #4: default
\def\rc@generic#1#2\@nil#3#4{%
  #3{#1\TR@TitleReference\@empty{#4}\@nil}{#1}#2\@nil
}
\def\rc@extract@{%
  \expandafter\@car\@gobble
}
\def\rc@extract@page{%
  \expandafter\@car\@gobbletwo
}
\def\rc@extract@name{%
  \expandafter\@car\@gobblefour\@empty
}
\def\rc@extract@anchor{%
  \expandafter\@car\@gobblefour
}
\def\rc@extract@url{%
  \expandafter\expandafter\expandafter\@car\expandafter
      \@gobble\@gobblefour
}
\def\rc@extract@title#1#2\@nil{%
  \rc@@extract@title#1%
}
\def\rc@@extract@title#1\TR@TitleReference#2#3#4\@nil{#3}
%</package>
%    \end{macrocode}
%
% \section{Installation}
%
% \subsection{Download}
%
% \paragraph{Package.} This package is available on
% CTAN\footnote{\url{ftp://ftp.ctan.org/tex-archive/}}:
% \begin{description}
% \item[\CTAN{macros/latex/contrib/oberdiek/refcount.dtx}] The source file.
% \item[\CTAN{macros/latex/contrib/oberdiek/refcount.pdf}] Documentation.
% \end{description}
%
%
% \paragraph{Bundle.} All the packages of the bundle `oberdiek'
% are also available in a TDS compliant ZIP archive. There
% the packages are already unpacked and the documentation files
% are generated. The files and directories obey the TDS standard.
% \begin{description}
% \item[\CTAN{install/macros/latex/contrib/oberdiek.tds.zip}]
% \end{description}
% \emph{TDS} refers to the standard ``A Directory Structure
% for \TeX\ Files'' (\CTAN{tds/tds.pdf}). Directories
% with \xfile{texmf} in their name are usually organized this way.
%
% \subsection{Bundle installation}
%
% \paragraph{Unpacking.} Unpack the \xfile{oberdiek.tds.zip} in the
% TDS tree (also known as \xfile{texmf} tree) of your choice.
% Example (linux):
% \begin{quote}
%   |unzip oberdiek.tds.zip -d ~/texmf|
% \end{quote}
%
% \paragraph{Script installation.}
% Check the directory \xfile{TDS:scripts/oberdiek/} for
% scripts that need further installation steps.
% Package \xpackage{attachfile2} comes with the Perl script
% \xfile{pdfatfi.pl} that should be installed in such a way
% that it can be called as \texttt{pdfatfi}.
% Example (linux):
% \begin{quote}
%   |chmod +x scripts/oberdiek/pdfatfi.pl|\\
%   |cp scripts/oberdiek/pdfatfi.pl /usr/local/bin/|
% \end{quote}
%
% \subsection{Package installation}
%
% \paragraph{Unpacking.} The \xfile{.dtx} file is a self-extracting
% \docstrip\ archive. The files are extracted by running the
% \xfile{.dtx} through \plainTeX:
% \begin{quote}
%   \verb|tex refcount.dtx|
% \end{quote}
%
% \paragraph{TDS.} Now the different files must be moved into
% the different directories in your installation TDS tree
% (also known as \xfile{texmf} tree):
% \begin{quote}
% \def\t{^^A
% \begin{tabular}{@{}>{\ttfamily}l@{ $\rightarrow$ }>{\ttfamily}l@{}}
%   refcount.sty & tex/latex/oberdiek/refcount.sty\\
%   refcount.pdf & doc/latex/oberdiek/refcount.pdf\\
%   refcount.dtx & source/latex/oberdiek/refcount.dtx\\
% \end{tabular}^^A
% }^^A
% \sbox0{\t}^^A
% \ifdim\wd0>\linewidth
%   \begingroup
%     \advance\linewidth by\leftmargin
%     \advance\linewidth by\rightmargin
%   \edef\x{\endgroup
%     \def\noexpand\lw{\the\linewidth}^^A
%   }\x
%   \def\lwbox{^^A
%     \leavevmode
%     \hbox to \linewidth{^^A
%       \kern-\leftmargin\relax
%       \hss
%       \usebox0
%       \hss
%       \kern-\rightmargin\relax
%     }^^A
%   }^^A
%   \ifdim\wd0>\lw
%     \sbox0{\small\t}^^A
%     \ifdim\wd0>\linewidth
%       \ifdim\wd0>\lw
%         \sbox0{\footnotesize\t}^^A
%         \ifdim\wd0>\linewidth
%           \ifdim\wd0>\lw
%             \sbox0{\scriptsize\t}^^A
%             \ifdim\wd0>\linewidth
%               \ifdim\wd0>\lw
%                 \sbox0{\tiny\t}^^A
%                 \ifdim\wd0>\linewidth
%                   \lwbox
%                 \else
%                   \usebox0
%                 \fi
%               \else
%                 \lwbox
%               \fi
%             \else
%               \usebox0
%             \fi
%           \else
%             \lwbox
%           \fi
%         \else
%           \usebox0
%         \fi
%       \else
%         \lwbox
%       \fi
%     \else
%       \usebox0
%     \fi
%   \else
%     \lwbox
%   \fi
% \else
%   \usebox0
% \fi
% \end{quote}
% If you have a \xfile{docstrip.cfg} that configures and enables \docstrip's
% TDS installing feature, then some files can already be in the right
% place, see the documentation of \docstrip.
%
% \subsection{Refresh file name databases}
%
% If your \TeX~distribution
% (\teTeX, \mikTeX, \dots) relies on file name databases, you must refresh
% these. For example, \teTeX\ users run \verb|texhash| or
% \verb|mktexlsr|.
%
% \subsection{Some details for the interested}
%
% \paragraph{Attached source.}
%
% The PDF documentation on CTAN also includes the
% \xfile{.dtx} source file. It can be extracted by
% AcrobatReader 6 or higher. Another option is \textsf{pdftk},
% e.g. unpack the file into the current directory:
% \begin{quote}
%   \verb|pdftk refcount.pdf unpack_files output .|
% \end{quote}
%
% \paragraph{Unpacking with \LaTeX.}
% The \xfile{.dtx} chooses its action depending on the format:
% \begin{description}
% \item[\plainTeX:] Run \docstrip\ and extract the files.
% \item[\LaTeX:] Generate the documentation.
% \end{description}
% If you insist on using \LaTeX\ for \docstrip\ (really,
% \docstrip\ does not need \LaTeX), then inform the autodetect routine
% about your intention:
% \begin{quote}
%   \verb|latex \let\install=y\input{refcount.dtx}|
% \end{quote}
% Do not forget to quote the argument according to the demands
% of your shell.
%
% \paragraph{Generating the documentation.}
% You can use both the \xfile{.dtx} or the \xfile{.drv} to generate
% the documentation. The process can be configured by the
% configuration file \xfile{ltxdoc.cfg}. For instance, put this
% line into this file, if you want to have A4 as paper format:
% \begin{quote}
%   \verb|\PassOptionsToClass{a4paper}{article}|
% \end{quote}
% An example follows how to generate the
% documentation with pdf\LaTeX:
% \begin{quote}
%\begin{verbatim}
%pdflatex refcount.dtx
%makeindex -s gind.ist refcount.idx
%pdflatex refcount.dtx
%makeindex -s gind.ist refcount.idx
%pdflatex refcount.dtx
%\end{verbatim}
% \end{quote}
%
% \begin{History}
%   \begin{Version}{1998/04/08 v1.0}
%   \item
%     First public release, written as answer in the
%     newsgroup \xnewsgroup{comp.text.tex}:
%     \URL{``\link{Re: Adding a \cs{ref} to a counter?}''}^^A
%     {http://groups.google.com/group/comp.text.tex/msg/c3f2a135ef5ee528}
%   \end{Version}
%   \begin{Version}{2000/09/07 v2.0}
%   \item
%     Documentation added.
%   \item
%     LPPL 1.2
%   \item
%     Package rewritten, new commands added.
%   \end{Version}
%   \begin{Version}{2006/02/20 v3.0}
%   \item
%     Support for \xpackage{hyperref} and \xpackage{nameref} improved.
%   \item
%     Support for \xpackage{titleref} and \xpackage{babel}'s shorthands added.
%   \item
%     New: \cs{refused}, \cs{getrefbykeydefault}
%   \end{Version}
%   \begin{Version}{2008/08/11 v3.1}
%   \item
%     Code is not changed.
%   \item
%     URLs updated.
%   \end{Version}
% \end{History}
%
% \PrintIndex
%
% \Finale
\endinput

%        (quote the arguments according to the demands of your shell)
%
% Documentation:
%    (a) If refcount.drv is present:
%           latex refcount.drv
%    (b) Without refcount.drv:
%           latex refcount.dtx; ...
%    The class ltxdoc loads the configuration file ltxdoc.cfg
%    if available. Here you can specify further options, e.g.
%    use A4 as paper format:
%       \PassOptionsToClass{a4paper}{article}
%
%    Programm calls to get the documentation (example):
%       pdflatex refcount.dtx
%       makeindex -s gind.ist refcount.idx
%       pdflatex refcount.dtx
%       makeindex -s gind.ist refcount.idx
%       pdflatex refcount.dtx
%
% Installation:
%    TDS:tex/latex/oberdiek/refcount.sty
%    TDS:doc/latex/oberdiek/refcount.pdf
%    TDS:source/latex/oberdiek/refcount.dtx
%
%<*ignore>
\begingroup
  \def\x{LaTeX2e}%
\expandafter\endgroup
\ifcase 0\ifx\install y1\fi\expandafter
         \ifx\csname processbatchFile\endcsname\relax\else1\fi
         \ifx\fmtname\x\else 1\fi\relax
\else\csname fi\endcsname
%</ignore>
%<*install>
\input docstrip.tex
\Msg{************************************************************************}
\Msg{* Installation}
\Msg{* Package: refcount 2008/08/11 v3.1 Data extraction from references (HO)}
\Msg{************************************************************************}

\keepsilent
\askforoverwritefalse

\let\MetaPrefix\relax
\preamble

This is a generated file.

Copyright (C) 1998, 2000, 2006, 2008 by
   Heiko Oberdiek <oberdiek@uni-freiburg.de>

This work may be distributed and/or modified under the
conditions of the LaTeX Project Public License, either
version 1.3 of this license or (at your option) any later
version. The latest version of this license is in
   http://www.latex-project.org/lppl.txt
and version 1.3 or later is part of all distributions of
LaTeX version 2005/12/01 or later.

This work has the LPPL maintenance status "maintained".

This Current Maintainer of this work is Heiko Oberdiek.

This work consists of the main source file refcount.dtx
and the derived files
   refcount.sty, refcount.pdf, refcount.ins, refcount.drv.

\endpreamble
\let\MetaPrefix\DoubleperCent

\generate{%
  \file{refcount.ins}{\from{refcount.dtx}{install}}%
  \file{refcount.drv}{\from{refcount.dtx}{driver}}%
  \usedir{tex/latex/oberdiek}%
  \file{refcount.sty}{\from{refcount.dtx}{package}}%
}

\obeyspaces
\Msg{************************************************************************}
\Msg{*}
\Msg{* To finish the installation you have to move the following}
\Msg{* file into a directory searched by TeX:}
\Msg{*}
\Msg{*     refcount.sty}
\Msg{*}
\Msg{* And install the following script file:}
\Msg{*}
\Msg{*     }
\Msg{*}
\Msg{* To produce the documentation run the file `refcount.drv'}
\Msg{* through LaTeX.}
\Msg{*}
\Msg{* Happy TeXing!}
\Msg{*}
\Msg{************************************************************************}

\endbatchfile
%</install>
%<*ignore>
\fi
%</ignore>
%<*driver>
\NeedsTeXFormat{LaTeX2e}
\ProvidesFile{refcount.drv}%
  [2008/08/11 v3.1 Data extraction from references (HO)]%
\documentclass{ltxdoc}
\usepackage{holtxdoc}[2008/08/11]
\begin{document}
  \DocInput{refcount.dtx}%
\end{document}
%</driver>
% \fi
%
% \CheckSum{198}
%
% \CharacterTable
%  {Upper-case    \A\B\C\D\E\F\G\H\I\J\K\L\M\N\O\P\Q\R\S\T\U\V\W\X\Y\Z
%   Lower-case    \a\b\c\d\e\f\g\h\i\j\k\l\m\n\o\p\q\r\s\t\u\v\w\x\y\z
%   Digits        \0\1\2\3\4\5\6\7\8\9
%   Exclamation   \!     Double quote  \"     Hash (number) \#
%   Dollar        \$     Percent       \%     Ampersand     \&
%   Acute accent  \'     Left paren    \(     Right paren   \)
%   Asterisk      \*     Plus          \+     Comma         \,
%   Minus         \-     Point         \.     Solidus       \/
%   Colon         \:     Semicolon     \;     Less than     \<
%   Equals        \=     Greater than  \>     Question mark \?
%   Commercial at \@     Left bracket  \[     Backslash     \\
%   Right bracket \]     Circumflex    \^     Underscore    \_
%   Grave accent  \`     Left brace    \{     Vertical bar  \|
%   Right brace   \}     Tilde         \~}
%
% \GetFileInfo{refcount.drv}
%
% \title{The \xpackage{refcount} package}
% \date{2008/08/11 v3.1}
% \author{Heiko Oberdiek\\\xemail{oberdiek@uni-freiburg.de}}
%
% \maketitle
%
% \begin{abstract}
% References are not numbers, however they often store numerical
% data such as section or page numbers. \cs{ref} or \cs{pageref}
% cannot be used for counter assignments or calculations because
% they are not expandable, generate warnings, or can even be links,
% The package provides expandable macros to extract the data
% from references. Packages \xpackage{hyperref}, \xpackage{nameref},
% \xpackage{titleref}, and \xpackage{babel} are supported.
% \end{abstract}
%
% \tableofcontents
%
% \section{Usage}
%
% \subsection{Setting counters}
%
% The following commands are similar to \LaTeX's
% \cs{setcounter} and \cs{addtocounter},
% but they extract the number value from a reference:
% \begin{quote}
%   \cs{setcounterref}, \cs{addtocounterref}\\
%   \cs{setcounterpageref}, \cs{addtocounterpageref}
% \end{quote}
% They take two arguments:
% \begin{quote}
%    \cs{...counter...ref} |{|\meta{\LaTeX\ counter}|}|
%    |{|\meta{reference}|}|
% \end{quote}
% An undefined references produces the usual LaTeX warning
% and its value is assumed to be zero.
% Example:
% \begin{quote}
%\begin{verbatim}
%\newcounter{ctrA}
%\newcounter{ctrB}
%\refstepcounter{ctrA}\label{ref:A}
%\setcounterref{ctrB}{ref:A}
%\addtocounterpageref{ctrB}{ref:A}
%\end{verbatim}
% \end{quote}
%
% \subsection{Expandable commands}
%
% These commands that can be used in expandible contexts
% (inside calculations, \cs{edef}, \cs{csname}, \cs{write}, \dots):
% \begin{quote}
%   \cs{getrefnumber}, \cs{getpagerefnumber}
% \end{quote}
% They take one argument, the reference:
% \begin{quote}
%   \cs{get...refnumber} |{|\meta{reference}|}|
% \end{quote}
% The default for undefined references can be changed
% with macro \cs{setrefcountdefault}, for example this
% package calls:
% \begin{quote}
%   \cs{setrefcountdefault}|{0}|
% \end{quote}
%
% Since version 2.0 of this package there is a new
% command:
% \begin{quote}
%   \cs{getrefbykeydefault} |{|\meta{reference}|}|
%   |{|\meta{key}|}| |{|\meta{default}|}|
% \end{quote}
% This generalized version allows the extraction
% of further properties of a reference than the
% two standard ones. Thus the following properties
% are supported, if they are available:
% \begin{quote}
% \begin{tabular}{@{}l|l|l@{}}
%    Key & Description & Package\\
% \hline
%   \meta{empty} & same as \cs{ref} & \LaTeX\\
%   |page| & same as \cs{pageref} & \LaTeX\\
%   |title| & section and caption titles & \xpackage{titleref}\\
%   |name| & section and caption titles & \xpackage{nameref}\\
%   |anchor| & anchor name & \xpackage{hyperref}\\
%   |url| & url/file & \xpackage{hyperref}/\xpackage{xr}
% \end{tabular}
% \end{quote}
%
% \subsection{Undefined references}
%
% Because warnings and assignments cannot be used in
% expandible contexts, undefined references do not
% produce a warning, their values are assumed to be zero.
% Example:
% \begin{quote}
%\begin{verbatim}
%\label{ref:here}% somewhere
%\refused{ref:here}% see below
%\ifodd\getpagerefnumber{ref:here}%
%  reference is on an odd page
%\else
%  reference is on an even page
%\fi
%\end{verbatim}
% \end{quote}
%
% In case of undefined references the user usually want's
% to be informed. Also \LaTeX\ prints a warning at
% the end of the \LaTeX\ run. To notify \LaTeX\ and
% get a normal warning, just use
% \begin{quote}
%   \cs{refused} |{|\meta{reference}|}|
% \end{quote}
% outside the expanding context. Example, see above.
%
% \subsection{Notes}
%
% \begin{itemize}
% \item
%   The method of extracting the number in this
%   package also works in cases, where the
%   reference cannot be used directly, because
%   a package such as \xpackage{hyperref} has added
%   extra stuff (hyper link), so that the reference cannot
%   be used as number any more.
% \item
%   If the reference does not contain a number,
%   assignments to a counter will fail of course.
% \end{itemize}
%
%
% \StopEventually{
% }
%
% \section{Implementation}
%
%    \begin{macrocode}
%<*package>
\NeedsTeXFormat{LaTeX2e}
\ProvidesPackage{refcount}
  [2008/08/11 v3.1 Data extraction from references (HO)]%

\def\setrefcountdefault#1{%
  \def\rc@default{#1}%
}
\setrefcountdefault{0}

% \def\@car#1#2\@nil{#1} % defined in LaTeX kernel
\def\rc@cartwo#1#2#3\@nil{#2}

% generic check without babel support
\long\def\rc@refused#1{%
  \expandafter\ifx\csname r@#1\endcsname\relax
    \protect\G@refundefinedtrue
    \@latex@warning{%
      Reference `#1' on page \thepage\space undefined%
    }%
  \fi
}

% user command, add babel support
\newcommand*{\refused}[1]{%
  \begingroup
    \csname @safe@activestrue\endcsname
    \rc@refused{#1}{}%
  \endgroup
}

% Generic command for "\{set,addto}counter{page,}ref":
% #1: \setcounter, \addtocounter
% #2: \@car (for \ref), \@cartwo (for \pageref)
% #3: LaTeX counter
% #4: reference
\def\rc@set#1#2#3#4{%
  \begingroup
    \csname @safe@activestrue\endcsname
    \rc@refused{#4}%
    \expandafter\rc@@set\csname r@#4\endcsname{#1}{#2}{#3}%
  \endgroup
}
% #1: \r@<...>
% #2: \setcounter, \addtocounter
% #3: \@car (for \ref), \@cartwo (for \pageref)
% #4: LaTeX counter
\def\rc@@set#1#2#3#4{%
  \ifx#1\relax
    #2{#4}{\rc@default}%
  \else
    #2{#4}{%
      \expandafter#3#1\rc@default\rc@default\@nil
    }%
  \fi
}

% user commands:

\newcommand*{\setcounterref}{\rc@set\setcounter\@car}
\newcommand*{\addtocounterref}{\rc@set\addtocounter\@car}
\newcommand*{\setcounterpageref}{\rc@set\setcounter\rc@cartwo}
\newcommand*{\addtocounterpageref}{\rc@set\addtocounter\rc@cartwo}

\newcommand*{\getrefnumber}[1]{%
  \expandafter\ifx\csname r@#1\endcsname\relax
    \rc@default
  \else
    \expandafter\expandafter\expandafter\@car
    \csname r@#1\endcsname\@nil
  \fi
}
\newcommand*{\getpagerefnumber}[1]{%
  \expandafter\ifx\csname r@#1\endcsname\relax
    \rc@default
  \else
    \expandafter\expandafter\expandafter\rc@cartwo
    \csname r@#1\endcsname\rc@default\rc@default\@nil
  \fi
}
\newcommand*{\getrefbykeydefault}[2]{%
  \expandafter\rc@getrefbykeydefault
    \csname r@#1\expandafter\endcsname
    \csname rc@extract@#2\endcsname
}
% #1: \r@<...>
% #2: \rc@extract@<...>
% #3: default
\def\rc@getrefbykeydefault#1#2#3{%
  \ifx#1\relax
    % reference is undefined
    #3%
  \else
    \ifx#2\relax
      % extract method is missing
      #3%
    \else
      \expandafter\rc@generic#1{#3}{#3}{#3}{#3}{#3}\@nil#2{#3}%
    \fi
  \fi
}
% #1: first item in \r@<...>
% #2: remaining items in \r@<...>
% #3: \rc@extract@<...>
% #4: default
\def\rc@generic#1#2\@nil#3#4{%
  #3{#1\TR@TitleReference\@empty{#4}\@nil}{#1}#2\@nil
}
\def\rc@extract@{%
  \expandafter\@car\@gobble
}
\def\rc@extract@page{%
  \expandafter\@car\@gobbletwo
}
\def\rc@extract@name{%
  \expandafter\@car\@gobblefour\@empty
}
\def\rc@extract@anchor{%
  \expandafter\@car\@gobblefour
}
\def\rc@extract@url{%
  \expandafter\expandafter\expandafter\@car\expandafter
      \@gobble\@gobblefour
}
\def\rc@extract@title#1#2\@nil{%
  \rc@@extract@title#1%
}
\def\rc@@extract@title#1\TR@TitleReference#2#3#4\@nil{#3}
%</package>
%    \end{macrocode}
%
% \section{Installation}
%
% \subsection{Download}
%
% \paragraph{Package.} This package is available on
% CTAN\footnote{\url{ftp://ftp.ctan.org/tex-archive/}}:
% \begin{description}
% \item[\CTAN{macros/latex/contrib/oberdiek/refcount.dtx}] The source file.
% \item[\CTAN{macros/latex/contrib/oberdiek/refcount.pdf}] Documentation.
% \end{description}
%
%
% \paragraph{Bundle.} All the packages of the bundle `oberdiek'
% are also available in a TDS compliant ZIP archive. There
% the packages are already unpacked and the documentation files
% are generated. The files and directories obey the TDS standard.
% \begin{description}
% \item[\CTAN{install/macros/latex/contrib/oberdiek.tds.zip}]
% \end{description}
% \emph{TDS} refers to the standard ``A Directory Structure
% for \TeX\ Files'' (\CTAN{tds/tds.pdf}). Directories
% with \xfile{texmf} in their name are usually organized this way.
%
% \subsection{Bundle installation}
%
% \paragraph{Unpacking.} Unpack the \xfile{oberdiek.tds.zip} in the
% TDS tree (also known as \xfile{texmf} tree) of your choice.
% Example (linux):
% \begin{quote}
%   |unzip oberdiek.tds.zip -d ~/texmf|
% \end{quote}
%
% \paragraph{Script installation.}
% Check the directory \xfile{TDS:scripts/oberdiek/} for
% scripts that need further installation steps.
% Package \xpackage{attachfile2} comes with the Perl script
% \xfile{pdfatfi.pl} that should be installed in such a way
% that it can be called as \texttt{pdfatfi}.
% Example (linux):
% \begin{quote}
%   |chmod +x scripts/oberdiek/pdfatfi.pl|\\
%   |cp scripts/oberdiek/pdfatfi.pl /usr/local/bin/|
% \end{quote}
%
% \subsection{Package installation}
%
% \paragraph{Unpacking.} The \xfile{.dtx} file is a self-extracting
% \docstrip\ archive. The files are extracted by running the
% \xfile{.dtx} through \plainTeX:
% \begin{quote}
%   \verb|tex refcount.dtx|
% \end{quote}
%
% \paragraph{TDS.} Now the different files must be moved into
% the different directories in your installation TDS tree
% (also known as \xfile{texmf} tree):
% \begin{quote}
% \def\t{^^A
% \begin{tabular}{@{}>{\ttfamily}l@{ $\rightarrow$ }>{\ttfamily}l@{}}
%   refcount.sty & tex/latex/oberdiek/refcount.sty\\
%   refcount.pdf & doc/latex/oberdiek/refcount.pdf\\
%   refcount.dtx & source/latex/oberdiek/refcount.dtx\\
% \end{tabular}^^A
% }^^A
% \sbox0{\t}^^A
% \ifdim\wd0>\linewidth
%   \begingroup
%     \advance\linewidth by\leftmargin
%     \advance\linewidth by\rightmargin
%   \edef\x{\endgroup
%     \def\noexpand\lw{\the\linewidth}^^A
%   }\x
%   \def\lwbox{^^A
%     \leavevmode
%     \hbox to \linewidth{^^A
%       \kern-\leftmargin\relax
%       \hss
%       \usebox0
%       \hss
%       \kern-\rightmargin\relax
%     }^^A
%   }^^A
%   \ifdim\wd0>\lw
%     \sbox0{\small\t}^^A
%     \ifdim\wd0>\linewidth
%       \ifdim\wd0>\lw
%         \sbox0{\footnotesize\t}^^A
%         \ifdim\wd0>\linewidth
%           \ifdim\wd0>\lw
%             \sbox0{\scriptsize\t}^^A
%             \ifdim\wd0>\linewidth
%               \ifdim\wd0>\lw
%                 \sbox0{\tiny\t}^^A
%                 \ifdim\wd0>\linewidth
%                   \lwbox
%                 \else
%                   \usebox0
%                 \fi
%               \else
%                 \lwbox
%               \fi
%             \else
%               \usebox0
%             \fi
%           \else
%             \lwbox
%           \fi
%         \else
%           \usebox0
%         \fi
%       \else
%         \lwbox
%       \fi
%     \else
%       \usebox0
%     \fi
%   \else
%     \lwbox
%   \fi
% \else
%   \usebox0
% \fi
% \end{quote}
% If you have a \xfile{docstrip.cfg} that configures and enables \docstrip's
% TDS installing feature, then some files can already be in the right
% place, see the documentation of \docstrip.
%
% \subsection{Refresh file name databases}
%
% If your \TeX~distribution
% (\teTeX, \mikTeX, \dots) relies on file name databases, you must refresh
% these. For example, \teTeX\ users run \verb|texhash| or
% \verb|mktexlsr|.
%
% \subsection{Some details for the interested}
%
% \paragraph{Attached source.}
%
% The PDF documentation on CTAN also includes the
% \xfile{.dtx} source file. It can be extracted by
% AcrobatReader 6 or higher. Another option is \textsf{pdftk},
% e.g. unpack the file into the current directory:
% \begin{quote}
%   \verb|pdftk refcount.pdf unpack_files output .|
% \end{quote}
%
% \paragraph{Unpacking with \LaTeX.}
% The \xfile{.dtx} chooses its action depending on the format:
% \begin{description}
% \item[\plainTeX:] Run \docstrip\ and extract the files.
% \item[\LaTeX:] Generate the documentation.
% \end{description}
% If you insist on using \LaTeX\ for \docstrip\ (really,
% \docstrip\ does not need \LaTeX), then inform the autodetect routine
% about your intention:
% \begin{quote}
%   \verb|latex \let\install=y% \iffalse meta-comment
%
% Copyright (C) 1998, 2000, 2006, 2008 by
%    Heiko Oberdiek <oberdiek@uni-freiburg.de>
%
% This work may be distributed and/or modified under the
% conditions of the LaTeX Project Public License, either
% version 1.3 of this license or (at your option) any later
% version. The latest version of this license is in
%    http://www.latex-project.org/lppl.txt
% and version 1.3 or later is part of all distributions of
% LaTeX version 2005/12/01 or later.
%
% This work has the LPPL maintenance status "maintained".
%
% This Current Maintainer of this work is Heiko Oberdiek.
%
% This work consists of the main source file refcount.dtx
% and the derived files
%    refcount.sty, refcount.pdf, refcount.ins, refcount.drv.
%
% Distribution:
%    CTAN:macros/latex/contrib/oberdiek/refcount.dtx
%    CTAN:macros/latex/contrib/oberdiek/refcount.pdf
%
% Unpacking:
%    (a) If refcount.ins is present:
%           tex refcount.ins
%    (b) Without refcount.ins:
%           tex refcount.dtx
%    (c) If you insist on using LaTeX
%           latex \let\install=y\input{refcount.dtx}
%        (quote the arguments according to the demands of your shell)
%
% Documentation:
%    (a) If refcount.drv is present:
%           latex refcount.drv
%    (b) Without refcount.drv:
%           latex refcount.dtx; ...
%    The class ltxdoc loads the configuration file ltxdoc.cfg
%    if available. Here you can specify further options, e.g.
%    use A4 as paper format:
%       \PassOptionsToClass{a4paper}{article}
%
%    Programm calls to get the documentation (example):
%       pdflatex refcount.dtx
%       makeindex -s gind.ist refcount.idx
%       pdflatex refcount.dtx
%       makeindex -s gind.ist refcount.idx
%       pdflatex refcount.dtx
%
% Installation:
%    TDS:tex/latex/oberdiek/refcount.sty
%    TDS:doc/latex/oberdiek/refcount.pdf
%    TDS:source/latex/oberdiek/refcount.dtx
%
%<*ignore>
\begingroup
  \def\x{LaTeX2e}%
\expandafter\endgroup
\ifcase 0\ifx\install y1\fi\expandafter
         \ifx\csname processbatchFile\endcsname\relax\else1\fi
         \ifx\fmtname\x\else 1\fi\relax
\else\csname fi\endcsname
%</ignore>
%<*install>
\input docstrip.tex
\Msg{************************************************************************}
\Msg{* Installation}
\Msg{* Package: refcount 2008/08/11 v3.1 Data extraction from references (HO)}
\Msg{************************************************************************}

\keepsilent
\askforoverwritefalse

\let\MetaPrefix\relax
\preamble

This is a generated file.

Copyright (C) 1998, 2000, 2006, 2008 by
   Heiko Oberdiek <oberdiek@uni-freiburg.de>

This work may be distributed and/or modified under the
conditions of the LaTeX Project Public License, either
version 1.3 of this license or (at your option) any later
version. The latest version of this license is in
   http://www.latex-project.org/lppl.txt
and version 1.3 or later is part of all distributions of
LaTeX version 2005/12/01 or later.

This work has the LPPL maintenance status "maintained".

This Current Maintainer of this work is Heiko Oberdiek.

This work consists of the main source file refcount.dtx
and the derived files
   refcount.sty, refcount.pdf, refcount.ins, refcount.drv.

\endpreamble
\let\MetaPrefix\DoubleperCent

\generate{%
  \file{refcount.ins}{\from{refcount.dtx}{install}}%
  \file{refcount.drv}{\from{refcount.dtx}{driver}}%
  \usedir{tex/latex/oberdiek}%
  \file{refcount.sty}{\from{refcount.dtx}{package}}%
}

\obeyspaces
\Msg{************************************************************************}
\Msg{*}
\Msg{* To finish the installation you have to move the following}
\Msg{* file into a directory searched by TeX:}
\Msg{*}
\Msg{*     refcount.sty}
\Msg{*}
\Msg{* And install the following script file:}
\Msg{*}
\Msg{*     }
\Msg{*}
\Msg{* To produce the documentation run the file `refcount.drv'}
\Msg{* through LaTeX.}
\Msg{*}
\Msg{* Happy TeXing!}
\Msg{*}
\Msg{************************************************************************}

\endbatchfile
%</install>
%<*ignore>
\fi
%</ignore>
%<*driver>
\NeedsTeXFormat{LaTeX2e}
\ProvidesFile{refcount.drv}%
  [2008/08/11 v3.1 Data extraction from references (HO)]%
\documentclass{ltxdoc}
\usepackage{holtxdoc}[2008/08/11]
\begin{document}
  \DocInput{refcount.dtx}%
\end{document}
%</driver>
% \fi
%
% \CheckSum{198}
%
% \CharacterTable
%  {Upper-case    \A\B\C\D\E\F\G\H\I\J\K\L\M\N\O\P\Q\R\S\T\U\V\W\X\Y\Z
%   Lower-case    \a\b\c\d\e\f\g\h\i\j\k\l\m\n\o\p\q\r\s\t\u\v\w\x\y\z
%   Digits        \0\1\2\3\4\5\6\7\8\9
%   Exclamation   \!     Double quote  \"     Hash (number) \#
%   Dollar        \$     Percent       \%     Ampersand     \&
%   Acute accent  \'     Left paren    \(     Right paren   \)
%   Asterisk      \*     Plus          \+     Comma         \,
%   Minus         \-     Point         \.     Solidus       \/
%   Colon         \:     Semicolon     \;     Less than     \<
%   Equals        \=     Greater than  \>     Question mark \?
%   Commercial at \@     Left bracket  \[     Backslash     \\
%   Right bracket \]     Circumflex    \^     Underscore    \_
%   Grave accent  \`     Left brace    \{     Vertical bar  \|
%   Right brace   \}     Tilde         \~}
%
% \GetFileInfo{refcount.drv}
%
% \title{The \xpackage{refcount} package}
% \date{2008/08/11 v3.1}
% \author{Heiko Oberdiek\\\xemail{oberdiek@uni-freiburg.de}}
%
% \maketitle
%
% \begin{abstract}
% References are not numbers, however they often store numerical
% data such as section or page numbers. \cs{ref} or \cs{pageref}
% cannot be used for counter assignments or calculations because
% they are not expandable, generate warnings, or can even be links,
% The package provides expandable macros to extract the data
% from references. Packages \xpackage{hyperref}, \xpackage{nameref},
% \xpackage{titleref}, and \xpackage{babel} are supported.
% \end{abstract}
%
% \tableofcontents
%
% \section{Usage}
%
% \subsection{Setting counters}
%
% The following commands are similar to \LaTeX's
% \cs{setcounter} and \cs{addtocounter},
% but they extract the number value from a reference:
% \begin{quote}
%   \cs{setcounterref}, \cs{addtocounterref}\\
%   \cs{setcounterpageref}, \cs{addtocounterpageref}
% \end{quote}
% They take two arguments:
% \begin{quote}
%    \cs{...counter...ref} |{|\meta{\LaTeX\ counter}|}|
%    |{|\meta{reference}|}|
% \end{quote}
% An undefined references produces the usual LaTeX warning
% and its value is assumed to be zero.
% Example:
% \begin{quote}
%\begin{verbatim}
%\newcounter{ctrA}
%\newcounter{ctrB}
%\refstepcounter{ctrA}\label{ref:A}
%\setcounterref{ctrB}{ref:A}
%\addtocounterpageref{ctrB}{ref:A}
%\end{verbatim}
% \end{quote}
%
% \subsection{Expandable commands}
%
% These commands that can be used in expandible contexts
% (inside calculations, \cs{edef}, \cs{csname}, \cs{write}, \dots):
% \begin{quote}
%   \cs{getrefnumber}, \cs{getpagerefnumber}
% \end{quote}
% They take one argument, the reference:
% \begin{quote}
%   \cs{get...refnumber} |{|\meta{reference}|}|
% \end{quote}
% The default for undefined references can be changed
% with macro \cs{setrefcountdefault}, for example this
% package calls:
% \begin{quote}
%   \cs{setrefcountdefault}|{0}|
% \end{quote}
%
% Since version 2.0 of this package there is a new
% command:
% \begin{quote}
%   \cs{getrefbykeydefault} |{|\meta{reference}|}|
%   |{|\meta{key}|}| |{|\meta{default}|}|
% \end{quote}
% This generalized version allows the extraction
% of further properties of a reference than the
% two standard ones. Thus the following properties
% are supported, if they are available:
% \begin{quote}
% \begin{tabular}{@{}l|l|l@{}}
%    Key & Description & Package\\
% \hline
%   \meta{empty} & same as \cs{ref} & \LaTeX\\
%   |page| & same as \cs{pageref} & \LaTeX\\
%   |title| & section and caption titles & \xpackage{titleref}\\
%   |name| & section and caption titles & \xpackage{nameref}\\
%   |anchor| & anchor name & \xpackage{hyperref}\\
%   |url| & url/file & \xpackage{hyperref}/\xpackage{xr}
% \end{tabular}
% \end{quote}
%
% \subsection{Undefined references}
%
% Because warnings and assignments cannot be used in
% expandible contexts, undefined references do not
% produce a warning, their values are assumed to be zero.
% Example:
% \begin{quote}
%\begin{verbatim}
%\label{ref:here}% somewhere
%\refused{ref:here}% see below
%\ifodd\getpagerefnumber{ref:here}%
%  reference is on an odd page
%\else
%  reference is on an even page
%\fi
%\end{verbatim}
% \end{quote}
%
% In case of undefined references the user usually want's
% to be informed. Also \LaTeX\ prints a warning at
% the end of the \LaTeX\ run. To notify \LaTeX\ and
% get a normal warning, just use
% \begin{quote}
%   \cs{refused} |{|\meta{reference}|}|
% \end{quote}
% outside the expanding context. Example, see above.
%
% \subsection{Notes}
%
% \begin{itemize}
% \item
%   The method of extracting the number in this
%   package also works in cases, where the
%   reference cannot be used directly, because
%   a package such as \xpackage{hyperref} has added
%   extra stuff (hyper link), so that the reference cannot
%   be used as number any more.
% \item
%   If the reference does not contain a number,
%   assignments to a counter will fail of course.
% \end{itemize}
%
%
% \StopEventually{
% }
%
% \section{Implementation}
%
%    \begin{macrocode}
%<*package>
\NeedsTeXFormat{LaTeX2e}
\ProvidesPackage{refcount}
  [2008/08/11 v3.1 Data extraction from references (HO)]%

\def\setrefcountdefault#1{%
  \def\rc@default{#1}%
}
\setrefcountdefault{0}

% \def\@car#1#2\@nil{#1} % defined in LaTeX kernel
\def\rc@cartwo#1#2#3\@nil{#2}

% generic check without babel support
\long\def\rc@refused#1{%
  \expandafter\ifx\csname r@#1\endcsname\relax
    \protect\G@refundefinedtrue
    \@latex@warning{%
      Reference `#1' on page \thepage\space undefined%
    }%
  \fi
}

% user command, add babel support
\newcommand*{\refused}[1]{%
  \begingroup
    \csname @safe@activestrue\endcsname
    \rc@refused{#1}{}%
  \endgroup
}

% Generic command for "\{set,addto}counter{page,}ref":
% #1: \setcounter, \addtocounter
% #2: \@car (for \ref), \@cartwo (for \pageref)
% #3: LaTeX counter
% #4: reference
\def\rc@set#1#2#3#4{%
  \begingroup
    \csname @safe@activestrue\endcsname
    \rc@refused{#4}%
    \expandafter\rc@@set\csname r@#4\endcsname{#1}{#2}{#3}%
  \endgroup
}
% #1: \r@<...>
% #2: \setcounter, \addtocounter
% #3: \@car (for \ref), \@cartwo (for \pageref)
% #4: LaTeX counter
\def\rc@@set#1#2#3#4{%
  \ifx#1\relax
    #2{#4}{\rc@default}%
  \else
    #2{#4}{%
      \expandafter#3#1\rc@default\rc@default\@nil
    }%
  \fi
}

% user commands:

\newcommand*{\setcounterref}{\rc@set\setcounter\@car}
\newcommand*{\addtocounterref}{\rc@set\addtocounter\@car}
\newcommand*{\setcounterpageref}{\rc@set\setcounter\rc@cartwo}
\newcommand*{\addtocounterpageref}{\rc@set\addtocounter\rc@cartwo}

\newcommand*{\getrefnumber}[1]{%
  \expandafter\ifx\csname r@#1\endcsname\relax
    \rc@default
  \else
    \expandafter\expandafter\expandafter\@car
    \csname r@#1\endcsname\@nil
  \fi
}
\newcommand*{\getpagerefnumber}[1]{%
  \expandafter\ifx\csname r@#1\endcsname\relax
    \rc@default
  \else
    \expandafter\expandafter\expandafter\rc@cartwo
    \csname r@#1\endcsname\rc@default\rc@default\@nil
  \fi
}
\newcommand*{\getrefbykeydefault}[2]{%
  \expandafter\rc@getrefbykeydefault
    \csname r@#1\expandafter\endcsname
    \csname rc@extract@#2\endcsname
}
% #1: \r@<...>
% #2: \rc@extract@<...>
% #3: default
\def\rc@getrefbykeydefault#1#2#3{%
  \ifx#1\relax
    % reference is undefined
    #3%
  \else
    \ifx#2\relax
      % extract method is missing
      #3%
    \else
      \expandafter\rc@generic#1{#3}{#3}{#3}{#3}{#3}\@nil#2{#3}%
    \fi
  \fi
}
% #1: first item in \r@<...>
% #2: remaining items in \r@<...>
% #3: \rc@extract@<...>
% #4: default
\def\rc@generic#1#2\@nil#3#4{%
  #3{#1\TR@TitleReference\@empty{#4}\@nil}{#1}#2\@nil
}
\def\rc@extract@{%
  \expandafter\@car\@gobble
}
\def\rc@extract@page{%
  \expandafter\@car\@gobbletwo
}
\def\rc@extract@name{%
  \expandafter\@car\@gobblefour\@empty
}
\def\rc@extract@anchor{%
  \expandafter\@car\@gobblefour
}
\def\rc@extract@url{%
  \expandafter\expandafter\expandafter\@car\expandafter
      \@gobble\@gobblefour
}
\def\rc@extract@title#1#2\@nil{%
  \rc@@extract@title#1%
}
\def\rc@@extract@title#1\TR@TitleReference#2#3#4\@nil{#3}
%</package>
%    \end{macrocode}
%
% \section{Installation}
%
% \subsection{Download}
%
% \paragraph{Package.} This package is available on
% CTAN\footnote{\url{ftp://ftp.ctan.org/tex-archive/}}:
% \begin{description}
% \item[\CTAN{macros/latex/contrib/oberdiek/refcount.dtx}] The source file.
% \item[\CTAN{macros/latex/contrib/oberdiek/refcount.pdf}] Documentation.
% \end{description}
%
%
% \paragraph{Bundle.} All the packages of the bundle `oberdiek'
% are also available in a TDS compliant ZIP archive. There
% the packages are already unpacked and the documentation files
% are generated. The files and directories obey the TDS standard.
% \begin{description}
% \item[\CTAN{install/macros/latex/contrib/oberdiek.tds.zip}]
% \end{description}
% \emph{TDS} refers to the standard ``A Directory Structure
% for \TeX\ Files'' (\CTAN{tds/tds.pdf}). Directories
% with \xfile{texmf} in their name are usually organized this way.
%
% \subsection{Bundle installation}
%
% \paragraph{Unpacking.} Unpack the \xfile{oberdiek.tds.zip} in the
% TDS tree (also known as \xfile{texmf} tree) of your choice.
% Example (linux):
% \begin{quote}
%   |unzip oberdiek.tds.zip -d ~/texmf|
% \end{quote}
%
% \paragraph{Script installation.}
% Check the directory \xfile{TDS:scripts/oberdiek/} for
% scripts that need further installation steps.
% Package \xpackage{attachfile2} comes with the Perl script
% \xfile{pdfatfi.pl} that should be installed in such a way
% that it can be called as \texttt{pdfatfi}.
% Example (linux):
% \begin{quote}
%   |chmod +x scripts/oberdiek/pdfatfi.pl|\\
%   |cp scripts/oberdiek/pdfatfi.pl /usr/local/bin/|
% \end{quote}
%
% \subsection{Package installation}
%
% \paragraph{Unpacking.} The \xfile{.dtx} file is a self-extracting
% \docstrip\ archive. The files are extracted by running the
% \xfile{.dtx} through \plainTeX:
% \begin{quote}
%   \verb|tex refcount.dtx|
% \end{quote}
%
% \paragraph{TDS.} Now the different files must be moved into
% the different directories in your installation TDS tree
% (also known as \xfile{texmf} tree):
% \begin{quote}
% \def\t{^^A
% \begin{tabular}{@{}>{\ttfamily}l@{ $\rightarrow$ }>{\ttfamily}l@{}}
%   refcount.sty & tex/latex/oberdiek/refcount.sty\\
%   refcount.pdf & doc/latex/oberdiek/refcount.pdf\\
%   refcount.dtx & source/latex/oberdiek/refcount.dtx\\
% \end{tabular}^^A
% }^^A
% \sbox0{\t}^^A
% \ifdim\wd0>\linewidth
%   \begingroup
%     \advance\linewidth by\leftmargin
%     \advance\linewidth by\rightmargin
%   \edef\x{\endgroup
%     \def\noexpand\lw{\the\linewidth}^^A
%   }\x
%   \def\lwbox{^^A
%     \leavevmode
%     \hbox to \linewidth{^^A
%       \kern-\leftmargin\relax
%       \hss
%       \usebox0
%       \hss
%       \kern-\rightmargin\relax
%     }^^A
%   }^^A
%   \ifdim\wd0>\lw
%     \sbox0{\small\t}^^A
%     \ifdim\wd0>\linewidth
%       \ifdim\wd0>\lw
%         \sbox0{\footnotesize\t}^^A
%         \ifdim\wd0>\linewidth
%           \ifdim\wd0>\lw
%             \sbox0{\scriptsize\t}^^A
%             \ifdim\wd0>\linewidth
%               \ifdim\wd0>\lw
%                 \sbox0{\tiny\t}^^A
%                 \ifdim\wd0>\linewidth
%                   \lwbox
%                 \else
%                   \usebox0
%                 \fi
%               \else
%                 \lwbox
%               \fi
%             \else
%               \usebox0
%             \fi
%           \else
%             \lwbox
%           \fi
%         \else
%           \usebox0
%         \fi
%       \else
%         \lwbox
%       \fi
%     \else
%       \usebox0
%     \fi
%   \else
%     \lwbox
%   \fi
% \else
%   \usebox0
% \fi
% \end{quote}
% If you have a \xfile{docstrip.cfg} that configures and enables \docstrip's
% TDS installing feature, then some files can already be in the right
% place, see the documentation of \docstrip.
%
% \subsection{Refresh file name databases}
%
% If your \TeX~distribution
% (\teTeX, \mikTeX, \dots) relies on file name databases, you must refresh
% these. For example, \teTeX\ users run \verb|texhash| or
% \verb|mktexlsr|.
%
% \subsection{Some details for the interested}
%
% \paragraph{Attached source.}
%
% The PDF documentation on CTAN also includes the
% \xfile{.dtx} source file. It can be extracted by
% AcrobatReader 6 or higher. Another option is \textsf{pdftk},
% e.g. unpack the file into the current directory:
% \begin{quote}
%   \verb|pdftk refcount.pdf unpack_files output .|
% \end{quote}
%
% \paragraph{Unpacking with \LaTeX.}
% The \xfile{.dtx} chooses its action depending on the format:
% \begin{description}
% \item[\plainTeX:] Run \docstrip\ and extract the files.
% \item[\LaTeX:] Generate the documentation.
% \end{description}
% If you insist on using \LaTeX\ for \docstrip\ (really,
% \docstrip\ does not need \LaTeX), then inform the autodetect routine
% about your intention:
% \begin{quote}
%   \verb|latex \let\install=y\input{refcount.dtx}|
% \end{quote}
% Do not forget to quote the argument according to the demands
% of your shell.
%
% \paragraph{Generating the documentation.}
% You can use both the \xfile{.dtx} or the \xfile{.drv} to generate
% the documentation. The process can be configured by the
% configuration file \xfile{ltxdoc.cfg}. For instance, put this
% line into this file, if you want to have A4 as paper format:
% \begin{quote}
%   \verb|\PassOptionsToClass{a4paper}{article}|
% \end{quote}
% An example follows how to generate the
% documentation with pdf\LaTeX:
% \begin{quote}
%\begin{verbatim}
%pdflatex refcount.dtx
%makeindex -s gind.ist refcount.idx
%pdflatex refcount.dtx
%makeindex -s gind.ist refcount.idx
%pdflatex refcount.dtx
%\end{verbatim}
% \end{quote}
%
% \begin{History}
%   \begin{Version}{1998/04/08 v1.0}
%   \item
%     First public release, written as answer in the
%     newsgroup \xnewsgroup{comp.text.tex}:
%     \URL{``\link{Re: Adding a \cs{ref} to a counter?}''}^^A
%     {http://groups.google.com/group/comp.text.tex/msg/c3f2a135ef5ee528}
%   \end{Version}
%   \begin{Version}{2000/09/07 v2.0}
%   \item
%     Documentation added.
%   \item
%     LPPL 1.2
%   \item
%     Package rewritten, new commands added.
%   \end{Version}
%   \begin{Version}{2006/02/20 v3.0}
%   \item
%     Support for \xpackage{hyperref} and \xpackage{nameref} improved.
%   \item
%     Support for \xpackage{titleref} and \xpackage{babel}'s shorthands added.
%   \item
%     New: \cs{refused}, \cs{getrefbykeydefault}
%   \end{Version}
%   \begin{Version}{2008/08/11 v3.1}
%   \item
%     Code is not changed.
%   \item
%     URLs updated.
%   \end{Version}
% \end{History}
%
% \PrintIndex
%
% \Finale
\endinput
|
% \end{quote}
% Do not forget to quote the argument according to the demands
% of your shell.
%
% \paragraph{Generating the documentation.}
% You can use both the \xfile{.dtx} or the \xfile{.drv} to generate
% the documentation. The process can be configured by the
% configuration file \xfile{ltxdoc.cfg}. For instance, put this
% line into this file, if you want to have A4 as paper format:
% \begin{quote}
%   \verb|\PassOptionsToClass{a4paper}{article}|
% \end{quote}
% An example follows how to generate the
% documentation with pdf\LaTeX:
% \begin{quote}
%\begin{verbatim}
%pdflatex refcount.dtx
%makeindex -s gind.ist refcount.idx
%pdflatex refcount.dtx
%makeindex -s gind.ist refcount.idx
%pdflatex refcount.dtx
%\end{verbatim}
% \end{quote}
%
% \begin{History}
%   \begin{Version}{1998/04/08 v1.0}
%   \item
%     First public release, written as answer in the
%     newsgroup \xnewsgroup{comp.text.tex}:
%     \URL{``\link{Re: Adding a \cs{ref} to a counter?}''}^^A
%     {http://groups.google.com/group/comp.text.tex/msg/c3f2a135ef5ee528}
%   \end{Version}
%   \begin{Version}{2000/09/07 v2.0}
%   \item
%     Documentation added.
%   \item
%     LPPL 1.2
%   \item
%     Package rewritten, new commands added.
%   \end{Version}
%   \begin{Version}{2006/02/20 v3.0}
%   \item
%     Support for \xpackage{hyperref} and \xpackage{nameref} improved.
%   \item
%     Support for \xpackage{titleref} and \xpackage{babel}'s shorthands added.
%   \item
%     New: \cs{refused}, \cs{getrefbykeydefault}
%   \end{Version}
%   \begin{Version}{2008/08/11 v3.1}
%   \item
%     Code is not changed.
%   \item
%     URLs updated.
%   \end{Version}
% \end{History}
%
% \PrintIndex
%
% \Finale
\endinput
|
% \end{quote}
% Do not forget to quote the argument according to the demands
% of your shell.
%
% \paragraph{Generating the documentation.}
% You can use both the \xfile{.dtx} or the \xfile{.drv} to generate
% the documentation. The process can be configured by the
% configuration file \xfile{ltxdoc.cfg}. For instance, put this
% line into this file, if you want to have A4 as paper format:
% \begin{quote}
%   \verb|\PassOptionsToClass{a4paper}{article}|
% \end{quote}
% An example follows how to generate the
% documentation with pdf\LaTeX:
% \begin{quote}
%\begin{verbatim}
%pdflatex refcount.dtx
%makeindex -s gind.ist refcount.idx
%pdflatex refcount.dtx
%makeindex -s gind.ist refcount.idx
%pdflatex refcount.dtx
%\end{verbatim}
% \end{quote}
%
% \begin{History}
%   \begin{Version}{1998/04/08 v1.0}
%   \item
%     First public release, written as answer in the
%     newsgroup \xnewsgroup{comp.text.tex}:
%     \URL{``\link{Re: Adding a \cs{ref} to a counter?}''}^^A
%     {http://groups.google.com/group/comp.text.tex/msg/c3f2a135ef5ee528}
%   \end{Version}
%   \begin{Version}{2000/09/07 v2.0}
%   \item
%     Documentation added.
%   \item
%     LPPL 1.2
%   \item
%     Package rewritten, new commands added.
%   \end{Version}
%   \begin{Version}{2006/02/20 v3.0}
%   \item
%     Support for \xpackage{hyperref} and \xpackage{nameref} improved.
%   \item
%     Support for \xpackage{titleref} and \xpackage{babel}'s shorthands added.
%   \item
%     New: \cs{refused}, \cs{getrefbykeydefault}
%   \end{Version}
%   \begin{Version}{2008/08/11 v3.1}
%   \item
%     Code is not changed.
%   \item
%     URLs updated.
%   \end{Version}
% \end{History}
%
% \PrintIndex
%
% \Finale
\endinput

%        (quote the arguments according to the demands of your shell)
%
% Documentation:
%    (a) If refcount.drv is present:
%           latex refcount.drv
%    (b) Without refcount.drv:
%           latex refcount.dtx; ...
%    The class ltxdoc loads the configuration file ltxdoc.cfg
%    if available. Here you can specify further options, e.g.
%    use A4 as paper format:
%       \PassOptionsToClass{a4paper}{article}
%
%    Programm calls to get the documentation (example):
%       pdflatex refcount.dtx
%       makeindex -s gind.ist refcount.idx
%       pdflatex refcount.dtx
%       makeindex -s gind.ist refcount.idx
%       pdflatex refcount.dtx
%
% Installation:
%    TDS:tex/latex/oberdiek/refcount.sty
%    TDS:doc/latex/oberdiek/refcount.pdf
%    TDS:source/latex/oberdiek/refcount.dtx
%
%<*ignore>
\begingroup
  \def\x{LaTeX2e}%
\expandafter\endgroup
\ifcase 0\ifx\install y1\fi\expandafter
         \ifx\csname processbatchFile\endcsname\relax\else1\fi
         \ifx\fmtname\x\else 1\fi\relax
\else\csname fi\endcsname
%</ignore>
%<*install>
\input docstrip.tex
\Msg{************************************************************************}
\Msg{* Installation}
\Msg{* Package: refcount 2008/08/11 v3.1 Data extraction from references (HO)}
\Msg{************************************************************************}

\keepsilent
\askforoverwritefalse

\let\MetaPrefix\relax
\preamble

This is a generated file.

Copyright (C) 1998, 2000, 2006, 2008 by
   Heiko Oberdiek <oberdiek@uni-freiburg.de>

This work may be distributed and/or modified under the
conditions of the LaTeX Project Public License, either
version 1.3 of this license or (at your option) any later
version. The latest version of this license is in
   http://www.latex-project.org/lppl.txt
and version 1.3 or later is part of all distributions of
LaTeX version 2005/12/01 or later.

This work has the LPPL maintenance status "maintained".

This Current Maintainer of this work is Heiko Oberdiek.

This work consists of the main source file refcount.dtx
and the derived files
   refcount.sty, refcount.pdf, refcount.ins, refcount.drv.

\endpreamble
\let\MetaPrefix\DoubleperCent

\generate{%
  \file{refcount.ins}{\from{refcount.dtx}{install}}%
  \file{refcount.drv}{\from{refcount.dtx}{driver}}%
  \usedir{tex/latex/oberdiek}%
  \file{refcount.sty}{\from{refcount.dtx}{package}}%
}

\obeyspaces
\Msg{************************************************************************}
\Msg{*}
\Msg{* To finish the installation you have to move the following}
\Msg{* file into a directory searched by TeX:}
\Msg{*}
\Msg{*     refcount.sty}
\Msg{*}
\Msg{* And install the following script file:}
\Msg{*}
\Msg{*     }
\Msg{*}
\Msg{* To produce the documentation run the file `refcount.drv'}
\Msg{* through LaTeX.}
\Msg{*}
\Msg{* Happy TeXing!}
\Msg{*}
\Msg{************************************************************************}

\endbatchfile
%</install>
%<*ignore>
\fi
%</ignore>
%<*driver>
\NeedsTeXFormat{LaTeX2e}
\ProvidesFile{refcount.drv}%
  [2008/08/11 v3.1 Data extraction from references (HO)]%
\documentclass{ltxdoc}
\usepackage{holtxdoc}[2008/08/11]
\begin{document}
  \DocInput{refcount.dtx}%
\end{document}
%</driver>
% \fi
%
% \CheckSum{198}
%
% \CharacterTable
%  {Upper-case    \A\B\C\D\E\F\G\H\I\J\K\L\M\N\O\P\Q\R\S\T\U\V\W\X\Y\Z
%   Lower-case    \a\b\c\d\e\f\g\h\i\j\k\l\m\n\o\p\q\r\s\t\u\v\w\x\y\z
%   Digits        \0\1\2\3\4\5\6\7\8\9
%   Exclamation   \!     Double quote  \"     Hash (number) \#
%   Dollar        \$     Percent       \%     Ampersand     \&
%   Acute accent  \'     Left paren    \(     Right paren   \)
%   Asterisk      \*     Plus          \+     Comma         \,
%   Minus         \-     Point         \.     Solidus       \/
%   Colon         \:     Semicolon     \;     Less than     \<
%   Equals        \=     Greater than  \>     Question mark \?
%   Commercial at \@     Left bracket  \[     Backslash     \\
%   Right bracket \]     Circumflex    \^     Underscore    \_
%   Grave accent  \`     Left brace    \{     Vertical bar  \|
%   Right brace   \}     Tilde         \~}
%
% \GetFileInfo{refcount.drv}
%
% \title{The \xpackage{refcount} package}
% \date{2008/08/11 v3.1}
% \author{Heiko Oberdiek\\\xemail{oberdiek@uni-freiburg.de}}
%
% \maketitle
%
% \begin{abstract}
% References are not numbers, however they often store numerical
% data such as section or page numbers. \cs{ref} or \cs{pageref}
% cannot be used for counter assignments or calculations because
% they are not expandable, generate warnings, or can even be links,
% The package provides expandable macros to extract the data
% from references. Packages \xpackage{hyperref}, \xpackage{nameref},
% \xpackage{titleref}, and \xpackage{babel} are supported.
% \end{abstract}
%
% \tableofcontents
%
% \section{Usage}
%
% \subsection{Setting counters}
%
% The following commands are similar to \LaTeX's
% \cs{setcounter} and \cs{addtocounter},
% but they extract the number value from a reference:
% \begin{quote}
%   \cs{setcounterref}, \cs{addtocounterref}\\
%   \cs{setcounterpageref}, \cs{addtocounterpageref}
% \end{quote}
% They take two arguments:
% \begin{quote}
%    \cs{...counter...ref} |{|\meta{\LaTeX\ counter}|}|
%    |{|\meta{reference}|}|
% \end{quote}
% An undefined references produces the usual LaTeX warning
% and its value is assumed to be zero.
% Example:
% \begin{quote}
%\begin{verbatim}
%\newcounter{ctrA}
%\newcounter{ctrB}
%\refstepcounter{ctrA}\label{ref:A}
%\setcounterref{ctrB}{ref:A}
%\addtocounterpageref{ctrB}{ref:A}
%\end{verbatim}
% \end{quote}
%
% \subsection{Expandable commands}
%
% These commands that can be used in expandible contexts
% (inside calculations, \cs{edef}, \cs{csname}, \cs{write}, \dots):
% \begin{quote}
%   \cs{getrefnumber}, \cs{getpagerefnumber}
% \end{quote}
% They take one argument, the reference:
% \begin{quote}
%   \cs{get...refnumber} |{|\meta{reference}|}|
% \end{quote}
% The default for undefined references can be changed
% with macro \cs{setrefcountdefault}, for example this
% package calls:
% \begin{quote}
%   \cs{setrefcountdefault}|{0}|
% \end{quote}
%
% Since version 2.0 of this package there is a new
% command:
% \begin{quote}
%   \cs{getrefbykeydefault} |{|\meta{reference}|}|
%   |{|\meta{key}|}| |{|\meta{default}|}|
% \end{quote}
% This generalized version allows the extraction
% of further properties of a reference than the
% two standard ones. Thus the following properties
% are supported, if they are available:
% \begin{quote}
% \begin{tabular}{@{}l|l|l@{}}
%    Key & Description & Package\\
% \hline
%   \meta{empty} & same as \cs{ref} & \LaTeX\\
%   |page| & same as \cs{pageref} & \LaTeX\\
%   |title| & section and caption titles & \xpackage{titleref}\\
%   |name| & section and caption titles & \xpackage{nameref}\\
%   |anchor| & anchor name & \xpackage{hyperref}\\
%   |url| & url/file & \xpackage{hyperref}/\xpackage{xr}
% \end{tabular}
% \end{quote}
%
% \subsection{Undefined references}
%
% Because warnings and assignments cannot be used in
% expandible contexts, undefined references do not
% produce a warning, their values are assumed to be zero.
% Example:
% \begin{quote}
%\begin{verbatim}
%\label{ref:here}% somewhere
%\refused{ref:here}% see below
%\ifodd\getpagerefnumber{ref:here}%
%  reference is on an odd page
%\else
%  reference is on an even page
%\fi
%\end{verbatim}
% \end{quote}
%
% In case of undefined references the user usually want's
% to be informed. Also \LaTeX\ prints a warning at
% the end of the \LaTeX\ run. To notify \LaTeX\ and
% get a normal warning, just use
% \begin{quote}
%   \cs{refused} |{|\meta{reference}|}|
% \end{quote}
% outside the expanding context. Example, see above.
%
% \subsection{Notes}
%
% \begin{itemize}
% \item
%   The method of extracting the number in this
%   package also works in cases, where the
%   reference cannot be used directly, because
%   a package such as \xpackage{hyperref} has added
%   extra stuff (hyper link), so that the reference cannot
%   be used as number any more.
% \item
%   If the reference does not contain a number,
%   assignments to a counter will fail of course.
% \end{itemize}
%
%
% \StopEventually{
% }
%
% \section{Implementation}
%
%    \begin{macrocode}
%<*package>
\NeedsTeXFormat{LaTeX2e}
\ProvidesPackage{refcount}
  [2008/08/11 v3.1 Data extraction from references (HO)]%

\def\setrefcountdefault#1{%
  \def\rc@default{#1}%
}
\setrefcountdefault{0}

% \def\@car#1#2\@nil{#1} % defined in LaTeX kernel
\def\rc@cartwo#1#2#3\@nil{#2}

% generic check without babel support
\long\def\rc@refused#1{%
  \expandafter\ifx\csname r@#1\endcsname\relax
    \protect\G@refundefinedtrue
    \@latex@warning{%
      Reference `#1' on page \thepage\space undefined%
    }%
  \fi
}

% user command, add babel support
\newcommand*{\refused}[1]{%
  \begingroup
    \csname @safe@activestrue\endcsname
    \rc@refused{#1}{}%
  \endgroup
}

% Generic command for "\{set,addto}counter{page,}ref":
% #1: \setcounter, \addtocounter
% #2: \@car (for \ref), \@cartwo (for \pageref)
% #3: LaTeX counter
% #4: reference
\def\rc@set#1#2#3#4{%
  \begingroup
    \csname @safe@activestrue\endcsname
    \rc@refused{#4}%
    \expandafter\rc@@set\csname r@#4\endcsname{#1}{#2}{#3}%
  \endgroup
}
% #1: \r@<...>
% #2: \setcounter, \addtocounter
% #3: \@car (for \ref), \@cartwo (for \pageref)
% #4: LaTeX counter
\def\rc@@set#1#2#3#4{%
  \ifx#1\relax
    #2{#4}{\rc@default}%
  \else
    #2{#4}{%
      \expandafter#3#1\rc@default\rc@default\@nil
    }%
  \fi
}

% user commands:

\newcommand*{\setcounterref}{\rc@set\setcounter\@car}
\newcommand*{\addtocounterref}{\rc@set\addtocounter\@car}
\newcommand*{\setcounterpageref}{\rc@set\setcounter\rc@cartwo}
\newcommand*{\addtocounterpageref}{\rc@set\addtocounter\rc@cartwo}

\newcommand*{\getrefnumber}[1]{%
  \expandafter\ifx\csname r@#1\endcsname\relax
    \rc@default
  \else
    \expandafter\expandafter\expandafter\@car
    \csname r@#1\endcsname\@nil
  \fi
}
\newcommand*{\getpagerefnumber}[1]{%
  \expandafter\ifx\csname r@#1\endcsname\relax
    \rc@default
  \else
    \expandafter\expandafter\expandafter\rc@cartwo
    \csname r@#1\endcsname\rc@default\rc@default\@nil
  \fi
}
\newcommand*{\getrefbykeydefault}[2]{%
  \expandafter\rc@getrefbykeydefault
    \csname r@#1\expandafter\endcsname
    \csname rc@extract@#2\endcsname
}
% #1: \r@<...>
% #2: \rc@extract@<...>
% #3: default
\def\rc@getrefbykeydefault#1#2#3{%
  \ifx#1\relax
    % reference is undefined
    #3%
  \else
    \ifx#2\relax
      % extract method is missing
      #3%
    \else
      \expandafter\rc@generic#1{#3}{#3}{#3}{#3}{#3}\@nil#2{#3}%
    \fi
  \fi
}
% #1: first item in \r@<...>
% #2: remaining items in \r@<...>
% #3: \rc@extract@<...>
% #4: default
\def\rc@generic#1#2\@nil#3#4{%
  #3{#1\TR@TitleReference\@empty{#4}\@nil}{#1}#2\@nil
}
\def\rc@extract@{%
  \expandafter\@car\@gobble
}
\def\rc@extract@page{%
  \expandafter\@car\@gobbletwo
}
\def\rc@extract@name{%
  \expandafter\@car\@gobblefour\@empty
}
\def\rc@extract@anchor{%
  \expandafter\@car\@gobblefour
}
\def\rc@extract@url{%
  \expandafter\expandafter\expandafter\@car\expandafter
      \@gobble\@gobblefour
}
\def\rc@extract@title#1#2\@nil{%
  \rc@@extract@title#1%
}
\def\rc@@extract@title#1\TR@TitleReference#2#3#4\@nil{#3}
%</package>
%    \end{macrocode}
%
% \section{Installation}
%
% \subsection{Download}
%
% \paragraph{Package.} This package is available on
% CTAN\footnote{\url{ftp://ftp.ctan.org/tex-archive/}}:
% \begin{description}
% \item[\CTAN{macros/latex/contrib/oberdiek/refcount.dtx}] The source file.
% \item[\CTAN{macros/latex/contrib/oberdiek/refcount.pdf}] Documentation.
% \end{description}
%
%
% \paragraph{Bundle.} All the packages of the bundle `oberdiek'
% are also available in a TDS compliant ZIP archive. There
% the packages are already unpacked and the documentation files
% are generated. The files and directories obey the TDS standard.
% \begin{description}
% \item[\CTAN{install/macros/latex/contrib/oberdiek.tds.zip}]
% \end{description}
% \emph{TDS} refers to the standard ``A Directory Structure
% for \TeX\ Files'' (\CTAN{tds/tds.pdf}). Directories
% with \xfile{texmf} in their name are usually organized this way.
%
% \subsection{Bundle installation}
%
% \paragraph{Unpacking.} Unpack the \xfile{oberdiek.tds.zip} in the
% TDS tree (also known as \xfile{texmf} tree) of your choice.
% Example (linux):
% \begin{quote}
%   |unzip oberdiek.tds.zip -d ~/texmf|
% \end{quote}
%
% \paragraph{Script installation.}
% Check the directory \xfile{TDS:scripts/oberdiek/} for
% scripts that need further installation steps.
% Package \xpackage{attachfile2} comes with the Perl script
% \xfile{pdfatfi.pl} that should be installed in such a way
% that it can be called as \texttt{pdfatfi}.
% Example (linux):
% \begin{quote}
%   |chmod +x scripts/oberdiek/pdfatfi.pl|\\
%   |cp scripts/oberdiek/pdfatfi.pl /usr/local/bin/|
% \end{quote}
%
% \subsection{Package installation}
%
% \paragraph{Unpacking.} The \xfile{.dtx} file is a self-extracting
% \docstrip\ archive. The files are extracted by running the
% \xfile{.dtx} through \plainTeX:
% \begin{quote}
%   \verb|tex refcount.dtx|
% \end{quote}
%
% \paragraph{TDS.} Now the different files must be moved into
% the different directories in your installation TDS tree
% (also known as \xfile{texmf} tree):
% \begin{quote}
% \def\t{^^A
% \begin{tabular}{@{}>{\ttfamily}l@{ $\rightarrow$ }>{\ttfamily}l@{}}
%   refcount.sty & tex/latex/oberdiek/refcount.sty\\
%   refcount.pdf & doc/latex/oberdiek/refcount.pdf\\
%   refcount.dtx & source/latex/oberdiek/refcount.dtx\\
% \end{tabular}^^A
% }^^A
% \sbox0{\t}^^A
% \ifdim\wd0>\linewidth
%   \begingroup
%     \advance\linewidth by\leftmargin
%     \advance\linewidth by\rightmargin
%   \edef\x{\endgroup
%     \def\noexpand\lw{\the\linewidth}^^A
%   }\x
%   \def\lwbox{^^A
%     \leavevmode
%     \hbox to \linewidth{^^A
%       \kern-\leftmargin\relax
%       \hss
%       \usebox0
%       \hss
%       \kern-\rightmargin\relax
%     }^^A
%   }^^A
%   \ifdim\wd0>\lw
%     \sbox0{\small\t}^^A
%     \ifdim\wd0>\linewidth
%       \ifdim\wd0>\lw
%         \sbox0{\footnotesize\t}^^A
%         \ifdim\wd0>\linewidth
%           \ifdim\wd0>\lw
%             \sbox0{\scriptsize\t}^^A
%             \ifdim\wd0>\linewidth
%               \ifdim\wd0>\lw
%                 \sbox0{\tiny\t}^^A
%                 \ifdim\wd0>\linewidth
%                   \lwbox
%                 \else
%                   \usebox0
%                 \fi
%               \else
%                 \lwbox
%               \fi
%             \else
%               \usebox0
%             \fi
%           \else
%             \lwbox
%           \fi
%         \else
%           \usebox0
%         \fi
%       \else
%         \lwbox
%       \fi
%     \else
%       \usebox0
%     \fi
%   \else
%     \lwbox
%   \fi
% \else
%   \usebox0
% \fi
% \end{quote}
% If you have a \xfile{docstrip.cfg} that configures and enables \docstrip's
% TDS installing feature, then some files can already be in the right
% place, see the documentation of \docstrip.
%
% \subsection{Refresh file name databases}
%
% If your \TeX~distribution
% (\teTeX, \mikTeX, \dots) relies on file name databases, you must refresh
% these. For example, \teTeX\ users run \verb|texhash| or
% \verb|mktexlsr|.
%
% \subsection{Some details for the interested}
%
% \paragraph{Attached source.}
%
% The PDF documentation on CTAN also includes the
% \xfile{.dtx} source file. It can be extracted by
% AcrobatReader 6 or higher. Another option is \textsf{pdftk},
% e.g. unpack the file into the current directory:
% \begin{quote}
%   \verb|pdftk refcount.pdf unpack_files output .|
% \end{quote}
%
% \paragraph{Unpacking with \LaTeX.}
% The \xfile{.dtx} chooses its action depending on the format:
% \begin{description}
% \item[\plainTeX:] Run \docstrip\ and extract the files.
% \item[\LaTeX:] Generate the documentation.
% \end{description}
% If you insist on using \LaTeX\ for \docstrip\ (really,
% \docstrip\ does not need \LaTeX), then inform the autodetect routine
% about your intention:
% \begin{quote}
%   \verb|latex \let\install=y% \iffalse meta-comment
%
% Copyright (C) 1998, 2000, 2006, 2008 by
%    Heiko Oberdiek <oberdiek@uni-freiburg.de>
%
% This work may be distributed and/or modified under the
% conditions of the LaTeX Project Public License, either
% version 1.3 of this license or (at your option) any later
% version. The latest version of this license is in
%    http://www.latex-project.org/lppl.txt
% and version 1.3 or later is part of all distributions of
% LaTeX version 2005/12/01 or later.
%
% This work has the LPPL maintenance status "maintained".
%
% This Current Maintainer of this work is Heiko Oberdiek.
%
% This work consists of the main source file refcount.dtx
% and the derived files
%    refcount.sty, refcount.pdf, refcount.ins, refcount.drv.
%
% Distribution:
%    CTAN:macros/latex/contrib/oberdiek/refcount.dtx
%    CTAN:macros/latex/contrib/oberdiek/refcount.pdf
%
% Unpacking:
%    (a) If refcount.ins is present:
%           tex refcount.ins
%    (b) Without refcount.ins:
%           tex refcount.dtx
%    (c) If you insist on using LaTeX
%           latex \let\install=y% \iffalse meta-comment
%
% Copyright (C) 1998, 2000, 2006, 2008 by
%    Heiko Oberdiek <oberdiek@uni-freiburg.de>
%
% This work may be distributed and/or modified under the
% conditions of the LaTeX Project Public License, either
% version 1.3 of this license or (at your option) any later
% version. The latest version of this license is in
%    http://www.latex-project.org/lppl.txt
% and version 1.3 or later is part of all distributions of
% LaTeX version 2005/12/01 or later.
%
% This work has the LPPL maintenance status "maintained".
%
% This Current Maintainer of this work is Heiko Oberdiek.
%
% This work consists of the main source file refcount.dtx
% and the derived files
%    refcount.sty, refcount.pdf, refcount.ins, refcount.drv.
%
% Distribution:
%    CTAN:macros/latex/contrib/oberdiek/refcount.dtx
%    CTAN:macros/latex/contrib/oberdiek/refcount.pdf
%
% Unpacking:
%    (a) If refcount.ins is present:
%           tex refcount.ins
%    (b) Without refcount.ins:
%           tex refcount.dtx
%    (c) If you insist on using LaTeX
%           latex \let\install=y% \iffalse meta-comment
%
% Copyright (C) 1998, 2000, 2006, 2008 by
%    Heiko Oberdiek <oberdiek@uni-freiburg.de>
%
% This work may be distributed and/or modified under the
% conditions of the LaTeX Project Public License, either
% version 1.3 of this license or (at your option) any later
% version. The latest version of this license is in
%    http://www.latex-project.org/lppl.txt
% and version 1.3 or later is part of all distributions of
% LaTeX version 2005/12/01 or later.
%
% This work has the LPPL maintenance status "maintained".
%
% This Current Maintainer of this work is Heiko Oberdiek.
%
% This work consists of the main source file refcount.dtx
% and the derived files
%    refcount.sty, refcount.pdf, refcount.ins, refcount.drv.
%
% Distribution:
%    CTAN:macros/latex/contrib/oberdiek/refcount.dtx
%    CTAN:macros/latex/contrib/oberdiek/refcount.pdf
%
% Unpacking:
%    (a) If refcount.ins is present:
%           tex refcount.ins
%    (b) Without refcount.ins:
%           tex refcount.dtx
%    (c) If you insist on using LaTeX
%           latex \let\install=y\input{refcount.dtx}
%        (quote the arguments according to the demands of your shell)
%
% Documentation:
%    (a) If refcount.drv is present:
%           latex refcount.drv
%    (b) Without refcount.drv:
%           latex refcount.dtx; ...
%    The class ltxdoc loads the configuration file ltxdoc.cfg
%    if available. Here you can specify further options, e.g.
%    use A4 as paper format:
%       \PassOptionsToClass{a4paper}{article}
%
%    Programm calls to get the documentation (example):
%       pdflatex refcount.dtx
%       makeindex -s gind.ist refcount.idx
%       pdflatex refcount.dtx
%       makeindex -s gind.ist refcount.idx
%       pdflatex refcount.dtx
%
% Installation:
%    TDS:tex/latex/oberdiek/refcount.sty
%    TDS:doc/latex/oberdiek/refcount.pdf
%    TDS:source/latex/oberdiek/refcount.dtx
%
%<*ignore>
\begingroup
  \def\x{LaTeX2e}%
\expandafter\endgroup
\ifcase 0\ifx\install y1\fi\expandafter
         \ifx\csname processbatchFile\endcsname\relax\else1\fi
         \ifx\fmtname\x\else 1\fi\relax
\else\csname fi\endcsname
%</ignore>
%<*install>
\input docstrip.tex
\Msg{************************************************************************}
\Msg{* Installation}
\Msg{* Package: refcount 2008/08/11 v3.1 Data extraction from references (HO)}
\Msg{************************************************************************}

\keepsilent
\askforoverwritefalse

\let\MetaPrefix\relax
\preamble

This is a generated file.

Copyright (C) 1998, 2000, 2006, 2008 by
   Heiko Oberdiek <oberdiek@uni-freiburg.de>

This work may be distributed and/or modified under the
conditions of the LaTeX Project Public License, either
version 1.3 of this license or (at your option) any later
version. The latest version of this license is in
   http://www.latex-project.org/lppl.txt
and version 1.3 or later is part of all distributions of
LaTeX version 2005/12/01 or later.

This work has the LPPL maintenance status "maintained".

This Current Maintainer of this work is Heiko Oberdiek.

This work consists of the main source file refcount.dtx
and the derived files
   refcount.sty, refcount.pdf, refcount.ins, refcount.drv.

\endpreamble
\let\MetaPrefix\DoubleperCent

\generate{%
  \file{refcount.ins}{\from{refcount.dtx}{install}}%
  \file{refcount.drv}{\from{refcount.dtx}{driver}}%
  \usedir{tex/latex/oberdiek}%
  \file{refcount.sty}{\from{refcount.dtx}{package}}%
}

\obeyspaces
\Msg{************************************************************************}
\Msg{*}
\Msg{* To finish the installation you have to move the following}
\Msg{* file into a directory searched by TeX:}
\Msg{*}
\Msg{*     refcount.sty}
\Msg{*}
\Msg{* And install the following script file:}
\Msg{*}
\Msg{*     }
\Msg{*}
\Msg{* To produce the documentation run the file `refcount.drv'}
\Msg{* through LaTeX.}
\Msg{*}
\Msg{* Happy TeXing!}
\Msg{*}
\Msg{************************************************************************}

\endbatchfile
%</install>
%<*ignore>
\fi
%</ignore>
%<*driver>
\NeedsTeXFormat{LaTeX2e}
\ProvidesFile{refcount.drv}%
  [2008/08/11 v3.1 Data extraction from references (HO)]%
\documentclass{ltxdoc}
\usepackage{holtxdoc}[2008/08/11]
\begin{document}
  \DocInput{refcount.dtx}%
\end{document}
%</driver>
% \fi
%
% \CheckSum{198}
%
% \CharacterTable
%  {Upper-case    \A\B\C\D\E\F\G\H\I\J\K\L\M\N\O\P\Q\R\S\T\U\V\W\X\Y\Z
%   Lower-case    \a\b\c\d\e\f\g\h\i\j\k\l\m\n\o\p\q\r\s\t\u\v\w\x\y\z
%   Digits        \0\1\2\3\4\5\6\7\8\9
%   Exclamation   \!     Double quote  \"     Hash (number) \#
%   Dollar        \$     Percent       \%     Ampersand     \&
%   Acute accent  \'     Left paren    \(     Right paren   \)
%   Asterisk      \*     Plus          \+     Comma         \,
%   Minus         \-     Point         \.     Solidus       \/
%   Colon         \:     Semicolon     \;     Less than     \<
%   Equals        \=     Greater than  \>     Question mark \?
%   Commercial at \@     Left bracket  \[     Backslash     \\
%   Right bracket \]     Circumflex    \^     Underscore    \_
%   Grave accent  \`     Left brace    \{     Vertical bar  \|
%   Right brace   \}     Tilde         \~}
%
% \GetFileInfo{refcount.drv}
%
% \title{The \xpackage{refcount} package}
% \date{2008/08/11 v3.1}
% \author{Heiko Oberdiek\\\xemail{oberdiek@uni-freiburg.de}}
%
% \maketitle
%
% \begin{abstract}
% References are not numbers, however they often store numerical
% data such as section or page numbers. \cs{ref} or \cs{pageref}
% cannot be used for counter assignments or calculations because
% they are not expandable, generate warnings, or can even be links,
% The package provides expandable macros to extract the data
% from references. Packages \xpackage{hyperref}, \xpackage{nameref},
% \xpackage{titleref}, and \xpackage{babel} are supported.
% \end{abstract}
%
% \tableofcontents
%
% \section{Usage}
%
% \subsection{Setting counters}
%
% The following commands are similar to \LaTeX's
% \cs{setcounter} and \cs{addtocounter},
% but they extract the number value from a reference:
% \begin{quote}
%   \cs{setcounterref}, \cs{addtocounterref}\\
%   \cs{setcounterpageref}, \cs{addtocounterpageref}
% \end{quote}
% They take two arguments:
% \begin{quote}
%    \cs{...counter...ref} |{|\meta{\LaTeX\ counter}|}|
%    |{|\meta{reference}|}|
% \end{quote}
% An undefined references produces the usual LaTeX warning
% and its value is assumed to be zero.
% Example:
% \begin{quote}
%\begin{verbatim}
%\newcounter{ctrA}
%\newcounter{ctrB}
%\refstepcounter{ctrA}\label{ref:A}
%\setcounterref{ctrB}{ref:A}
%\addtocounterpageref{ctrB}{ref:A}
%\end{verbatim}
% \end{quote}
%
% \subsection{Expandable commands}
%
% These commands that can be used in expandible contexts
% (inside calculations, \cs{edef}, \cs{csname}, \cs{write}, \dots):
% \begin{quote}
%   \cs{getrefnumber}, \cs{getpagerefnumber}
% \end{quote}
% They take one argument, the reference:
% \begin{quote}
%   \cs{get...refnumber} |{|\meta{reference}|}|
% \end{quote}
% The default for undefined references can be changed
% with macro \cs{setrefcountdefault}, for example this
% package calls:
% \begin{quote}
%   \cs{setrefcountdefault}|{0}|
% \end{quote}
%
% Since version 2.0 of this package there is a new
% command:
% \begin{quote}
%   \cs{getrefbykeydefault} |{|\meta{reference}|}|
%   |{|\meta{key}|}| |{|\meta{default}|}|
% \end{quote}
% This generalized version allows the extraction
% of further properties of a reference than the
% two standard ones. Thus the following properties
% are supported, if they are available:
% \begin{quote}
% \begin{tabular}{@{}l|l|l@{}}
%    Key & Description & Package\\
% \hline
%   \meta{empty} & same as \cs{ref} & \LaTeX\\
%   |page| & same as \cs{pageref} & \LaTeX\\
%   |title| & section and caption titles & \xpackage{titleref}\\
%   |name| & section and caption titles & \xpackage{nameref}\\
%   |anchor| & anchor name & \xpackage{hyperref}\\
%   |url| & url/file & \xpackage{hyperref}/\xpackage{xr}
% \end{tabular}
% \end{quote}
%
% \subsection{Undefined references}
%
% Because warnings and assignments cannot be used in
% expandible contexts, undefined references do not
% produce a warning, their values are assumed to be zero.
% Example:
% \begin{quote}
%\begin{verbatim}
%\label{ref:here}% somewhere
%\refused{ref:here}% see below
%\ifodd\getpagerefnumber{ref:here}%
%  reference is on an odd page
%\else
%  reference is on an even page
%\fi
%\end{verbatim}
% \end{quote}
%
% In case of undefined references the user usually want's
% to be informed. Also \LaTeX\ prints a warning at
% the end of the \LaTeX\ run. To notify \LaTeX\ and
% get a normal warning, just use
% \begin{quote}
%   \cs{refused} |{|\meta{reference}|}|
% \end{quote}
% outside the expanding context. Example, see above.
%
% \subsection{Notes}
%
% \begin{itemize}
% \item
%   The method of extracting the number in this
%   package also works in cases, where the
%   reference cannot be used directly, because
%   a package such as \xpackage{hyperref} has added
%   extra stuff (hyper link), so that the reference cannot
%   be used as number any more.
% \item
%   If the reference does not contain a number,
%   assignments to a counter will fail of course.
% \end{itemize}
%
%
% \StopEventually{
% }
%
% \section{Implementation}
%
%    \begin{macrocode}
%<*package>
\NeedsTeXFormat{LaTeX2e}
\ProvidesPackage{refcount}
  [2008/08/11 v3.1 Data extraction from references (HO)]%

\def\setrefcountdefault#1{%
  \def\rc@default{#1}%
}
\setrefcountdefault{0}

% \def\@car#1#2\@nil{#1} % defined in LaTeX kernel
\def\rc@cartwo#1#2#3\@nil{#2}

% generic check without babel support
\long\def\rc@refused#1{%
  \expandafter\ifx\csname r@#1\endcsname\relax
    \protect\G@refundefinedtrue
    \@latex@warning{%
      Reference `#1' on page \thepage\space undefined%
    }%
  \fi
}

% user command, add babel support
\newcommand*{\refused}[1]{%
  \begingroup
    \csname @safe@activestrue\endcsname
    \rc@refused{#1}{}%
  \endgroup
}

% Generic command for "\{set,addto}counter{page,}ref":
% #1: \setcounter, \addtocounter
% #2: \@car (for \ref), \@cartwo (for \pageref)
% #3: LaTeX counter
% #4: reference
\def\rc@set#1#2#3#4{%
  \begingroup
    \csname @safe@activestrue\endcsname
    \rc@refused{#4}%
    \expandafter\rc@@set\csname r@#4\endcsname{#1}{#2}{#3}%
  \endgroup
}
% #1: \r@<...>
% #2: \setcounter, \addtocounter
% #3: \@car (for \ref), \@cartwo (for \pageref)
% #4: LaTeX counter
\def\rc@@set#1#2#3#4{%
  \ifx#1\relax
    #2{#4}{\rc@default}%
  \else
    #2{#4}{%
      \expandafter#3#1\rc@default\rc@default\@nil
    }%
  \fi
}

% user commands:

\newcommand*{\setcounterref}{\rc@set\setcounter\@car}
\newcommand*{\addtocounterref}{\rc@set\addtocounter\@car}
\newcommand*{\setcounterpageref}{\rc@set\setcounter\rc@cartwo}
\newcommand*{\addtocounterpageref}{\rc@set\addtocounter\rc@cartwo}

\newcommand*{\getrefnumber}[1]{%
  \expandafter\ifx\csname r@#1\endcsname\relax
    \rc@default
  \else
    \expandafter\expandafter\expandafter\@car
    \csname r@#1\endcsname\@nil
  \fi
}
\newcommand*{\getpagerefnumber}[1]{%
  \expandafter\ifx\csname r@#1\endcsname\relax
    \rc@default
  \else
    \expandafter\expandafter\expandafter\rc@cartwo
    \csname r@#1\endcsname\rc@default\rc@default\@nil
  \fi
}
\newcommand*{\getrefbykeydefault}[2]{%
  \expandafter\rc@getrefbykeydefault
    \csname r@#1\expandafter\endcsname
    \csname rc@extract@#2\endcsname
}
% #1: \r@<...>
% #2: \rc@extract@<...>
% #3: default
\def\rc@getrefbykeydefault#1#2#3{%
  \ifx#1\relax
    % reference is undefined
    #3%
  \else
    \ifx#2\relax
      % extract method is missing
      #3%
    \else
      \expandafter\rc@generic#1{#3}{#3}{#3}{#3}{#3}\@nil#2{#3}%
    \fi
  \fi
}
% #1: first item in \r@<...>
% #2: remaining items in \r@<...>
% #3: \rc@extract@<...>
% #4: default
\def\rc@generic#1#2\@nil#3#4{%
  #3{#1\TR@TitleReference\@empty{#4}\@nil}{#1}#2\@nil
}
\def\rc@extract@{%
  \expandafter\@car\@gobble
}
\def\rc@extract@page{%
  \expandafter\@car\@gobbletwo
}
\def\rc@extract@name{%
  \expandafter\@car\@gobblefour\@empty
}
\def\rc@extract@anchor{%
  \expandafter\@car\@gobblefour
}
\def\rc@extract@url{%
  \expandafter\expandafter\expandafter\@car\expandafter
      \@gobble\@gobblefour
}
\def\rc@extract@title#1#2\@nil{%
  \rc@@extract@title#1%
}
\def\rc@@extract@title#1\TR@TitleReference#2#3#4\@nil{#3}
%</package>
%    \end{macrocode}
%
% \section{Installation}
%
% \subsection{Download}
%
% \paragraph{Package.} This package is available on
% CTAN\footnote{\url{ftp://ftp.ctan.org/tex-archive/}}:
% \begin{description}
% \item[\CTAN{macros/latex/contrib/oberdiek/refcount.dtx}] The source file.
% \item[\CTAN{macros/latex/contrib/oberdiek/refcount.pdf}] Documentation.
% \end{description}
%
%
% \paragraph{Bundle.} All the packages of the bundle `oberdiek'
% are also available in a TDS compliant ZIP archive. There
% the packages are already unpacked and the documentation files
% are generated. The files and directories obey the TDS standard.
% \begin{description}
% \item[\CTAN{install/macros/latex/contrib/oberdiek.tds.zip}]
% \end{description}
% \emph{TDS} refers to the standard ``A Directory Structure
% for \TeX\ Files'' (\CTAN{tds/tds.pdf}). Directories
% with \xfile{texmf} in their name are usually organized this way.
%
% \subsection{Bundle installation}
%
% \paragraph{Unpacking.} Unpack the \xfile{oberdiek.tds.zip} in the
% TDS tree (also known as \xfile{texmf} tree) of your choice.
% Example (linux):
% \begin{quote}
%   |unzip oberdiek.tds.zip -d ~/texmf|
% \end{quote}
%
% \paragraph{Script installation.}
% Check the directory \xfile{TDS:scripts/oberdiek/} for
% scripts that need further installation steps.
% Package \xpackage{attachfile2} comes with the Perl script
% \xfile{pdfatfi.pl} that should be installed in such a way
% that it can be called as \texttt{pdfatfi}.
% Example (linux):
% \begin{quote}
%   |chmod +x scripts/oberdiek/pdfatfi.pl|\\
%   |cp scripts/oberdiek/pdfatfi.pl /usr/local/bin/|
% \end{quote}
%
% \subsection{Package installation}
%
% \paragraph{Unpacking.} The \xfile{.dtx} file is a self-extracting
% \docstrip\ archive. The files are extracted by running the
% \xfile{.dtx} through \plainTeX:
% \begin{quote}
%   \verb|tex refcount.dtx|
% \end{quote}
%
% \paragraph{TDS.} Now the different files must be moved into
% the different directories in your installation TDS tree
% (also known as \xfile{texmf} tree):
% \begin{quote}
% \def\t{^^A
% \begin{tabular}{@{}>{\ttfamily}l@{ $\rightarrow$ }>{\ttfamily}l@{}}
%   refcount.sty & tex/latex/oberdiek/refcount.sty\\
%   refcount.pdf & doc/latex/oberdiek/refcount.pdf\\
%   refcount.dtx & source/latex/oberdiek/refcount.dtx\\
% \end{tabular}^^A
% }^^A
% \sbox0{\t}^^A
% \ifdim\wd0>\linewidth
%   \begingroup
%     \advance\linewidth by\leftmargin
%     \advance\linewidth by\rightmargin
%   \edef\x{\endgroup
%     \def\noexpand\lw{\the\linewidth}^^A
%   }\x
%   \def\lwbox{^^A
%     \leavevmode
%     \hbox to \linewidth{^^A
%       \kern-\leftmargin\relax
%       \hss
%       \usebox0
%       \hss
%       \kern-\rightmargin\relax
%     }^^A
%   }^^A
%   \ifdim\wd0>\lw
%     \sbox0{\small\t}^^A
%     \ifdim\wd0>\linewidth
%       \ifdim\wd0>\lw
%         \sbox0{\footnotesize\t}^^A
%         \ifdim\wd0>\linewidth
%           \ifdim\wd0>\lw
%             \sbox0{\scriptsize\t}^^A
%             \ifdim\wd0>\linewidth
%               \ifdim\wd0>\lw
%                 \sbox0{\tiny\t}^^A
%                 \ifdim\wd0>\linewidth
%                   \lwbox
%                 \else
%                   \usebox0
%                 \fi
%               \else
%                 \lwbox
%               \fi
%             \else
%               \usebox0
%             \fi
%           \else
%             \lwbox
%           \fi
%         \else
%           \usebox0
%         \fi
%       \else
%         \lwbox
%       \fi
%     \else
%       \usebox0
%     \fi
%   \else
%     \lwbox
%   \fi
% \else
%   \usebox0
% \fi
% \end{quote}
% If you have a \xfile{docstrip.cfg} that configures and enables \docstrip's
% TDS installing feature, then some files can already be in the right
% place, see the documentation of \docstrip.
%
% \subsection{Refresh file name databases}
%
% If your \TeX~distribution
% (\teTeX, \mikTeX, \dots) relies on file name databases, you must refresh
% these. For example, \teTeX\ users run \verb|texhash| or
% \verb|mktexlsr|.
%
% \subsection{Some details for the interested}
%
% \paragraph{Attached source.}
%
% The PDF documentation on CTAN also includes the
% \xfile{.dtx} source file. It can be extracted by
% AcrobatReader 6 or higher. Another option is \textsf{pdftk},
% e.g. unpack the file into the current directory:
% \begin{quote}
%   \verb|pdftk refcount.pdf unpack_files output .|
% \end{quote}
%
% \paragraph{Unpacking with \LaTeX.}
% The \xfile{.dtx} chooses its action depending on the format:
% \begin{description}
% \item[\plainTeX:] Run \docstrip\ and extract the files.
% \item[\LaTeX:] Generate the documentation.
% \end{description}
% If you insist on using \LaTeX\ for \docstrip\ (really,
% \docstrip\ does not need \LaTeX), then inform the autodetect routine
% about your intention:
% \begin{quote}
%   \verb|latex \let\install=y\input{refcount.dtx}|
% \end{quote}
% Do not forget to quote the argument according to the demands
% of your shell.
%
% \paragraph{Generating the documentation.}
% You can use both the \xfile{.dtx} or the \xfile{.drv} to generate
% the documentation. The process can be configured by the
% configuration file \xfile{ltxdoc.cfg}. For instance, put this
% line into this file, if you want to have A4 as paper format:
% \begin{quote}
%   \verb|\PassOptionsToClass{a4paper}{article}|
% \end{quote}
% An example follows how to generate the
% documentation with pdf\LaTeX:
% \begin{quote}
%\begin{verbatim}
%pdflatex refcount.dtx
%makeindex -s gind.ist refcount.idx
%pdflatex refcount.dtx
%makeindex -s gind.ist refcount.idx
%pdflatex refcount.dtx
%\end{verbatim}
% \end{quote}
%
% \begin{History}
%   \begin{Version}{1998/04/08 v1.0}
%   \item
%     First public release, written as answer in the
%     newsgroup \xnewsgroup{comp.text.tex}:
%     \URL{``\link{Re: Adding a \cs{ref} to a counter?}''}^^A
%     {http://groups.google.com/group/comp.text.tex/msg/c3f2a135ef5ee528}
%   \end{Version}
%   \begin{Version}{2000/09/07 v2.0}
%   \item
%     Documentation added.
%   \item
%     LPPL 1.2
%   \item
%     Package rewritten, new commands added.
%   \end{Version}
%   \begin{Version}{2006/02/20 v3.0}
%   \item
%     Support for \xpackage{hyperref} and \xpackage{nameref} improved.
%   \item
%     Support for \xpackage{titleref} and \xpackage{babel}'s shorthands added.
%   \item
%     New: \cs{refused}, \cs{getrefbykeydefault}
%   \end{Version}
%   \begin{Version}{2008/08/11 v3.1}
%   \item
%     Code is not changed.
%   \item
%     URLs updated.
%   \end{Version}
% \end{History}
%
% \PrintIndex
%
% \Finale
\endinput

%        (quote the arguments according to the demands of your shell)
%
% Documentation:
%    (a) If refcount.drv is present:
%           latex refcount.drv
%    (b) Without refcount.drv:
%           latex refcount.dtx; ...
%    The class ltxdoc loads the configuration file ltxdoc.cfg
%    if available. Here you can specify further options, e.g.
%    use A4 as paper format:
%       \PassOptionsToClass{a4paper}{article}
%
%    Programm calls to get the documentation (example):
%       pdflatex refcount.dtx
%       makeindex -s gind.ist refcount.idx
%       pdflatex refcount.dtx
%       makeindex -s gind.ist refcount.idx
%       pdflatex refcount.dtx
%
% Installation:
%    TDS:tex/latex/oberdiek/refcount.sty
%    TDS:doc/latex/oberdiek/refcount.pdf
%    TDS:source/latex/oberdiek/refcount.dtx
%
%<*ignore>
\begingroup
  \def\x{LaTeX2e}%
\expandafter\endgroup
\ifcase 0\ifx\install y1\fi\expandafter
         \ifx\csname processbatchFile\endcsname\relax\else1\fi
         \ifx\fmtname\x\else 1\fi\relax
\else\csname fi\endcsname
%</ignore>
%<*install>
\input docstrip.tex
\Msg{************************************************************************}
\Msg{* Installation}
\Msg{* Package: refcount 2008/08/11 v3.1 Data extraction from references (HO)}
\Msg{************************************************************************}

\keepsilent
\askforoverwritefalse

\let\MetaPrefix\relax
\preamble

This is a generated file.

Copyright (C) 1998, 2000, 2006, 2008 by
   Heiko Oberdiek <oberdiek@uni-freiburg.de>

This work may be distributed and/or modified under the
conditions of the LaTeX Project Public License, either
version 1.3 of this license or (at your option) any later
version. The latest version of this license is in
   http://www.latex-project.org/lppl.txt
and version 1.3 or later is part of all distributions of
LaTeX version 2005/12/01 or later.

This work has the LPPL maintenance status "maintained".

This Current Maintainer of this work is Heiko Oberdiek.

This work consists of the main source file refcount.dtx
and the derived files
   refcount.sty, refcount.pdf, refcount.ins, refcount.drv.

\endpreamble
\let\MetaPrefix\DoubleperCent

\generate{%
  \file{refcount.ins}{\from{refcount.dtx}{install}}%
  \file{refcount.drv}{\from{refcount.dtx}{driver}}%
  \usedir{tex/latex/oberdiek}%
  \file{refcount.sty}{\from{refcount.dtx}{package}}%
}

\obeyspaces
\Msg{************************************************************************}
\Msg{*}
\Msg{* To finish the installation you have to move the following}
\Msg{* file into a directory searched by TeX:}
\Msg{*}
\Msg{*     refcount.sty}
\Msg{*}
\Msg{* And install the following script file:}
\Msg{*}
\Msg{*     }
\Msg{*}
\Msg{* To produce the documentation run the file `refcount.drv'}
\Msg{* through LaTeX.}
\Msg{*}
\Msg{* Happy TeXing!}
\Msg{*}
\Msg{************************************************************************}

\endbatchfile
%</install>
%<*ignore>
\fi
%</ignore>
%<*driver>
\NeedsTeXFormat{LaTeX2e}
\ProvidesFile{refcount.drv}%
  [2008/08/11 v3.1 Data extraction from references (HO)]%
\documentclass{ltxdoc}
\usepackage{holtxdoc}[2008/08/11]
\begin{document}
  \DocInput{refcount.dtx}%
\end{document}
%</driver>
% \fi
%
% \CheckSum{198}
%
% \CharacterTable
%  {Upper-case    \A\B\C\D\E\F\G\H\I\J\K\L\M\N\O\P\Q\R\S\T\U\V\W\X\Y\Z
%   Lower-case    \a\b\c\d\e\f\g\h\i\j\k\l\m\n\o\p\q\r\s\t\u\v\w\x\y\z
%   Digits        \0\1\2\3\4\5\6\7\8\9
%   Exclamation   \!     Double quote  \"     Hash (number) \#
%   Dollar        \$     Percent       \%     Ampersand     \&
%   Acute accent  \'     Left paren    \(     Right paren   \)
%   Asterisk      \*     Plus          \+     Comma         \,
%   Minus         \-     Point         \.     Solidus       \/
%   Colon         \:     Semicolon     \;     Less than     \<
%   Equals        \=     Greater than  \>     Question mark \?
%   Commercial at \@     Left bracket  \[     Backslash     \\
%   Right bracket \]     Circumflex    \^     Underscore    \_
%   Grave accent  \`     Left brace    \{     Vertical bar  \|
%   Right brace   \}     Tilde         \~}
%
% \GetFileInfo{refcount.drv}
%
% \title{The \xpackage{refcount} package}
% \date{2008/08/11 v3.1}
% \author{Heiko Oberdiek\\\xemail{oberdiek@uni-freiburg.de}}
%
% \maketitle
%
% \begin{abstract}
% References are not numbers, however they often store numerical
% data such as section or page numbers. \cs{ref} or \cs{pageref}
% cannot be used for counter assignments or calculations because
% they are not expandable, generate warnings, or can even be links,
% The package provides expandable macros to extract the data
% from references. Packages \xpackage{hyperref}, \xpackage{nameref},
% \xpackage{titleref}, and \xpackage{babel} are supported.
% \end{abstract}
%
% \tableofcontents
%
% \section{Usage}
%
% \subsection{Setting counters}
%
% The following commands are similar to \LaTeX's
% \cs{setcounter} and \cs{addtocounter},
% but they extract the number value from a reference:
% \begin{quote}
%   \cs{setcounterref}, \cs{addtocounterref}\\
%   \cs{setcounterpageref}, \cs{addtocounterpageref}
% \end{quote}
% They take two arguments:
% \begin{quote}
%    \cs{...counter...ref} |{|\meta{\LaTeX\ counter}|}|
%    |{|\meta{reference}|}|
% \end{quote}
% An undefined references produces the usual LaTeX warning
% and its value is assumed to be zero.
% Example:
% \begin{quote}
%\begin{verbatim}
%\newcounter{ctrA}
%\newcounter{ctrB}
%\refstepcounter{ctrA}\label{ref:A}
%\setcounterref{ctrB}{ref:A}
%\addtocounterpageref{ctrB}{ref:A}
%\end{verbatim}
% \end{quote}
%
% \subsection{Expandable commands}
%
% These commands that can be used in expandible contexts
% (inside calculations, \cs{edef}, \cs{csname}, \cs{write}, \dots):
% \begin{quote}
%   \cs{getrefnumber}, \cs{getpagerefnumber}
% \end{quote}
% They take one argument, the reference:
% \begin{quote}
%   \cs{get...refnumber} |{|\meta{reference}|}|
% \end{quote}
% The default for undefined references can be changed
% with macro \cs{setrefcountdefault}, for example this
% package calls:
% \begin{quote}
%   \cs{setrefcountdefault}|{0}|
% \end{quote}
%
% Since version 2.0 of this package there is a new
% command:
% \begin{quote}
%   \cs{getrefbykeydefault} |{|\meta{reference}|}|
%   |{|\meta{key}|}| |{|\meta{default}|}|
% \end{quote}
% This generalized version allows the extraction
% of further properties of a reference than the
% two standard ones. Thus the following properties
% are supported, if they are available:
% \begin{quote}
% \begin{tabular}{@{}l|l|l@{}}
%    Key & Description & Package\\
% \hline
%   \meta{empty} & same as \cs{ref} & \LaTeX\\
%   |page| & same as \cs{pageref} & \LaTeX\\
%   |title| & section and caption titles & \xpackage{titleref}\\
%   |name| & section and caption titles & \xpackage{nameref}\\
%   |anchor| & anchor name & \xpackage{hyperref}\\
%   |url| & url/file & \xpackage{hyperref}/\xpackage{xr}
% \end{tabular}
% \end{quote}
%
% \subsection{Undefined references}
%
% Because warnings and assignments cannot be used in
% expandible contexts, undefined references do not
% produce a warning, their values are assumed to be zero.
% Example:
% \begin{quote}
%\begin{verbatim}
%\label{ref:here}% somewhere
%\refused{ref:here}% see below
%\ifodd\getpagerefnumber{ref:here}%
%  reference is on an odd page
%\else
%  reference is on an even page
%\fi
%\end{verbatim}
% \end{quote}
%
% In case of undefined references the user usually want's
% to be informed. Also \LaTeX\ prints a warning at
% the end of the \LaTeX\ run. To notify \LaTeX\ and
% get a normal warning, just use
% \begin{quote}
%   \cs{refused} |{|\meta{reference}|}|
% \end{quote}
% outside the expanding context. Example, see above.
%
% \subsection{Notes}
%
% \begin{itemize}
% \item
%   The method of extracting the number in this
%   package also works in cases, where the
%   reference cannot be used directly, because
%   a package such as \xpackage{hyperref} has added
%   extra stuff (hyper link), so that the reference cannot
%   be used as number any more.
% \item
%   If the reference does not contain a number,
%   assignments to a counter will fail of course.
% \end{itemize}
%
%
% \StopEventually{
% }
%
% \section{Implementation}
%
%    \begin{macrocode}
%<*package>
\NeedsTeXFormat{LaTeX2e}
\ProvidesPackage{refcount}
  [2008/08/11 v3.1 Data extraction from references (HO)]%

\def\setrefcountdefault#1{%
  \def\rc@default{#1}%
}
\setrefcountdefault{0}

% \def\@car#1#2\@nil{#1} % defined in LaTeX kernel
\def\rc@cartwo#1#2#3\@nil{#2}

% generic check without babel support
\long\def\rc@refused#1{%
  \expandafter\ifx\csname r@#1\endcsname\relax
    \protect\G@refundefinedtrue
    \@latex@warning{%
      Reference `#1' on page \thepage\space undefined%
    }%
  \fi
}

% user command, add babel support
\newcommand*{\refused}[1]{%
  \begingroup
    \csname @safe@activestrue\endcsname
    \rc@refused{#1}{}%
  \endgroup
}

% Generic command for "\{set,addto}counter{page,}ref":
% #1: \setcounter, \addtocounter
% #2: \@car (for \ref), \@cartwo (for \pageref)
% #3: LaTeX counter
% #4: reference
\def\rc@set#1#2#3#4{%
  \begingroup
    \csname @safe@activestrue\endcsname
    \rc@refused{#4}%
    \expandafter\rc@@set\csname r@#4\endcsname{#1}{#2}{#3}%
  \endgroup
}
% #1: \r@<...>
% #2: \setcounter, \addtocounter
% #3: \@car (for \ref), \@cartwo (for \pageref)
% #4: LaTeX counter
\def\rc@@set#1#2#3#4{%
  \ifx#1\relax
    #2{#4}{\rc@default}%
  \else
    #2{#4}{%
      \expandafter#3#1\rc@default\rc@default\@nil
    }%
  \fi
}

% user commands:

\newcommand*{\setcounterref}{\rc@set\setcounter\@car}
\newcommand*{\addtocounterref}{\rc@set\addtocounter\@car}
\newcommand*{\setcounterpageref}{\rc@set\setcounter\rc@cartwo}
\newcommand*{\addtocounterpageref}{\rc@set\addtocounter\rc@cartwo}

\newcommand*{\getrefnumber}[1]{%
  \expandafter\ifx\csname r@#1\endcsname\relax
    \rc@default
  \else
    \expandafter\expandafter\expandafter\@car
    \csname r@#1\endcsname\@nil
  \fi
}
\newcommand*{\getpagerefnumber}[1]{%
  \expandafter\ifx\csname r@#1\endcsname\relax
    \rc@default
  \else
    \expandafter\expandafter\expandafter\rc@cartwo
    \csname r@#1\endcsname\rc@default\rc@default\@nil
  \fi
}
\newcommand*{\getrefbykeydefault}[2]{%
  \expandafter\rc@getrefbykeydefault
    \csname r@#1\expandafter\endcsname
    \csname rc@extract@#2\endcsname
}
% #1: \r@<...>
% #2: \rc@extract@<...>
% #3: default
\def\rc@getrefbykeydefault#1#2#3{%
  \ifx#1\relax
    % reference is undefined
    #3%
  \else
    \ifx#2\relax
      % extract method is missing
      #3%
    \else
      \expandafter\rc@generic#1{#3}{#3}{#3}{#3}{#3}\@nil#2{#3}%
    \fi
  \fi
}
% #1: first item in \r@<...>
% #2: remaining items in \r@<...>
% #3: \rc@extract@<...>
% #4: default
\def\rc@generic#1#2\@nil#3#4{%
  #3{#1\TR@TitleReference\@empty{#4}\@nil}{#1}#2\@nil
}
\def\rc@extract@{%
  \expandafter\@car\@gobble
}
\def\rc@extract@page{%
  \expandafter\@car\@gobbletwo
}
\def\rc@extract@name{%
  \expandafter\@car\@gobblefour\@empty
}
\def\rc@extract@anchor{%
  \expandafter\@car\@gobblefour
}
\def\rc@extract@url{%
  \expandafter\expandafter\expandafter\@car\expandafter
      \@gobble\@gobblefour
}
\def\rc@extract@title#1#2\@nil{%
  \rc@@extract@title#1%
}
\def\rc@@extract@title#1\TR@TitleReference#2#3#4\@nil{#3}
%</package>
%    \end{macrocode}
%
% \section{Installation}
%
% \subsection{Download}
%
% \paragraph{Package.} This package is available on
% CTAN\footnote{\url{ftp://ftp.ctan.org/tex-archive/}}:
% \begin{description}
% \item[\CTAN{macros/latex/contrib/oberdiek/refcount.dtx}] The source file.
% \item[\CTAN{macros/latex/contrib/oberdiek/refcount.pdf}] Documentation.
% \end{description}
%
%
% \paragraph{Bundle.} All the packages of the bundle `oberdiek'
% are also available in a TDS compliant ZIP archive. There
% the packages are already unpacked and the documentation files
% are generated. The files and directories obey the TDS standard.
% \begin{description}
% \item[\CTAN{install/macros/latex/contrib/oberdiek.tds.zip}]
% \end{description}
% \emph{TDS} refers to the standard ``A Directory Structure
% for \TeX\ Files'' (\CTAN{tds/tds.pdf}). Directories
% with \xfile{texmf} in their name are usually organized this way.
%
% \subsection{Bundle installation}
%
% \paragraph{Unpacking.} Unpack the \xfile{oberdiek.tds.zip} in the
% TDS tree (also known as \xfile{texmf} tree) of your choice.
% Example (linux):
% \begin{quote}
%   |unzip oberdiek.tds.zip -d ~/texmf|
% \end{quote}
%
% \paragraph{Script installation.}
% Check the directory \xfile{TDS:scripts/oberdiek/} for
% scripts that need further installation steps.
% Package \xpackage{attachfile2} comes with the Perl script
% \xfile{pdfatfi.pl} that should be installed in such a way
% that it can be called as \texttt{pdfatfi}.
% Example (linux):
% \begin{quote}
%   |chmod +x scripts/oberdiek/pdfatfi.pl|\\
%   |cp scripts/oberdiek/pdfatfi.pl /usr/local/bin/|
% \end{quote}
%
% \subsection{Package installation}
%
% \paragraph{Unpacking.} The \xfile{.dtx} file is a self-extracting
% \docstrip\ archive. The files are extracted by running the
% \xfile{.dtx} through \plainTeX:
% \begin{quote}
%   \verb|tex refcount.dtx|
% \end{quote}
%
% \paragraph{TDS.} Now the different files must be moved into
% the different directories in your installation TDS tree
% (also known as \xfile{texmf} tree):
% \begin{quote}
% \def\t{^^A
% \begin{tabular}{@{}>{\ttfamily}l@{ $\rightarrow$ }>{\ttfamily}l@{}}
%   refcount.sty & tex/latex/oberdiek/refcount.sty\\
%   refcount.pdf & doc/latex/oberdiek/refcount.pdf\\
%   refcount.dtx & source/latex/oberdiek/refcount.dtx\\
% \end{tabular}^^A
% }^^A
% \sbox0{\t}^^A
% \ifdim\wd0>\linewidth
%   \begingroup
%     \advance\linewidth by\leftmargin
%     \advance\linewidth by\rightmargin
%   \edef\x{\endgroup
%     \def\noexpand\lw{\the\linewidth}^^A
%   }\x
%   \def\lwbox{^^A
%     \leavevmode
%     \hbox to \linewidth{^^A
%       \kern-\leftmargin\relax
%       \hss
%       \usebox0
%       \hss
%       \kern-\rightmargin\relax
%     }^^A
%   }^^A
%   \ifdim\wd0>\lw
%     \sbox0{\small\t}^^A
%     \ifdim\wd0>\linewidth
%       \ifdim\wd0>\lw
%         \sbox0{\footnotesize\t}^^A
%         \ifdim\wd0>\linewidth
%           \ifdim\wd0>\lw
%             \sbox0{\scriptsize\t}^^A
%             \ifdim\wd0>\linewidth
%               \ifdim\wd0>\lw
%                 \sbox0{\tiny\t}^^A
%                 \ifdim\wd0>\linewidth
%                   \lwbox
%                 \else
%                   \usebox0
%                 \fi
%               \else
%                 \lwbox
%               \fi
%             \else
%               \usebox0
%             \fi
%           \else
%             \lwbox
%           \fi
%         \else
%           \usebox0
%         \fi
%       \else
%         \lwbox
%       \fi
%     \else
%       \usebox0
%     \fi
%   \else
%     \lwbox
%   \fi
% \else
%   \usebox0
% \fi
% \end{quote}
% If you have a \xfile{docstrip.cfg} that configures and enables \docstrip's
% TDS installing feature, then some files can already be in the right
% place, see the documentation of \docstrip.
%
% \subsection{Refresh file name databases}
%
% If your \TeX~distribution
% (\teTeX, \mikTeX, \dots) relies on file name databases, you must refresh
% these. For example, \teTeX\ users run \verb|texhash| or
% \verb|mktexlsr|.
%
% \subsection{Some details for the interested}
%
% \paragraph{Attached source.}
%
% The PDF documentation on CTAN also includes the
% \xfile{.dtx} source file. It can be extracted by
% AcrobatReader 6 or higher. Another option is \textsf{pdftk},
% e.g. unpack the file into the current directory:
% \begin{quote}
%   \verb|pdftk refcount.pdf unpack_files output .|
% \end{quote}
%
% \paragraph{Unpacking with \LaTeX.}
% The \xfile{.dtx} chooses its action depending on the format:
% \begin{description}
% \item[\plainTeX:] Run \docstrip\ and extract the files.
% \item[\LaTeX:] Generate the documentation.
% \end{description}
% If you insist on using \LaTeX\ for \docstrip\ (really,
% \docstrip\ does not need \LaTeX), then inform the autodetect routine
% about your intention:
% \begin{quote}
%   \verb|latex \let\install=y% \iffalse meta-comment
%
% Copyright (C) 1998, 2000, 2006, 2008 by
%    Heiko Oberdiek <oberdiek@uni-freiburg.de>
%
% This work may be distributed and/or modified under the
% conditions of the LaTeX Project Public License, either
% version 1.3 of this license or (at your option) any later
% version. The latest version of this license is in
%    http://www.latex-project.org/lppl.txt
% and version 1.3 or later is part of all distributions of
% LaTeX version 2005/12/01 or later.
%
% This work has the LPPL maintenance status "maintained".
%
% This Current Maintainer of this work is Heiko Oberdiek.
%
% This work consists of the main source file refcount.dtx
% and the derived files
%    refcount.sty, refcount.pdf, refcount.ins, refcount.drv.
%
% Distribution:
%    CTAN:macros/latex/contrib/oberdiek/refcount.dtx
%    CTAN:macros/latex/contrib/oberdiek/refcount.pdf
%
% Unpacking:
%    (a) If refcount.ins is present:
%           tex refcount.ins
%    (b) Without refcount.ins:
%           tex refcount.dtx
%    (c) If you insist on using LaTeX
%           latex \let\install=y\input{refcount.dtx}
%        (quote the arguments according to the demands of your shell)
%
% Documentation:
%    (a) If refcount.drv is present:
%           latex refcount.drv
%    (b) Without refcount.drv:
%           latex refcount.dtx; ...
%    The class ltxdoc loads the configuration file ltxdoc.cfg
%    if available. Here you can specify further options, e.g.
%    use A4 as paper format:
%       \PassOptionsToClass{a4paper}{article}
%
%    Programm calls to get the documentation (example):
%       pdflatex refcount.dtx
%       makeindex -s gind.ist refcount.idx
%       pdflatex refcount.dtx
%       makeindex -s gind.ist refcount.idx
%       pdflatex refcount.dtx
%
% Installation:
%    TDS:tex/latex/oberdiek/refcount.sty
%    TDS:doc/latex/oberdiek/refcount.pdf
%    TDS:source/latex/oberdiek/refcount.dtx
%
%<*ignore>
\begingroup
  \def\x{LaTeX2e}%
\expandafter\endgroup
\ifcase 0\ifx\install y1\fi\expandafter
         \ifx\csname processbatchFile\endcsname\relax\else1\fi
         \ifx\fmtname\x\else 1\fi\relax
\else\csname fi\endcsname
%</ignore>
%<*install>
\input docstrip.tex
\Msg{************************************************************************}
\Msg{* Installation}
\Msg{* Package: refcount 2008/08/11 v3.1 Data extraction from references (HO)}
\Msg{************************************************************************}

\keepsilent
\askforoverwritefalse

\let\MetaPrefix\relax
\preamble

This is a generated file.

Copyright (C) 1998, 2000, 2006, 2008 by
   Heiko Oberdiek <oberdiek@uni-freiburg.de>

This work may be distributed and/or modified under the
conditions of the LaTeX Project Public License, either
version 1.3 of this license or (at your option) any later
version. The latest version of this license is in
   http://www.latex-project.org/lppl.txt
and version 1.3 or later is part of all distributions of
LaTeX version 2005/12/01 or later.

This work has the LPPL maintenance status "maintained".

This Current Maintainer of this work is Heiko Oberdiek.

This work consists of the main source file refcount.dtx
and the derived files
   refcount.sty, refcount.pdf, refcount.ins, refcount.drv.

\endpreamble
\let\MetaPrefix\DoubleperCent

\generate{%
  \file{refcount.ins}{\from{refcount.dtx}{install}}%
  \file{refcount.drv}{\from{refcount.dtx}{driver}}%
  \usedir{tex/latex/oberdiek}%
  \file{refcount.sty}{\from{refcount.dtx}{package}}%
}

\obeyspaces
\Msg{************************************************************************}
\Msg{*}
\Msg{* To finish the installation you have to move the following}
\Msg{* file into a directory searched by TeX:}
\Msg{*}
\Msg{*     refcount.sty}
\Msg{*}
\Msg{* And install the following script file:}
\Msg{*}
\Msg{*     }
\Msg{*}
\Msg{* To produce the documentation run the file `refcount.drv'}
\Msg{* through LaTeX.}
\Msg{*}
\Msg{* Happy TeXing!}
\Msg{*}
\Msg{************************************************************************}

\endbatchfile
%</install>
%<*ignore>
\fi
%</ignore>
%<*driver>
\NeedsTeXFormat{LaTeX2e}
\ProvidesFile{refcount.drv}%
  [2008/08/11 v3.1 Data extraction from references (HO)]%
\documentclass{ltxdoc}
\usepackage{holtxdoc}[2008/08/11]
\begin{document}
  \DocInput{refcount.dtx}%
\end{document}
%</driver>
% \fi
%
% \CheckSum{198}
%
% \CharacterTable
%  {Upper-case    \A\B\C\D\E\F\G\H\I\J\K\L\M\N\O\P\Q\R\S\T\U\V\W\X\Y\Z
%   Lower-case    \a\b\c\d\e\f\g\h\i\j\k\l\m\n\o\p\q\r\s\t\u\v\w\x\y\z
%   Digits        \0\1\2\3\4\5\6\7\8\9
%   Exclamation   \!     Double quote  \"     Hash (number) \#
%   Dollar        \$     Percent       \%     Ampersand     \&
%   Acute accent  \'     Left paren    \(     Right paren   \)
%   Asterisk      \*     Plus          \+     Comma         \,
%   Minus         \-     Point         \.     Solidus       \/
%   Colon         \:     Semicolon     \;     Less than     \<
%   Equals        \=     Greater than  \>     Question mark \?
%   Commercial at \@     Left bracket  \[     Backslash     \\
%   Right bracket \]     Circumflex    \^     Underscore    \_
%   Grave accent  \`     Left brace    \{     Vertical bar  \|
%   Right brace   \}     Tilde         \~}
%
% \GetFileInfo{refcount.drv}
%
% \title{The \xpackage{refcount} package}
% \date{2008/08/11 v3.1}
% \author{Heiko Oberdiek\\\xemail{oberdiek@uni-freiburg.de}}
%
% \maketitle
%
% \begin{abstract}
% References are not numbers, however they often store numerical
% data such as section or page numbers. \cs{ref} or \cs{pageref}
% cannot be used for counter assignments or calculations because
% they are not expandable, generate warnings, or can even be links,
% The package provides expandable macros to extract the data
% from references. Packages \xpackage{hyperref}, \xpackage{nameref},
% \xpackage{titleref}, and \xpackage{babel} are supported.
% \end{abstract}
%
% \tableofcontents
%
% \section{Usage}
%
% \subsection{Setting counters}
%
% The following commands are similar to \LaTeX's
% \cs{setcounter} and \cs{addtocounter},
% but they extract the number value from a reference:
% \begin{quote}
%   \cs{setcounterref}, \cs{addtocounterref}\\
%   \cs{setcounterpageref}, \cs{addtocounterpageref}
% \end{quote}
% They take two arguments:
% \begin{quote}
%    \cs{...counter...ref} |{|\meta{\LaTeX\ counter}|}|
%    |{|\meta{reference}|}|
% \end{quote}
% An undefined references produces the usual LaTeX warning
% and its value is assumed to be zero.
% Example:
% \begin{quote}
%\begin{verbatim}
%\newcounter{ctrA}
%\newcounter{ctrB}
%\refstepcounter{ctrA}\label{ref:A}
%\setcounterref{ctrB}{ref:A}
%\addtocounterpageref{ctrB}{ref:A}
%\end{verbatim}
% \end{quote}
%
% \subsection{Expandable commands}
%
% These commands that can be used in expandible contexts
% (inside calculations, \cs{edef}, \cs{csname}, \cs{write}, \dots):
% \begin{quote}
%   \cs{getrefnumber}, \cs{getpagerefnumber}
% \end{quote}
% They take one argument, the reference:
% \begin{quote}
%   \cs{get...refnumber} |{|\meta{reference}|}|
% \end{quote}
% The default for undefined references can be changed
% with macro \cs{setrefcountdefault}, for example this
% package calls:
% \begin{quote}
%   \cs{setrefcountdefault}|{0}|
% \end{quote}
%
% Since version 2.0 of this package there is a new
% command:
% \begin{quote}
%   \cs{getrefbykeydefault} |{|\meta{reference}|}|
%   |{|\meta{key}|}| |{|\meta{default}|}|
% \end{quote}
% This generalized version allows the extraction
% of further properties of a reference than the
% two standard ones. Thus the following properties
% are supported, if they are available:
% \begin{quote}
% \begin{tabular}{@{}l|l|l@{}}
%    Key & Description & Package\\
% \hline
%   \meta{empty} & same as \cs{ref} & \LaTeX\\
%   |page| & same as \cs{pageref} & \LaTeX\\
%   |title| & section and caption titles & \xpackage{titleref}\\
%   |name| & section and caption titles & \xpackage{nameref}\\
%   |anchor| & anchor name & \xpackage{hyperref}\\
%   |url| & url/file & \xpackage{hyperref}/\xpackage{xr}
% \end{tabular}
% \end{quote}
%
% \subsection{Undefined references}
%
% Because warnings and assignments cannot be used in
% expandible contexts, undefined references do not
% produce a warning, their values are assumed to be zero.
% Example:
% \begin{quote}
%\begin{verbatim}
%\label{ref:here}% somewhere
%\refused{ref:here}% see below
%\ifodd\getpagerefnumber{ref:here}%
%  reference is on an odd page
%\else
%  reference is on an even page
%\fi
%\end{verbatim}
% \end{quote}
%
% In case of undefined references the user usually want's
% to be informed. Also \LaTeX\ prints a warning at
% the end of the \LaTeX\ run. To notify \LaTeX\ and
% get a normal warning, just use
% \begin{quote}
%   \cs{refused} |{|\meta{reference}|}|
% \end{quote}
% outside the expanding context. Example, see above.
%
% \subsection{Notes}
%
% \begin{itemize}
% \item
%   The method of extracting the number in this
%   package also works in cases, where the
%   reference cannot be used directly, because
%   a package such as \xpackage{hyperref} has added
%   extra stuff (hyper link), so that the reference cannot
%   be used as number any more.
% \item
%   If the reference does not contain a number,
%   assignments to a counter will fail of course.
% \end{itemize}
%
%
% \StopEventually{
% }
%
% \section{Implementation}
%
%    \begin{macrocode}
%<*package>
\NeedsTeXFormat{LaTeX2e}
\ProvidesPackage{refcount}
  [2008/08/11 v3.1 Data extraction from references (HO)]%

\def\setrefcountdefault#1{%
  \def\rc@default{#1}%
}
\setrefcountdefault{0}

% \def\@car#1#2\@nil{#1} % defined in LaTeX kernel
\def\rc@cartwo#1#2#3\@nil{#2}

% generic check without babel support
\long\def\rc@refused#1{%
  \expandafter\ifx\csname r@#1\endcsname\relax
    \protect\G@refundefinedtrue
    \@latex@warning{%
      Reference `#1' on page \thepage\space undefined%
    }%
  \fi
}

% user command, add babel support
\newcommand*{\refused}[1]{%
  \begingroup
    \csname @safe@activestrue\endcsname
    \rc@refused{#1}{}%
  \endgroup
}

% Generic command for "\{set,addto}counter{page,}ref":
% #1: \setcounter, \addtocounter
% #2: \@car (for \ref), \@cartwo (for \pageref)
% #3: LaTeX counter
% #4: reference
\def\rc@set#1#2#3#4{%
  \begingroup
    \csname @safe@activestrue\endcsname
    \rc@refused{#4}%
    \expandafter\rc@@set\csname r@#4\endcsname{#1}{#2}{#3}%
  \endgroup
}
% #1: \r@<...>
% #2: \setcounter, \addtocounter
% #3: \@car (for \ref), \@cartwo (for \pageref)
% #4: LaTeX counter
\def\rc@@set#1#2#3#4{%
  \ifx#1\relax
    #2{#4}{\rc@default}%
  \else
    #2{#4}{%
      \expandafter#3#1\rc@default\rc@default\@nil
    }%
  \fi
}

% user commands:

\newcommand*{\setcounterref}{\rc@set\setcounter\@car}
\newcommand*{\addtocounterref}{\rc@set\addtocounter\@car}
\newcommand*{\setcounterpageref}{\rc@set\setcounter\rc@cartwo}
\newcommand*{\addtocounterpageref}{\rc@set\addtocounter\rc@cartwo}

\newcommand*{\getrefnumber}[1]{%
  \expandafter\ifx\csname r@#1\endcsname\relax
    \rc@default
  \else
    \expandafter\expandafter\expandafter\@car
    \csname r@#1\endcsname\@nil
  \fi
}
\newcommand*{\getpagerefnumber}[1]{%
  \expandafter\ifx\csname r@#1\endcsname\relax
    \rc@default
  \else
    \expandafter\expandafter\expandafter\rc@cartwo
    \csname r@#1\endcsname\rc@default\rc@default\@nil
  \fi
}
\newcommand*{\getrefbykeydefault}[2]{%
  \expandafter\rc@getrefbykeydefault
    \csname r@#1\expandafter\endcsname
    \csname rc@extract@#2\endcsname
}
% #1: \r@<...>
% #2: \rc@extract@<...>
% #3: default
\def\rc@getrefbykeydefault#1#2#3{%
  \ifx#1\relax
    % reference is undefined
    #3%
  \else
    \ifx#2\relax
      % extract method is missing
      #3%
    \else
      \expandafter\rc@generic#1{#3}{#3}{#3}{#3}{#3}\@nil#2{#3}%
    \fi
  \fi
}
% #1: first item in \r@<...>
% #2: remaining items in \r@<...>
% #3: \rc@extract@<...>
% #4: default
\def\rc@generic#1#2\@nil#3#4{%
  #3{#1\TR@TitleReference\@empty{#4}\@nil}{#1}#2\@nil
}
\def\rc@extract@{%
  \expandafter\@car\@gobble
}
\def\rc@extract@page{%
  \expandafter\@car\@gobbletwo
}
\def\rc@extract@name{%
  \expandafter\@car\@gobblefour\@empty
}
\def\rc@extract@anchor{%
  \expandafter\@car\@gobblefour
}
\def\rc@extract@url{%
  \expandafter\expandafter\expandafter\@car\expandafter
      \@gobble\@gobblefour
}
\def\rc@extract@title#1#2\@nil{%
  \rc@@extract@title#1%
}
\def\rc@@extract@title#1\TR@TitleReference#2#3#4\@nil{#3}
%</package>
%    \end{macrocode}
%
% \section{Installation}
%
% \subsection{Download}
%
% \paragraph{Package.} This package is available on
% CTAN\footnote{\url{ftp://ftp.ctan.org/tex-archive/}}:
% \begin{description}
% \item[\CTAN{macros/latex/contrib/oberdiek/refcount.dtx}] The source file.
% \item[\CTAN{macros/latex/contrib/oberdiek/refcount.pdf}] Documentation.
% \end{description}
%
%
% \paragraph{Bundle.} All the packages of the bundle `oberdiek'
% are also available in a TDS compliant ZIP archive. There
% the packages are already unpacked and the documentation files
% are generated. The files and directories obey the TDS standard.
% \begin{description}
% \item[\CTAN{install/macros/latex/contrib/oberdiek.tds.zip}]
% \end{description}
% \emph{TDS} refers to the standard ``A Directory Structure
% for \TeX\ Files'' (\CTAN{tds/tds.pdf}). Directories
% with \xfile{texmf} in their name are usually organized this way.
%
% \subsection{Bundle installation}
%
% \paragraph{Unpacking.} Unpack the \xfile{oberdiek.tds.zip} in the
% TDS tree (also known as \xfile{texmf} tree) of your choice.
% Example (linux):
% \begin{quote}
%   |unzip oberdiek.tds.zip -d ~/texmf|
% \end{quote}
%
% \paragraph{Script installation.}
% Check the directory \xfile{TDS:scripts/oberdiek/} for
% scripts that need further installation steps.
% Package \xpackage{attachfile2} comes with the Perl script
% \xfile{pdfatfi.pl} that should be installed in such a way
% that it can be called as \texttt{pdfatfi}.
% Example (linux):
% \begin{quote}
%   |chmod +x scripts/oberdiek/pdfatfi.pl|\\
%   |cp scripts/oberdiek/pdfatfi.pl /usr/local/bin/|
% \end{quote}
%
% \subsection{Package installation}
%
% \paragraph{Unpacking.} The \xfile{.dtx} file is a self-extracting
% \docstrip\ archive. The files are extracted by running the
% \xfile{.dtx} through \plainTeX:
% \begin{quote}
%   \verb|tex refcount.dtx|
% \end{quote}
%
% \paragraph{TDS.} Now the different files must be moved into
% the different directories in your installation TDS tree
% (also known as \xfile{texmf} tree):
% \begin{quote}
% \def\t{^^A
% \begin{tabular}{@{}>{\ttfamily}l@{ $\rightarrow$ }>{\ttfamily}l@{}}
%   refcount.sty & tex/latex/oberdiek/refcount.sty\\
%   refcount.pdf & doc/latex/oberdiek/refcount.pdf\\
%   refcount.dtx & source/latex/oberdiek/refcount.dtx\\
% \end{tabular}^^A
% }^^A
% \sbox0{\t}^^A
% \ifdim\wd0>\linewidth
%   \begingroup
%     \advance\linewidth by\leftmargin
%     \advance\linewidth by\rightmargin
%   \edef\x{\endgroup
%     \def\noexpand\lw{\the\linewidth}^^A
%   }\x
%   \def\lwbox{^^A
%     \leavevmode
%     \hbox to \linewidth{^^A
%       \kern-\leftmargin\relax
%       \hss
%       \usebox0
%       \hss
%       \kern-\rightmargin\relax
%     }^^A
%   }^^A
%   \ifdim\wd0>\lw
%     \sbox0{\small\t}^^A
%     \ifdim\wd0>\linewidth
%       \ifdim\wd0>\lw
%         \sbox0{\footnotesize\t}^^A
%         \ifdim\wd0>\linewidth
%           \ifdim\wd0>\lw
%             \sbox0{\scriptsize\t}^^A
%             \ifdim\wd0>\linewidth
%               \ifdim\wd0>\lw
%                 \sbox0{\tiny\t}^^A
%                 \ifdim\wd0>\linewidth
%                   \lwbox
%                 \else
%                   \usebox0
%                 \fi
%               \else
%                 \lwbox
%               \fi
%             \else
%               \usebox0
%             \fi
%           \else
%             \lwbox
%           \fi
%         \else
%           \usebox0
%         \fi
%       \else
%         \lwbox
%       \fi
%     \else
%       \usebox0
%     \fi
%   \else
%     \lwbox
%   \fi
% \else
%   \usebox0
% \fi
% \end{quote}
% If you have a \xfile{docstrip.cfg} that configures and enables \docstrip's
% TDS installing feature, then some files can already be in the right
% place, see the documentation of \docstrip.
%
% \subsection{Refresh file name databases}
%
% If your \TeX~distribution
% (\teTeX, \mikTeX, \dots) relies on file name databases, you must refresh
% these. For example, \teTeX\ users run \verb|texhash| or
% \verb|mktexlsr|.
%
% \subsection{Some details for the interested}
%
% \paragraph{Attached source.}
%
% The PDF documentation on CTAN also includes the
% \xfile{.dtx} source file. It can be extracted by
% AcrobatReader 6 or higher. Another option is \textsf{pdftk},
% e.g. unpack the file into the current directory:
% \begin{quote}
%   \verb|pdftk refcount.pdf unpack_files output .|
% \end{quote}
%
% \paragraph{Unpacking with \LaTeX.}
% The \xfile{.dtx} chooses its action depending on the format:
% \begin{description}
% \item[\plainTeX:] Run \docstrip\ and extract the files.
% \item[\LaTeX:] Generate the documentation.
% \end{description}
% If you insist on using \LaTeX\ for \docstrip\ (really,
% \docstrip\ does not need \LaTeX), then inform the autodetect routine
% about your intention:
% \begin{quote}
%   \verb|latex \let\install=y\input{refcount.dtx}|
% \end{quote}
% Do not forget to quote the argument according to the demands
% of your shell.
%
% \paragraph{Generating the documentation.}
% You can use both the \xfile{.dtx} or the \xfile{.drv} to generate
% the documentation. The process can be configured by the
% configuration file \xfile{ltxdoc.cfg}. For instance, put this
% line into this file, if you want to have A4 as paper format:
% \begin{quote}
%   \verb|\PassOptionsToClass{a4paper}{article}|
% \end{quote}
% An example follows how to generate the
% documentation with pdf\LaTeX:
% \begin{quote}
%\begin{verbatim}
%pdflatex refcount.dtx
%makeindex -s gind.ist refcount.idx
%pdflatex refcount.dtx
%makeindex -s gind.ist refcount.idx
%pdflatex refcount.dtx
%\end{verbatim}
% \end{quote}
%
% \begin{History}
%   \begin{Version}{1998/04/08 v1.0}
%   \item
%     First public release, written as answer in the
%     newsgroup \xnewsgroup{comp.text.tex}:
%     \URL{``\link{Re: Adding a \cs{ref} to a counter?}''}^^A
%     {http://groups.google.com/group/comp.text.tex/msg/c3f2a135ef5ee528}
%   \end{Version}
%   \begin{Version}{2000/09/07 v2.0}
%   \item
%     Documentation added.
%   \item
%     LPPL 1.2
%   \item
%     Package rewritten, new commands added.
%   \end{Version}
%   \begin{Version}{2006/02/20 v3.0}
%   \item
%     Support for \xpackage{hyperref} and \xpackage{nameref} improved.
%   \item
%     Support for \xpackage{titleref} and \xpackage{babel}'s shorthands added.
%   \item
%     New: \cs{refused}, \cs{getrefbykeydefault}
%   \end{Version}
%   \begin{Version}{2008/08/11 v3.1}
%   \item
%     Code is not changed.
%   \item
%     URLs updated.
%   \end{Version}
% \end{History}
%
% \PrintIndex
%
% \Finale
\endinput
|
% \end{quote}
% Do not forget to quote the argument according to the demands
% of your shell.
%
% \paragraph{Generating the documentation.}
% You can use both the \xfile{.dtx} or the \xfile{.drv} to generate
% the documentation. The process can be configured by the
% configuration file \xfile{ltxdoc.cfg}. For instance, put this
% line into this file, if you want to have A4 as paper format:
% \begin{quote}
%   \verb|\PassOptionsToClass{a4paper}{article}|
% \end{quote}
% An example follows how to generate the
% documentation with pdf\LaTeX:
% \begin{quote}
%\begin{verbatim}
%pdflatex refcount.dtx
%makeindex -s gind.ist refcount.idx
%pdflatex refcount.dtx
%makeindex -s gind.ist refcount.idx
%pdflatex refcount.dtx
%\end{verbatim}
% \end{quote}
%
% \begin{History}
%   \begin{Version}{1998/04/08 v1.0}
%   \item
%     First public release, written as answer in the
%     newsgroup \xnewsgroup{comp.text.tex}:
%     \URL{``\link{Re: Adding a \cs{ref} to a counter?}''}^^A
%     {http://groups.google.com/group/comp.text.tex/msg/c3f2a135ef5ee528}
%   \end{Version}
%   \begin{Version}{2000/09/07 v2.0}
%   \item
%     Documentation added.
%   \item
%     LPPL 1.2
%   \item
%     Package rewritten, new commands added.
%   \end{Version}
%   \begin{Version}{2006/02/20 v3.0}
%   \item
%     Support for \xpackage{hyperref} and \xpackage{nameref} improved.
%   \item
%     Support for \xpackage{titleref} and \xpackage{babel}'s shorthands added.
%   \item
%     New: \cs{refused}, \cs{getrefbykeydefault}
%   \end{Version}
%   \begin{Version}{2008/08/11 v3.1}
%   \item
%     Code is not changed.
%   \item
%     URLs updated.
%   \end{Version}
% \end{History}
%
% \PrintIndex
%
% \Finale
\endinput

%        (quote the arguments according to the demands of your shell)
%
% Documentation:
%    (a) If refcount.drv is present:
%           latex refcount.drv
%    (b) Without refcount.drv:
%           latex refcount.dtx; ...
%    The class ltxdoc loads the configuration file ltxdoc.cfg
%    if available. Here you can specify further options, e.g.
%    use A4 as paper format:
%       \PassOptionsToClass{a4paper}{article}
%
%    Programm calls to get the documentation (example):
%       pdflatex refcount.dtx
%       makeindex -s gind.ist refcount.idx
%       pdflatex refcount.dtx
%       makeindex -s gind.ist refcount.idx
%       pdflatex refcount.dtx
%
% Installation:
%    TDS:tex/latex/oberdiek/refcount.sty
%    TDS:doc/latex/oberdiek/refcount.pdf
%    TDS:source/latex/oberdiek/refcount.dtx
%
%<*ignore>
\begingroup
  \def\x{LaTeX2e}%
\expandafter\endgroup
\ifcase 0\ifx\install y1\fi\expandafter
         \ifx\csname processbatchFile\endcsname\relax\else1\fi
         \ifx\fmtname\x\else 1\fi\relax
\else\csname fi\endcsname
%</ignore>
%<*install>
\input docstrip.tex
\Msg{************************************************************************}
\Msg{* Installation}
\Msg{* Package: refcount 2008/08/11 v3.1 Data extraction from references (HO)}
\Msg{************************************************************************}

\keepsilent
\askforoverwritefalse

\let\MetaPrefix\relax
\preamble

This is a generated file.

Copyright (C) 1998, 2000, 2006, 2008 by
   Heiko Oberdiek <oberdiek@uni-freiburg.de>

This work may be distributed and/or modified under the
conditions of the LaTeX Project Public License, either
version 1.3 of this license or (at your option) any later
version. The latest version of this license is in
   http://www.latex-project.org/lppl.txt
and version 1.3 or later is part of all distributions of
LaTeX version 2005/12/01 or later.

This work has the LPPL maintenance status "maintained".

This Current Maintainer of this work is Heiko Oberdiek.

This work consists of the main source file refcount.dtx
and the derived files
   refcount.sty, refcount.pdf, refcount.ins, refcount.drv.

\endpreamble
\let\MetaPrefix\DoubleperCent

\generate{%
  \file{refcount.ins}{\from{refcount.dtx}{install}}%
  \file{refcount.drv}{\from{refcount.dtx}{driver}}%
  \usedir{tex/latex/oberdiek}%
  \file{refcount.sty}{\from{refcount.dtx}{package}}%
}

\obeyspaces
\Msg{************************************************************************}
\Msg{*}
\Msg{* To finish the installation you have to move the following}
\Msg{* file into a directory searched by TeX:}
\Msg{*}
\Msg{*     refcount.sty}
\Msg{*}
\Msg{* And install the following script file:}
\Msg{*}
\Msg{*     }
\Msg{*}
\Msg{* To produce the documentation run the file `refcount.drv'}
\Msg{* through LaTeX.}
\Msg{*}
\Msg{* Happy TeXing!}
\Msg{*}
\Msg{************************************************************************}

\endbatchfile
%</install>
%<*ignore>
\fi
%</ignore>
%<*driver>
\NeedsTeXFormat{LaTeX2e}
\ProvidesFile{refcount.drv}%
  [2008/08/11 v3.1 Data extraction from references (HO)]%
\documentclass{ltxdoc}
\usepackage{holtxdoc}[2008/08/11]
\begin{document}
  \DocInput{refcount.dtx}%
\end{document}
%</driver>
% \fi
%
% \CheckSum{198}
%
% \CharacterTable
%  {Upper-case    \A\B\C\D\E\F\G\H\I\J\K\L\M\N\O\P\Q\R\S\T\U\V\W\X\Y\Z
%   Lower-case    \a\b\c\d\e\f\g\h\i\j\k\l\m\n\o\p\q\r\s\t\u\v\w\x\y\z
%   Digits        \0\1\2\3\4\5\6\7\8\9
%   Exclamation   \!     Double quote  \"     Hash (number) \#
%   Dollar        \$     Percent       \%     Ampersand     \&
%   Acute accent  \'     Left paren    \(     Right paren   \)
%   Asterisk      \*     Plus          \+     Comma         \,
%   Minus         \-     Point         \.     Solidus       \/
%   Colon         \:     Semicolon     \;     Less than     \<
%   Equals        \=     Greater than  \>     Question mark \?
%   Commercial at \@     Left bracket  \[     Backslash     \\
%   Right bracket \]     Circumflex    \^     Underscore    \_
%   Grave accent  \`     Left brace    \{     Vertical bar  \|
%   Right brace   \}     Tilde         \~}
%
% \GetFileInfo{refcount.drv}
%
% \title{The \xpackage{refcount} package}
% \date{2008/08/11 v3.1}
% \author{Heiko Oberdiek\\\xemail{oberdiek@uni-freiburg.de}}
%
% \maketitle
%
% \begin{abstract}
% References are not numbers, however they often store numerical
% data such as section or page numbers. \cs{ref} or \cs{pageref}
% cannot be used for counter assignments or calculations because
% they are not expandable, generate warnings, or can even be links,
% The package provides expandable macros to extract the data
% from references. Packages \xpackage{hyperref}, \xpackage{nameref},
% \xpackage{titleref}, and \xpackage{babel} are supported.
% \end{abstract}
%
% \tableofcontents
%
% \section{Usage}
%
% \subsection{Setting counters}
%
% The following commands are similar to \LaTeX's
% \cs{setcounter} and \cs{addtocounter},
% but they extract the number value from a reference:
% \begin{quote}
%   \cs{setcounterref}, \cs{addtocounterref}\\
%   \cs{setcounterpageref}, \cs{addtocounterpageref}
% \end{quote}
% They take two arguments:
% \begin{quote}
%    \cs{...counter...ref} |{|\meta{\LaTeX\ counter}|}|
%    |{|\meta{reference}|}|
% \end{quote}
% An undefined references produces the usual LaTeX warning
% and its value is assumed to be zero.
% Example:
% \begin{quote}
%\begin{verbatim}
%\newcounter{ctrA}
%\newcounter{ctrB}
%\refstepcounter{ctrA}\label{ref:A}
%\setcounterref{ctrB}{ref:A}
%\addtocounterpageref{ctrB}{ref:A}
%\end{verbatim}
% \end{quote}
%
% \subsection{Expandable commands}
%
% These commands that can be used in expandible contexts
% (inside calculations, \cs{edef}, \cs{csname}, \cs{write}, \dots):
% \begin{quote}
%   \cs{getrefnumber}, \cs{getpagerefnumber}
% \end{quote}
% They take one argument, the reference:
% \begin{quote}
%   \cs{get...refnumber} |{|\meta{reference}|}|
% \end{quote}
% The default for undefined references can be changed
% with macro \cs{setrefcountdefault}, for example this
% package calls:
% \begin{quote}
%   \cs{setrefcountdefault}|{0}|
% \end{quote}
%
% Since version 2.0 of this package there is a new
% command:
% \begin{quote}
%   \cs{getrefbykeydefault} |{|\meta{reference}|}|
%   |{|\meta{key}|}| |{|\meta{default}|}|
% \end{quote}
% This generalized version allows the extraction
% of further properties of a reference than the
% two standard ones. Thus the following properties
% are supported, if they are available:
% \begin{quote}
% \begin{tabular}{@{}l|l|l@{}}
%    Key & Description & Package\\
% \hline
%   \meta{empty} & same as \cs{ref} & \LaTeX\\
%   |page| & same as \cs{pageref} & \LaTeX\\
%   |title| & section and caption titles & \xpackage{titleref}\\
%   |name| & section and caption titles & \xpackage{nameref}\\
%   |anchor| & anchor name & \xpackage{hyperref}\\
%   |url| & url/file & \xpackage{hyperref}/\xpackage{xr}
% \end{tabular}
% \end{quote}
%
% \subsection{Undefined references}
%
% Because warnings and assignments cannot be used in
% expandible contexts, undefined references do not
% produce a warning, their values are assumed to be zero.
% Example:
% \begin{quote}
%\begin{verbatim}
%\label{ref:here}% somewhere
%\refused{ref:here}% see below
%\ifodd\getpagerefnumber{ref:here}%
%  reference is on an odd page
%\else
%  reference is on an even page
%\fi
%\end{verbatim}
% \end{quote}
%
% In case of undefined references the user usually want's
% to be informed. Also \LaTeX\ prints a warning at
% the end of the \LaTeX\ run. To notify \LaTeX\ and
% get a normal warning, just use
% \begin{quote}
%   \cs{refused} |{|\meta{reference}|}|
% \end{quote}
% outside the expanding context. Example, see above.
%
% \subsection{Notes}
%
% \begin{itemize}
% \item
%   The method of extracting the number in this
%   package also works in cases, where the
%   reference cannot be used directly, because
%   a package such as \xpackage{hyperref} has added
%   extra stuff (hyper link), so that the reference cannot
%   be used as number any more.
% \item
%   If the reference does not contain a number,
%   assignments to a counter will fail of course.
% \end{itemize}
%
%
% \StopEventually{
% }
%
% \section{Implementation}
%
%    \begin{macrocode}
%<*package>
\NeedsTeXFormat{LaTeX2e}
\ProvidesPackage{refcount}
  [2008/08/11 v3.1 Data extraction from references (HO)]%

\def\setrefcountdefault#1{%
  \def\rc@default{#1}%
}
\setrefcountdefault{0}

% \def\@car#1#2\@nil{#1} % defined in LaTeX kernel
\def\rc@cartwo#1#2#3\@nil{#2}

% generic check without babel support
\long\def\rc@refused#1{%
  \expandafter\ifx\csname r@#1\endcsname\relax
    \protect\G@refundefinedtrue
    \@latex@warning{%
      Reference `#1' on page \thepage\space undefined%
    }%
  \fi
}

% user command, add babel support
\newcommand*{\refused}[1]{%
  \begingroup
    \csname @safe@activestrue\endcsname
    \rc@refused{#1}{}%
  \endgroup
}

% Generic command for "\{set,addto}counter{page,}ref":
% #1: \setcounter, \addtocounter
% #2: \@car (for \ref), \@cartwo (for \pageref)
% #3: LaTeX counter
% #4: reference
\def\rc@set#1#2#3#4{%
  \begingroup
    \csname @safe@activestrue\endcsname
    \rc@refused{#4}%
    \expandafter\rc@@set\csname r@#4\endcsname{#1}{#2}{#3}%
  \endgroup
}
% #1: \r@<...>
% #2: \setcounter, \addtocounter
% #3: \@car (for \ref), \@cartwo (for \pageref)
% #4: LaTeX counter
\def\rc@@set#1#2#3#4{%
  \ifx#1\relax
    #2{#4}{\rc@default}%
  \else
    #2{#4}{%
      \expandafter#3#1\rc@default\rc@default\@nil
    }%
  \fi
}

% user commands:

\newcommand*{\setcounterref}{\rc@set\setcounter\@car}
\newcommand*{\addtocounterref}{\rc@set\addtocounter\@car}
\newcommand*{\setcounterpageref}{\rc@set\setcounter\rc@cartwo}
\newcommand*{\addtocounterpageref}{\rc@set\addtocounter\rc@cartwo}

\newcommand*{\getrefnumber}[1]{%
  \expandafter\ifx\csname r@#1\endcsname\relax
    \rc@default
  \else
    \expandafter\expandafter\expandafter\@car
    \csname r@#1\endcsname\@nil
  \fi
}
\newcommand*{\getpagerefnumber}[1]{%
  \expandafter\ifx\csname r@#1\endcsname\relax
    \rc@default
  \else
    \expandafter\expandafter\expandafter\rc@cartwo
    \csname r@#1\endcsname\rc@default\rc@default\@nil
  \fi
}
\newcommand*{\getrefbykeydefault}[2]{%
  \expandafter\rc@getrefbykeydefault
    \csname r@#1\expandafter\endcsname
    \csname rc@extract@#2\endcsname
}
% #1: \r@<...>
% #2: \rc@extract@<...>
% #3: default
\def\rc@getrefbykeydefault#1#2#3{%
  \ifx#1\relax
    % reference is undefined
    #3%
  \else
    \ifx#2\relax
      % extract method is missing
      #3%
    \else
      \expandafter\rc@generic#1{#3}{#3}{#3}{#3}{#3}\@nil#2{#3}%
    \fi
  \fi
}
% #1: first item in \r@<...>
% #2: remaining items in \r@<...>
% #3: \rc@extract@<...>
% #4: default
\def\rc@generic#1#2\@nil#3#4{%
  #3{#1\TR@TitleReference\@empty{#4}\@nil}{#1}#2\@nil
}
\def\rc@extract@{%
  \expandafter\@car\@gobble
}
\def\rc@extract@page{%
  \expandafter\@car\@gobbletwo
}
\def\rc@extract@name{%
  \expandafter\@car\@gobblefour\@empty
}
\def\rc@extract@anchor{%
  \expandafter\@car\@gobblefour
}
\def\rc@extract@url{%
  \expandafter\expandafter\expandafter\@car\expandafter
      \@gobble\@gobblefour
}
\def\rc@extract@title#1#2\@nil{%
  \rc@@extract@title#1%
}
\def\rc@@extract@title#1\TR@TitleReference#2#3#4\@nil{#3}
%</package>
%    \end{macrocode}
%
% \section{Installation}
%
% \subsection{Download}
%
% \paragraph{Package.} This package is available on
% CTAN\footnote{\url{ftp://ftp.ctan.org/tex-archive/}}:
% \begin{description}
% \item[\CTAN{macros/latex/contrib/oberdiek/refcount.dtx}] The source file.
% \item[\CTAN{macros/latex/contrib/oberdiek/refcount.pdf}] Documentation.
% \end{description}
%
%
% \paragraph{Bundle.} All the packages of the bundle `oberdiek'
% are also available in a TDS compliant ZIP archive. There
% the packages are already unpacked and the documentation files
% are generated. The files and directories obey the TDS standard.
% \begin{description}
% \item[\CTAN{install/macros/latex/contrib/oberdiek.tds.zip}]
% \end{description}
% \emph{TDS} refers to the standard ``A Directory Structure
% for \TeX\ Files'' (\CTAN{tds/tds.pdf}). Directories
% with \xfile{texmf} in their name are usually organized this way.
%
% \subsection{Bundle installation}
%
% \paragraph{Unpacking.} Unpack the \xfile{oberdiek.tds.zip} in the
% TDS tree (also known as \xfile{texmf} tree) of your choice.
% Example (linux):
% \begin{quote}
%   |unzip oberdiek.tds.zip -d ~/texmf|
% \end{quote}
%
% \paragraph{Script installation.}
% Check the directory \xfile{TDS:scripts/oberdiek/} for
% scripts that need further installation steps.
% Package \xpackage{attachfile2} comes with the Perl script
% \xfile{pdfatfi.pl} that should be installed in such a way
% that it can be called as \texttt{pdfatfi}.
% Example (linux):
% \begin{quote}
%   |chmod +x scripts/oberdiek/pdfatfi.pl|\\
%   |cp scripts/oberdiek/pdfatfi.pl /usr/local/bin/|
% \end{quote}
%
% \subsection{Package installation}
%
% \paragraph{Unpacking.} The \xfile{.dtx} file is a self-extracting
% \docstrip\ archive. The files are extracted by running the
% \xfile{.dtx} through \plainTeX:
% \begin{quote}
%   \verb|tex refcount.dtx|
% \end{quote}
%
% \paragraph{TDS.} Now the different files must be moved into
% the different directories in your installation TDS tree
% (also known as \xfile{texmf} tree):
% \begin{quote}
% \def\t{^^A
% \begin{tabular}{@{}>{\ttfamily}l@{ $\rightarrow$ }>{\ttfamily}l@{}}
%   refcount.sty & tex/latex/oberdiek/refcount.sty\\
%   refcount.pdf & doc/latex/oberdiek/refcount.pdf\\
%   refcount.dtx & source/latex/oberdiek/refcount.dtx\\
% \end{tabular}^^A
% }^^A
% \sbox0{\t}^^A
% \ifdim\wd0>\linewidth
%   \begingroup
%     \advance\linewidth by\leftmargin
%     \advance\linewidth by\rightmargin
%   \edef\x{\endgroup
%     \def\noexpand\lw{\the\linewidth}^^A
%   }\x
%   \def\lwbox{^^A
%     \leavevmode
%     \hbox to \linewidth{^^A
%       \kern-\leftmargin\relax
%       \hss
%       \usebox0
%       \hss
%       \kern-\rightmargin\relax
%     }^^A
%   }^^A
%   \ifdim\wd0>\lw
%     \sbox0{\small\t}^^A
%     \ifdim\wd0>\linewidth
%       \ifdim\wd0>\lw
%         \sbox0{\footnotesize\t}^^A
%         \ifdim\wd0>\linewidth
%           \ifdim\wd0>\lw
%             \sbox0{\scriptsize\t}^^A
%             \ifdim\wd0>\linewidth
%               \ifdim\wd0>\lw
%                 \sbox0{\tiny\t}^^A
%                 \ifdim\wd0>\linewidth
%                   \lwbox
%                 \else
%                   \usebox0
%                 \fi
%               \else
%                 \lwbox
%               \fi
%             \else
%               \usebox0
%             \fi
%           \else
%             \lwbox
%           \fi
%         \else
%           \usebox0
%         \fi
%       \else
%         \lwbox
%       \fi
%     \else
%       \usebox0
%     \fi
%   \else
%     \lwbox
%   \fi
% \else
%   \usebox0
% \fi
% \end{quote}
% If you have a \xfile{docstrip.cfg} that configures and enables \docstrip's
% TDS installing feature, then some files can already be in the right
% place, see the documentation of \docstrip.
%
% \subsection{Refresh file name databases}
%
% If your \TeX~distribution
% (\teTeX, \mikTeX, \dots) relies on file name databases, you must refresh
% these. For example, \teTeX\ users run \verb|texhash| or
% \verb|mktexlsr|.
%
% \subsection{Some details for the interested}
%
% \paragraph{Attached source.}
%
% The PDF documentation on CTAN also includes the
% \xfile{.dtx} source file. It can be extracted by
% AcrobatReader 6 or higher. Another option is \textsf{pdftk},
% e.g. unpack the file into the current directory:
% \begin{quote}
%   \verb|pdftk refcount.pdf unpack_files output .|
% \end{quote}
%
% \paragraph{Unpacking with \LaTeX.}
% The \xfile{.dtx} chooses its action depending on the format:
% \begin{description}
% \item[\plainTeX:] Run \docstrip\ and extract the files.
% \item[\LaTeX:] Generate the documentation.
% \end{description}
% If you insist on using \LaTeX\ for \docstrip\ (really,
% \docstrip\ does not need \LaTeX), then inform the autodetect routine
% about your intention:
% \begin{quote}
%   \verb|latex \let\install=y% \iffalse meta-comment
%
% Copyright (C) 1998, 2000, 2006, 2008 by
%    Heiko Oberdiek <oberdiek@uni-freiburg.de>
%
% This work may be distributed and/or modified under the
% conditions of the LaTeX Project Public License, either
% version 1.3 of this license or (at your option) any later
% version. The latest version of this license is in
%    http://www.latex-project.org/lppl.txt
% and version 1.3 or later is part of all distributions of
% LaTeX version 2005/12/01 or later.
%
% This work has the LPPL maintenance status "maintained".
%
% This Current Maintainer of this work is Heiko Oberdiek.
%
% This work consists of the main source file refcount.dtx
% and the derived files
%    refcount.sty, refcount.pdf, refcount.ins, refcount.drv.
%
% Distribution:
%    CTAN:macros/latex/contrib/oberdiek/refcount.dtx
%    CTAN:macros/latex/contrib/oberdiek/refcount.pdf
%
% Unpacking:
%    (a) If refcount.ins is present:
%           tex refcount.ins
%    (b) Without refcount.ins:
%           tex refcount.dtx
%    (c) If you insist on using LaTeX
%           latex \let\install=y% \iffalse meta-comment
%
% Copyright (C) 1998, 2000, 2006, 2008 by
%    Heiko Oberdiek <oberdiek@uni-freiburg.de>
%
% This work may be distributed and/or modified under the
% conditions of the LaTeX Project Public License, either
% version 1.3 of this license or (at your option) any later
% version. The latest version of this license is in
%    http://www.latex-project.org/lppl.txt
% and version 1.3 or later is part of all distributions of
% LaTeX version 2005/12/01 or later.
%
% This work has the LPPL maintenance status "maintained".
%
% This Current Maintainer of this work is Heiko Oberdiek.
%
% This work consists of the main source file refcount.dtx
% and the derived files
%    refcount.sty, refcount.pdf, refcount.ins, refcount.drv.
%
% Distribution:
%    CTAN:macros/latex/contrib/oberdiek/refcount.dtx
%    CTAN:macros/latex/contrib/oberdiek/refcount.pdf
%
% Unpacking:
%    (a) If refcount.ins is present:
%           tex refcount.ins
%    (b) Without refcount.ins:
%           tex refcount.dtx
%    (c) If you insist on using LaTeX
%           latex \let\install=y\input{refcount.dtx}
%        (quote the arguments according to the demands of your shell)
%
% Documentation:
%    (a) If refcount.drv is present:
%           latex refcount.drv
%    (b) Without refcount.drv:
%           latex refcount.dtx; ...
%    The class ltxdoc loads the configuration file ltxdoc.cfg
%    if available. Here you can specify further options, e.g.
%    use A4 as paper format:
%       \PassOptionsToClass{a4paper}{article}
%
%    Programm calls to get the documentation (example):
%       pdflatex refcount.dtx
%       makeindex -s gind.ist refcount.idx
%       pdflatex refcount.dtx
%       makeindex -s gind.ist refcount.idx
%       pdflatex refcount.dtx
%
% Installation:
%    TDS:tex/latex/oberdiek/refcount.sty
%    TDS:doc/latex/oberdiek/refcount.pdf
%    TDS:source/latex/oberdiek/refcount.dtx
%
%<*ignore>
\begingroup
  \def\x{LaTeX2e}%
\expandafter\endgroup
\ifcase 0\ifx\install y1\fi\expandafter
         \ifx\csname processbatchFile\endcsname\relax\else1\fi
         \ifx\fmtname\x\else 1\fi\relax
\else\csname fi\endcsname
%</ignore>
%<*install>
\input docstrip.tex
\Msg{************************************************************************}
\Msg{* Installation}
\Msg{* Package: refcount 2008/08/11 v3.1 Data extraction from references (HO)}
\Msg{************************************************************************}

\keepsilent
\askforoverwritefalse

\let\MetaPrefix\relax
\preamble

This is a generated file.

Copyright (C) 1998, 2000, 2006, 2008 by
   Heiko Oberdiek <oberdiek@uni-freiburg.de>

This work may be distributed and/or modified under the
conditions of the LaTeX Project Public License, either
version 1.3 of this license or (at your option) any later
version. The latest version of this license is in
   http://www.latex-project.org/lppl.txt
and version 1.3 or later is part of all distributions of
LaTeX version 2005/12/01 or later.

This work has the LPPL maintenance status "maintained".

This Current Maintainer of this work is Heiko Oberdiek.

This work consists of the main source file refcount.dtx
and the derived files
   refcount.sty, refcount.pdf, refcount.ins, refcount.drv.

\endpreamble
\let\MetaPrefix\DoubleperCent

\generate{%
  \file{refcount.ins}{\from{refcount.dtx}{install}}%
  \file{refcount.drv}{\from{refcount.dtx}{driver}}%
  \usedir{tex/latex/oberdiek}%
  \file{refcount.sty}{\from{refcount.dtx}{package}}%
}

\obeyspaces
\Msg{************************************************************************}
\Msg{*}
\Msg{* To finish the installation you have to move the following}
\Msg{* file into a directory searched by TeX:}
\Msg{*}
\Msg{*     refcount.sty}
\Msg{*}
\Msg{* And install the following script file:}
\Msg{*}
\Msg{*     }
\Msg{*}
\Msg{* To produce the documentation run the file `refcount.drv'}
\Msg{* through LaTeX.}
\Msg{*}
\Msg{* Happy TeXing!}
\Msg{*}
\Msg{************************************************************************}

\endbatchfile
%</install>
%<*ignore>
\fi
%</ignore>
%<*driver>
\NeedsTeXFormat{LaTeX2e}
\ProvidesFile{refcount.drv}%
  [2008/08/11 v3.1 Data extraction from references (HO)]%
\documentclass{ltxdoc}
\usepackage{holtxdoc}[2008/08/11]
\begin{document}
  \DocInput{refcount.dtx}%
\end{document}
%</driver>
% \fi
%
% \CheckSum{198}
%
% \CharacterTable
%  {Upper-case    \A\B\C\D\E\F\G\H\I\J\K\L\M\N\O\P\Q\R\S\T\U\V\W\X\Y\Z
%   Lower-case    \a\b\c\d\e\f\g\h\i\j\k\l\m\n\o\p\q\r\s\t\u\v\w\x\y\z
%   Digits        \0\1\2\3\4\5\6\7\8\9
%   Exclamation   \!     Double quote  \"     Hash (number) \#
%   Dollar        \$     Percent       \%     Ampersand     \&
%   Acute accent  \'     Left paren    \(     Right paren   \)
%   Asterisk      \*     Plus          \+     Comma         \,
%   Minus         \-     Point         \.     Solidus       \/
%   Colon         \:     Semicolon     \;     Less than     \<
%   Equals        \=     Greater than  \>     Question mark \?
%   Commercial at \@     Left bracket  \[     Backslash     \\
%   Right bracket \]     Circumflex    \^     Underscore    \_
%   Grave accent  \`     Left brace    \{     Vertical bar  \|
%   Right brace   \}     Tilde         \~}
%
% \GetFileInfo{refcount.drv}
%
% \title{The \xpackage{refcount} package}
% \date{2008/08/11 v3.1}
% \author{Heiko Oberdiek\\\xemail{oberdiek@uni-freiburg.de}}
%
% \maketitle
%
% \begin{abstract}
% References are not numbers, however they often store numerical
% data such as section or page numbers. \cs{ref} or \cs{pageref}
% cannot be used for counter assignments or calculations because
% they are not expandable, generate warnings, or can even be links,
% The package provides expandable macros to extract the data
% from references. Packages \xpackage{hyperref}, \xpackage{nameref},
% \xpackage{titleref}, and \xpackage{babel} are supported.
% \end{abstract}
%
% \tableofcontents
%
% \section{Usage}
%
% \subsection{Setting counters}
%
% The following commands are similar to \LaTeX's
% \cs{setcounter} and \cs{addtocounter},
% but they extract the number value from a reference:
% \begin{quote}
%   \cs{setcounterref}, \cs{addtocounterref}\\
%   \cs{setcounterpageref}, \cs{addtocounterpageref}
% \end{quote}
% They take two arguments:
% \begin{quote}
%    \cs{...counter...ref} |{|\meta{\LaTeX\ counter}|}|
%    |{|\meta{reference}|}|
% \end{quote}
% An undefined references produces the usual LaTeX warning
% and its value is assumed to be zero.
% Example:
% \begin{quote}
%\begin{verbatim}
%\newcounter{ctrA}
%\newcounter{ctrB}
%\refstepcounter{ctrA}\label{ref:A}
%\setcounterref{ctrB}{ref:A}
%\addtocounterpageref{ctrB}{ref:A}
%\end{verbatim}
% \end{quote}
%
% \subsection{Expandable commands}
%
% These commands that can be used in expandible contexts
% (inside calculations, \cs{edef}, \cs{csname}, \cs{write}, \dots):
% \begin{quote}
%   \cs{getrefnumber}, \cs{getpagerefnumber}
% \end{quote}
% They take one argument, the reference:
% \begin{quote}
%   \cs{get...refnumber} |{|\meta{reference}|}|
% \end{quote}
% The default for undefined references can be changed
% with macro \cs{setrefcountdefault}, for example this
% package calls:
% \begin{quote}
%   \cs{setrefcountdefault}|{0}|
% \end{quote}
%
% Since version 2.0 of this package there is a new
% command:
% \begin{quote}
%   \cs{getrefbykeydefault} |{|\meta{reference}|}|
%   |{|\meta{key}|}| |{|\meta{default}|}|
% \end{quote}
% This generalized version allows the extraction
% of further properties of a reference than the
% two standard ones. Thus the following properties
% are supported, if they are available:
% \begin{quote}
% \begin{tabular}{@{}l|l|l@{}}
%    Key & Description & Package\\
% \hline
%   \meta{empty} & same as \cs{ref} & \LaTeX\\
%   |page| & same as \cs{pageref} & \LaTeX\\
%   |title| & section and caption titles & \xpackage{titleref}\\
%   |name| & section and caption titles & \xpackage{nameref}\\
%   |anchor| & anchor name & \xpackage{hyperref}\\
%   |url| & url/file & \xpackage{hyperref}/\xpackage{xr}
% \end{tabular}
% \end{quote}
%
% \subsection{Undefined references}
%
% Because warnings and assignments cannot be used in
% expandible contexts, undefined references do not
% produce a warning, their values are assumed to be zero.
% Example:
% \begin{quote}
%\begin{verbatim}
%\label{ref:here}% somewhere
%\refused{ref:here}% see below
%\ifodd\getpagerefnumber{ref:here}%
%  reference is on an odd page
%\else
%  reference is on an even page
%\fi
%\end{verbatim}
% \end{quote}
%
% In case of undefined references the user usually want's
% to be informed. Also \LaTeX\ prints a warning at
% the end of the \LaTeX\ run. To notify \LaTeX\ and
% get a normal warning, just use
% \begin{quote}
%   \cs{refused} |{|\meta{reference}|}|
% \end{quote}
% outside the expanding context. Example, see above.
%
% \subsection{Notes}
%
% \begin{itemize}
% \item
%   The method of extracting the number in this
%   package also works in cases, where the
%   reference cannot be used directly, because
%   a package such as \xpackage{hyperref} has added
%   extra stuff (hyper link), so that the reference cannot
%   be used as number any more.
% \item
%   If the reference does not contain a number,
%   assignments to a counter will fail of course.
% \end{itemize}
%
%
% \StopEventually{
% }
%
% \section{Implementation}
%
%    \begin{macrocode}
%<*package>
\NeedsTeXFormat{LaTeX2e}
\ProvidesPackage{refcount}
  [2008/08/11 v3.1 Data extraction from references (HO)]%

\def\setrefcountdefault#1{%
  \def\rc@default{#1}%
}
\setrefcountdefault{0}

% \def\@car#1#2\@nil{#1} % defined in LaTeX kernel
\def\rc@cartwo#1#2#3\@nil{#2}

% generic check without babel support
\long\def\rc@refused#1{%
  \expandafter\ifx\csname r@#1\endcsname\relax
    \protect\G@refundefinedtrue
    \@latex@warning{%
      Reference `#1' on page \thepage\space undefined%
    }%
  \fi
}

% user command, add babel support
\newcommand*{\refused}[1]{%
  \begingroup
    \csname @safe@activestrue\endcsname
    \rc@refused{#1}{}%
  \endgroup
}

% Generic command for "\{set,addto}counter{page,}ref":
% #1: \setcounter, \addtocounter
% #2: \@car (for \ref), \@cartwo (for \pageref)
% #3: LaTeX counter
% #4: reference
\def\rc@set#1#2#3#4{%
  \begingroup
    \csname @safe@activestrue\endcsname
    \rc@refused{#4}%
    \expandafter\rc@@set\csname r@#4\endcsname{#1}{#2}{#3}%
  \endgroup
}
% #1: \r@<...>
% #2: \setcounter, \addtocounter
% #3: \@car (for \ref), \@cartwo (for \pageref)
% #4: LaTeX counter
\def\rc@@set#1#2#3#4{%
  \ifx#1\relax
    #2{#4}{\rc@default}%
  \else
    #2{#4}{%
      \expandafter#3#1\rc@default\rc@default\@nil
    }%
  \fi
}

% user commands:

\newcommand*{\setcounterref}{\rc@set\setcounter\@car}
\newcommand*{\addtocounterref}{\rc@set\addtocounter\@car}
\newcommand*{\setcounterpageref}{\rc@set\setcounter\rc@cartwo}
\newcommand*{\addtocounterpageref}{\rc@set\addtocounter\rc@cartwo}

\newcommand*{\getrefnumber}[1]{%
  \expandafter\ifx\csname r@#1\endcsname\relax
    \rc@default
  \else
    \expandafter\expandafter\expandafter\@car
    \csname r@#1\endcsname\@nil
  \fi
}
\newcommand*{\getpagerefnumber}[1]{%
  \expandafter\ifx\csname r@#1\endcsname\relax
    \rc@default
  \else
    \expandafter\expandafter\expandafter\rc@cartwo
    \csname r@#1\endcsname\rc@default\rc@default\@nil
  \fi
}
\newcommand*{\getrefbykeydefault}[2]{%
  \expandafter\rc@getrefbykeydefault
    \csname r@#1\expandafter\endcsname
    \csname rc@extract@#2\endcsname
}
% #1: \r@<...>
% #2: \rc@extract@<...>
% #3: default
\def\rc@getrefbykeydefault#1#2#3{%
  \ifx#1\relax
    % reference is undefined
    #3%
  \else
    \ifx#2\relax
      % extract method is missing
      #3%
    \else
      \expandafter\rc@generic#1{#3}{#3}{#3}{#3}{#3}\@nil#2{#3}%
    \fi
  \fi
}
% #1: first item in \r@<...>
% #2: remaining items in \r@<...>
% #3: \rc@extract@<...>
% #4: default
\def\rc@generic#1#2\@nil#3#4{%
  #3{#1\TR@TitleReference\@empty{#4}\@nil}{#1}#2\@nil
}
\def\rc@extract@{%
  \expandafter\@car\@gobble
}
\def\rc@extract@page{%
  \expandafter\@car\@gobbletwo
}
\def\rc@extract@name{%
  \expandafter\@car\@gobblefour\@empty
}
\def\rc@extract@anchor{%
  \expandafter\@car\@gobblefour
}
\def\rc@extract@url{%
  \expandafter\expandafter\expandafter\@car\expandafter
      \@gobble\@gobblefour
}
\def\rc@extract@title#1#2\@nil{%
  \rc@@extract@title#1%
}
\def\rc@@extract@title#1\TR@TitleReference#2#3#4\@nil{#3}
%</package>
%    \end{macrocode}
%
% \section{Installation}
%
% \subsection{Download}
%
% \paragraph{Package.} This package is available on
% CTAN\footnote{\url{ftp://ftp.ctan.org/tex-archive/}}:
% \begin{description}
% \item[\CTAN{macros/latex/contrib/oberdiek/refcount.dtx}] The source file.
% \item[\CTAN{macros/latex/contrib/oberdiek/refcount.pdf}] Documentation.
% \end{description}
%
%
% \paragraph{Bundle.} All the packages of the bundle `oberdiek'
% are also available in a TDS compliant ZIP archive. There
% the packages are already unpacked and the documentation files
% are generated. The files and directories obey the TDS standard.
% \begin{description}
% \item[\CTAN{install/macros/latex/contrib/oberdiek.tds.zip}]
% \end{description}
% \emph{TDS} refers to the standard ``A Directory Structure
% for \TeX\ Files'' (\CTAN{tds/tds.pdf}). Directories
% with \xfile{texmf} in their name are usually organized this way.
%
% \subsection{Bundle installation}
%
% \paragraph{Unpacking.} Unpack the \xfile{oberdiek.tds.zip} in the
% TDS tree (also known as \xfile{texmf} tree) of your choice.
% Example (linux):
% \begin{quote}
%   |unzip oberdiek.tds.zip -d ~/texmf|
% \end{quote}
%
% \paragraph{Script installation.}
% Check the directory \xfile{TDS:scripts/oberdiek/} for
% scripts that need further installation steps.
% Package \xpackage{attachfile2} comes with the Perl script
% \xfile{pdfatfi.pl} that should be installed in such a way
% that it can be called as \texttt{pdfatfi}.
% Example (linux):
% \begin{quote}
%   |chmod +x scripts/oberdiek/pdfatfi.pl|\\
%   |cp scripts/oberdiek/pdfatfi.pl /usr/local/bin/|
% \end{quote}
%
% \subsection{Package installation}
%
% \paragraph{Unpacking.} The \xfile{.dtx} file is a self-extracting
% \docstrip\ archive. The files are extracted by running the
% \xfile{.dtx} through \plainTeX:
% \begin{quote}
%   \verb|tex refcount.dtx|
% \end{quote}
%
% \paragraph{TDS.} Now the different files must be moved into
% the different directories in your installation TDS tree
% (also known as \xfile{texmf} tree):
% \begin{quote}
% \def\t{^^A
% \begin{tabular}{@{}>{\ttfamily}l@{ $\rightarrow$ }>{\ttfamily}l@{}}
%   refcount.sty & tex/latex/oberdiek/refcount.sty\\
%   refcount.pdf & doc/latex/oberdiek/refcount.pdf\\
%   refcount.dtx & source/latex/oberdiek/refcount.dtx\\
% \end{tabular}^^A
% }^^A
% \sbox0{\t}^^A
% \ifdim\wd0>\linewidth
%   \begingroup
%     \advance\linewidth by\leftmargin
%     \advance\linewidth by\rightmargin
%   \edef\x{\endgroup
%     \def\noexpand\lw{\the\linewidth}^^A
%   }\x
%   \def\lwbox{^^A
%     \leavevmode
%     \hbox to \linewidth{^^A
%       \kern-\leftmargin\relax
%       \hss
%       \usebox0
%       \hss
%       \kern-\rightmargin\relax
%     }^^A
%   }^^A
%   \ifdim\wd0>\lw
%     \sbox0{\small\t}^^A
%     \ifdim\wd0>\linewidth
%       \ifdim\wd0>\lw
%         \sbox0{\footnotesize\t}^^A
%         \ifdim\wd0>\linewidth
%           \ifdim\wd0>\lw
%             \sbox0{\scriptsize\t}^^A
%             \ifdim\wd0>\linewidth
%               \ifdim\wd0>\lw
%                 \sbox0{\tiny\t}^^A
%                 \ifdim\wd0>\linewidth
%                   \lwbox
%                 \else
%                   \usebox0
%                 \fi
%               \else
%                 \lwbox
%               \fi
%             \else
%               \usebox0
%             \fi
%           \else
%             \lwbox
%           \fi
%         \else
%           \usebox0
%         \fi
%       \else
%         \lwbox
%       \fi
%     \else
%       \usebox0
%     \fi
%   \else
%     \lwbox
%   \fi
% \else
%   \usebox0
% \fi
% \end{quote}
% If you have a \xfile{docstrip.cfg} that configures and enables \docstrip's
% TDS installing feature, then some files can already be in the right
% place, see the documentation of \docstrip.
%
% \subsection{Refresh file name databases}
%
% If your \TeX~distribution
% (\teTeX, \mikTeX, \dots) relies on file name databases, you must refresh
% these. For example, \teTeX\ users run \verb|texhash| or
% \verb|mktexlsr|.
%
% \subsection{Some details for the interested}
%
% \paragraph{Attached source.}
%
% The PDF documentation on CTAN also includes the
% \xfile{.dtx} source file. It can be extracted by
% AcrobatReader 6 or higher. Another option is \textsf{pdftk},
% e.g. unpack the file into the current directory:
% \begin{quote}
%   \verb|pdftk refcount.pdf unpack_files output .|
% \end{quote}
%
% \paragraph{Unpacking with \LaTeX.}
% The \xfile{.dtx} chooses its action depending on the format:
% \begin{description}
% \item[\plainTeX:] Run \docstrip\ and extract the files.
% \item[\LaTeX:] Generate the documentation.
% \end{description}
% If you insist on using \LaTeX\ for \docstrip\ (really,
% \docstrip\ does not need \LaTeX), then inform the autodetect routine
% about your intention:
% \begin{quote}
%   \verb|latex \let\install=y\input{refcount.dtx}|
% \end{quote}
% Do not forget to quote the argument according to the demands
% of your shell.
%
% \paragraph{Generating the documentation.}
% You can use both the \xfile{.dtx} or the \xfile{.drv} to generate
% the documentation. The process can be configured by the
% configuration file \xfile{ltxdoc.cfg}. For instance, put this
% line into this file, if you want to have A4 as paper format:
% \begin{quote}
%   \verb|\PassOptionsToClass{a4paper}{article}|
% \end{quote}
% An example follows how to generate the
% documentation with pdf\LaTeX:
% \begin{quote}
%\begin{verbatim}
%pdflatex refcount.dtx
%makeindex -s gind.ist refcount.idx
%pdflatex refcount.dtx
%makeindex -s gind.ist refcount.idx
%pdflatex refcount.dtx
%\end{verbatim}
% \end{quote}
%
% \begin{History}
%   \begin{Version}{1998/04/08 v1.0}
%   \item
%     First public release, written as answer in the
%     newsgroup \xnewsgroup{comp.text.tex}:
%     \URL{``\link{Re: Adding a \cs{ref} to a counter?}''}^^A
%     {http://groups.google.com/group/comp.text.tex/msg/c3f2a135ef5ee528}
%   \end{Version}
%   \begin{Version}{2000/09/07 v2.0}
%   \item
%     Documentation added.
%   \item
%     LPPL 1.2
%   \item
%     Package rewritten, new commands added.
%   \end{Version}
%   \begin{Version}{2006/02/20 v3.0}
%   \item
%     Support for \xpackage{hyperref} and \xpackage{nameref} improved.
%   \item
%     Support for \xpackage{titleref} and \xpackage{babel}'s shorthands added.
%   \item
%     New: \cs{refused}, \cs{getrefbykeydefault}
%   \end{Version}
%   \begin{Version}{2008/08/11 v3.1}
%   \item
%     Code is not changed.
%   \item
%     URLs updated.
%   \end{Version}
% \end{History}
%
% \PrintIndex
%
% \Finale
\endinput

%        (quote the arguments according to the demands of your shell)
%
% Documentation:
%    (a) If refcount.drv is present:
%           latex refcount.drv
%    (b) Without refcount.drv:
%           latex refcount.dtx; ...
%    The class ltxdoc loads the configuration file ltxdoc.cfg
%    if available. Here you can specify further options, e.g.
%    use A4 as paper format:
%       \PassOptionsToClass{a4paper}{article}
%
%    Programm calls to get the documentation (example):
%       pdflatex refcount.dtx
%       makeindex -s gind.ist refcount.idx
%       pdflatex refcount.dtx
%       makeindex -s gind.ist refcount.idx
%       pdflatex refcount.dtx
%
% Installation:
%    TDS:tex/latex/oberdiek/refcount.sty
%    TDS:doc/latex/oberdiek/refcount.pdf
%    TDS:source/latex/oberdiek/refcount.dtx
%
%<*ignore>
\begingroup
  \def\x{LaTeX2e}%
\expandafter\endgroup
\ifcase 0\ifx\install y1\fi\expandafter
         \ifx\csname processbatchFile\endcsname\relax\else1\fi
         \ifx\fmtname\x\else 1\fi\relax
\else\csname fi\endcsname
%</ignore>
%<*install>
\input docstrip.tex
\Msg{************************************************************************}
\Msg{* Installation}
\Msg{* Package: refcount 2008/08/11 v3.1 Data extraction from references (HO)}
\Msg{************************************************************************}

\keepsilent
\askforoverwritefalse

\let\MetaPrefix\relax
\preamble

This is a generated file.

Copyright (C) 1998, 2000, 2006, 2008 by
   Heiko Oberdiek <oberdiek@uni-freiburg.de>

This work may be distributed and/or modified under the
conditions of the LaTeX Project Public License, either
version 1.3 of this license or (at your option) any later
version. The latest version of this license is in
   http://www.latex-project.org/lppl.txt
and version 1.3 or later is part of all distributions of
LaTeX version 2005/12/01 or later.

This work has the LPPL maintenance status "maintained".

This Current Maintainer of this work is Heiko Oberdiek.

This work consists of the main source file refcount.dtx
and the derived files
   refcount.sty, refcount.pdf, refcount.ins, refcount.drv.

\endpreamble
\let\MetaPrefix\DoubleperCent

\generate{%
  \file{refcount.ins}{\from{refcount.dtx}{install}}%
  \file{refcount.drv}{\from{refcount.dtx}{driver}}%
  \usedir{tex/latex/oberdiek}%
  \file{refcount.sty}{\from{refcount.dtx}{package}}%
}

\obeyspaces
\Msg{************************************************************************}
\Msg{*}
\Msg{* To finish the installation you have to move the following}
\Msg{* file into a directory searched by TeX:}
\Msg{*}
\Msg{*     refcount.sty}
\Msg{*}
\Msg{* And install the following script file:}
\Msg{*}
\Msg{*     }
\Msg{*}
\Msg{* To produce the documentation run the file `refcount.drv'}
\Msg{* through LaTeX.}
\Msg{*}
\Msg{* Happy TeXing!}
\Msg{*}
\Msg{************************************************************************}

\endbatchfile
%</install>
%<*ignore>
\fi
%</ignore>
%<*driver>
\NeedsTeXFormat{LaTeX2e}
\ProvidesFile{refcount.drv}%
  [2008/08/11 v3.1 Data extraction from references (HO)]%
\documentclass{ltxdoc}
\usepackage{holtxdoc}[2008/08/11]
\begin{document}
  \DocInput{refcount.dtx}%
\end{document}
%</driver>
% \fi
%
% \CheckSum{198}
%
% \CharacterTable
%  {Upper-case    \A\B\C\D\E\F\G\H\I\J\K\L\M\N\O\P\Q\R\S\T\U\V\W\X\Y\Z
%   Lower-case    \a\b\c\d\e\f\g\h\i\j\k\l\m\n\o\p\q\r\s\t\u\v\w\x\y\z
%   Digits        \0\1\2\3\4\5\6\7\8\9
%   Exclamation   \!     Double quote  \"     Hash (number) \#
%   Dollar        \$     Percent       \%     Ampersand     \&
%   Acute accent  \'     Left paren    \(     Right paren   \)
%   Asterisk      \*     Plus          \+     Comma         \,
%   Minus         \-     Point         \.     Solidus       \/
%   Colon         \:     Semicolon     \;     Less than     \<
%   Equals        \=     Greater than  \>     Question mark \?
%   Commercial at \@     Left bracket  \[     Backslash     \\
%   Right bracket \]     Circumflex    \^     Underscore    \_
%   Grave accent  \`     Left brace    \{     Vertical bar  \|
%   Right brace   \}     Tilde         \~}
%
% \GetFileInfo{refcount.drv}
%
% \title{The \xpackage{refcount} package}
% \date{2008/08/11 v3.1}
% \author{Heiko Oberdiek\\\xemail{oberdiek@uni-freiburg.de}}
%
% \maketitle
%
% \begin{abstract}
% References are not numbers, however they often store numerical
% data such as section or page numbers. \cs{ref} or \cs{pageref}
% cannot be used for counter assignments or calculations because
% they are not expandable, generate warnings, or can even be links,
% The package provides expandable macros to extract the data
% from references. Packages \xpackage{hyperref}, \xpackage{nameref},
% \xpackage{titleref}, and \xpackage{babel} are supported.
% \end{abstract}
%
% \tableofcontents
%
% \section{Usage}
%
% \subsection{Setting counters}
%
% The following commands are similar to \LaTeX's
% \cs{setcounter} and \cs{addtocounter},
% but they extract the number value from a reference:
% \begin{quote}
%   \cs{setcounterref}, \cs{addtocounterref}\\
%   \cs{setcounterpageref}, \cs{addtocounterpageref}
% \end{quote}
% They take two arguments:
% \begin{quote}
%    \cs{...counter...ref} |{|\meta{\LaTeX\ counter}|}|
%    |{|\meta{reference}|}|
% \end{quote}
% An undefined references produces the usual LaTeX warning
% and its value is assumed to be zero.
% Example:
% \begin{quote}
%\begin{verbatim}
%\newcounter{ctrA}
%\newcounter{ctrB}
%\refstepcounter{ctrA}\label{ref:A}
%\setcounterref{ctrB}{ref:A}
%\addtocounterpageref{ctrB}{ref:A}
%\end{verbatim}
% \end{quote}
%
% \subsection{Expandable commands}
%
% These commands that can be used in expandible contexts
% (inside calculations, \cs{edef}, \cs{csname}, \cs{write}, \dots):
% \begin{quote}
%   \cs{getrefnumber}, \cs{getpagerefnumber}
% \end{quote}
% They take one argument, the reference:
% \begin{quote}
%   \cs{get...refnumber} |{|\meta{reference}|}|
% \end{quote}
% The default for undefined references can be changed
% with macro \cs{setrefcountdefault}, for example this
% package calls:
% \begin{quote}
%   \cs{setrefcountdefault}|{0}|
% \end{quote}
%
% Since version 2.0 of this package there is a new
% command:
% \begin{quote}
%   \cs{getrefbykeydefault} |{|\meta{reference}|}|
%   |{|\meta{key}|}| |{|\meta{default}|}|
% \end{quote}
% This generalized version allows the extraction
% of further properties of a reference than the
% two standard ones. Thus the following properties
% are supported, if they are available:
% \begin{quote}
% \begin{tabular}{@{}l|l|l@{}}
%    Key & Description & Package\\
% \hline
%   \meta{empty} & same as \cs{ref} & \LaTeX\\
%   |page| & same as \cs{pageref} & \LaTeX\\
%   |title| & section and caption titles & \xpackage{titleref}\\
%   |name| & section and caption titles & \xpackage{nameref}\\
%   |anchor| & anchor name & \xpackage{hyperref}\\
%   |url| & url/file & \xpackage{hyperref}/\xpackage{xr}
% \end{tabular}
% \end{quote}
%
% \subsection{Undefined references}
%
% Because warnings and assignments cannot be used in
% expandible contexts, undefined references do not
% produce a warning, their values are assumed to be zero.
% Example:
% \begin{quote}
%\begin{verbatim}
%\label{ref:here}% somewhere
%\refused{ref:here}% see below
%\ifodd\getpagerefnumber{ref:here}%
%  reference is on an odd page
%\else
%  reference is on an even page
%\fi
%\end{verbatim}
% \end{quote}
%
% In case of undefined references the user usually want's
% to be informed. Also \LaTeX\ prints a warning at
% the end of the \LaTeX\ run. To notify \LaTeX\ and
% get a normal warning, just use
% \begin{quote}
%   \cs{refused} |{|\meta{reference}|}|
% \end{quote}
% outside the expanding context. Example, see above.
%
% \subsection{Notes}
%
% \begin{itemize}
% \item
%   The method of extracting the number in this
%   package also works in cases, where the
%   reference cannot be used directly, because
%   a package such as \xpackage{hyperref} has added
%   extra stuff (hyper link), so that the reference cannot
%   be used as number any more.
% \item
%   If the reference does not contain a number,
%   assignments to a counter will fail of course.
% \end{itemize}
%
%
% \StopEventually{
% }
%
% \section{Implementation}
%
%    \begin{macrocode}
%<*package>
\NeedsTeXFormat{LaTeX2e}
\ProvidesPackage{refcount}
  [2008/08/11 v3.1 Data extraction from references (HO)]%

\def\setrefcountdefault#1{%
  \def\rc@default{#1}%
}
\setrefcountdefault{0}

% \def\@car#1#2\@nil{#1} % defined in LaTeX kernel
\def\rc@cartwo#1#2#3\@nil{#2}

% generic check without babel support
\long\def\rc@refused#1{%
  \expandafter\ifx\csname r@#1\endcsname\relax
    \protect\G@refundefinedtrue
    \@latex@warning{%
      Reference `#1' on page \thepage\space undefined%
    }%
  \fi
}

% user command, add babel support
\newcommand*{\refused}[1]{%
  \begingroup
    \csname @safe@activestrue\endcsname
    \rc@refused{#1}{}%
  \endgroup
}

% Generic command for "\{set,addto}counter{page,}ref":
% #1: \setcounter, \addtocounter
% #2: \@car (for \ref), \@cartwo (for \pageref)
% #3: LaTeX counter
% #4: reference
\def\rc@set#1#2#3#4{%
  \begingroup
    \csname @safe@activestrue\endcsname
    \rc@refused{#4}%
    \expandafter\rc@@set\csname r@#4\endcsname{#1}{#2}{#3}%
  \endgroup
}
% #1: \r@<...>
% #2: \setcounter, \addtocounter
% #3: \@car (for \ref), \@cartwo (for \pageref)
% #4: LaTeX counter
\def\rc@@set#1#2#3#4{%
  \ifx#1\relax
    #2{#4}{\rc@default}%
  \else
    #2{#4}{%
      \expandafter#3#1\rc@default\rc@default\@nil
    }%
  \fi
}

% user commands:

\newcommand*{\setcounterref}{\rc@set\setcounter\@car}
\newcommand*{\addtocounterref}{\rc@set\addtocounter\@car}
\newcommand*{\setcounterpageref}{\rc@set\setcounter\rc@cartwo}
\newcommand*{\addtocounterpageref}{\rc@set\addtocounter\rc@cartwo}

\newcommand*{\getrefnumber}[1]{%
  \expandafter\ifx\csname r@#1\endcsname\relax
    \rc@default
  \else
    \expandafter\expandafter\expandafter\@car
    \csname r@#1\endcsname\@nil
  \fi
}
\newcommand*{\getpagerefnumber}[1]{%
  \expandafter\ifx\csname r@#1\endcsname\relax
    \rc@default
  \else
    \expandafter\expandafter\expandafter\rc@cartwo
    \csname r@#1\endcsname\rc@default\rc@default\@nil
  \fi
}
\newcommand*{\getrefbykeydefault}[2]{%
  \expandafter\rc@getrefbykeydefault
    \csname r@#1\expandafter\endcsname
    \csname rc@extract@#2\endcsname
}
% #1: \r@<...>
% #2: \rc@extract@<...>
% #3: default
\def\rc@getrefbykeydefault#1#2#3{%
  \ifx#1\relax
    % reference is undefined
    #3%
  \else
    \ifx#2\relax
      % extract method is missing
      #3%
    \else
      \expandafter\rc@generic#1{#3}{#3}{#3}{#3}{#3}\@nil#2{#3}%
    \fi
  \fi
}
% #1: first item in \r@<...>
% #2: remaining items in \r@<...>
% #3: \rc@extract@<...>
% #4: default
\def\rc@generic#1#2\@nil#3#4{%
  #3{#1\TR@TitleReference\@empty{#4}\@nil}{#1}#2\@nil
}
\def\rc@extract@{%
  \expandafter\@car\@gobble
}
\def\rc@extract@page{%
  \expandafter\@car\@gobbletwo
}
\def\rc@extract@name{%
  \expandafter\@car\@gobblefour\@empty
}
\def\rc@extract@anchor{%
  \expandafter\@car\@gobblefour
}
\def\rc@extract@url{%
  \expandafter\expandafter\expandafter\@car\expandafter
      \@gobble\@gobblefour
}
\def\rc@extract@title#1#2\@nil{%
  \rc@@extract@title#1%
}
\def\rc@@extract@title#1\TR@TitleReference#2#3#4\@nil{#3}
%</package>
%    \end{macrocode}
%
% \section{Installation}
%
% \subsection{Download}
%
% \paragraph{Package.} This package is available on
% CTAN\footnote{\url{ftp://ftp.ctan.org/tex-archive/}}:
% \begin{description}
% \item[\CTAN{macros/latex/contrib/oberdiek/refcount.dtx}] The source file.
% \item[\CTAN{macros/latex/contrib/oberdiek/refcount.pdf}] Documentation.
% \end{description}
%
%
% \paragraph{Bundle.} All the packages of the bundle `oberdiek'
% are also available in a TDS compliant ZIP archive. There
% the packages are already unpacked and the documentation files
% are generated. The files and directories obey the TDS standard.
% \begin{description}
% \item[\CTAN{install/macros/latex/contrib/oberdiek.tds.zip}]
% \end{description}
% \emph{TDS} refers to the standard ``A Directory Structure
% for \TeX\ Files'' (\CTAN{tds/tds.pdf}). Directories
% with \xfile{texmf} in their name are usually organized this way.
%
% \subsection{Bundle installation}
%
% \paragraph{Unpacking.} Unpack the \xfile{oberdiek.tds.zip} in the
% TDS tree (also known as \xfile{texmf} tree) of your choice.
% Example (linux):
% \begin{quote}
%   |unzip oberdiek.tds.zip -d ~/texmf|
% \end{quote}
%
% \paragraph{Script installation.}
% Check the directory \xfile{TDS:scripts/oberdiek/} for
% scripts that need further installation steps.
% Package \xpackage{attachfile2} comes with the Perl script
% \xfile{pdfatfi.pl} that should be installed in such a way
% that it can be called as \texttt{pdfatfi}.
% Example (linux):
% \begin{quote}
%   |chmod +x scripts/oberdiek/pdfatfi.pl|\\
%   |cp scripts/oberdiek/pdfatfi.pl /usr/local/bin/|
% \end{quote}
%
% \subsection{Package installation}
%
% \paragraph{Unpacking.} The \xfile{.dtx} file is a self-extracting
% \docstrip\ archive. The files are extracted by running the
% \xfile{.dtx} through \plainTeX:
% \begin{quote}
%   \verb|tex refcount.dtx|
% \end{quote}
%
% \paragraph{TDS.} Now the different files must be moved into
% the different directories in your installation TDS tree
% (also known as \xfile{texmf} tree):
% \begin{quote}
% \def\t{^^A
% \begin{tabular}{@{}>{\ttfamily}l@{ $\rightarrow$ }>{\ttfamily}l@{}}
%   refcount.sty & tex/latex/oberdiek/refcount.sty\\
%   refcount.pdf & doc/latex/oberdiek/refcount.pdf\\
%   refcount.dtx & source/latex/oberdiek/refcount.dtx\\
% \end{tabular}^^A
% }^^A
% \sbox0{\t}^^A
% \ifdim\wd0>\linewidth
%   \begingroup
%     \advance\linewidth by\leftmargin
%     \advance\linewidth by\rightmargin
%   \edef\x{\endgroup
%     \def\noexpand\lw{\the\linewidth}^^A
%   }\x
%   \def\lwbox{^^A
%     \leavevmode
%     \hbox to \linewidth{^^A
%       \kern-\leftmargin\relax
%       \hss
%       \usebox0
%       \hss
%       \kern-\rightmargin\relax
%     }^^A
%   }^^A
%   \ifdim\wd0>\lw
%     \sbox0{\small\t}^^A
%     \ifdim\wd0>\linewidth
%       \ifdim\wd0>\lw
%         \sbox0{\footnotesize\t}^^A
%         \ifdim\wd0>\linewidth
%           \ifdim\wd0>\lw
%             \sbox0{\scriptsize\t}^^A
%             \ifdim\wd0>\linewidth
%               \ifdim\wd0>\lw
%                 \sbox0{\tiny\t}^^A
%                 \ifdim\wd0>\linewidth
%                   \lwbox
%                 \else
%                   \usebox0
%                 \fi
%               \else
%                 \lwbox
%               \fi
%             \else
%               \usebox0
%             \fi
%           \else
%             \lwbox
%           \fi
%         \else
%           \usebox0
%         \fi
%       \else
%         \lwbox
%       \fi
%     \else
%       \usebox0
%     \fi
%   \else
%     \lwbox
%   \fi
% \else
%   \usebox0
% \fi
% \end{quote}
% If you have a \xfile{docstrip.cfg} that configures and enables \docstrip's
% TDS installing feature, then some files can already be in the right
% place, see the documentation of \docstrip.
%
% \subsection{Refresh file name databases}
%
% If your \TeX~distribution
% (\teTeX, \mikTeX, \dots) relies on file name databases, you must refresh
% these. For example, \teTeX\ users run \verb|texhash| or
% \verb|mktexlsr|.
%
% \subsection{Some details for the interested}
%
% \paragraph{Attached source.}
%
% The PDF documentation on CTAN also includes the
% \xfile{.dtx} source file. It can be extracted by
% AcrobatReader 6 or higher. Another option is \textsf{pdftk},
% e.g. unpack the file into the current directory:
% \begin{quote}
%   \verb|pdftk refcount.pdf unpack_files output .|
% \end{quote}
%
% \paragraph{Unpacking with \LaTeX.}
% The \xfile{.dtx} chooses its action depending on the format:
% \begin{description}
% \item[\plainTeX:] Run \docstrip\ and extract the files.
% \item[\LaTeX:] Generate the documentation.
% \end{description}
% If you insist on using \LaTeX\ for \docstrip\ (really,
% \docstrip\ does not need \LaTeX), then inform the autodetect routine
% about your intention:
% \begin{quote}
%   \verb|latex \let\install=y% \iffalse meta-comment
%
% Copyright (C) 1998, 2000, 2006, 2008 by
%    Heiko Oberdiek <oberdiek@uni-freiburg.de>
%
% This work may be distributed and/or modified under the
% conditions of the LaTeX Project Public License, either
% version 1.3 of this license or (at your option) any later
% version. The latest version of this license is in
%    http://www.latex-project.org/lppl.txt
% and version 1.3 or later is part of all distributions of
% LaTeX version 2005/12/01 or later.
%
% This work has the LPPL maintenance status "maintained".
%
% This Current Maintainer of this work is Heiko Oberdiek.
%
% This work consists of the main source file refcount.dtx
% and the derived files
%    refcount.sty, refcount.pdf, refcount.ins, refcount.drv.
%
% Distribution:
%    CTAN:macros/latex/contrib/oberdiek/refcount.dtx
%    CTAN:macros/latex/contrib/oberdiek/refcount.pdf
%
% Unpacking:
%    (a) If refcount.ins is present:
%           tex refcount.ins
%    (b) Without refcount.ins:
%           tex refcount.dtx
%    (c) If you insist on using LaTeX
%           latex \let\install=y\input{refcount.dtx}
%        (quote the arguments according to the demands of your shell)
%
% Documentation:
%    (a) If refcount.drv is present:
%           latex refcount.drv
%    (b) Without refcount.drv:
%           latex refcount.dtx; ...
%    The class ltxdoc loads the configuration file ltxdoc.cfg
%    if available. Here you can specify further options, e.g.
%    use A4 as paper format:
%       \PassOptionsToClass{a4paper}{article}
%
%    Programm calls to get the documentation (example):
%       pdflatex refcount.dtx
%       makeindex -s gind.ist refcount.idx
%       pdflatex refcount.dtx
%       makeindex -s gind.ist refcount.idx
%       pdflatex refcount.dtx
%
% Installation:
%    TDS:tex/latex/oberdiek/refcount.sty
%    TDS:doc/latex/oberdiek/refcount.pdf
%    TDS:source/latex/oberdiek/refcount.dtx
%
%<*ignore>
\begingroup
  \def\x{LaTeX2e}%
\expandafter\endgroup
\ifcase 0\ifx\install y1\fi\expandafter
         \ifx\csname processbatchFile\endcsname\relax\else1\fi
         \ifx\fmtname\x\else 1\fi\relax
\else\csname fi\endcsname
%</ignore>
%<*install>
\input docstrip.tex
\Msg{************************************************************************}
\Msg{* Installation}
\Msg{* Package: refcount 2008/08/11 v3.1 Data extraction from references (HO)}
\Msg{************************************************************************}

\keepsilent
\askforoverwritefalse

\let\MetaPrefix\relax
\preamble

This is a generated file.

Copyright (C) 1998, 2000, 2006, 2008 by
   Heiko Oberdiek <oberdiek@uni-freiburg.de>

This work may be distributed and/or modified under the
conditions of the LaTeX Project Public License, either
version 1.3 of this license or (at your option) any later
version. The latest version of this license is in
   http://www.latex-project.org/lppl.txt
and version 1.3 or later is part of all distributions of
LaTeX version 2005/12/01 or later.

This work has the LPPL maintenance status "maintained".

This Current Maintainer of this work is Heiko Oberdiek.

This work consists of the main source file refcount.dtx
and the derived files
   refcount.sty, refcount.pdf, refcount.ins, refcount.drv.

\endpreamble
\let\MetaPrefix\DoubleperCent

\generate{%
  \file{refcount.ins}{\from{refcount.dtx}{install}}%
  \file{refcount.drv}{\from{refcount.dtx}{driver}}%
  \usedir{tex/latex/oberdiek}%
  \file{refcount.sty}{\from{refcount.dtx}{package}}%
}

\obeyspaces
\Msg{************************************************************************}
\Msg{*}
\Msg{* To finish the installation you have to move the following}
\Msg{* file into a directory searched by TeX:}
\Msg{*}
\Msg{*     refcount.sty}
\Msg{*}
\Msg{* And install the following script file:}
\Msg{*}
\Msg{*     }
\Msg{*}
\Msg{* To produce the documentation run the file `refcount.drv'}
\Msg{* through LaTeX.}
\Msg{*}
\Msg{* Happy TeXing!}
\Msg{*}
\Msg{************************************************************************}

\endbatchfile
%</install>
%<*ignore>
\fi
%</ignore>
%<*driver>
\NeedsTeXFormat{LaTeX2e}
\ProvidesFile{refcount.drv}%
  [2008/08/11 v3.1 Data extraction from references (HO)]%
\documentclass{ltxdoc}
\usepackage{holtxdoc}[2008/08/11]
\begin{document}
  \DocInput{refcount.dtx}%
\end{document}
%</driver>
% \fi
%
% \CheckSum{198}
%
% \CharacterTable
%  {Upper-case    \A\B\C\D\E\F\G\H\I\J\K\L\M\N\O\P\Q\R\S\T\U\V\W\X\Y\Z
%   Lower-case    \a\b\c\d\e\f\g\h\i\j\k\l\m\n\o\p\q\r\s\t\u\v\w\x\y\z
%   Digits        \0\1\2\3\4\5\6\7\8\9
%   Exclamation   \!     Double quote  \"     Hash (number) \#
%   Dollar        \$     Percent       \%     Ampersand     \&
%   Acute accent  \'     Left paren    \(     Right paren   \)
%   Asterisk      \*     Plus          \+     Comma         \,
%   Minus         \-     Point         \.     Solidus       \/
%   Colon         \:     Semicolon     \;     Less than     \<
%   Equals        \=     Greater than  \>     Question mark \?
%   Commercial at \@     Left bracket  \[     Backslash     \\
%   Right bracket \]     Circumflex    \^     Underscore    \_
%   Grave accent  \`     Left brace    \{     Vertical bar  \|
%   Right brace   \}     Tilde         \~}
%
% \GetFileInfo{refcount.drv}
%
% \title{The \xpackage{refcount} package}
% \date{2008/08/11 v3.1}
% \author{Heiko Oberdiek\\\xemail{oberdiek@uni-freiburg.de}}
%
% \maketitle
%
% \begin{abstract}
% References are not numbers, however they often store numerical
% data such as section or page numbers. \cs{ref} or \cs{pageref}
% cannot be used for counter assignments or calculations because
% they are not expandable, generate warnings, or can even be links,
% The package provides expandable macros to extract the data
% from references. Packages \xpackage{hyperref}, \xpackage{nameref},
% \xpackage{titleref}, and \xpackage{babel} are supported.
% \end{abstract}
%
% \tableofcontents
%
% \section{Usage}
%
% \subsection{Setting counters}
%
% The following commands are similar to \LaTeX's
% \cs{setcounter} and \cs{addtocounter},
% but they extract the number value from a reference:
% \begin{quote}
%   \cs{setcounterref}, \cs{addtocounterref}\\
%   \cs{setcounterpageref}, \cs{addtocounterpageref}
% \end{quote}
% They take two arguments:
% \begin{quote}
%    \cs{...counter...ref} |{|\meta{\LaTeX\ counter}|}|
%    |{|\meta{reference}|}|
% \end{quote}
% An undefined references produces the usual LaTeX warning
% and its value is assumed to be zero.
% Example:
% \begin{quote}
%\begin{verbatim}
%\newcounter{ctrA}
%\newcounter{ctrB}
%\refstepcounter{ctrA}\label{ref:A}
%\setcounterref{ctrB}{ref:A}
%\addtocounterpageref{ctrB}{ref:A}
%\end{verbatim}
% \end{quote}
%
% \subsection{Expandable commands}
%
% These commands that can be used in expandible contexts
% (inside calculations, \cs{edef}, \cs{csname}, \cs{write}, \dots):
% \begin{quote}
%   \cs{getrefnumber}, \cs{getpagerefnumber}
% \end{quote}
% They take one argument, the reference:
% \begin{quote}
%   \cs{get...refnumber} |{|\meta{reference}|}|
% \end{quote}
% The default for undefined references can be changed
% with macro \cs{setrefcountdefault}, for example this
% package calls:
% \begin{quote}
%   \cs{setrefcountdefault}|{0}|
% \end{quote}
%
% Since version 2.0 of this package there is a new
% command:
% \begin{quote}
%   \cs{getrefbykeydefault} |{|\meta{reference}|}|
%   |{|\meta{key}|}| |{|\meta{default}|}|
% \end{quote}
% This generalized version allows the extraction
% of further properties of a reference than the
% two standard ones. Thus the following properties
% are supported, if they are available:
% \begin{quote}
% \begin{tabular}{@{}l|l|l@{}}
%    Key & Description & Package\\
% \hline
%   \meta{empty} & same as \cs{ref} & \LaTeX\\
%   |page| & same as \cs{pageref} & \LaTeX\\
%   |title| & section and caption titles & \xpackage{titleref}\\
%   |name| & section and caption titles & \xpackage{nameref}\\
%   |anchor| & anchor name & \xpackage{hyperref}\\
%   |url| & url/file & \xpackage{hyperref}/\xpackage{xr}
% \end{tabular}
% \end{quote}
%
% \subsection{Undefined references}
%
% Because warnings and assignments cannot be used in
% expandible contexts, undefined references do not
% produce a warning, their values are assumed to be zero.
% Example:
% \begin{quote}
%\begin{verbatim}
%\label{ref:here}% somewhere
%\refused{ref:here}% see below
%\ifodd\getpagerefnumber{ref:here}%
%  reference is on an odd page
%\else
%  reference is on an even page
%\fi
%\end{verbatim}
% \end{quote}
%
% In case of undefined references the user usually want's
% to be informed. Also \LaTeX\ prints a warning at
% the end of the \LaTeX\ run. To notify \LaTeX\ and
% get a normal warning, just use
% \begin{quote}
%   \cs{refused} |{|\meta{reference}|}|
% \end{quote}
% outside the expanding context. Example, see above.
%
% \subsection{Notes}
%
% \begin{itemize}
% \item
%   The method of extracting the number in this
%   package also works in cases, where the
%   reference cannot be used directly, because
%   a package such as \xpackage{hyperref} has added
%   extra stuff (hyper link), so that the reference cannot
%   be used as number any more.
% \item
%   If the reference does not contain a number,
%   assignments to a counter will fail of course.
% \end{itemize}
%
%
% \StopEventually{
% }
%
% \section{Implementation}
%
%    \begin{macrocode}
%<*package>
\NeedsTeXFormat{LaTeX2e}
\ProvidesPackage{refcount}
  [2008/08/11 v3.1 Data extraction from references (HO)]%

\def\setrefcountdefault#1{%
  \def\rc@default{#1}%
}
\setrefcountdefault{0}

% \def\@car#1#2\@nil{#1} % defined in LaTeX kernel
\def\rc@cartwo#1#2#3\@nil{#2}

% generic check without babel support
\long\def\rc@refused#1{%
  \expandafter\ifx\csname r@#1\endcsname\relax
    \protect\G@refundefinedtrue
    \@latex@warning{%
      Reference `#1' on page \thepage\space undefined%
    }%
  \fi
}

% user command, add babel support
\newcommand*{\refused}[1]{%
  \begingroup
    \csname @safe@activestrue\endcsname
    \rc@refused{#1}{}%
  \endgroup
}

% Generic command for "\{set,addto}counter{page,}ref":
% #1: \setcounter, \addtocounter
% #2: \@car (for \ref), \@cartwo (for \pageref)
% #3: LaTeX counter
% #4: reference
\def\rc@set#1#2#3#4{%
  \begingroup
    \csname @safe@activestrue\endcsname
    \rc@refused{#4}%
    \expandafter\rc@@set\csname r@#4\endcsname{#1}{#2}{#3}%
  \endgroup
}
% #1: \r@<...>
% #2: \setcounter, \addtocounter
% #3: \@car (for \ref), \@cartwo (for \pageref)
% #4: LaTeX counter
\def\rc@@set#1#2#3#4{%
  \ifx#1\relax
    #2{#4}{\rc@default}%
  \else
    #2{#4}{%
      \expandafter#3#1\rc@default\rc@default\@nil
    }%
  \fi
}

% user commands:

\newcommand*{\setcounterref}{\rc@set\setcounter\@car}
\newcommand*{\addtocounterref}{\rc@set\addtocounter\@car}
\newcommand*{\setcounterpageref}{\rc@set\setcounter\rc@cartwo}
\newcommand*{\addtocounterpageref}{\rc@set\addtocounter\rc@cartwo}

\newcommand*{\getrefnumber}[1]{%
  \expandafter\ifx\csname r@#1\endcsname\relax
    \rc@default
  \else
    \expandafter\expandafter\expandafter\@car
    \csname r@#1\endcsname\@nil
  \fi
}
\newcommand*{\getpagerefnumber}[1]{%
  \expandafter\ifx\csname r@#1\endcsname\relax
    \rc@default
  \else
    \expandafter\expandafter\expandafter\rc@cartwo
    \csname r@#1\endcsname\rc@default\rc@default\@nil
  \fi
}
\newcommand*{\getrefbykeydefault}[2]{%
  \expandafter\rc@getrefbykeydefault
    \csname r@#1\expandafter\endcsname
    \csname rc@extract@#2\endcsname
}
% #1: \r@<...>
% #2: \rc@extract@<...>
% #3: default
\def\rc@getrefbykeydefault#1#2#3{%
  \ifx#1\relax
    % reference is undefined
    #3%
  \else
    \ifx#2\relax
      % extract method is missing
      #3%
    \else
      \expandafter\rc@generic#1{#3}{#3}{#3}{#3}{#3}\@nil#2{#3}%
    \fi
  \fi
}
% #1: first item in \r@<...>
% #2: remaining items in \r@<...>
% #3: \rc@extract@<...>
% #4: default
\def\rc@generic#1#2\@nil#3#4{%
  #3{#1\TR@TitleReference\@empty{#4}\@nil}{#1}#2\@nil
}
\def\rc@extract@{%
  \expandafter\@car\@gobble
}
\def\rc@extract@page{%
  \expandafter\@car\@gobbletwo
}
\def\rc@extract@name{%
  \expandafter\@car\@gobblefour\@empty
}
\def\rc@extract@anchor{%
  \expandafter\@car\@gobblefour
}
\def\rc@extract@url{%
  \expandafter\expandafter\expandafter\@car\expandafter
      \@gobble\@gobblefour
}
\def\rc@extract@title#1#2\@nil{%
  \rc@@extract@title#1%
}
\def\rc@@extract@title#1\TR@TitleReference#2#3#4\@nil{#3}
%</package>
%    \end{macrocode}
%
% \section{Installation}
%
% \subsection{Download}
%
% \paragraph{Package.} This package is available on
% CTAN\footnote{\url{ftp://ftp.ctan.org/tex-archive/}}:
% \begin{description}
% \item[\CTAN{macros/latex/contrib/oberdiek/refcount.dtx}] The source file.
% \item[\CTAN{macros/latex/contrib/oberdiek/refcount.pdf}] Documentation.
% \end{description}
%
%
% \paragraph{Bundle.} All the packages of the bundle `oberdiek'
% are also available in a TDS compliant ZIP archive. There
% the packages are already unpacked and the documentation files
% are generated. The files and directories obey the TDS standard.
% \begin{description}
% \item[\CTAN{install/macros/latex/contrib/oberdiek.tds.zip}]
% \end{description}
% \emph{TDS} refers to the standard ``A Directory Structure
% for \TeX\ Files'' (\CTAN{tds/tds.pdf}). Directories
% with \xfile{texmf} in their name are usually organized this way.
%
% \subsection{Bundle installation}
%
% \paragraph{Unpacking.} Unpack the \xfile{oberdiek.tds.zip} in the
% TDS tree (also known as \xfile{texmf} tree) of your choice.
% Example (linux):
% \begin{quote}
%   |unzip oberdiek.tds.zip -d ~/texmf|
% \end{quote}
%
% \paragraph{Script installation.}
% Check the directory \xfile{TDS:scripts/oberdiek/} for
% scripts that need further installation steps.
% Package \xpackage{attachfile2} comes with the Perl script
% \xfile{pdfatfi.pl} that should be installed in such a way
% that it can be called as \texttt{pdfatfi}.
% Example (linux):
% \begin{quote}
%   |chmod +x scripts/oberdiek/pdfatfi.pl|\\
%   |cp scripts/oberdiek/pdfatfi.pl /usr/local/bin/|
% \end{quote}
%
% \subsection{Package installation}
%
% \paragraph{Unpacking.} The \xfile{.dtx} file is a self-extracting
% \docstrip\ archive. The files are extracted by running the
% \xfile{.dtx} through \plainTeX:
% \begin{quote}
%   \verb|tex refcount.dtx|
% \end{quote}
%
% \paragraph{TDS.} Now the different files must be moved into
% the different directories in your installation TDS tree
% (also known as \xfile{texmf} tree):
% \begin{quote}
% \def\t{^^A
% \begin{tabular}{@{}>{\ttfamily}l@{ $\rightarrow$ }>{\ttfamily}l@{}}
%   refcount.sty & tex/latex/oberdiek/refcount.sty\\
%   refcount.pdf & doc/latex/oberdiek/refcount.pdf\\
%   refcount.dtx & source/latex/oberdiek/refcount.dtx\\
% \end{tabular}^^A
% }^^A
% \sbox0{\t}^^A
% \ifdim\wd0>\linewidth
%   \begingroup
%     \advance\linewidth by\leftmargin
%     \advance\linewidth by\rightmargin
%   \edef\x{\endgroup
%     \def\noexpand\lw{\the\linewidth}^^A
%   }\x
%   \def\lwbox{^^A
%     \leavevmode
%     \hbox to \linewidth{^^A
%       \kern-\leftmargin\relax
%       \hss
%       \usebox0
%       \hss
%       \kern-\rightmargin\relax
%     }^^A
%   }^^A
%   \ifdim\wd0>\lw
%     \sbox0{\small\t}^^A
%     \ifdim\wd0>\linewidth
%       \ifdim\wd0>\lw
%         \sbox0{\footnotesize\t}^^A
%         \ifdim\wd0>\linewidth
%           \ifdim\wd0>\lw
%             \sbox0{\scriptsize\t}^^A
%             \ifdim\wd0>\linewidth
%               \ifdim\wd0>\lw
%                 \sbox0{\tiny\t}^^A
%                 \ifdim\wd0>\linewidth
%                   \lwbox
%                 \else
%                   \usebox0
%                 \fi
%               \else
%                 \lwbox
%               \fi
%             \else
%               \usebox0
%             \fi
%           \else
%             \lwbox
%           \fi
%         \else
%           \usebox0
%         \fi
%       \else
%         \lwbox
%       \fi
%     \else
%       \usebox0
%     \fi
%   \else
%     \lwbox
%   \fi
% \else
%   \usebox0
% \fi
% \end{quote}
% If you have a \xfile{docstrip.cfg} that configures and enables \docstrip's
% TDS installing feature, then some files can already be in the right
% place, see the documentation of \docstrip.
%
% \subsection{Refresh file name databases}
%
% If your \TeX~distribution
% (\teTeX, \mikTeX, \dots) relies on file name databases, you must refresh
% these. For example, \teTeX\ users run \verb|texhash| or
% \verb|mktexlsr|.
%
% \subsection{Some details for the interested}
%
% \paragraph{Attached source.}
%
% The PDF documentation on CTAN also includes the
% \xfile{.dtx} source file. It can be extracted by
% AcrobatReader 6 or higher. Another option is \textsf{pdftk},
% e.g. unpack the file into the current directory:
% \begin{quote}
%   \verb|pdftk refcount.pdf unpack_files output .|
% \end{quote}
%
% \paragraph{Unpacking with \LaTeX.}
% The \xfile{.dtx} chooses its action depending on the format:
% \begin{description}
% \item[\plainTeX:] Run \docstrip\ and extract the files.
% \item[\LaTeX:] Generate the documentation.
% \end{description}
% If you insist on using \LaTeX\ for \docstrip\ (really,
% \docstrip\ does not need \LaTeX), then inform the autodetect routine
% about your intention:
% \begin{quote}
%   \verb|latex \let\install=y\input{refcount.dtx}|
% \end{quote}
% Do not forget to quote the argument according to the demands
% of your shell.
%
% \paragraph{Generating the documentation.}
% You can use both the \xfile{.dtx} or the \xfile{.drv} to generate
% the documentation. The process can be configured by the
% configuration file \xfile{ltxdoc.cfg}. For instance, put this
% line into this file, if you want to have A4 as paper format:
% \begin{quote}
%   \verb|\PassOptionsToClass{a4paper}{article}|
% \end{quote}
% An example follows how to generate the
% documentation with pdf\LaTeX:
% \begin{quote}
%\begin{verbatim}
%pdflatex refcount.dtx
%makeindex -s gind.ist refcount.idx
%pdflatex refcount.dtx
%makeindex -s gind.ist refcount.idx
%pdflatex refcount.dtx
%\end{verbatim}
% \end{quote}
%
% \begin{History}
%   \begin{Version}{1998/04/08 v1.0}
%   \item
%     First public release, written as answer in the
%     newsgroup \xnewsgroup{comp.text.tex}:
%     \URL{``\link{Re: Adding a \cs{ref} to a counter?}''}^^A
%     {http://groups.google.com/group/comp.text.tex/msg/c3f2a135ef5ee528}
%   \end{Version}
%   \begin{Version}{2000/09/07 v2.0}
%   \item
%     Documentation added.
%   \item
%     LPPL 1.2
%   \item
%     Package rewritten, new commands added.
%   \end{Version}
%   \begin{Version}{2006/02/20 v3.0}
%   \item
%     Support for \xpackage{hyperref} and \xpackage{nameref} improved.
%   \item
%     Support for \xpackage{titleref} and \xpackage{babel}'s shorthands added.
%   \item
%     New: \cs{refused}, \cs{getrefbykeydefault}
%   \end{Version}
%   \begin{Version}{2008/08/11 v3.1}
%   \item
%     Code is not changed.
%   \item
%     URLs updated.
%   \end{Version}
% \end{History}
%
% \PrintIndex
%
% \Finale
\endinput
|
% \end{quote}
% Do not forget to quote the argument according to the demands
% of your shell.
%
% \paragraph{Generating the documentation.}
% You can use both the \xfile{.dtx} or the \xfile{.drv} to generate
% the documentation. The process can be configured by the
% configuration file \xfile{ltxdoc.cfg}. For instance, put this
% line into this file, if you want to have A4 as paper format:
% \begin{quote}
%   \verb|\PassOptionsToClass{a4paper}{article}|
% \end{quote}
% An example follows how to generate the
% documentation with pdf\LaTeX:
% \begin{quote}
%\begin{verbatim}
%pdflatex refcount.dtx
%makeindex -s gind.ist refcount.idx
%pdflatex refcount.dtx
%makeindex -s gind.ist refcount.idx
%pdflatex refcount.dtx
%\end{verbatim}
% \end{quote}
%
% \begin{History}
%   \begin{Version}{1998/04/08 v1.0}
%   \item
%     First public release, written as answer in the
%     newsgroup \xnewsgroup{comp.text.tex}:
%     \URL{``\link{Re: Adding a \cs{ref} to a counter?}''}^^A
%     {http://groups.google.com/group/comp.text.tex/msg/c3f2a135ef5ee528}
%   \end{Version}
%   \begin{Version}{2000/09/07 v2.0}
%   \item
%     Documentation added.
%   \item
%     LPPL 1.2
%   \item
%     Package rewritten, new commands added.
%   \end{Version}
%   \begin{Version}{2006/02/20 v3.0}
%   \item
%     Support for \xpackage{hyperref} and \xpackage{nameref} improved.
%   \item
%     Support for \xpackage{titleref} and \xpackage{babel}'s shorthands added.
%   \item
%     New: \cs{refused}, \cs{getrefbykeydefault}
%   \end{Version}
%   \begin{Version}{2008/08/11 v3.1}
%   \item
%     Code is not changed.
%   \item
%     URLs updated.
%   \end{Version}
% \end{History}
%
% \PrintIndex
%
% \Finale
\endinput
|
% \end{quote}
% Do not forget to quote the argument according to the demands
% of your shell.
%
% \paragraph{Generating the documentation.}
% You can use both the \xfile{.dtx} or the \xfile{.drv} to generate
% the documentation. The process can be configured by the
% configuration file \xfile{ltxdoc.cfg}. For instance, put this
% line into this file, if you want to have A4 as paper format:
% \begin{quote}
%   \verb|\PassOptionsToClass{a4paper}{article}|
% \end{quote}
% An example follows how to generate the
% documentation with pdf\LaTeX:
% \begin{quote}
%\begin{verbatim}
%pdflatex refcount.dtx
%makeindex -s gind.ist refcount.idx
%pdflatex refcount.dtx
%makeindex -s gind.ist refcount.idx
%pdflatex refcount.dtx
%\end{verbatim}
% \end{quote}
%
% \begin{History}
%   \begin{Version}{1998/04/08 v1.0}
%   \item
%     First public release, written as answer in the
%     newsgroup \xnewsgroup{comp.text.tex}:
%     \URL{``\link{Re: Adding a \cs{ref} to a counter?}''}^^A
%     {http://groups.google.com/group/comp.text.tex/msg/c3f2a135ef5ee528}
%   \end{Version}
%   \begin{Version}{2000/09/07 v2.0}
%   \item
%     Documentation added.
%   \item
%     LPPL 1.2
%   \item
%     Package rewritten, new commands added.
%   \end{Version}
%   \begin{Version}{2006/02/20 v3.0}
%   \item
%     Support for \xpackage{hyperref} and \xpackage{nameref} improved.
%   \item
%     Support for \xpackage{titleref} and \xpackage{babel}'s shorthands added.
%   \item
%     New: \cs{refused}, \cs{getrefbykeydefault}
%   \end{Version}
%   \begin{Version}{2008/08/11 v3.1}
%   \item
%     Code is not changed.
%   \item
%     URLs updated.
%   \end{Version}
% \end{History}
%
% \PrintIndex
%
% \Finale
\endinput
|
% \end{quote}
% Do not forget to quote the argument according to the demands
% of your shell.
%
% \paragraph{Generating the documentation.}
% You can use both the \xfile{.dtx} or the \xfile{.drv} to generate
% the documentation. The process can be configured by the
% configuration file \xfile{ltxdoc.cfg}. For instance, put this
% line into this file, if you want to have A4 as paper format:
% \begin{quote}
%   \verb|\PassOptionsToClass{a4paper}{article}|
% \end{quote}
% An example follows how to generate the
% documentation with pdf\LaTeX:
% \begin{quote}
%\begin{verbatim}
%pdflatex refcount.dtx
%makeindex -s gind.ist refcount.idx
%pdflatex refcount.dtx
%makeindex -s gind.ist refcount.idx
%pdflatex refcount.dtx
%\end{verbatim}
% \end{quote}
%
% \begin{History}
%   \begin{Version}{1998/04/08 v1.0}
%   \item
%     First public release, written as answer in the
%     newsgroup \xnewsgroup{comp.text.tex}:
%     \URL{``\link{Re: Adding a \cs{ref} to a counter?}''}^^A
%     {http://groups.google.com/group/comp.text.tex/msg/c3f2a135ef5ee528}
%   \end{Version}
%   \begin{Version}{2000/09/07 v2.0}
%   \item
%     Documentation added.
%   \item
%     LPPL 1.2
%   \item
%     Package rewritten, new commands added.
%   \end{Version}
%   \begin{Version}{2006/02/20 v3.0}
%   \item
%     Support for \xpackage{hyperref} and \xpackage{nameref} improved.
%   \item
%     Support for \xpackage{titleref} and \xpackage{babel}'s shorthands added.
%   \item
%     New: \cs{refused}, \cs{getrefbykeydefault}
%   \end{Version}
%   \begin{Version}{2008/08/11 v3.1}
%   \item
%     Code is not changed.
%   \item
%     URLs updated.
%   \end{Version}
% \end{History}
%
% \PrintIndex
%
% \Finale
\endinput
