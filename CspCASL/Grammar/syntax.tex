\documentclass[11pt]{article}

\newcommand{\Mod}[1]{\mathbf{Mod}(#1)}
\newcommand{\Sen}[1]{\mathbf{Sen}(#1)}
\newcommand{\Sig}{\mathbf{Sig}}
\newcommand{\PF}{\mathit{PF}}
\newcommand{\TF}{\mathit{TF}}

% \Gram is modified wrt the language summary so that it works in
% italics as well as roman.
%\newcommand{\Gram}[1]{{\textup{\texttt{#1}}}}

\newenvironment{Grammar}
 {\texonly{\footnotesize\setlength{\exampleindent}{0pt}}
  \begin{example}}{\end{example}\ignorespaces}

\newenvironment{AbstractGrammar}
 {\par\texonly{\smallskip\samepage}\begin{Grammar}}{\end{Grammar}\par}

\newenvironment{ConcreteDisplay}
 {\nopagebreak\begin{quote}\caslInit}{\end{quote}\pagebreak[3]}

\newcommand{\DisplayLatexInput}[3]
 {The sign displayed as \texorhtml{`\Math{\caslInit{#1}}'}{#2 in \LaTeX}
 is input as `\Gram{#3}'.} 

\newcommand{\DisplayISOInput}[3]
 {The sign displayed as `\Math{\caslInit{#1}}' may be input as
 `\Math{#2}' in ISO Latin-1, or as `\Gram{#3}' in ASCII.} 

%%%%%%%%%%%%%%%%%%%%%%%%%%%%%%%%%%%%%%%%%%%%%%%%%%%%%%%%%%%%%%%%
%
% Text within an "explanation" environment, which is printed in
% indented italics, is the semantics.  Semantic definitions should be
% kept as close as possible to the text that presents the relevant
% construct.

\newenvironment{explanation}
        {\begin{quote}\em}
        {\end{quote}\normalsize}

% Text within a "doubt" environment, which is printed in indented
% small roman font, is for (e.g.) comments on inadequacies in the
% semantics.

\newcounter{doubt}[section]
\renewcommand{\thedoubt}{\thesection.\arabic{doubt}}
\newenvironment{doubt}
        {\begin{quote}\upshape\footnotesize\refstepcounter{doubt}%
         Doubt \thedoubt:\ }
        {\end{quote}\normalsize}

% 
%%%%%%%%%%%%%%%%%%%%%%%%%%%%%%%%%%%%%%%%%%%%%%%%%%%%%%%%%%%%%%%%

%%%%%%%%%%%%%%%%%%%%%%%%%%%%%%%%%%%%%%%%%%%%%%%%%%%%%%%%%%%%%%%%
%
%    Macros for Basic specifications.  Many of them are also used by
%    later parts.
%%%%%%%%%%%%%%%%%%%%%%%%%%%%%%%%%%%%%%%%%%%%%%%%%%%%%%%%%%%%%%%%
\newcommand{\FunName}{\mathit{FunName}}
\newcommand{\PredName}{\mathit{PredName}}
\newcommand{\FinSet}[1]{\mathit{FinSet}(#1)}
\newcommand{\Set}[1]{\mathit{Set}(#1)}
\newcommand{\FinSeq}[1]{\mathit{FinSeq}(#1)}
\newcommand{\ang}[1]{\langle #1 \rangle}
\newcommand{\harpoon}{\rightharpoonup}
\newcommand{\PartialFun}[2]{#1 \harpoon #2}
\newcommand{\TotalFun}[2]{#1 \to #2}
\newcommand{\Dom}[1]{\mathit{Dom}(#1)}
\newcommand{\app}[2]{#1(#2)}
\newcommand{\FinMap}[2]{#1 \stackrel{\scriptstyle\mathrm{fin}}{\rightarrow} #2}
\newcommand{\complete}[2]{\mathit{complete}(#1,#2)}
\newcommand{\qua}[2]{#1 \: \textrm{qua} \: #2}
\newcommand{\unit}{\mathit{unit}}
\newcommand{\Signature}{\mathit{Signature}}
\newcommand{\SemModel}{\mathbf{Model}}
\newcommand{\SortSet}{\mathit{SortSet}}
\newcommand{\orelse}{\textrm{or}}
\newcommand{\ws}{\mathit{ws}}
\newcommand{\FunProfile}{\mathit{FunProfile}}
\newcommand{\FunNameSet}{\mathit{FunNameSet}}
\newcommand{\FunSet}{\mathit{FunSet}}
\newcommand{\PredProfile}{\mathit{PredProfile}}
\newcommand{\PredNameSet}{\mathit{PredNameSet}}
\newcommand{\PredSet}{\mathit{PredSet}}
% \newcommand{\Qual}{\mathit{Qual}}
% \newcommand{\sortqual}{\mathit{sort}}
% \newcommand{\tfunqual}{\mathit{tfun}}
% \newcommand{\pfunqual}{\mathit{pfun}}
% \newcommand{\predqual}{\mathit{pred}}
\newcommand{\Sym}{\mathit{Sym}}
\newcommand{\totqual}{\mathsf{t}}
\newcommand{\parqual}{\mathsf{p}}
\newcommand{\SigFragment}{\mathit{SigFragment}}
\newcommand{\sorts}[1]{\mathit{sorts}(#1)}
\newcommand{\Extension}{\mathit{Extension}}
\newcommand{\QualFunName}{\mathit{QualFunName}}
\newcommand{\QualPredName}{\mathit{QualPredName}}
\newcommand{\SortMap}{\mathit{SMap}}
\newcommand{\TotalFunMap}{\mathit{TFMap}}
\newcommand{\PartialFunMap}{\mathit{PFMap}}
\newcommand{\PredMap}{\mathit{PMap}}
\newcommand{\SignatureMorphism}{\mathit{SignatureMorphism}}
\newcommand{\hook}{\hookrightarrow}
\newcommand{\identity}{\mathit{id}}
\newcommand{\SigCategory}{\Sig}
\newcommand{\SemCarrier}{\mathbf{Carrier}}
\newcommand{\SemCarriers}{\mathbf{Carriers}}
\newcommand{\SemPartialFun}{\mathbf{PartialFun}}
\newcommand{\SemPartialFuns}{\mathbf{PartialFuns}}
\newcommand{\SemPred}{\mathbf{Pred}}
\newcommand{\SemPreds}{\mathbf{Preds}}
\newcommand{\Mset}{{\cal M}}
\newcommand{\SemModelClass}{\mathbf{ModelClass}}
\newcommand{\Homomorphism}{\mathbf{Homomorphism}}
\newcommand{\reduct}[2]{#1|_{#2}}
\newcommand{\ModFunctor}{\mathbf{Mod}}
\newcommand{\CatCategory}{\mathbf{Cat}}
\newcommand{\op}{\mathit{op}}
\newcommand{\SynVar}{\mathit{Var}}
\newcommand{\Variables}{\mathit{Variables}}
\newcommand{\QualVarName}{\mathit{QualVarName}}
\newcommand{\FQTerm}{\mathit{FQTerm}}
\newcommand{\conditional}[3]{#1 \rightarrow #2 \mid #3}
\newcommand{\sort}[1]{\mathit{sort}(#1)}
\newcommand{\SynFormula}{\mathit{Formula}}
\newcommand{\strongeq}{\stackrel{\scriptstyle\mathrm{s}}{=}}
\newcommand{\existseq}{\stackrel{\scriptstyle\mathrm{e}}{=}}
\newcommand{\false}{\mathit{false}}
\newcommand{\true}{\mathit{true}}
\newcommand{\abbreviates}{\textrm{abbreviates}}
\newcommand{\Constraint}{\mathit{Constraint}}
\newcommand{\Sentence}{\mathit{Sentence}}
\newcommand{\SenFunctor}{\mathbf{Sen}}
\newcommand{\SetCategory}{\mathbf{Set}}
\newcommand{\Enrichment}{\mathit{Enrichment}}
\newcommand{\Assignment}{\mathbf{Assignment}}
\newcommand{\lsem}{\mathopen{\lbrack\!\lbrack}}
\newcommand{\rsem}{\mathopen{\rbrack\!\rbrack}}
\newcommand{\termvalue}[2]{\lsem #1 \rsem_{#2}}
\newcommand{\infrule}[2]{\frac{#1}{#2}}
\newcommand{\satisfies}[3]{#1 \models_{#2} #3}
\newcommand{\notsatisfies}[3]{#1 \not\models_{#2} #3}
\newcommand{\staticsem}[3]{#1 \vdash #2 \rhd #3}
\newcommand{\modelsem}[3]{#1 \vdash #2 \Rightarrow #3}
\newcommand{\rulesection}[1]{\[\fbox{$#1$}\]}
\newcommand{\disjointfunctions}[1]{\textit{disjoint-functions\/}(#1)}
\newcommand{\consistentselectors}{\textit{consistent-selectors}\:\:}
\newcommand{\injective}[1]{\mathit{injective}(#1)}
\newcommand{\disjoint}[2]{\mathit{disjoint}(#1,#2)}
\newcommand{\undef}[2]{\mathit{undef}(#1,#2)}
\newcommand{\calC}{{\cal C}}

\usepackage{cofidoc}
\usepackage{casl}
\usepackage{alltt}
\usepackage{xspace}

\newcommand{\point} {\mbox{ $\scriptstyle\bullet$ }}

%\newcommand{\undef}{\mbox{\it undef}\xspace}

\newcommand{\tick}{tick}

\newcommand{\ar}{ \ - \! \! >}
\newcommand{\lar}{ <  \! \! - \ }
\newcommand{\lrar}{ <  \! \! - \!\! > }
\newcommand{\ec}{ \, [\,] \, }
\newcommand{\ic}{ \, |^\sim| \,}
\newcommand{\genp}[1]{ \, [| \, #1 \, |] \,}
\newcommand{\alphp}[2]{ \, [ \, #1 \, || \, #2 \,] \,}
\newcommand{\synp}{ \, || \, }
\newcommand{\intp}{ \, ||| \, }

\newcommand{\traces}{\mbox{\it traces}}
\newcommand{\free}{\mbox{\it free}\,}

\newcommand{\Seq}{\mbox{\it Seq}\,}
\newcommand{\IntC}{\mbox{\it IntC}\,}
\newcommand{\ExtC}{\mbox{\it ExtC}\,}
\newcommand{\GenP}{\mbox{\it GenP}\,}
\newcommand{\Synch}{\mbox{\it Synch}\,}
\newcommand{\Int}{\mbox{\it Int}\,}

\newcommand{\va}{\mbox{\it var}\,}

\newcommand{\CSP}{\textrm{\textsc{Csp}}\xspace}
\newcommand{\CC} {\textrm{\textsc{Csp}-\textsc{Casl}}\xspace}

\newcommand{\Gram}[1]{\texttt{#1}}
\newcommand{\ul}[1]{\underline{#1}}
\newcommand{\annodef}{ {\bf \%def}}
\newcommand{\annocons}{ {\bf \%cons}}

%\newenvironment{Grammar}
% {\texonly{\footnotesize\setlength{\exampleindent}{0pt}}
%  \begin{example}}{\end{example}\normalsize\ignorespaces}


\title{Concrete and Abstract Syntax of \CC}

\begin{document}

\maketitle

\section{\CC without Channels}

\subsection{Overall Specification}

{\bf hier wird sich noch viel aendern!!!}

\begin{Grammar}
CSP-CASL-SPEC ::= data DATA-DEFN process PROCESS-DEFN end/

DATA-DEFN     ::=   \ul{BASIC-SPEC}

PROCESS-DEFN  ::=   PROCESS 
                  | REC-PROCESS
                  | var/vars \ul{VAR-DECL}; ...; \ul{VAR-DECL};/ . PROCESS 
                  | var/vars \ul{VAR-DECL}; ...; \ul{VAR-DECL};/ . REC-PROCESS 
\end{Grammar}

\subsection{Basic Processes}

\begin{Grammar}
PROCESS ::=   (PROCESS)
            | Skip
            | Stop
            | EVENT -> PROCESS
            | [] \ul{VAR}: EVENT-SET -> PROCESS
            | PROCESS ; PROCESS
            | PROCESS [] PROCESS
            | PROCESS |~| PROCESS
            | PROCESS [| EVENT-SET |] PROCESS
            | PROCESS [ EVENT-Set || EVENT-SET ] PROCESS
            | PROCESS || PROCESS
            | PROCESS ||| PROCESS
            | PROCESS \back EVENT-SET
            | PROCESS [[CSP-RENAMING]]
            | if \ul{FORMULA} then PROCESS else PROCESS
\end{Grammar}

This context-free grammar for input syntax is quite ambiguous. For a
\Gram{PROCESS} a unique syntax tree is requiered. This can be achieved
by brackets. Beyond that ambiguities of this grammar are resolved by
precedences\footnote{This definition is an extension of the FDR
conventions as documented e.g.~in \cite{roscoe98}: all its rules are
preserved, new precedences have been defined for hiding (the same
precedence as renaming) and multiple prefix (the same precedence as
prefix).}:
\begin{itemize}
  
\item Renaming $\_\_[[\_\_]]$ and hiding $\_\_ \backslash \_\_$ have the
highest precedence.
  
\item 
%
Prefix $\_\_\ar\_\_$, 
%
multiple prefix $[]\_\_: \_\_ \ar \_\_$, and 
%
sequential composition $\_\_;\_\_$ have
lower precedence.
  
\item External choice $\_\_ \ec \_\_$ and internal choice $\_\_ \ic \_\_$
  have still lower precendence.
  
\item The parallel operators $\_\_[|\_\_|]\_\_,$ $\_\_\synp\_\_,$ and
$\_\_\intp\_\_$ have lower precedence than all the above.
  
\item The conditional $\_\_<\_\_>\_\_$ has the lowest precedence.

\end{itemize}

%% Precedence:s Roscoe, S. 512; FDR Manual 2.78, page 65.

To avoid brackets in case of consecutive occurences of the same
operator the following operators are considered as left
associative\footnote{As these operators turn out to be associative,
  this convention gives not rise to any problem with the semantics.}:
\begin{itemize}
\item $\_\_;\_\_,$
\item $\_\_\ec\_\_,$
\item $\_\_\ic\_\_,$
\item $\_\_\synp\_\_,$ and
\item $\_\_\intp\_\_.$
\end{itemize}
This means that, for instance, $P_1;P_2;P_3$ is parsed as
$(P_1;P_2);P_3.$ The prefix operator is assumed to be right
associative. 

{\bf right assoc f�r prefix wirklich notwendig?}

The non terminals  
%
\Gram{FORMULA}, 
\Gram{OP-Name}, 
\Gram{SORT},
\Gram{TERM}, and
\Gram{VAR}
%
of the above grammar refer to appendix C of the \CASL language summary
\cite{CASL/Summary00}. The others are explained in the following
sections.

\subsubsection{Event Sets}\label{subsec:event_sets}

\begin{Grammar}
EVENT-SET ::= \ul{SORT}
\end{Grammar}

\subsubsection{Prefix and Multiple Prefix}

\begin{Grammar}
EVENT ::= \ul{TERM}
\end{Grammar}

\subsubsection{Predefined Macros}

With the above operators one can define:
\begin{enumerate}

\item ``time-out operator'':
$
P \, [> Q := (P \ic Stop) \ec Q
$
%% Roscoe, S.80.

\item ``boolean guard'':
$
F \ \& \ P := P < F > Stop
$
%% Roscoe, S.508

\end{enumerate}

These operators are added to the input syntax of \CC, but do not
belong to its core language. Each occurence of them is replaced
according to their above definition.

These operators are parsed with the following precedences:

\begin{enumerate}

\item Time-out has the same precedence as the choice operators.

\item Boolean guard has the same precedence as the conditional.

\end{enumerate}

\section{{\sc CSP-CASL} with Channels}\label{chap:csp_with_channels}

A \Gram{CSP-CASL-CHAN-SPEC} consists of a data part, a channel part,
and a process part:

\begin{Grammar}
CSP-CASL-CHAN-SPEC ::= data DATA-DEFN 
                       channel/channels CHANNEL-DECL
                       process PROCESS-DEFN
                       end/
\end{Grammar}

\subsection{Channel Declaration}\label{sec:channel_decl}

A channel declaration is a non-empty list of \Gram{CHANNEL-ITEM}s.

\begin{Grammar}
CHANNEL-DECL ::= CHANNEL-ITEM ; ... ; CHANNEL-ITEM ;/
\end{Grammar}

\begin{Grammar}
CHANNEL-ITEM ::=  CHANNEL-NAME, ..., CHANNEL-NAME : \ul{SORT}             
CHANNEL-NAME ::=  \ul{SIMPLE-ID}
\end{Grammar}

\subsection{Extensions in the Process Part}

\subsubsection{Event sets}\label{subs:event_sets}

\begin{Grammar}
EVENT-SET ::=  ... 
              | \{| CHANNEL-NAME |\}
\end{Grammar}

Assuming there is a \Gram{CHANNEL-ITEM}
\begin{casl}\(c:s\)\end{casl} declared in the channel part,
the \Gram{EVENT-SET}
%
\begin{center}
\begin{casl}
\( \{| c |\} \)
\end{casl}
\end{center} 
within the process part of a \Gram{CSP-CASL-CHAN-SPEC} specification
is translated into
\begin{center}
\begin{casl}
\(CHANNEL\_c\) .
\end{casl} 
\end{center}

\subsubsection{Prefix}\label{subs:prefix}

{\sc CSP-CASL} with Channels extends the set of possible events
\Gram{EVENT} by writing and reading on an (indexed) channel:

\begin{Grammar}
EVENT ::=   ... 
          | CHANNEL-NAME ! \ul{TERM} 
          | CHANNEL-NAME ? \ul{VAR} : \ul{SORT}
\end{Grammar}

\subsubsection{Linked Parallel}\label{subs:linkedParallel}

\begin{Grammar}
PROCESS ::=   ...
            | PROCESS [ CHANNEL-NAME <-> CHANNEL-NAME ] PROCESS
              ...
\end{Grammar}

\subsubsection{Renaming}\label{subs:renaming}

\begin{Grammar}
  CHANNEL-RENAMING ::= CHANNEL-NAME <- CHANNEL-NAME
\end{Grammar}


\begin{Grammar}
CSP-RENAMING ::=   ...  
                 | CHANNEL-RENAMING
\end{Grammar}

\newpage
\section{Complete Concrete Syntax: without Channels}

\begin{Grammar}
CSP-CASL-SPEC ::=   data DATA-DEFN process PROCESS-DEFN end/

DATA-DEFN     ::=   \ul{BASIC-SPEC}

PROCESS-DEFN  ::=   PROCESS 

PROCESS ::=         (PROCESS)
                  | PROCESS-NAME
                  | PROCESS-NAME (\ul{TERM})
                  | Skip
                  | Stop
                  | EVENT -> PROCESS
                  | [] \ul{VAR}: EVENT-SET -> PROCESS
                  | PROCESS ; PROCESS
                  | PROCESS [] PROCESS
                  | PROCESS |~| PROCESS
                  | PROCESS [| EVENT-SET |] PROCESS
                  | PROCESS [ EVENT-Set || EVENT-SET ] PROCESS
                  | PROCESS || PROCESS
                  | PROCESS ||| PROCESS
                  | PROCESS \back EVENT-SET
                  | PROCESS [[CSP-RENAMING]]
                  | if \ul{FORMULA} then PROCESS else PROCESS
                  | PROCESS [> PROCESS
                  | \ul{FORMULA} & PROCESS

EVENT-SET ::=       \ul{SORT}

EVENT ::=           \ul{TERM}

CSP-RENAMING  ::=  SORT-RENAMING

SORT-RENAMING ::=  \ul{OP-NAME}, ..., \ul{OP-NAME}


PROCESS-NAME ::=    \ul{SIMPLE-ID}
\end{Grammar}

\newpage
\section{Complete Concrete Syntax: with Channels}

\begin{Grammar}
CSP-CASL-C-SPEC ::= data DATA-DEFN 
                    channel/channels CHANNEL-DECL
                    process PROCESS-DEFN
                    end/

DATA-DEFN     ::=   \ul{BASIC-SPEC}

CHANNEL-DECL  ::=   CHANNEL-ITEM ; ... ; CHANNEL-ITEM ;/

CHANNEL-ITEM ::=    CHANNEL-NAME, ..., CHANNEL-NAME : \ul{SORT}             

CHANNEL-NAME ::=    \ul{SIMPLE-ID}

PROCESS-DEFN ::=    PROCESS 

PROCESS ::=         (PROCESS)
                  | PROCESS-NAME
                  | PROCESS-NAME (\ul{TERM})
                  | Skip
                  | Stop
                  | EVENT -> PROCESS
                  | [] \ul{VAR}: EVENT-SET -> PROCESS
                  | PROCESS ; PROCESS
                  | PROCESS [] PROCESS
                  | PROCESS |~| PROCESS
                  | PROCESS [| EVENT-SET |] PROCESS
                  | PROCESS [ EVENT-Set || EVENT-SET ] PROCESS
                  | PROCESS || PROCESS
                  | PROCESS ||| PROCESS
                  | PROCESS \back EVENT-SET
                  | PROCESS [[CSP-RENAMING]]
                  | if \ul{FORMULA} then PROCESS else PROCESS
                  | PROCESS [> PROCESS
                  | \ul{FORMULA} & PROCESS
                  | PROCESS [ CHANNEL-NAME <-> CHANNEL-NAME ] PROCESS


EVENT-SET ::=       \ul{SORT}
                  | \{| CHANNEL-NAME |\}

EVENT ::=           \ul{TERM}
                  | CHANNEL-NAME ! \ul{TERM} 
                  | CHANNEL-NAME ? \ul{VAR} : \ul{SORT}

CSP-RENAMING ::=    SORT-RENAMING
                  | CHANNEL-RENAMING

SORT-RENAMING ::=  \ul{OP-NAME}, ..., \ul{OP-NAME}

CHANNEL-RENAMING ::= CHANNEL-NAME <- CHANNEL-NAME

PROCESS-NAME ::=    \ul{SIMPLE-ID}
\end{Grammar}


\newpage
\section{Complete Abstract Syntax: with Channels}

\begin{Grammar}
CSP-CASL-C-SPEC ::= cc_c_spec DATA-DEFN CHANNEL-DECL PROCESS-DEFN

DATA-DEFN     ::=   \ul{BASIC-SPEC}

CHANNEL-DECL  ::=   ch_items CHANNEL-ITEM*

CHANNEL-ITEM  ::=   ch_decl CHANNEL-NAME+ \ul{SORT}

CHANNEL-NAME ::=    \ul{SIMPLE-ID}

PROCESS-DEFN ::=    basic PROCESS 

PROCESS ::=         named_process PROCESS-NAME
                  | generic_named_process PROCESS-NAME \ul{TERM}
                  | skip
                  | stop
                  | prefix EVENT PROCESS
                  | multiple_prefix \ul{VAR} EVENT-SET PROCESS
                  | sequential PROCESS+ 
                  | external_choice PROCESS+
                  | internal_choice PROCESS+
                  | alphabet_parallel PROCESS EVENT-SET PROCESS
                  | general_parallel PROCESS EVENT-SET EVENT-SET PROCESS
                  | synchronous_parallel PROCESS+
                  | interleaving_parallel PROCESS+
                  | hiding PROCESS EVENT-SET
                  | renaming PROCESS CSP-RENAMING
                  | conditional \ul{FORMULA} PROCESS PROCESS
                  | channel_parallel PROCESS CHANNEL-NAME CHANNEL-NAME PROCESS

EVENT-SET ::=       sort \ul{SORT}
                  | channel CHANNEL-NAME

EVENT ::=           term \ul{TERM}
                  | send CHANNEL-NAME \ul{TERM} 
                  | receive CHANNEL-NAME \ul{VAR} \ul{SORT}

CSP-RENAMING ::=    SORT-RENAMING
                  | CHANNEL-RENAMING

SORT-RENAMING ::=  op_list \ul{OP-NAME}+

CHANNEL-RENAMING ::= channel_renaming CHANNEL-NAME CHANNEL-NAME

PROCESS-NAME ::=    \ul{SIMPLE-ID}
\end{Grammar}


\end{document}







