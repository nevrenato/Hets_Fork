%!TEX root = main.tex

In this section we explain how the Maude specifications introduced in
\Hets are parsed in order to obtain a term following the abstract syntax
described in
the previous section. We are able to implement this parsing in Maude
itself thanks to Maude metalevel \cite[Chapter 14]{maude-book}, a
module that allows the programmer to use Maude entities such as modules, equations,
or rules as usual data by efficiently implementing the \emph{reflective}
capabilities of rewriting logic \cite{ClavelMeseguerPalomino07}.

The function \verb"haskellify" receives a module (the first parameter
stands for the original module, while the second one contains the
flattened one) and returns a list of quoted identifiers creating an
object of type \verb"Spec", that can be read by Haskell since this data
type derives the class \verb"Read":

{\codesize
\begin{verbatim}
  op haskellify : Module Module -> QidList .
  ceq haskellify(M, M') = 
      'SpecMod '`( 'Module haskellifyHeader(H) ' ' 
      '`[ haskellifyImports(IL) comma(IL, SS)
          haskellifySorts(SS) comma(IL, SS, SSDS)
          haskellifySubsorts(SSDS) comma(IL, SS, SSDS, ODS)
          haskellifyOpDeclSet(M', ODS) comma(IL, SS, SSDS, ODS, MAS)
          haskellifyMembAxSet(M', MAS) comma(IL, SS, SSDS, ODS, MAS, EqS)
          haskellifyEqSet(M', EqS) '`] '`) '\n '@#$endHetsSpec$#@ '\n
    if fmod H is IL sorts SS . SSDS ODS MAS EqS endfm := M .
\end{verbatim}
}

This function prints the keyword \verb"SpecMod" and uses the \verb"haskellify"
auxiliary functions to print the different parts of the module.
The functions \verb"comma" introduce a comma whenever it is necessary.
Since all the ``haskellify'' functions are very similar, we describe
them by using \verb"haskellifyImports" as example. This function traverses
all the imports in the list and applies the auxiliary function
\verb"haskellifyImport" to each of them:

{\codesize
\begin{verbatim}
  op haskellifyImports : ImportList -> QidList .
  eq haskellifyImports(nil) = nil .
  eq haskellifyImports(I IL) = 'ImportStmnt ' '`( haskellifyImport(I) '`)
                               comma(IL) haskellifyImports(IL) .
\end{verbatim}
}

This auxiliary function distinguishes between the importation modes,
using the appropriate keyword for each of them:

{\codesize
\begin{verbatim}
  op haskellifyImport : Import -> QidList .
  eq haskellifyImport(protecting ME .) = 'Protecting haskellifyME(ME) .
  eq haskellifyImport(including ME .) = 'Including haskellifyME(ME) .
  eq haskellifyImport(extending ME .) = 'Extending haskellifyME(ME) .
\end{verbatim}
}

\noindent where \verb"haskellifyME" is in charge of printing the 
module expression. When it is just an identifier, it prints it 
preceded by the word \verb"ModId":

{\codesize
\begin{verbatim}
  op haskellifyME : ModuleExpression -> QidList .
  eq haskellifyME(Q) = ' '`( 'ModExp ' '`( 'ModId qid2token(Q) '`) '`) ' .
\end{verbatim}
}

The summation module expression recursively prints the summands, and uses
the keyword \verb"SummationModExp" to indicate the type of module expression:

{\codesize
\begin{verbatim}
  eq haskellifyME(ME + ME') = ' '`( 'SummationModExp haskellifyME(ME) 
                              haskellifyME(ME')  '`) ' .
\end{verbatim}
}

To print a renaming we recursively apply \verb"haskellifyME" for the inner
module expression and then we use the auxiliary function \verb"haskellifyMaps" to 
print the renamings. In this case we use the constant \verb"no-module" as argument
because it will only be used when parsing mappings from views, since it may
be needed to parse terms:

{\codesize
\begin{verbatim}
  eq haskellifyME(ME * (RNMS)) = ' '`( 'RenamingModExp haskellifyME(ME) 
                                 '`[ haskellifyMaps(no-module, no-module, RNMS) '`]  '`) ' .
\end{verbatim}
}

Finally, an instantiation is printed by using an auxiliary function \verb"haskellifyPL"
in charge of the parameters:

{\codesize
\begin{verbatim}
  eq haskellifyME(ME {PL}) = ' '`( 'InstantiationModExp haskellifyME(ME) 
                             '`[ haskellifyPL(PL)  '`]  '`) ' .
\end{verbatim}
}
