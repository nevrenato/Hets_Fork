%!TEX root = main.tex

We show in this section how the comorphism
from Maude to \CASL described in Section \ref{sec:comoprh} is implemented.
The function in charge of computing the comorphism is
\verb"maude2casl", that returns the \CASL signature and sentences
given the Maude signature and sentences:

{\codesize
\begin{verbatim}
maude2casl :: MSign.Sign -> [Named MSentence.Sentence] -> (CSign.CASLSign, 
                                                           [Named CAS.CASLFORMULA])
\end{verbatim}
}

This function splits the work into different stages:

\begin{itemize}

\item
The function \verb"rewPredicates" generates the \verb"rew" predicates for
each sort to simulate the rewrite rules in the Maude specification.

\item
The function \verb"rewPredicatesSens" creates the formulas associated to
the \verb"rew" predicates created above, stating that they are reflexive
and transitive.

\item
The \CASL operators are obtained from the Maude operators:

\begin{itemize}
\item
The function \verb"translateOps" splits the Maude operator map
into a tuple of \CASL operators and \CASL associative operators,
(which are required for parsing purposes).

\item
Since \CASL does not allow the definition
of polymorphic operators, these operators are removed from the map
with \verb"deleteUniversal" and for each one of these Maude operators we
create a set of \CASL operators with all the possible profiles with
\verb"universalOps".

\end{itemize}

\item \CASL sentences are obtained from the Maude sentences and from
predefined \CASL libraries:

\begin{itemize}
\item In the computation of the \CASL formulas we split Maude sentences in
equations defined without the \verb"owise" attribute, equations defined
with \verb"owise", and the rest of statements with the function
\verb"splitOwiseEqs".
\item The equations defined without the \verb"owise" attribute are
translated as universally quantified equations, as shown in Section
\ref{sec:comoprh}, with \verb"noOwiseSen2Formula".
\item Equations with the \verb"owise" attribute are translated using
a negative existential quantification, as we will show later, with
the function \verb"owiseSen2Formula". This function requires as additional
parameter the definition of the formulas defined without the \verb"owise"
attribute, in order to state that the equations defined with \verb"owise"
are applied when the rest of possible equations cannot.
\item The rest of statements, namely memberships and rules, are translated
with the function \verb"mb_rl2formula".
\item There are some built-in operators in Maude that are not defined by
means of equations. To allow the user to reason about them we provide
some libraries with the definitions of these operators as \CASL formulas,
obtained with \verb"loadLibraries".
\end{itemize}

\item Finally, the \CASL symbols are created:

\begin{itemize}
\item The kinds are translated to symbols with \verb"kinds2syms".
\item The operators are translated with \verb"ops2symbols".
\item The symbol predicates are obtained with \verb"preds2syms".
\end{itemize}

\end{itemize}

{\codesize
\begin{verbatim}
maude2casl msign nsens = (csign { CSign.sortSet = cs,
                            CSign.sortRel = sbs',
                            CSign.opMap = cops',
                            CSign.assocOps = assoc_ops,
                            CSign.predMap = preds,
                            CSign.declaredSymbols = syms }, new_sens)
   where csign = CSign.emptySign ()
         ss = MSign.sorts msign
         ss' = Set.map sym2id ss
         mk = kindMapId $ MSign.kindRel msign
         sbs = MSign.subsorts msign
         sbs' = maudeSbs2caslSbs sbs mk
         cs = Set.union ss' $ kindsFromMap mk
         preds = rewPredicates cs
         rs = rewPredicatesSens cs
         ops = deleteUniversal $ MSign.ops msign
         ksyms = kinds2syms cs
         (cops, assoc_ops, _) = translateOps mk ops
         cops' = universalOps cs cops $ booleanImported ops
         rs' = rewPredicatesCongSens cops'
         pred_forms = loadLibraries (MSign.sorts msign) ops
         ops_syms = ops2symbols cops'
         (no_owise_sens, owise_sens, mbs_rls_sens) = splitOwiseEqs nsens
         no_owise_forms = map (noOwiseSen2Formula mk) no_owise_sens
         owise_forms = map (owiseSen2Formula mk no_owise_forms) owise_sens
         mb_rl_forms = map (mb_rl2formula mk) mbs_rls_sens
         preds_syms = preds2syms preds
         syms = Set.union ksyms $ Set.union ops_syms preds_syms
         new_sens = concat [rs, rs', no_owise_forms, owise_forms,
                            mb_rl_forms, pred_forms]
\end{verbatim}
}

The \verb"rew" predicates are declared with the function
\verb"rewPredicates", that traverses the set of sorts applying
the function \verb"rewPredicate":

{\codesize
\begin{verbatim}
rewPredicates :: Set.Set Id -> Map.Map Id (Set.Set CSign.PredType)
rewPredicates = Set.fold rewPredicate Map.empty
\end{verbatim}
}

This function defines a binary predicate using as identifier the constant
\verb"rewID" and the sort as type of the arguments:

{\codesize
\begin{verbatim}
rewPredicate :: Id -> Map.Map Id (Set.Set CSign.PredType)
                -> Map.Map Id (Set.Set CSign.PredType)
rewPredicate sort m = Map.insertWith (Set.union) rewID ar m
   where ar = Set.singleton $ CSign.PredType [sort, sort]
\end{verbatim}
}

Once these predicates have been declared, we have to introduce
formulas to state their properties. The function \verb"rewPredicatesSens"
accomplishes this task by traversing the set of sorts and applying
\verb"rewPredicateSens":

{\codesize
\begin{verbatim}
rewPredicatesSens :: Set.Set Id -> [Named CAS.CASLFORMULA]
rewPredicatesSens = Set.fold rewPredicateSens []
\end{verbatim}
}

This function generates the formulas for each sort:

{\codesize
\begin{verbatim}
rewPredicateSens :: Id -> [Named CAS.CASLFORMULA] -> [Named CAS.CASLFORMULA]
rewPredicateSens sort acc = ref : trans : acc
        where ref = reflSen sort
              trans = transSen sort
\end{verbatim}
}

We describe the formula for the reflexivity, being the formula for the
transitivity analogous. A new variable of the required sort is created with
the auxiliary function \verb"newVar", then the qualified predicate
name is created with the \verb"rewID" constant and applied to the
variable. Finally, the formula is named with the prefix \verb"rew_refl_"
followed by the name of the sort:

{\codesize
\begin{verbatim}
reflSen :: Id -> Named CAS.CASLFORMULA
reflSen sort = makeNamed name $ quantifyUniversally form
        where v = newVar sort
              pred_type = CAS.Pred_type [sort, sort] nullRange
              pn = CAS.Qual_pred_name rewID pred_type nullRange
              form = CAS.Predication pn [v, v] nullRange
              name = "rew_refl_" ++ show sort
\end{verbatim}
}

The function \verb"translateOps" traverses the map of Maude operators,
applying to each of them the function \verb"translateOpDeclSet":

{\codesize
\begin{verbatim}
translateOps :: IdMap -> MSign.OpMap -> OpTransTuple
translateOps im = Map.fold (translateOpDeclSet im) (Map.empty, Map.empty, Set.empty)
\end{verbatim}
}

Since the values in the Maude operator map are sets of operator declarations
the auxiliary function \verb"translateOpDeclSet" has to traverse these sets, applying
\verb"translateOpDecl" to each operator declaration:

{\codesize
\begin{verbatim}
translateOpDeclSet :: IdMap -> MSign.OpDeclSet -> OpTransTuple -> OpTransTuple
translateOpDeclSet im ods tpl = Set.fold (translateOpDecl im) tpl ods
\end{verbatim}
}

The function \verb"translateOpDecl" receives an operator declaration,
that consists of all the operators declared with the same profile at
the kind level. The function traverses these operators, transforming
them into \CASL operators with the function \verb"ops2pred" and returning
a tuple containing the operators, the associative operators, and the
constructors:

{\codesize
\begin{verbatim}
translateOpDecl :: IdMap -> MSign.OpDecl -> OpTransTuple -> OpTransTuple
translateOpDecl im (syms, ats) (ops, assoc_ops, cs) = case tl of
                      [] -> (ops', assoc_ops', cs')
                      _ -> translateOpDecl im (syms', ats) (ops', assoc_ops', cs')
      where sym = head $ Set.toList syms
            tl = tail $ Set.toList syms
            syms' = Set.fromList tl
            (cop_id, ot, _) = fromJust $ maudeSym2CASLOp im sym
            cop_type = Set.singleton ot
            ops' = Map.insertWith (Set.union) cop_id cop_type ops
            assoc_ops' = if any MAS.assoc ats
                         then Map.insertWith (Set.union) cop_id cop_type assoc_ops
                         else assoc_ops
            cs' = if any MAS.ctor ats
                  then Set.insert (Component cop_id ot) cs
                  else cs
\end{verbatim}
}

As said above, Maude equations that are not defined with the \verb"owise"
attribute are translated to \CASL with \verb"noOwiseSen2Formula". This
function extracts the current equation from the named sentence, translates
it with \verb"noOwiseEq2Formula" and creates a new named sentence
with the resulting formula:

{\codesize
\begin{verbatim}
noOwiseSen2Formula ::  IdMap -> Named MSentence.Sentence -> Named CAS.CASLFORMULA
noOwiseSen2Formula im s = s'
       where MSentence.Equation eq = sentence s
             sen' = noOwiseEq2Formula im eq
             s' = s { sentence = sen' }
\end{verbatim}
}

The function \verb"noOwiseEq2Formula" distinguishes whether the equation
is conditional or not. In both cases, the Maude terms in the equation
are translated into \CASL terms with \verb"maudeTerm2caslTerm", and a
strong equation is used to create a formula. If the equation has no
conditions this formula is universally quantified and returned as result,
while if it has conditions each of them generates a formula and their
conjunction, computed with \verb"conds2formula", will be used as premise
of the equality formula:

{\codesize
\begin{verbatim}
noOwiseEq2Formula :: IdMap -> MAS.Equation -> CAS.CASLFORMULA
noOwiseEq2Formula im (MAS.Eq t t' [] _) = quantifyUniversally form
      where ct = maudeTerm2caslTerm im t
            ct' = maudeTerm2caslTerm im t'
            form = CAS.Strong_equation ct ct' nullRange
noOwiseEq2Formula im (MAS.Eq t t' conds@(_:_) _) = quantifyUniversally form
      where ct = maudeTerm2caslTerm im t
            ct' = maudeTerm2caslTerm im t'
            conds_form = conds2formula im conds
            concl_form = CAS.Strong_equation ct ct' nullRange
            form = createImpForm conds_form concl_form
\end{verbatim}
}

\verb"maudeTerm2caslTerm" is defined for each Maude term:

\begin{itemize}

\item Variables are translated into qualified \CASL variables, and their
type is translated to the corresponding type in \CASL:

{\codesize
\begin{verbatim}
maudeTerm2caslTerm :: IdMap -> MAS.Term -> CAS.CASLTERM
maudeTerm2caslTerm im (MAS.Var q ty) = CAS.Qual_var q ty' nullRange
        where ty' = maudeType2caslSort ty im
\end{verbatim}
}

\item Constants are translated as functions applied
to the empty list of arguments:

{\codesize
\begin{verbatim}
maudeTerm2caslTerm im (MAS.Const q ty) = CAS.Application op [] nullRange
        where name = token2id q
              ty' = maudeType2caslSort ty im
              op_type = CAS.Op_type CAS.Total [] ty' nullRange
              op = CAS.Qual_op_name name op_type nullRange
\end{verbatim}
}

\item The application of an operator to a list of terms is translated
into another application, translating recursively the arguments into
valid \CASL terms:

{\codesize
\begin{verbatim}
maudeTerm2caslTerm im (MAS.Apply q ts ty) = CAS.Application op tts nullRange
        where name = token2id q
              tts = map (maudeTerm2caslTerm im) ts
              ty' = maudeType2caslSort ty im
              types_tts = getTypes tts
              op_type = CAS.Op_type CAS.Total types_tts ty' nullRange
              op = CAS.Qual_op_name name op_type nullRange
\end{verbatim}
}

\end{itemize}

The conditions are translated into a conjunction with \verb"conds2formula",
that traverses the conditions applying \verb"cond2formula" to each of them,
and then creates the conjunction of the obtained formulas:

{\codesize
\begin{verbatim}
conds2formula :: IdMap -> [MAS.Condition] -> CAS.CASLFORMULA
conds2formula im conds = CAS.Conjunction forms nullRange
        where forms = map (cond2formula im) conds
\end{verbatim}
}

\begin{itemize}

\item Both equality and matching conditions are translated into
strong equations:

{\codesize
\begin{verbatim}
cond2formula :: IdMap -> MAS.Condition -> CAS.CASLFORMULA
cond2formula im (MAS.EqCond t t') = CAS.Strong_equation ct ct' nullRange
       where ct = maudeTerm2caslTerm im t
             ct' = maudeTerm2caslTerm im t'
cond2formula im (MAS.MatchCond t t') = CAS.Strong_equation ct ct' nullRange
       where ct = maudeTerm2caslTerm im t
             ct' = maudeTerm2caslTerm im t'
\end{verbatim}
}

\item Membership conditions are translated into \CASL memberships by translating
the term and the sort:

{\codesize
\begin{verbatim}
cond2formula im (MAS.MbCond t s) = CAS.Membership ct s' nullRange
      where ct = maudeTerm2caslTerm im t
            s' = token2id $ getName s
\end{verbatim}
}

\item Rewrite conditions are translated into formulas by using both terms
as arguments of the corresponding \verb"rew" predicate: 

{\codesize
\begin{verbatim}
cond2formula im (MAS.RwCond t t') = CAS.Predication pred_name [ct, ct'] nullRange
       where ct = maudeTerm2caslTerm im t
             ct' = maudeTerm2caslTerm im t'
             ty = token2id $ getName $ MAS.getTermType t
             kind = Map.findWithDefault (errorId "rw cond to formula") ty im
             pred_type = CAS.Pred_type [kind, kind] nullRange
             pred_name = CAS.Qual_pred_name rewID pred_type nullRange
\end{verbatim}
}

\end{itemize}

The equations defined with the \verb"owise" attribute are translated
with \verb"owiseSen2Formula", that traverses them and applies
\verb"owiseEq2Formula" to the inner equation:

{\codesize
\begin{verbatim}
owiseSen2Formula ::  IdMap -> [Named CAS.CASLFORMULA] -> Named MSentence.Sentence 
                     -> Named CAS.CASLFORMULA
owiseSen2Formula im owise_forms s = s'
       where MSentence.Equation eq = sentence s
             sen' = owiseEq2Formula im owise_forms eq
             s' = s { sentence = sen' }
\end{verbatim}
}

This function receives all the formulas defined without the \verb"owise"
attribute and, for each formula with the same operator in the lefthand
side as the current equation (obtained with \verb"getLeftApp"), it
generates with \verb"existencialNegationOtherEqs" a negative existential
quantification stating that the arguments do not match or the condition
does not hold that is used as premise of the equation:

{\codesize
\begin{verbatim}
owiseEq2Formula :: IdMap -> [Named CAS.CASLFORMULA] -> MAS.Equation -> CAS.CASLFORMULA
owiseEq2Formula im no_owise_form eq = form
      where (eq_form, vars) = noQuantification $ noOwiseEq2Formula im eq
            (op, ts, _) = fromJust $ getLeftApp eq_form
            ex_form = existencialNegationOtherEqs op ts no_owise_form
            imp_form = createImpForm ex_form eq_form
            form = CAS.Quantification CAS.Universal vars imp_form nullRange
\end{verbatim}
}

%The rest of Maude sentences are translated in a similar way to the one
%shown for the conditions above.
%The rest of the sentences generated in the comorphism are obtained
%from external libraries with the function \verb"readLib". We describe
%below how sentences defining the behavior of the natural numbers are
%loaded: once the library is obtained, we transform the theory sentences
%into named sentences with \verb"toNamedList" and then we ``coerce''
%them with \verb"coerceSens" to indicate that they are \CASL sentences.
%Finally, the sentence about the generators is filtered and the result
%returned:

%{\codesize
%\begin{verbatim}
%loadNaturalNatSens :: [Named CAS.CASLFORMULA]
%loadNaturalNatSens = 
%       let lib = head $ unsafePerformIO $ readLib "Maude/MaudeNumbers.casl"
%       in case lib of
%           G_theory lid _ _ thSens _ -> let sens = toNamedList thSens
%                                        in do
%                                            sens' <- coerceSens lid CASL "" sens
%                                            filter (not . ctorCons) sens'
%\end{verbatim}
%}

The translation from sorts, operators, and predicates to symbols
works in a similar way to the transformations shown above, so we only
describe  how the predicate symbols are obtained. The function
\verb"preds2syms" traverses the map of predicates and inserts each
obtained symbol into the set with \verb"pred2sym":

{\codesize
\begin{verbatim}
preds2syms :: Map.Map Id (Set.Set CSign.PredType) -> Set.Set CSign.Symbol
preds2syms = Map.foldWithKey pred2sym Set.empty
\end{verbatim}
}

This function traverses the set of predicate types and creates the
symbol corresponding to each one with \verb"createSym4id":

{\codesize
\begin{verbatim}
pred2sym :: Id -> Set.Set CSign.PredType -> Set.Set CSign.Symbol -> Set.Set CSign.Symbol
pred2sym pn spt acc = Set.fold (createSym4id pn) acc spt
\end{verbatim}
}

\verb"createSym4id" generates the symbol and inserts it into the
accumulated set:

{\codesize
\begin{verbatim}
createSym4id :: Id -> CSign.PredType -> Set.Set CSign.Symbol -> Set.Set CSign.Symbol
createSym4id pn pt acc = Set.insert sym acc
      where sym = CSign.Symbol pn $ CSign.PredAsItemType pt
\end{verbatim}
}





















