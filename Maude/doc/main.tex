\documentclass[10pt,a4paper]{article}
\usepackage{geometry}
\geometry{a4paper}
\geometry{margin=2.5cm,nohead}
\usepackage{amsfonts}
\usepackage[leqno]{amsmath}
\usepackage{rotating}
\usepackage[english]{babel}
\usepackage{url}
\usepackage{proof}
\usepackage{alltt}
\usepackage{hyperref}
\usepackage{verbatim}
\usepackage{amssymb}
\usepackage{xspace}
\usepackage{xypic}
\usepackage{hetcasl}
\usepackage{wrapfig}

\usepackage{graphicx}
\usepackage{proof}
\usepackage{hetcasl}

% Images
%\renewcommand{\topfraction}{0.95} 
%\renewcommand{\textfraction}{0.1} 
%\renewcommand{\floatpagefraction}{0.75} 

% Texttt curly brackets
\newcommand{\ttlcb}{\texttt{\char'173}}
\newcommand{\ttrcb}{\texttt{\char'175}}

\def\squareforqed{\hbox{\rlap{$\sqcap$}$\sqcup$}}
\def\qed{\ifmmode\squareforqed\else{\unskip\nobreak\hfil
\penalty50\hskip1em\null\nobreak\hfil\squareforqed
\parfillskip=0pt\finalhyphendemerits=0\endgraf}\fi}

\newtheorem{definition}{Definition} %[chapter]
\newtheorem{proposition}{Proposition} %[chapter]
\newtheorem{property}{Property} %[chapter]
\newtheorem{lemma}{Lemma} %[chapter]
\newtheorem{theorem}{Theorem} %[chapter]
\newenvironment{proof}{\par\addvspace{\bigskipamount}%
\noindent\textit{Proof.}\ }{\qed\par\addvspace{\bigskipamount}}

\newcommand{\stroke}{|}
\newcommand{\forget}[1]{|_{#1}}
\newcommand{\Mod}{\mathbf{Mod}}
\newcommand{\Sen}{\mathbf{Sen}}
\newcommand{\red}{\upharpoonright}

\newcommand{\codesize}{\small}
\newcommand{\field}[1]{\mathbb{#1}}
\newcommand{\vertex}[1]{\|\,#1\,\|}
\newcommand{\mt}{{\small$\blacktriangle$}}
\newcommand{\leaf}[1]{{\small$\blacktriangle\;$}#1{\small$\;\blacktriangle$}}

\newcommand{\nodo}[1]{\rule[-.5ex]{0pt}{2.4ex}\,#1\,}
%\newcommand{\lab}[1]{\!\raisebox{0.5ex}{\scriptsize\textsf{#1}}}
\newcommand{\lab}[1]{\!{\scriptsize\textsf{#1}}}
\newcommand{\labb}[2]{\!{\scriptsize\textsf{#1}_\textsf{#2}}}
\newcommand{\labapt}[2]{{\scriptstyle\;\textsf{#1}_\textsf{#2}}}

\newcommand{\model}{\mathcal{A}} % use: $\model$
\newcommand{\stepRule}[1]{\scriptstyle{ #1}}
\newcommand{\den}[1]{[\![#1]\!]_\model}  % use: $\den{e}$

\newcommand{\bigunion}{\mathop{ \mathgroup\symoperators \bigcup}}
\newcommand{\fracc}[2]{\begin{array}{c}{#1}\\ \hline {#2} \end{array}}

\newcommand{\Hets}{\textmd{\textsc{Hets}}\xspace}
\newcommand{\CASL}{\textmd{\textsc{Casl}}\xspace}
\newcommand{\CoFI}{\textmd{\textsc{CoFI}}\xspace}

\newcommand{\free}[2]{\mbox{\textbf{free} $#1$ \textbf{along} $#2$}}

\newcommand{\rewrites}{\Rightarrow}

%% macros for development graphs

\newcommand{\glinka}[3]{\!\xymatrix{{#1} \ar@{=>}[r]^{#2} & {#3}}\!\!}
\newcommand{\longglinka}[3]{\!\xymatrix{#1 \ar@{=>}[rr]^{#2} && #3}\!\!}
\newcommand{\llinka}[3]{\!\xymatrix{#1 \ar[r]^{#2} & #3}\!\!}
\newcommand{\hlinka}[3]{\!\xymatrix{#1 \ar@{=>}[r]^{#2}_{\mathit{hide}} & #3}\!\!}
\newcommand{\longhlinka}[3]{\!\xymatrix{#1 \ar@{=>}[rr]^{#2}_{\mathit{hide}} && #3}\!\!}
\newcommand{\flinka}[3]{\!\xymatrix{#1 \ar@{=>}[r]^{#2}_{\mathit{free}} & #3}\!\!}
\newcommand{\longflinka}[3]{\!\xymatrix{#1 \ar@{=>}[rr]^{#2}_{\mathit{free}} && #3}\!\!}

\newcommand{\tglinka}[3]{\!\xymatrix{#1 \ar@{==>}[r]^{#2} & #3}\!\!}
\newcommand{\longtglinka}[3]{\!\xymatrix{#1 \ar@{==>}[rr]^{#2} && #3}\!\!}
\newcommand{\thlinka}[4]{\!\xymatrix{#1 \ar@{==>}[r]^{#2}_{{\mathit{hide}}\ #3} & #4}\!\!}
\newcommand{\longthlinka}[4]{\!\xymatrix{#1 \ar@{==>}[rr]^{#2}_{{\mathit{hide}}\ #3} && #4}\!\!}
\newcommand{\tflinka}[4]{\!\xymatrix{#1 \ar@{==>}[r]^{#2}_{\mathit{free}\ #3} & #4}\!\!}
\newcommand{\tllinka}[3]{\!\xymatrix{#1 \ar@{-->}[r]^{#2} & #3}\!\!}

\newcommand{\gclinka}[3]{\!\xymatrix{#1 \ar@{=>}[r]^{#2}_{\mathit{cons}} & #3}\!\!}
\newcommand{\gdlinka}[3]{\!\xymatrix{#1 \ar@{=>}[r]^{#2}_{\mathit{def}} & #3}\!\!}
\newcommand{\gmlinka}[3]{\!\xymatrix{#1 \ar@{=>}[r]^{#2}_{\mathit{mono}} & #3}\!\!}

\newcommand{\tclinka}[3]{\!\xymatrix{#1 \ar@{==>}[r]^{#2}_{\mathit{cons}} & #3}\!\!}
\newcommand{\tdlinka}[3]{\!\xymatrix{#1 \ar@{==>}[r]^{#2}_{\mathit{def}} & #3}\!\!}
\newcommand{\tmlinka}[3]{\!\xymatrix{#1 \ar@{==>}[r]^{#2}_{\mathit{mono}} & #3}\!\!}

\newcommand{\longtclinka}[3]{\!\xymatrix{#1 \ar@{==>}[rr]^{#2}_{\mathit{cons}} && #3}\!\!}
\newcommand{\longtdlinka}[3]{\!\xymatrix{#1 \ar@{==>}[rr]^{#2}_{\mathit{def}} && #3}\!\!}
\newcommand{\longtmlinka}[3]{\!\xymatrix{#1 \ar@{==>}[rr]^{#2}_{\mathit{mono}} && #3}\!\!}

\newcommand{\greaches}[1]{\!\xymatrix{\ar@{=>>}[r]^{#1} & }\!\!}
\newcommand{\lreaches}[1]{\!\xymatrix{\ar@{>=>>}[r]^{#1} & }\!\!}

\newtheorem{fact}{Fact}

\newcommand{\mi}[1]{\mathit{#1}}

\graphicspath{{images/}}

\title{Integrating Maude into Hets%
%\thanks{Research supported by MEC Spanish project
%\emph{DESAFIOS} (TIN2006-15660-C02-01) and Comunidad de
%Madrid program \emph{PROMESAS} (S�0505/TIC/0407).}
}

\author{Mihai Codescu, Till Mossakowski,
Adri\'an Riesco, and Christian Maeder\\[.7cm]
\normalsize Technical Report 07/10\\[1ex]
  \normalsize\textit{Departamento de Sistemas Inform\'aticos y Computaci\'on}\\
  \normalsize\textit{Universidad Complutense de Madrid}\\[.4cm]
  September 2010
  }


\date{}

%\institute{}

%%%%%%%%%%%%%%%%%%%%%%%%%%%%%%%%%%%%%%%%%%%%%%%%%%%%%%%%%%
\begin{document}
%
\maketitle

\thispagestyle{empty}
\newpage
\thispagestyle{empty}
\mbox{}\vfill

\begin{abstract}
Maude modules can be understood as models that can be formally analyzed
and verified with respect to different properties expressing various
formal requirements.
However, Maude lacks the formal tools to perform some of these analyses
and thus they can only be done by hand.
The Heterogeneous Tool Set \Hets is an institution-based
combination of different logics and corresponding rewriting, model
checking, and proof tools.
%
We present in this paper an integration of Maude into \Hets that allows
to use the logics and tools already integrated in \Hets with
Maude specifications. To achieve such integration we have defined an
institution for Maude based on preordered algebras and a comorphism
between Maude and \CASL, the central logic in \Hets.

\smallskip

\noindent\textbf{Keywords:} rewriting logic, heterogeneous specifications,
Maude, \CASL
\end{abstract}

\vfill

\begin{small}
\tableofcontents
\end{small}

\vfill
\mbox{}
\newpage
%\setcounter{page}{1}

\section{Introduction}\label{sec:intro}
\input{intro}

\section{Rewriting logic and Maude}\label{sec:maude}
%!TEX root = main.tex

As mentioned in the introduction, Maude modules are executable rewriting logic specifications.
Rewriting logic \cite{Meseguer92-tcs} is a logic of change very suitable
for the specification of concurrent systems that is parameterized
by an underlying equational logic, for which Maude uses \emph{membership
equational logic} \cite{BouhoulaJouannaudMeseguer00,Meseguer97},
which, in addition to equations, allows the statement of membership
axioms characterizing the elements of a sort. In the following sections we
present both logics and how their specifications are represented as Maude modules.

\subsection{Membership equational logic} \label{mel-section}

A \emph{signature} in membership equational logic is a triple $(K,\Sigma, S)$
(just $\Sigma$ in the following),\
with $K$ a set of {\em kinds},
$\Sigma = \{\Sigma_{k_1\ldots k_n,k}\}_{(k_1\ldots k_n,k)\in K^{*}\times K}$ a
many-kinded signature, and $S =
\{S_{k}\}_{k\in K}$ a pairwise disjoint $K$-kinded family of sets of
\emph{sorts}.
The kind of a sort $s$ is denoted by $[s]$.
We write $T_{\Sigma,k}$ and $T_{\Sigma,k}(X)$ to denote respectively the set
of ground
$\Sigma$-terms with kind $k$ and of $\Sigma$-terms with kind $k$ over variables
in $X$, where $X = \{ x_1:k_1, \dots, x_n:k_n\}$ is a set of $K$-kinded
variables.
Intuitively, terms with a kind but without a sort represent undefined or error
elements.

The atomic formulas of membership equational logic are either \emph{equations}
$t = t'$, where $t$ and $t'$ are $\Sigma$-terms of the same kind, or
\emph{membership axioms} of the form $t : s$, where the term $t$
has kind $k$ and $s \in S_k$.
\emph{Sentences} are universally-quantified Horn clauses of the
form $(\forall X)\, A_0 \Leftarrow A_1 \wedge \ldots \wedge A_n$,
where each $A_i$ is  either an equation or a membership axiom, and $X$ is a
set of $K$-kinded variables containing all the variables in the $A_i$.
A \emph{specification} is a pair $(\Sigma,E)$, where $E$ is a set
of sentences in membership equational logic over the signature $\Sigma$.

Models of membership equational logic specifications are
\emph{$\Sigma$-algebras} $\model$ consisting of a set $A_k$ for each kind $k \in K$,
a function $A_f : A_{k_1}\times \dots \times A_{k_n} \longrightarrow A_k$ for each 
operator $f \in \Sigma_{k_1 \dots k_n, k}$, and
a subset $A_s \subseteq A_k$ for each sort $s\in S_k$. The meaning
$\den{t}$ of a term $t$ in an algebra $\model$ is inductively defined as usual.
Then, an algebra $\model$ satisfies an equation $t = t'$  (or the equation holds
in the algebra), denoted $\model \models t = t'$, when both terms have the same meaning:
$\den{t} = \den{t'}$. In the same way, satisfaction of a membership is defined as:
$\model \models t : s$ when $\den{t} \in A_s$.

A membership equational logic specification $(\Sigma,E)$ has an initial model 
$\mathcal{T}_{\Sigma/E}$ whose elements
are $E$-equivalence classes of terms $[t]$.
We refer to \cite{BouhoulaJouannaudMeseguer00,Meseguer97} for a detailed presentation of
$(\Sigma,E)$-algebras,
sound and complete deduction rules, as well as the construction of
initial and free algebras.

Since the membership equational logic specifications that we consider are assumed
to satisfy the executability requirements of confluence, termination, and 
sort-decreasingness, their equations $t=t'$ can be oriented from left to right,
$t \rightarrow t'$. Such a statement holds in an algebra, denoted 
$\model \models t \rightarrow t'$, exactly when $\model \models t = t'$, i.e., when
$\den{t} = \den{t'}$. Moreover, under those assumptions an equational condition $u = v$
in a conditional equation can be checked by finding a common term $t$ such
that $u \rightarrow t$ and $v \rightarrow t$. Since these are \emph{executability}
requirements they are not necessary for using \Hets; in fact, these are some of
the properties we expect to check in the near future.

\subsection{Maude functional modules} \label{maudefmod}

Maude functional modules \cite[Chapter 4]{maude-book}, introduced 
with syntax \texttt{fmod ...\ endfm}, are executable membership
equational specifications and their semantics is given by the corresponding
initial membership algebra in the class of algebras satisfying the specification.

In a functional module we can declare sorts (by means of keyword
\texttt{sort}(\texttt{s})); subsort relations between sorts
(\texttt{subsort}); operators (\texttt{op}) for building values of these
sorts, giving the sorts of their arguments and result, and which may have
attributes such as being associative (\texttt{assoc}) or commutative
(\texttt{comm}), for example; memberships (\texttt{mb}) asserting that a term
has a sort; and equations (\texttt{eq}) identifying terms.  
Both memberships and equations can be conditional (\texttt{cmb} and \texttt{ceq}).

Maude does automatic kind inference from the sorts declared by the user and
their subsort relations.  Kinds are \emph{not} declared explicitly, and
correspond to the connected components of the subsort relation.
The kind corresponding to a sort \texttt{s} is denoted \texttt{[s]}.
For example, if we have sorts \texttt{Nat} for natural numbers and \texttt{NzNat} 
for nonzero natural numbers with a subsort \texttt{NzNat < Nat}, then 
\texttt{[NzNat]} = \texttt{[Nat]}.

An operator declaration like

{\codesize
\begin{verbatim}
 op _div_ : Nat NzNat -> Nat .
\end{verbatim}
}

\noindent
is logically understood as a declaration at the kind level

{\codesize
\begin{verbatim}
 op _div_ : [Nat] [Nat] -> [Nat] .
\end{verbatim}
}

\noindent
together with the conditional membership axiom

{\codesize
\begin{verbatim}
 cmb N div M : Nat if N : Nat and M : NzNat .
\end{verbatim}
}

A subsort declaration \texttt{NzNat < Nat} is logically understood as
the conditional membership axiom

{\codesize
\begin{verbatim}
 cmb N : Nat if N : NzNat .
\end{verbatim}
}

\subsection{Rewriting logic}

Rewriting logic extends equational logic by introducing the notion of \emph{rewrites} 
corresponding to
transitions between states; that is, while equations are interpreted as equalities and therefore
they are symmetric, rewrites denote changes which can be irreversible. 

A rewriting logic specification, or \emph{rewrite theory}, has the form
$\mathcal{R} = (\Sigma,E,R)$, where $(\Sigma,E)$ is an equational specification
and $R$ is a set of \emph{rules} as described below. From this definition,
one can see that rewriting logic is built on top of equational logic, so
that rewriting logic is parameterized
with respect to the version of the underlying equational logic; in our
case, Maude uses membership equational logic, as described in the
previous sections. A rule in $R$ has the general conditional
form\footnote{There is no need for the condition listing first equations,
then  memberships, and then rewrites: this is just a notational
abbreviation, they can be listed in any order.} 
\[
(\forall X) \; t \Rightarrow t' \Leftarrow \bigwedge_{i=1}^{n} u_i = u'_i \wedge
                      \bigwedge_{j=1}^{m}  v_j : s_j \wedge
                      \bigwedge_{k=1}^{l} w_k \Rightarrow w'_k
\]
where the head is a rewrite and the conditions can be equations,
memberships, and rewrites; both sides of a rewrite must have the same kind. 
From these rewrite rules, one can deduce rewrites of the form
$t \Rightarrow t'$ by means of general deduction rules introduced
in \cite{Meseguer92-tcs} (for a generalization see also \cite{BruniMeseguer06}).

Models of rewrite theories are called \emph{$\mathcal{R}$-systems}.
Such systems are defined as categories that possess a
$(\Sigma,E)$-algebra structure, together with a natural transformation
for each rule in the set $R$. More intuitively, the idea is that we have a
$(\Sigma,E)$-algebra, as described in Section~\ref{mel-section}, with
transitions between the elements in each set $A_k$; moreover, these
transitions must satisfy several additional requirements, including that
there are identity transitions for each element, that transitions can
be sequentially composed, that the operations in the signature $\Sigma$
are also appropriately defined for the transitions, and that we have
enough transitions corresponding to the rules in $R$. Then, if we keep in
this context the notation $\model$ to denote an $\mathcal{R}$-system, a
rewrite $t \Rightarrow t'$ is satisfied by $\model$, 
denoted $\model \models t \Rightarrow t'$, when there is a transition
$\den{t} \rightarrow_\model \den{t'}$ in the system between the
corresponding meanings of both sides of the rewrite, where $\rightarrow_\model$
will be our notation for such transitions. 

The rewriting logic deduction rules introduced in \cite{Meseguer92-tcs}
are sound and complete with respect to this notion of model. Moreover,
they can be used to build initial and free models; see \cite{Meseguer92-tcs}
for details.

\subsection{Maude system modules}

Maude system modules \cite[Chapter 6]{maude-book}, introduced with
syntax \texttt{mod ...\ endm}, are executable rewrite 
theories and their semantics is given by the initial system in the class of 
systems corresponding to the rewrite theory.  A system module can contain all the
declarations of a functional module and, in addition, declarations for
rules (\texttt{rl}) and conditional rules (\texttt{crl}).

The executability requirements for equations and memberships in a system
module are the same as those of functional modules, namely, confluence,
termination, and sort-decreasingness. With respect to rules, the satisfaction
of all the conditions in a conditional rewrite rule is attempted sequentially 
from left to right, solving rewrite conditions by means of search; 
for this reason, we can have new variables in such conditions but they
must become instantiated along this process of solving from left to right
(see \cite{maude-book} for details). Furthermore, the strategy followed
by Maude in rewriting with rules is to compute the normal form of a term
with respect to the equations before applying a rule. This strategy is
guaranteed not to miss any rewrites when the rules are \emph{coherent}
with respect to the equations \cite{eq-rl-rwl,maude-book}. In a way
quite analogous to confluence, this coherence requirement means that, given
a term $t$, for each rewrite of it using a rule in $R$ to some term $t'$,
if $u$ is the normal form of $t$ with respect to the equations and
memberships in $E$, then there is a rewrite of $u$ with some rule in
$R$ to a term $u'$ such that $u' =_E t'$ (that is, the equation $t' = u'$
can be deduced from $E$).

\subsection{Advanced features}\label{subsec:adv_feat}

In addition to the modules presented thus far, we present in this section
some other Maude features that will be used throughout this paper. More
information on these topics can be found in \cite{maude-book}.

\subsubsection{Module operations}

To ease the specification of large systems, Maude provides several
mechanisms to structure its modules. We describe in this section these
procedures, that will be used later to build the development graphs.

Maude modules can import other modules in three different modes:
\begin{itemize}
\item
The \verb"protecting" mode (abbreviated as \verb"pr") indicates that \emph{no
junk and no confusion} can be added to the imported module, where junk refers to
new terms in canonical form while confusion implies that different canonical
terms in the initial module are made equal by equations in the imported module.

\item
The \verb"extending" mode (abbreviated as \verb"ex") indicates that junk is
allowed but confusion is forbidden.

\item
The \verb"including" mode (abbreviated as \verb"inc") allows both
junk and confusion.

\end{itemize}

More specifically, these importation modes do not import modules but
\emph{module expressions} that, in addition to a single module identifier,
can be:

\begin{itemize}
\item
A summation of two module expressions $\mathit{ME}_1 \,\verb"+"\,\mathit{ME}_2$,
which creates a new module that includes all the information in its summands.

\item
A renaming $\mathit{ME} \,\verb"*"\, \verb"("\mathit{Renaming}\verb")"$, where 
$\mathit{Renaming}$ is a list of renamings. They can be renaming of sorts:

$$
\verb"sort"\; \mathit{sort}_1\; \verb"to"\; \mathit{sort}_2\; \verb"."
$$

of operators, distinguishing whether it renames all the operators
with the given identifier (when the attributes are modified, only
\verb"prec", \verb"gather", and \verb"format" are allowed)

$$
\verb"op"\; \mathit{id}_1\; \verb"to"\; \mathit{id}_2\; \verb"."
$$
\vspace{-3ex}
$$
\verb"op"\; \mathit{id}_1\; \verb"to"\; \mathit{id}_2\; \verb"[" \mathit{atts} \verb"]"
\;  \verb"."
$$

\noindent or it renames the operators of the given arity:

$$
\verb"op"\; \mathit{id}_1 : \mathit{arity}\; \verb"->" \; \mathit{coarity}\;
\verb"to"\; \mathit{id}_2\; \verb"."
$$
\vspace{-3ex}
$$
\verb"op"\; \mathit{id}_1 : \mathit{arity}\; \verb"->" \; \mathit{coarity}\;
\verb"to"\; \mathit{id}_2\; \verb"[" \mathit{atts} \verb"]" \; \verb"."
$$

or of labels:

$$
\verb"label"\; \mathit{label}_1\; \verb"to"\; \mathit{label}_2\; \verb"."
$$

\end{itemize}

\subsubsection{Theories}\label{subsec:theories}

Theories are used to declare module interfaces, namely the syntactic
and semantic properties to be satisfied by the actual parameter modules
used in an instantiation. As for modules, Maude supports two different types
of theories: functional theories and system theories, with the same structure
of their module counterparts, but with a different semantics. Functional
theories  are declared with the keywords \verb"fth ... endfth", and
system theories with the keywords \verb"th ... endth". Both of them can
have sorts, subsort relationships, operators, variables, membership axioms,
and equations, and can import other theories or modules. System theories can
also have rules. Although there is no restriction on the operator attributes
that can be used in a theory, there are some subtle restrictions and
issues regarding the mapping of such operators (see Section
\ref{subsec:views}).
%
Like functional modules, functional theories are membership equational
logic theories, but they do not need to be Church-Rosser and terminating.
%and  therefore some or all of their statements may be declared with the
%\verb"nonexec" attribute and can only be executed in a controlled way.

For example, we can define a theory for some processes. First, we indicate
that a sort for processes is required:

{\codesize
\begin{verbatim}
fth PROCESS is
 pr BOOL .

 sort Process .
\end{verbatim}
}

Then, we state that two operators, one updating the processes and another
one checking whether a process has finished, have to be defined:
 
{\codesize
\begin{verbatim}
 op update : Process -> Process .
 op finished : Process -> Bool .
\end{verbatim}
}

Finally, we define an operator \verb"_<_" over processes that is required
to be irreflexive and transitive:

{\codesize
\begin{verbatim}
 vars X Y Z : Process .
 
 op _<_ : Process Process -> Bool .
 eq X < X = false [nonexec label irreflexive] .
 ceq X < Z = true if X < Y /\ Y < Z [nonexec label transitive] .
endfth
\end{verbatim}
}

\subsubsection{Views}\label{subsec:views}

We use views to specify how a particular target module or theory
satisfies a source theory. In general, there may be several ways in which 
such requirements might be satisfied by the target module or
theory; that is, there can be many different views, each specifying a
particular interpretation of the source theory in the target.
In the definition of a view we have to indicate its name, the source
theory, the target module or theory, and the mapping of each sort and
operator in the source theory. The source and target of a
view can be any module expression, with the source module expression
evaluating to a theory and the target module expression evaluating to a
module or a theory.
Each view declaration has an associated set of proof obligations, namely, for
each axiom in the source theory it should be the case that the axiom's
translation by the view holds true in the target. Since the target can
be a module interpreted initially, verifying such proof obligations may
in general require inductive proof techniques. Such proof obligations
are not discharged or checked by the system. 

The mappings allowed in views are:

\begin{itemize}

\item
Mappings between sorts:

$$
\verb"sort"\; \mathit{sort}_1\; \verb"to"\; \mathit{sort}_2\; \verb"."
$$

\item
Mappings between operators, where the user can specify the arity and coarity
of the operators to disambiguate them:

$$
\verb"op"\; \mathit{id}_1\; \verb"to"\; \mathit{id}_2\; \verb"."
$$
\vspace{-3ex}
$$
\verb"op"\; \mathit{id}_1 : \mathit{arity}\; \verb"->" \; \mathit{coarity}\;
\verb"to"\; \mathit{id}_2\; \verb"."
$$

\item
In addition to these mappings, the user can map a term $\mathit{term}_1$, that
can only be a single operator applied to variables, to any term $\mathit{term}_2$
in the target module, where the sorts of the variables in the first term have been
translated by using the sort mappings. Note that in that case the arity of the
operator in the source theory and the one in the target module can be different:

$$
\verb"op"\; \mathit{term}_1\; \verb"to term"\; \mathit{term}_2\; .
$$

\end{itemize}

Notice that we cannot map labels, and thus we cannot identify the statements in
the theory with those in the target module.

We can now create a view \verb"NatProcess" from the theory \verb"PROCESS" in the
previous section to \verb"NAT", the predefined modules for natural numbers:

{\codesize
\begin{verbatim}
 view NatProcess from PROCESS to NAT is
\end{verbatim}
}

We need a sort in \verb"NAT" to identify processes. We use \verb"Nat", the sort
for natural numbers:

{\codesize
\begin{verbatim}
  sort Process to Nat .
\end{verbatim}
}

Since we identify now processes with natural numbers, we can \verb"update" a process
by applying the successor function, which is declared as \verb"s_" in \verb"NAT":

{\codesize
\begin{verbatim}
  op update to s_ .
\end{verbatim}
}

We map the operator \verb"finished?" in a different way: we create a term
with this operator with a variable as argument, and it is mapped to a term
in the syntax of the target module. In that case we consider a process has
finished if it reaches \verb"100":

{\codesize
\begin{verbatim}
  op finished?(P:Process) to term P:Nat < 100 .
\end{verbatim}
}

Since the \verb"NAT" module already contains an operator \verb"_<_" it is not
necessary to explicitly indicate the mapping.

\subsubsection{Parameterized modules}\label{subsec:pmod}

Maude modules can be parameterized. A parameterized
system module has syntax

$$
\verb"mod M{" X_1 :: T_1 , \ldots , X_n :: T_n \verb"} is ... endm"
$$

\noindent with $n \geq 1$. Parameterized functional modules have completely
analogous syntax.

The \verb"{"$X_1 :: T_1 , \ldots , X_n :: T_n$\verb"}" part is called the
interface, where each pair $X_i :: T_i$ is a parameter, each $X_i$ is an
identifier---the parameter name or parameter label---, and each $T_i$ is
an expression that yields a theory---the parameter theory. Each parameter
name in an interface must be unique, although there is no uniqueness
restriction on the parameter theories of a module. The parameter theories
of a functional module must be functional theories.

In a parameterized module $M$, all the sorts and statement labels
coming from theories in its interface must be qualified by their names. Thus,
given a parameter $X_i :: T_i$, each sort $S$ in $T_i$ must be
qualified as $X_i\texttt{\$}S$, and each label $l$ of a statement occurring in
$T_i$ must be qualified as $X_i\texttt{\$}l$. In fact, the parameterized module
$M$ is flattened as follows. For each parameter $X_i :: T_i$, 
a renamed copy of the theory $T_i$, called $X_i :: T_i$ is included.
The renaming  maps each sort $S$ to $X_i\texttt{\$}S$, and each label $l$
of a statement occurring in $T_i$ to $X_i\texttt{\$}l$. The renaming has
no effect on importations of modules. Thus, if $T_i$ includes a theory $T'$,
when the renamed
theory $X_i :: T_i$ is created and included into $M$, the renamed
theory $X_i :: T'$ will also be created and included into $X_i :: T_i$. 
However, the renaming will have no effect on modules imported by either the
$T_i$ or $T'$; for example, if \verb"BOOL" is imported by one of these
theories, it is not renamed, but imported in the same way into $M$.
%
Moreover, sorts declared in parameterized modules can also be parameterized,
and these may duplicate, omit, or reorder parameters.

The parameters in parameterized modules are bound to the formal parameters
by \emph{instantiation}. The 
instantiation requires a view from each formal parameter to its corresponding 
actual parameter. Each such view is then used to bind the names of sorts, 
operators, etc. in the formal parameters to the corresponding sorts, operators 
(or expressions), etc. in the actual target.
The instantiation of a parameterized module must be made with views 
explicitly defined previously.

We can define a parameterized modules for multisets of the processes shown in
Section \ref{subsec:views}. This module defines the sort \verb"MSet{X}" for
multisets, which is a supersort of Process:

{\codesize
\begin{verbatim}
fmod PROCESS_MSET{X :: PROCESS} is
 sort MSet{X} .
 subsort X$Process < MSet{X} .
\end{verbatim}
}

The constructors of multisets are \verb"empty" for the empty multiset and
the juxtaposition operator \verb"__" for bigger multisets:

{\codesize
\begin{verbatim}
 op empty : -> MSet{X} [ctor] .
 op __ : MSet{X} MSet{X} -> MSet{X} [ctor assoc id: empty] .
\end{verbatim}
}

We can also use the operators declared in the view. For example, we can remove
a process from the multiset if it is finished: 

{\codesize
\begin{verbatim}
 var P : X$Process .
 var MS : MSet{X} .
 
 ceq P MS = MS if finished?(P) .
endfm
\end{verbatim}
}

We can use the view \verb"NatProcess" to instantiate this parameterized module
and create lists of processes identified as natural numbers.

{\codesize
\begin{verbatim}
fmod NAT_PROCSES_MSET is
 pr PROCESS_MSET{NatProcess} .
endfm
\end{verbatim}
}












\section{\Hets}\label{sec:hets}
\setlength{\unitlength}{1740sp}%
%
\begingroup\makeatletter\ifx\SetFigFont\undefined%
\gdef\SetFigFont#1#2#3#4#5{%
  \reset@font\fontsize{#1}{#2pt}%
  \fontfamily{#3}\fontseries{#4}\fontshape{#5}%
  \selectfont}%
\fi\endgroup%
\begin{picture}(12534,8879)(439,-8907)
\thinlines
{\color[rgb]{0,0,0}\put(2458,-5596){\vector( 0,-1){630}}
}%
{\color[rgb]{0,0,0}\put(2458,-4799){\vector( 0,-1){270}}
}%
{\color[rgb]{0,0,0}\put(2458,-4269){\line( 0,-1){260}}
}%
{\color[rgb]{0,0,0}\put(2458,-3645){\vector( 0,-1){270}}
}%
{\color[rgb]{0,0,0}\put(2458,-3115){\line( 0,-1){260}}
}%
{\color[rgb]{0,0,0}\put(2458,-2491){\vector( 0,-1){270}}
}%
{\color[rgb]{0,0,0}\put(2458,-1961){\line( 0,-1){260}}
}%
{\color[rgb]{0,0,0}\put(10773,-5551){\vector( 0,-1){900}}
}%
{\color[rgb]{0,0,0}\put(10773,-4799){\vector( 0,-1){270}}
}%
{\color[rgb]{0,0,0}\put(10773,-4269){\line( 0,-1){260}}
}%
{\color[rgb]{0,0,0}\put(10773,-3645){\vector( 0,-1){270}}
}%
{\color[rgb]{0,0,0}\put(10773,-3115){\line( 0,-1){260}}
}%
{\color[rgb]{0,0,0}\put(10773,-2491){\vector( 0,-1){270}}
}%
{\color[rgb]{0,0,0}\put(10773,-1961){\line( 0,-1){260}}
}%
{\color[rgb]{0,0,0}\put(6436,-3391){\line(-2, 3){595.385}}
}%
{\color[rgb]{0,0,0}\put(6571,-3391){\line( 0, 1){1530}}
}%
{\color[rgb]{0,0,0}\put(6706,-3391){\line( 2, 3){595.385}}
}%
{\color[rgb]{0,0,0}\put(7039,-3523){\line( 3, 2){488.077}}
}%
{\color[rgb]{0,0,0}\put(6226,-3548){\line(-3, 2){488.077}}
}%
{\color[rgb]{0,0,0}\put(5791,-4046){\line( 3, 2){405}}
}%
{\color[rgb]{0,0,0}\put(7253,-4056){\line(-3, 2){405}}
}%
{\color[rgb]{0,0,0}\put(6795,-4557){\line( 5, 2){411.207}}
}%
{\color[rgb]{0,0,0}\put(6283,-4558){\line(-5, 2){411.207}}
}%
{\color[rgb]{0,0,0}\put(5691,-5146){\line( 3, 2){405}}
}%
{\color[rgb]{0,0,0}\put(7403,-5146){\line(-3, 2){405}}
}%
{\color[rgb]{0,0,0}\put(6571,-5056){\line( 0, 1){180}}
}%
{\color[rgb]{0,0,0}\put(10773,-2755){\oval(3785,5632)}
%\put(9106,-164){\oval(210,210)[tl]}
%\put(12541,-5446){\oval(210,210)[br]}
%\put(12541,-164){\oval(210,210)[tr]}
%\put(9106,-5551){\line( 1, 0){3435}}
%\put(9106,-59){\line( 1, 0){3435}}
%\put(9001,-5446){\line( 0, 1){5282}}
%\put(12646,-5446){\line( 0, 1){5282}}
}%
{\color[rgb]{0,0,0}\put(2458,-2755){\oval(3785,5632)}
%{\color[rgb]{0,0,0}\put(781,-5491){\oval(210,210)[bl]}
%\put(781,-145){\oval(210,210)[tl]}
%\put(4306,-5491){\oval(210,210)[br]}
%\put(4306,-145){\oval(210,210)[tr]}
%\put(781,-5596){\line( 1, 0){3525}}
%\put(781,-40){\line( 1, 0){3525}}
%\put(676,-5491){\line( 0, 1){5346}}
%\put(4411,-5491){\line( 0, 1){5346}}
}%
{\color[rgb]{0,0,0}\put(6623,-2755){\oval(3785,5632)}
%{\color[rgb]{0,0,0}\put(4966,-5491){\oval(210,210)[bl]}
%\put(4966,-166){\oval(210,210)[tl]}
%\put(8430,-5491){\oval(210,210)[br]}
%\put(8430,-166){\oval(210,210)[tr]}
%\put(4966,-5596){\line( 1, 0){3464}}
%\put(4966,-61){\line( 1, 0){3464}}
%\put(4861,-5491){\line( 0, 1){5325}}
%\put(8535,-5491){\line( 0, 1){5325}}
}%
{\color[rgb]{0,0,0}\put(4361,-2851){\line( 1, 0){350}}
}%
{\color[rgb]{0,0,0}\put(8521,-2851){\vector( 1, 0){350}}
}%
{\color[rgb]{0,0,0}\put(751,-7276){\oval(330,330)[bl]}
\put(751,-6526){\oval(330,330)[tl]}
\put(2941,-7276){\oval(330,330)[br]}
\put(2941,-6526){\oval(330,330)[tr]}
\put(751,-7441){\line( 1, 0){2190}}
\put(751,-6361){\line( 1, 0){2190}}
\put(586,-7276){\line( 0, 1){750}}
\put(3106,-7276){\line( 0, 1){750}}
}%
{\color[rgb]{0,0,0}\put(586,-8760){\oval(270,270)[bl]}
\put(586,-6361){\oval(270,270)[tl]}
\put(5851,-8760){\oval(270,270)[br]}
\put(5851,-6361){\oval(270,270)[tr]}
\put(586,-8895){\line( 1, 0){5265}}
\put(586,-6226){\line( 1, 0){5265}}
\put(451,-8760){\line( 0, 1){2399}}
\put(5986,-8760){\line( 0, 1){2399}}
}%
{\color[rgb]{0,0,0}\put(1726,-8551){\oval(300,300)[bl]}
\put(1726,-7906){\oval(300,300)[tl]}
\put(4411,-8551){\oval(300,300)[br]}
\put(4411,-7906){\oval(300,300)[tr]}
\put(1726,-8701){\line( 1, 0){2685}}
\put(1726,-7756){\line( 1, 0){2685}}
\put(1576,-8551){\line( 0, 1){645}}
\put(4561,-8551){\line( 0, 1){645}}
}%
{\color[rgb]{0,0,0}\put(8537,-8101){\oval(300,300)[bl]}
\put(8537,-6646){\oval(300,300)[tl]}
\put(12811,-8101){\oval(300,300)[br]}
\put(12811,-6646){\oval(300,300)[tr]}
\put(8537,-8251){\line( 1, 0){4274}}
\put(8537,-6496){\line( 1, 0){4274}}
\put(8387,-8101){\line( 0, 1){1455}}
\put(12961,-8101){\line( 0, 1){1455}}
}%
{\color[rgb]{0,0,0}\put(3481,-7290){\oval(300,300)[bl]}
\put(3481,-6531){\oval(300,300)[tl]}
\put(5656,-7290){\oval(300,300)[br]}
\put(5656,-6531){\oval(300,300)[tr]}
\put(3481,-7440){\line( 1, 0){2175}}
\put(3481,-6381){\line( 1, 0){2175}}
\put(3331,-7290){\line( 0, 1){759}}
\put(5806,-7290){\line( 0, 1){759}}
}%
{\color[rgb]{0,0,0}\put(5986,-7441){\vector( 1, 0){2385}}
}%
\put(2135,-1861){\makebox(0,0)[lb]{\smash{\SetFigFont{7}{8.4}{\sfdefault}{\mddefault}{\updefault}{\color[rgb]{0,0,0}Text}%
}}}
\put(2106,-2438){\makebox(0,0)[lb]{\smash{\SetFigFont{7}{8.4}{\sfdefault}{\mddefault}{\updefault}{\color[rgb]{0,0,0}\emph{Parser}}%
}}}
\put(1441,-3015){\makebox(0,0)[lb]{\smash{\SetFigFont{7}{8.4}{\sfdefault}{\mddefault}{\updefault}{\color[rgb]{0,0,0}Abstract syntax}%
}}}
\put(1616,-3592){\makebox(0,0)[lb]{\smash{\SetFigFont{7}{8.4}{\sfdefault}{\mddefault}{\updefault}{\color[rgb]{0,0,0}\emph{Static analysis}}%
}}}
\put(1131,-4169){\makebox(0,0)[lb]{\smash{\SetFigFont{7}{8.4}{\sfdefault}{\mddefault}{\updefault}{\color[rgb]{0,0,0}(Signature, Sentences)}%
}}}
\put(1896,-4746){\makebox(0,0)[lb]{\smash{\SetFigFont{7}{8.4}{\sfdefault}{\mddefault}{\updefault}{\color[rgb]{0,0,0}\emph{Interfaces}}%
}}}
\put(1641,-5323){\makebox(0,0)[lb]{\smash{\SetFigFont{7}{8.4}{\sfdefault}{\mddefault}{\updefault}{\color[rgb]{0,0,0}XML, ATerms}%
}}}
\put(6301,-3706){\makebox(0,0)[lb]{\smash{\SetFigFont{7}{8.4}{\sfdefault}{\mddefault}{\updefault}{\color[rgb]{0,0,0}CASL}%
}}}
\put(5356,-2356){\makebox(0,0)[lb]{\smash{\SetFigFont{7}{8.4}{\sfdefault}{\mddefault}{\updefault}{\color[rgb]{0,0,0}CoCASL}%
}}}
\put(6976,-2356){\makebox(0,0)[lb]{\smash{\SetFigFont{7}{8.4}{\sfdefault}{\mddefault}{\updefault}{\color[rgb]{0,0,0}CASL-LTL}%
}}}
\put(6121,-1726){\makebox(0,0)[lb]{\smash{\SetFigFont{7}{8.4}{\sfdefault}{\mddefault}{\updefault}{\color[rgb]{0,0,0}CSP-CASL}%
}}}
\put(7201,-3121){\makebox(0,0)[lb]{\smash{\SetFigFont{7}{8.4}{\sfdefault}{\mddefault}{\updefault}{\color[rgb]{0,0,0}SB-CASL}%
}}}
\put(5041,-3166){\makebox(0,0)[lb]{\smash{\SetFigFont{7}{8.4}{\sfdefault}{\mddefault}{\updefault}{\color[rgb]{0,0,0}HasCASL}%
}}}
\put(5086,-4336){\makebox(0,0)[lb]{\smash{\SetFigFont{7}{8.4}{\sfdefault}{\mddefault}{\updefault}{\color[rgb]{0,0,0}SubFOL$^=$}%
}}}
\put(7246,-4336){\makebox(0,0)[lb]{\smash{\SetFigFont{7}{8.4}{\sfdefault}{\mddefault}{\updefault}{\color[rgb]{0,0,0}PFOL$^=$}%
}}}
\put(6346,-4786){\makebox(0,0)[lb]{\smash{\SetFigFont{7}{8.4}{\sfdefault}{\mddefault}{\updefault}{\color[rgb]{0,0,0}FOL$^=$}%
}}}
\put(6346,-5371){\makebox(0,0)[lb]{\smash{\SetFigFont{7}{8.4}{\sfdefault}{\mddefault}{\updefault}{\color[rgb]{0,0,0}Horn$^=$}%
}}}
\put(5400,-5371){\makebox(0,0)[lb]{\smash{\SetFigFont{7}{8.4}{\sfdefault}{\mddefault}{\updefault}{\color[rgb]{0,0,0}$\bullet$}%
}}}
\put(7600,-5371){\makebox(0,0)[lb]{\smash{\SetFigFont{7}{8.4}{\sfdefault}{\mddefault}{\updefault}{\color[rgb]{0,0,0}$\bullet$}%
}}}
\put(1006,-421){\makebox(0,0)[lb]{\smash{\SetFigFont{7}{8.4}{\rmdefault}{\bfdefault}{\updefault}{\color[rgb]{0,0,0}Basic specifications}%
}}}
\put(826,-826){\makebox(0,0)[lb]{\smash{\SetFigFont{7}{8.4}{\rmdefault}{\bfdefault}{\updefault}{\color[rgb]{0,0,0}(logic-specific tools for}%
}}}
\put(826,-1231){\makebox(0,0)[lb]{\smash{\SetFigFont{7}{8.4}{\rmdefault}{\bfdefault}{\updefault}{\color[rgb]{0,0,0}CASL and extensions)}%
}}}
\put(5411,-421){\makebox(0,0)[lb]{\smash{\SetFigFont{7}{8.4}{\rmdefault}{\bfdefault}{\updefault}{\color[rgb]{0,0,0}Graph of CASL}%
}}}
\put(9591,-421){\makebox(0,0)[lb]{\smash{\SetFigFont{7}{8.4}{\rmdefault}{\bfdefault}{\updefault}{\color[rgb]{0,0,0}Structured and}%
}}}
\put(9771,-826){\makebox(0,0)[lb]{\smash{\SetFigFont{7}{8.4}{\rmdefault}{\bfdefault}{\updefault}{\color[rgb]{0,0,0}architectural}%
}}}
\put(9796,-1231){\makebox(0,0)[lb]{\smash{\SetFigFont{7}{8.4}{\rmdefault}{\bfdefault}{\updefault}{\color[rgb]{0,0,0}specifications}%
}}}
\put(10501,-1861){\makebox(0,0)[lb]{\smash{\SetFigFont{7}{8.4}{\sfdefault}{\mddefault}{\updefault}{\color[rgb]{0,0,0}Text}%
}}}
\put(10441,-2428){\makebox(0,0)[lb]{\smash{\SetFigFont{7}{8.4}{\sfdefault}{\mddefault}{\updefault}{\color[rgb]{0,0,0}\emph{Parser}}%
}}}
\put(9886,-3015){\makebox(0,0)[lb]{\smash{\SetFigFont{7}{8.4}{\sfdefault}{\mddefault}{\updefault}{\color[rgb]{0,0,0}Abstract syntax}%
}}}
\put(9936,-3592){\makebox(0,0)[lb]{\smash{\SetFigFont{7}{8.4}{\sfdefault}{\mddefault}{\updefault}{\color[rgb]{0,0,0}\emph{Static analysis}}%
}}}
\put(9641,-4169){\makebox(0,0)[lb]{\smash{\SetFigFont{7}{8.4}{\sfdefault}{\mddefault}{\updefault}{\color[rgb]{0,0,0}Development graph}%
}}}
\put(10216,-4746){\makebox(0,0)[lb]{\smash{\SetFigFont{7}{8.4}{\sfdefault}{\mddefault}{\updefault}{\color[rgb]{0,0,0}\emph{Interfaces}}%
}}}
\put(9946,-5323){\makebox(0,0)[lb]{\smash{\SetFigFont{7}{8.4}{\sfdefault}{\mddefault}{\updefault}{\color[rgb]{0,0,0}XML, ATerms}%
}}}
\put(2476,-8476){\makebox(0,0)[lb]{\smash{\SetFigFont{7}{8.4}{\sfdefault}{\mddefault}{\updefault}{\color[rgb]{0,0,0}(e.g. CCC)}%
}}}
\put(1936,-8161){\makebox(0,0)[lb]{\smash{\SetFigFont{7}{8.4}{\sfdefault}{\mddefault}{\updefault}{\color[rgb]{0,0,0}Consistency checker}%
}}}
\put(846,-7171){\makebox(0,0)[lb]{\smash{\SetFigFont{7}{8.4}{\sfdefault}{\mddefault}{\updefault}{\color[rgb]{0,0,0}(e.g. HOL-CASL)}%
}}}
\put(936,-6766){\makebox(0,0)[lb]{\smash{\SetFigFont{7}{8.4}{\sfdefault}{\mddefault}{\updefault}{\color[rgb]{0,0,0}Theorem prover}%
}}}
\put(8806,-7846){\makebox(0,0)[lb]{\smash{\SetFigFont{7}{8.4}{\sfdefault}{\mddefault}{\updefault}{\color[rgb]{0,0,0}Management of proofs \& change}%
}}}
\put(9091,-7351){\makebox(0,0)[lb]{\smash{\SetFigFont{7}{8.4}{\sfdefault}{\mddefault}{\updefault}{\color[rgb]{0,0,0}Heterogeneous proof engine}%
}}}
\put(10206,-6946){\makebox(0,0)[lb]{\smash{\SetFigFont{7}{8.4}{\rmdefault}{\bfdefault}{\updefault}{\color[rgb]{0,0,0}MAYA}%
}}}
\put(3501,-7216){\makebox(0,0)[lb]{\smash{\SetFigFont{7}{8.4}{\sfdefault}{\mddefault}{\updefault}{\color[rgb]{0,0,0}(e.g. ELAN-CASL)}%
}}}
\put(4051,-6766){\makebox(0,0)[lb]{\smash{\SetFigFont{7}{8.4}{\sfdefault}{\mddefault}{\updefault}{\color[rgb]{0,0,0}Rewriter}%
}}}
\put(5106,-1231){\makebox(0,0)[lb]{\smash{\SetFigFont{7}{8.4}{\rmdefault}{\bfdefault}{\updefault}{\color[rgb]{0,0,0}proposed extensions}%
}}}
\put(5280,-826){\makebox(0,0)[lb]{\smash{\SetFigFont{7}{8.4}{\rmdefault}{\bfdefault}{\updefault}{\color[rgb]{0,0,0}sublanguages and}%
}}}
\end{picture}


\section{Relating the Maude and \CASL logics}\label{sec:comoprh}
\input{inst}

\section{Building development graphs}\label{sec:dg}
%!TEX root = main.tex

We describe in this section how Maude structuring mechanisms
described in Section \ref{sec:maude}
are translated into development graphs. Then, we explain how these development
graphs are normalized to deal with freeness constraints.

\subsection{Creating the development graph}
We describe here how Maude modules, theories, and views are translated into
development graphs, illustrating it with an example.

\subsubsection{Modules}

Each Maude module generates two nodes in the development
graph. The first one contains the theory equipped with the usual
loose semantics. The second one, linked
to the first one with a free definition link (whose signature morphism
is detailed in Section \ref{sec:free}), contains the same signature but
no local axioms and stands for the free models of the theory.
Note that Maude theories only generate one node, since their initial
semantics is not used by Maude specifications.

The model class of parameterized modules
consists of free extensions of the models of their parameters, that are
persistent on sorts, but not on kinds. This notion of freeness has been 
studied in \cite{BouhoulaJouannaudMeseguer00} under assumptions like existence of top sorts for kinds
and sorted variables in formulas; our results hold under similar
hypotheses. We use non-persistent free links to link these modules with
their corresponding theories.

\subsubsection{Module expressions}\label{subsec:me}

Maude module expressions allow to combine and modify the information
contained in Maude modules:

\begin{itemize}

\item
When the module expression is a simple identifier the development
graph remains unchanged.

\item
The summation of the module expressions $\mathit{ME}_1$ and
$\mathit{ME}_2$ generates a new node in the development graph
$(\mathit{ME}_1 + \mathit{ME}_2)$ with
the union of the information in both summands. A definition link
is also created between the original expressions and the resulting one.

\item
The renaming expression $\mathit{ME} * (R)$ creates a morphism with
the information given in $R$ that will be used to label the link between
the node standing for the module expression and the node importing it.

\end{itemize}

\subsubsection{Importations}

As explained above, each Maude module generates two nodes in the development
graph;
when importing a module, we will select between these nodes depending on the
chosen importation mode:
\begin{itemize}

\item
The \verb"protecting" mode generates a non-persistent free link between
the current node and the node standing for the free semantics of the
included one. We use the same links for the parameters in
parameterized modules.

\item
The \verb"extending" mode generates a global link with the annotation
\textsf{PCons?}, that stands for proof-theoretic conservativity and that
can be checked with a special conservativity checker that is
integrated into \Hets.

\item
The \verb"including" mode generates a global definition link between the
current node  and the node standing for the loose semantics of the
included one.
\end{itemize}

\subsubsection{Views}\label{subsec:dg_views}

Maude views have a theory as source and either a module or a theory
as target. All the sorts and the operators declared in the source theory
have to be mapped to sorts and operators in the target.

As seen in Section \ref{subsec:views}, a particular case of mapping
between operators is the mapping between
terms, that has the general form $\verb"op" \; e \; \verb"to term" \; t$.
%where $e$ is a term consisting of a 
%single operator applied to variables declared either on-the-fly or with
%variable declarations in the same view and the target term is any term
%with variables, those in the source $e$ in the corresponding sorts
%resulting from the mapping.
Since this shortcut allows to map operators with different profiles,
in these cases it generates an auxiliary node with the signature of the
target specification extended by an extra operator of the appropriate arity;
this node will be used as new target.

Views generate a theorem
link between the theory and the module satisfying it.
Note that an instantiation generates some implicit morphisms and modifies
the ones stated in the views, see Section \ref{subsec:adv_feat} for details:
\begin{itemize}

\item
Sorts and labels are qualified by the parameter name in order to distinguish
different labels/sorts with the same name defined in different theories. Thus,
the mapping indicated by the view (more specifically, the source sorts) is
modified depending on the name of the parameter.

\item
As explained in the Section \ref{subsec:pmod}, parameterized modules can
define parameterized sorts, that is, sorts that use the parameters as part of
the sort name and hence they are modified by the mapping in the view.
Moreover, when the target of a view is a theory the identifiers of these sorts
are extended with the name of the view and the name of the new parameter.
%
Thus, the sort morphism is extended with these new renamings.

\end{itemize}


%%%%%% PAPER

\subsubsection{Development graph: An example\label{subsubsec:dg_ex}}

We illustrate how to build the development graph with an example. Consider
the following Maude specification:

{\codesize
\begin{verbatim}
fmod M1 is                                         fmod M2 is
 sort S1 .                                           sort S2 .
 op _+_ : S1 S1 -> S1 [comm] .                     endfm
endfm

th T is                                            mod M3{X :: T} is
 sort S1 .                                          sort S4 .
 op _._ : S1 S1 -> S1 .                            endm
 eq V1:S1 . V2:S1 = V2:S1 . V1:S1 [nonexec] . 
endth

mod M is                                           view V from T to M is
 ex M1 + M2 * (sort S2 to S) .                      op _._ to _+_ .
endm                                               endv
\end{verbatim}
}

\begin{figure}[t]
\begin{center}
\includegraphics[scale=.47]{dg}
\caption{Development Graph for Maude Specifications\label{fig:dg}}
\end{center}
\end{figure}

\noindent \Hets builds the graph shown in Figure \ref{fig:dg},
where the following steps take place:
\begin{itemize}
\item
Each module has generated a node with its name and
another primed one that contains the initial model, while both of them
are linked with a non-persistent free link (in blue in the illustration). Note that
theory \verb"T" did not generate this primed node.

\item
The summation expression has created a new node that includes the theories
of \verb"M1" and \verb"M2", importing the latter with a renaming; this new
node, since it is imported in \verb"extending" mode, uses a link with the
\textsf{PCons?} annotation.

\item
There is a theorem link (red link in the figure) between \verb"T" and the
free (here, initial) model of
\verb"M". This link is labeled with the mapping defined in the view \verb"V",
namely \verb"op _._ to _+_ .".

\item
The parameterized module \verb"M3" includes the theory of its parameter
with a renaming, that qualifies the sort. Note that these nodes are connected
by means of a non-persistent free link.
\end{itemize}

It is straightforward to show:
\begin{theorem}
The translation of Maude modules into development graphs is
semantics-preserving.
\end{theorem}

Once the development graph is built, we can apply the (logic
independent) calculus rules that reduce global theorem links to local
theorem links, which are in turn discharged by local theorem proving
\cite{MAH-05-a}.  This can be used to prove Maude views, like e.g.\
``natural numbers are a total order.'' For example, we could prove
the view \verb"V" above


We show in the next
section how we deal with the freeness constraints imposed by free 
definition links.

\subsection{Normalization of free definition links}
\label{sec:free}

Maude uses initial and free semantics intensively. The semantics of
freeness is, as mentioned, different from the one used in \CASL in
that the free extensions of models are required to be persistent
only on sorts and new error elements can be added on the
interpretation of kinds. Attempts to design the translation to \CASL
in such a way that Maude free links would be translated to usual free
definition links in \CASL have been unsuccessful. We decided thus to
introduce a special type of links to represent Maude's freeness in
\CASL.  In order not to break the development graph calculus, we need
a way to normalize them. The idea is to replace them with a
semantically equivalent development graph in \CASL. The main idea is
to make a free extension persistent by duplicating parameter sorts
appropriately, such that the parameter is always explicitly included
in the free extension.

For any Maude signature $\Sigma$, let us define
an extension $\Sigma^\# = (S^\#, \leq^\#, F^\#, P^\#)$ of the
translation $\Phi(\Sigma)$ of $\Sigma$ to \CASL as follows:

\begin{itemize}

 \item $S^\#$ unites the sorts of $\Phi(\Sigma)$ and the set 
       $\{[s] \mid s \in \mi{Sorts}(\Sigma)\}$;

 \item $\leq^\#$ extends the subsort relation $\leq$ with pairs
       $(s, [s])$ for each sort $s$ and $([s],[s'])$ for any sorts $s \leq s'$;

 \item $F^\#$ adds the function symbols $\{f:[w] \rightarrow [s]\}$ for all
       function symbols on sorts $f:w \rightarrow s $;\footnote{$[x_1 \ldots x_n]$
       is defined to be $[x_1] \ldots [x_n]$.} and
 \item $P^\#$ adds the predicate symbol $rew$ on all new sorts.
\end{itemize}

Now, we consider a Maude non-persistent free definition link and let
$\sigma: \Sigma \rightarrow \Sigma'$ be the morphism labeling it.%
\footnote{In Maude, this would usually be an injective renaming.}
We define a \CASL signature morphism 
 $\sigma^\# : \Phi(\Sigma) \rightarrow \Sigma'^\#$: on sorts, 
 $\sigma^\#(s) := \sigma^{sort}(s)$ and $\sigma^\#([s]):=[\sigma^{sort}(s)]$;
 on operation symbols, we can define $\sigma^ \#(f) := 
 \sigma^{op}(f)$ and this is correct because the operation symbols were
 introduced in $\Sigma'^\#$; $rew$ is mapped identically.


The normalization of Maude freeness is then illustrated in Figure \ref{nf}.
Given a free non-persistent definition link $\flinka{M}{\sigma}{N}$, with 
$\sigma:\Sigma\rightarrow \Sigma_N$, we first take the translation of the nodes
to \CASL (nodes $M'$ and $N'$) and then introduce a new node, $K$, labeled with 
$\Sigma^\#_N$, a global definition link from $M'$ to $M''$ labeled with the 
inclusion $\iota_N$ of $\Sigma_N$ in $\Sigma^\#_N$, a free definition link from 
$M''$ to $K$ labeled with 
$\sigma^\#$ and a hiding definition link from $K$ to $N'$ labeled with the
inclusion $\iota_N$.\footnote{The arrows without labels in Figure \ref{nf} 
correspond to heterogeneous links from Maude to \CASL.}

\begin{figure}
$$
\xymatrix{
M \ar@{=>}[rr]_{n.p.\mathit{free}}^{\sigma} \ar@{=>}[d]& & N \ar@{=>}[dd]\\
M'\ar@{=>}[d]^{\iota_N}& & \\
M''\ar@{=>}[r]_{\mathit{free}}^{\sigma^\#} & K \ar@{=>}[r]_{\mathit{hide}}^{\iota_n} & N'
}
$$
\caption{Normalization of Maude free links}\label{nf}
\end{figure}

Notice that the models of $N$ are Maude reducts of \CASL models of $K$, 
reduced along the inclusion $\iota_N$. 

The next step is to eliminate \CASL free definition links. 
The idea is to use then a transformation specific to the second-order 
extension of \CASL to normalize freeness. 
The intuition behind this construction is that
it mimics the quotient term algebra construction,
that is, the free model is specified as the homomorphic image
of an absolutely free model (i.e.\ term model).

We are going to make use of the following known facts \cite{Reichel87}:

\begin{fact}

Extensions of theories in Horn form admit free extensions of models.

\end{fact}

\begin{fact}

Extensions of theories in Horn form are monomorphic.

\end{fact}

Given a free definition link $\flinka{M}{\sigma}{N}$, with 
$\sigma:\Sigma\rightarrow \Sigma^N$ such that $\mathit{Th}(M)$ is in Horn 
form, replace it with 
\xymatrix{
M \ar@{=>}[r]^{\mathit{incl}} &
K \ar@{=>}[r]^{\mathit{incl}}_{\mathit{hide}} &
N'
}, where $N'$ has the same signature as $N$, $\mathit{incl}$ denotes inclusions and 
the node $K$ is constructed as follows. 

The signature $\Sigma^K$ consists of the signature $\Sigma^M$ disjointly 
united with a copy of $\Sigma^M$, denoted $\iota(\Sigma_M)$ which makes all function symbols total 
(let us denote $\iota(x)$ the corresponding symbol in this copy for each
symbol $x$ from the signature $\Sigma^M$) and augmented with new operations 
$h: \iota(s) \rightarrow? \, s$, for any sort $s$ of $\Sigma^M$
%(and $\to?$ indicating it is a partial function)
and $\mathit{make}_s:s\rightarrow \iota(s)$, for any sort $s$ of the source
signature $\Sigma$ of the morphism $\sigma$  labelling the free definition link.

The axioms $\psi^K$ of the node $K$ consist of:

\begin{itemize}

\item
sentences imposing the bijectivity of \textit{make};

\item axiomatization of the sorts in $\iota(\Sigma_M)$ as free types
with all operations as constructors, including $\mathit{make}$ for the sorts
in $\iota(\Sigma)$;

\item homomorphism conditions for $h$:
 $$ h(\iota(f)(x_1, \dots, x_n)) = f(h(x_1), \dots, h(x_n)) $$
 
and

$$\iota(p)(t_1, \dots, t_n) \Rightarrow p(h(t_1), \dots, h(t_n))$$

\item surjectivity of homomorphisms:

$$\forall y : s . \exists x:\iota(s) . h(x) \EEQ y$$

\item a second-order formula saying that the kernel of $h$ ($\mathit{ker}(h)$)
is the least partial predicative congruence\footnote
{A \emph{partial predicative congruence} consists of a symmetric 
and transitive binary relation for each sort and a relation
of appropriate type for each predicate symbol.} satisfying
$Th(M)$. This is done by quantifying over a predicate symbol for each sort
for the binary relation and one predicate symbol for each relation symbol
as follows: 

$$\begin{array}{l}
 \forall \{P_s : \iota(s),  \iota(s)\}_{s \in Sorts(\Sigma_M)} ,
           \{P_{p:w} : \iota(w)\} _{p:w \in \Sigma_M} \\
 .~ 
\mathit{symmetry}
 \land \mathit{transitivity}
 \land  \mathit{congruence}
\land \mathit{satThM}
\implies \mathit{largerThenKerH}
\end{array}
$$

where $\mathit{symmetry}$ stands for
$$\bigwedge_{s\in Sorts(\Sigma^M)} \forall x:{\iota(s)},y:{\iota(s)}.P_s(x,y)\implies P_s(y, x),$$
$\mathit{transitivity}$ stands for
$$\bigwedge_{s\in Sorts(\Sigma^M)} \forall x:{\iota(s)},y:{\iota(s)},z:{\iota(s)}.P_s(x, y)\land P_s(y, z)\implies P_s(x, z),$$
$\mathit{congruence}$ is the conjunction of
$$
\begin{array}{l}
\bigwedge_{f_{w\rightarrow s}\in\Sigma^M} \forall x_1\ldots x_n:{\iota(w)},y_1\ldots y_n:{\iota(w)}\,.\,\,\\
D(\iota(f_{w,s})(\bar{x}))\land D(\iota(f_{w,s})(\bar{y}))\land P_w(\bar{x},\bar{y})
\implies P_s(\iota(f_{w,s})(\bar{x}),\iota(f_{w,s})(\bar{y}))
\end{array}
$$ 

and

$$
\begin{array}{l}
\bigwedge_{p_w \in\Sigma^M} \forall x_1\ldots x_n:{\iota(w)},y_1\ldots y_n:{\iota(w)}\,.\,\,\\
D(\iota(f_{w,s})(\bar{x}))\land D(\iota(f_{w,s})(\bar{y}))\land P_w(\bar{x},\bar{y})
\implies P_{p:w}(\bar{x}) \Leftrightarrow P_{p:w}(\bar{y})
\end{array}
$$
\noindent where $D$ indicates definedness. $\mathit{satThM}$ stands for
$$Th(M)[\EEQ/P_s; p:w/P_{p:w}; D(t)/P_s(t,t); t=u/P_s(t,u)\lor(\neg P_s(t,t)\land\neg P_s(u,u))]$$
where, for a set of formulas $\Psi$, $\Psi[sy_1/sy'_1;\ldots ;sy_n/sy'_n]$
denotes the simultaneous substitution of $sy'_i$ for $sy_i$ in
all formulas of $\Psi$ (while possibly instantiating the meta-variables
$t$ and $u$).
%
Finally $\mathit{largerThenKerH}$ stands for
$$\begin{array}{l}
\bigwedge_{s\in Sorts(\Sigma^M)} \forall x:{\iota(s)},y:{\iota(s)}.h(x)\EEQ h(y)\implies P_s(x, y)\\
\bigwedge \land_{p_w\in\Sigma^M} \forall \bar{x}:{\iota(w)}.\iota(p:w)(\bar{x})
\implies P_{p:w}(\bar{x})
\end{array}
$$


\end{itemize}

\begin{proposition}

The models of the nodes $N$ and $N'$ are
the same.

\end{proposition}%\ednote{AR: Proof in appendix?}

{\noindent\it Proof.}
%
Let $n$ be a $N$-model. To prove that $n$ is also a $N'$-model,
we need to show that it has a $K$-expansion.

Let us define the following $\Sigma_K$ model, denoted $k$:

\begin{itemize}

\item on $\Sigma_M$, $k$ coincides with $n$;

\item on $\iota(\Sigma_M)$, the interpretation of sorts and function symbols 
is given by the free types axioms (i.e. sorts are interpreted as set of terms, 
operations $\iota(f)$ map terms $t_1, \ldots, t_n$ to the term
$\iota(f)(t_1, \ldots, t_n)$).  We define interpretation of predicates after defining $h$;

\item $make$ assigns to each $x$ the term $make(x)$;

\item the homomorphism $h$ is defined inductively as follows:

 \begin{itemize}

  \item $h(\mathit{make}(x)) = x$, if $x \in n_s$ and $s\in \mathit{Sorts}(\Sigma)$;

 \item $h(\mathit{make}(t)) = h(t)$, otherwise;

 \item $h(\iota(f)(t_1, \dots, t_n))$ is defined iff $f(h(t_1), \ldots, h(t_n))$
       is defined in $n$
       and then  $h(\iota(f)(t_1, \ldots, t_n)) = f(h(t_1), \ldots, h(t_n))$;
 \end{itemize}

\item for predicates in $\iota(\Sigma_M)$ we define
         $\iota(p)(t_1, \ldots, t_n)$ iff $p(h(t_1), \ldots, h(t_n))$.

\end{itemize}

Notice that the first three types of axioms of the node $K$ hold by construction and also
notice that $ker(h)$ satisfies $Th(M)$ because $n$ is a $M$-model.
The surjectivity of $h$ and the minimality of $ker(h)$ are exactly the
``no junk'' and the ``no confusion'' properties of the free model $n$.  

For the other inclusion, let $n'$ be a model of $N'$, $n_0$ be its $\Sigma$-reduct and
$k'$ a $K$-expansion of $n'$.
Using the fact that the theory of $M$ is in Horn form, we get an expansion of $n_0$ to a
$\sigma$-free model $n$. We have seen that all free models are also models of $N'$ and moreover
we have seen that $ker(k_h)$ is the least predicative congruence satisfying $Th(M)$. The free types 
axioms of $K$ fix the interpretation of $\iota(\Sigma_M)$ and therefore $ker(k'_h)$ and
$ker(k_h)$ are both minimal on the same set, and must be the same. This and the surjectivity
of $k_h$ and $k'_h$ allow us to define easily
an isomorphism between $n$ and $n'$ and because $n'$ is isomorphic with a free model it must be free as well.

\qed































\section{An example: Reversing lists}\label{ex:rev}
\documentclass{entcs} \usepackage{entcsmacro}
\usepackage{graphicx}
\usepackage{mathpartir}
\usepackage{amsmath,amssymb}
\newcommand{\hearts}{\heartsuit}
\newcommand{\prems}{\mathit{prems}}
\newcommand{\eval}{\mathit{eval}}

\newcommand{\mi}[1]{\mathit{#1}}

% defs.tex ---
% for WADT99
% copied from Tarlecki's definitions for frocos'98
% modified from moving defs.tex

\newcommand{\ASP}{\mathit{ASP}}
\newcommand{\SP}{\mathit{SP}}
\newcommand{\UDD}{\mathit{UDD}}
\newcommand{\UT}{\mathit{T}}
\newcommand{\UN}{\mathit{U}}
\newcommand{\UDecl}{\mathit{Dcl}}
\newcommand{\UDefn}{\mathit{Dfn}}

\newcommand{\parspec}[3]{#1{\stackrel{#2}{\longrightarrow}}#3}

\newcommand{\archspec}[2]{\textbf{arch spec }#1\textbf{ result }#2}
\newcommand{\decl}[2]{#1\colon#2}
\newcommand{\dfn}[2]{#1=#2}
\newcommand{\amlg}[4]{#1\textbf{ with }#2\textbf{ and }#3\textbf{ with }#4}
\newcommand{\appl}[3]{#1\lbrack#2\textbf{ fit }#3\rbrack}
\newcommand{\gris}{\mathrel{::=}}


\newcommand{\Mset}{\mathcal{M}}
\newcommand{\Fset}{\mathcal{F}}

\newcommand{\Dgm}{\mathit{Diag}}
\newcommand{\Shape}[1]{\mathit{Shape}(#1)}
\newcommand{\Nodes}[1]{\mathit{Nodes}(#1)}
\newcommand{\Edges}[1]{\mathit{Edges}(#1)}
%\newcommand{\restrict}[2]{\reduct{#1}{#2}}

\newcommand{\fm}[2]{\langle#1\rangle_{#2}}

\newcommand{\incl}[2]{\iota_{{#1}\subseteq{#2}}}

\newcommand{\st}{\mathit{st}}
\newcommand{\stC}{\mathit{C}_{\st}}
\newcommand{\stB}{\mathit{B}_{\st}}
\newcommand{\stP}{\mathit{P}_{\st}}
\newcommand{\emptystC}{\stC^\emptyset}
\newcommand{\emptyUC}{\UC^\emptyset}
\newcommand{\UC}{\mathcal{C}}
\newcommand{\UEv}{\mathit{UEv}}
\newcommand{\Uset}{\mathcal{V}}
\newcommand{\UU}{\mathit{V}}

\newcommand{\EstC}{\mathcal{C}_{\st}}
\newcommand{\emptyEstC}{\EstC^\emptyset}
\newcommand{\EstB}{\mathcal{B}_{\st}}
\newcommand{\SD}{D}

\newcommand{\CstC}{\mathfrak{C}_{\st}}

\newcommand{\getstctx}{\mathit{ctx}}

\newcommand{\env}{\mathit{E}}

\newcommand{\lambdaexp}[2]{\lambda#1\cdot#2}
\newcommand{\reduct}[2]{#1|_{#2}}

\newlength{\croutw}
\newlength{\crouth}
\newcommand{\crossout}[1]%
        {\settowidth{\croutw}{$#1$}\settoheight{\crouth}{$#1$}#1%
        \hspace{-1.0\croutw}\raisebox{0.3\crouth}{\rule{\croutw}{0.1ex}}}

\newcommand{\commentout}[1]{\ignorespaces}
\newcommand{\underscore}{\rule{3mm}{0.005in}}

\newcommand{\extaticsem}[3]{#1 \vdash #2 \extsmb #3}
\newcommand{\extsmb}{\mathrel{\rhd\hspace{-0.5em}\rhd}}

\newcommand{\rulesection}[1]{\[\fbox{$#1$}\]}
\newcommand{\infrule}[2]{\frac{#1}{#2}}
\newcommand{\staticsem}[3]{#1 \vdash #2 \rhd #3}
\newcommand{\modelsem}[3]{#1 \vdash #2 \Rightarrow #3}
\newcommand{\staticsembreak}[3]{\begin{array}{r}#1 \vdash #2 \qquad\\ 
                                                \rhd #3 \end{array}}
\newcommand{\modelsembreak}[3]{\begin{array}{r}#1 \vdash #2 \qquad\\
                                                 \Rightarrow #3 \end{array}}



% Defs from Till 


%\newenvironment{proof}{\begin{trivlist}\item[\textbf{Proof. 
%}]}{\end{trivlist}}

%\newcommand{\qed} {\hfill{$\Box$}}

\newcommand{\ttsize}{\footnotesize }
\newcommand{\NT}[1]{{\ttsize\texttt{#1}}}

%\newcommand{\gram}[1]{{\texttt{#1}}}
%\newenvironment{grammar}
% {\texonly{\footnotesize}
%  \begin{example}}{\end{example}\normalsize}
%\newenvironment{slgrammar}
% {\texonly{\footnotesize}
%  \begin{example}\texorhtml{\slshape}{\it}}{\end{example}\normalsize}


\newcommand{\cofidirectory}{}


%symbols etc
\newcommand{\parrightarrow}{\p\rightarrow}
\newcommand{\p}[1]{\mathrel{\ooalign{\hfil$\mapstochar\mkern 
5mu$\hfil\cr$#1$}}}
%\newsavebox{\pararrbox}
%\savebox{\pararrbox}{$\ \rightarrow\!\!\!\!\!\!{\raisebox{.15cm}{\tiny
%p}}\ \ $}
%\newcommand{\classicalpararrow}{{\usebox{\pararrbox}}}
\newcommand{\totrightarrow}{\rightarrow}


\newcommand{\X}{CASL}
%\newcommand{\congr}{\equiv}
%\newcommand{\TSX}{{T_{\Sigma}(X)}}
\newcommand{\LSX}{{W_{\Sigma}(X)}}

\newcommand{\Powfin}{\mathcal{P}_\omega}
\newcommand{\gen}[1]{\mid_{#1}}
\newcommand{\cogen}[1]{\mid^{#1}}
\newcommand{\induce}[2]{\mid^{#1}_{#2}}
\newcommand{\inducefrom}[1]{\!\mid_{#1}}
\newcommand{\induceto}[1]{\mid^{#1}}

%\makeindex

%\newcommand{\Mod}{\mathbf{Mod}}
\newcommand{\sen}{\mathbf{Sen}}

%\newcommand{\Sen}{\mathbf{Sen}}

%\newcommand{\Sign}{\mathbf{Sign}}
%\newcommand{\Set}{\mathbf{Set}}
%\newcommand{\ttsize}{\footnotesize }
%\newcommand{\NT}[1]{{\ttsize\texttt{#1}}}
\newcommand{\PF}{\mathit{PF}}
\newcommand{\TF}{\mathit{TF}}

\newcommand{\SY}{\mathit SY}
\newcommand{\SYs}{\mathit SYs}
\newcommand{\RSY}{\mathit RSY}
\newcommand{\RSYs}{\mathit RSYs}
\newcommand{\Sym}{\mathit{Sym}}
\newcommand{\RawSym}{\mathit{RawSym}}
\newcommand{\RawSymMap}{\mathit{RawSymMap}}
\newcommand{\IDAsRawSym}{\mathit{IDAsRawSym}}
\newcommand{\SymAsRawSym}{\mathit{SymAsRawSym}}

%\newcommand{\dom}{\mathit{dom}}
%\newcommand{\graph}{\mathit{graph}}
%% incompatible with xypic -- dirk


%\newenvironment{grammar}
% {\ttsize\begin{alltt}}{\end{alltt}}



\newcommand{\TSX}{{T_{\Sigma^\#}(X)}}

\newcommand{\bit}{\begin{itemize}}
\newcommand{\eit}{\end{itemize}}
\newcommand{\Equivalent}{\Leftrightarrow}
%\newcommand{\Implies}{\Rightarrow}
\newcommand{\logand}{\wedge}
\newcommand{\logor}{\vee}
%\newcommand{\union}{\cup}
\newcommand{\intersection}{\cap}
\newcommand{\comp}{\circ}
\newcommand{\suchthat}{\mid}
\newcommand{\congr}{\equiv}
\newcommand{\disjoint}{\uplus}
\newcommand{\buildset}[1]{\{#1\}}
\newcommand{\Si}{\Sigma}
\newcommand{\al}{\alpha}
\newcommand{\pp}[2]{#1_1\commadots #1_{#2}}
%\newcommand{\map}[2]{\colon#1\!\longrightarrow\!#2}
\newcommand{\mmap}[2]{#1\!\longrightarrow\!#2}
\newcommand{\anddots}{\logand\,\cdots\,\logand}
\newcommand{\lldots}{\,\ldots\,}
\newcommand{\ccdots}{\,\cdots\,}
\newcommand{\commadots}{,\lldots,}
\newcommand{\sigspec}[1]{\langle #1 \rangle}
\newcommand{\eq}{\stackrel{\mbox{\scriptsize e}}{=}}
\newcommand{\weq}{\stackrel{\mbox{\scriptsize w}}{=}}

\newcommand{\mBox}[1]{\, \mbox{#1} \,}

\newcommand{\engtwocase}[3]{
\left\{
\begin{array}{ll}
  #1,&\mBox{if }#2\\
  #3,&\mBox{otherwise}
\end{array}\right.}

\newcommand{\threecase}[5]{
\left\{
\begin{array}{ll}
  #1,&\mbox{if }#2\\
  #3,&\mbox{if }#4\\
  #5,&\mbox{otherwise}
\end{array}\right.}

\newcommand{\threefullcase}[6]{
\left\{
\begin{array}{ll}
  #1,&\mbox{if }#2\\
  #3,&\mbox{if }#4\\
  #5,&\mbox{if }#6
\end{array}\right.}

\newcommand{\delete}[1]{$\overline{\mbox{#1}}$}
\newcommand{\new}[1]{{\it\tiny #1}}

\newcommand{\mbs}[1]{\mbox{$#1$}}




\newcommand{\ASL}[1]{[\![#1]\!]}

\newcommand{\sym}{\mathit{sym}}
\newcommand{\totqual}{\mathsf{t}}
\newcommand{\parqual}{\mathsf{p}}
\newcommand{\ws}{\mathit{ws}}
\newcommand{\FinSet}[1]{\mathit{FinSet}(#1)}
\newcommand{\RawSymOf}{\mathit{RawSymOf}}
\newcommand{\Gram}[1]{{\textup{\texttt{#1}}}}

\newcommand{\implicitqualkind}{\mathit{implicit}}
\newcommand{\sortqualkind}{\mathit{sort}}
\newcommand{\funqualkind}{\mathit{fun}}
\newcommand{\predqualkind}{\mathit{pred}}
\newcommand{\SymKind}{\mathit{SymKind}}

%\newcommand{\Sort}{\mathit{Sort}}
\newcommand{\QualFunName}{\mathit{QualFunName}}
\newcommand{\QualPredName}{\mathit{QualPredName}}



% Lutz's definitions


\hfuzz1pc
\sloppy
%\frenchspacing


%\spnewtheorem{thm}{Theorem}[section]{\bfseries}{\itshape}
%\spnewtheorem{cor}[theorem]{Corollary}{\bfseries}{\itshape}
%\spnewtheorem{lem}[theorem]{Lemma}{\bfseries}{\itshape}
%\spnewtheorem{prop}[theorem]{Proposition}{\bfseries}{\itshape}
%\spnewtheorem{defn}[theorem]{Definition}{\bfseries}{\upshape}
%\spnewtheorem{rem}[theorem]{Remark}{\bfseries}{\upshape}
%\spnewtheorem{expl}[theorem]{Example}{\bfseries}{\upshape}
%\spnewtheorem{thmdefn}[theorem]{Theorem and Definition}{\bfseries}{\itshape}
%\spnewtheorem{propdefn}[theorem]{Proposition and Definition}{\bfseries}{\itshape}
%\spnewtheorem{assumption}[theorem]{Assumption}{\bfseries}{\upshape}
%\spnewtheorem{algorithm}[theorem]{Algorithm}{\bfseries}{\upshape}

%% Other environments
%\newenvironment{mycd}
%       {\begin{displaymath}\renewcommand{\arraystretch}{0.5}\begin{CD}}
%       {\end{CD}\end{displaymath}}
\newenvironment{rmenumerate}%
        {\begin{enumerate}\renewcommand{\labelenumi}{(\roman{enumi})}}%
        {\end{enumerate}}
\newenvironment{alenumerate}%
        {\begin{enumerate}\renewcommand{\labelenumi}{(\alph{enumi})}}%
        {\end{enumerate}}
\newenvironment{Alenumerate}%
        {\begin{enumerate}\renewcommand{\labelenumi}{\Alph{enumi})}}%
        {\end{enumerate}}
\newenvironment{numenumerate}%
        {\begin{enumerate}\renewcommand{\labelenumi}{\arabic{enumi})}}%
        {\end{enumerate}}
%\newenvironment{defn}%
%       {\begin{definition}\em}%
%       {\end{definition}}
%\newenvironment{rem}%
%       {\begin{remark}\em}%
%       {\end{remark}}
%\renewenvironment{example}%
%       {\begin{expl}\em}%
%       {\end{expl}}
%\newenvironment{texteqnarray}%
%       {\setlength{\itemsep}{0pt}%
%       \setlength{\topsep}{0pt}%
%       \begin{trivlist}\item[]\begin{displaymath}%
%       \renewcommand{\arraystretch}{1.2}\begin{array}{rcll}}%
%       {\end{array}\end{displaymath}\end{trivlist}}
%\newenvironment{acknowledge}%
%       {\begin{list}{}{\setlength{\leftmargin}{0pt}%
%                       \setlength{\topsep}{1cm}}% 
%       \item{\bf Acknowledgements:}}%
%       {\end{list}}

%\newcommand{\qed}{\hfill\openbox}
%\newenvironment{proof}
%       {\begin{trivlist}\item{\sc Proof: }}
%       {\qed\end{trivlist}}



%% Math definitions
\newcommand{\Cat}{\mathbf}
\newcommand{\Cls}{\mathcal}
\newcommand{\Opname}{\mathrm}

\newcommand{\Ob}{{\Opname{Ob}\,}}
\newcommand{\Mor}{{\Opname{Mor}\,}}
\newcommand{\Epi}{{\Opname{Epi}\,}}
\newcommand{\Mono}{{\Opname{Mono}\,}}
\newcommand{\Iso}{{\Opname{Iso}\,}}
\newcommand{\Ident}{{\Opname{Ident}\,}}
\newcommand{\Op}{{op}}
\newcommand{\colim}{{\Opname{colim}\,}}

\newcommand{\lrule}[3]{(#1)\;\;\infrule{#2}{#3}}
\newcommand{\lrrule}[3]{(#1)\;\;\infrule{\underline{#2}}{#3}}
\newcommand{\laxiom}[2]{(#1)\;\;#2}

\newcommand{\BA}{{\Cat A}}
%\newcommand{\BB}{{\Cat B}}
\newcommand{\BC}{{\Cat C}}
\newcommand{\BD}{{\Cat D}}
\newcommand{\BI}{{\Cat I}}
\newcommand{\BK}{{\Cat K}}
\newcommand{\BL}{{\Cat L}}
\newcommand{\BM}{{\Cat M}}
\newcommand{\BN}{{\Cat N}}
\newcommand{\BS}{{\Cat S}}
\newcommand{\BW}{{\Cat W}}
\newcommand{\Btwo}{{\Cat 2}}

\newcommand{\CC}{{\Cls C}}
\newcommand{\CE}{{\Cls E}}
%\newcommand{\CK}{{\Cls K}}
% \newcommand{\CL}{{\Cls L}}
\newcommand{\CM}{{\Cls M}}
\newcommand{\CO}{{\Cls O}}
\newcommand{\CP}{{\Cls P}}
\newcommand{\CS}{{\Cls S}}
\newcommand{\CT}{{\Cls T}}
\newcommand{\eps}{\varepsilon}

\newcommand{\integers}{{\mathbb Z}}
\newcommand{\tensor}{\otimes}

%\newcommand{\openbox}{\leavevmode
%  \hbox to.77778em{%
%  \hfil\vrule
%  \vbox to.675em{\hrule width.6em\vfil\hrule}%
%  \vrule\hfil}}
\newcommand{\mystrut}[1]{\rule[#1]{0cm}{0.1cm}}



%\newcommand{\red}{\vdash}
\newcommand{\idred}{\red}
\newcommand{\step}{\red_c}
% \newcommand{\stepeq}{\succeq}
\newcommand{\downto}{\succeq}
\newcommand{\upto}{\preceq}
\newcommand{\moremonicthan}{\Rightarrow}
\newcommand{\into}{\hookrightarrow}
\newcommand{\impl}{\Rightarrow}
\newcommand{\id}{{id}}
\newcommand{\ev}{{ev}}
\newcommand{\Hom}{{hom}}
\newcommand{\Dom}{{dom\, }}
\newcommand{\Cod}{{cod}}
\newcommand{\adj}{\dashv}
\newcommand{\powerset}{{\mathcal P}}
\newcommand{\finpowerset}{{\powerset_\omega}}
\newcommand{\kVec}{\mbox{$k$-$\Cat{Vec}$}}
\newcommand{\termObj}{1}

\newcommand{\map}[2]{:#1\to #2}
\newcommand{\restr}[2]{{#1}|_{#2}}
\newcommand{\wordbrace}[1]{\underbrace{\hspace{1.5cm}}_{\displaystyle{#1}}}
\newcommand{\forget}[1]{|_{#1}}
\newcommand{\argument}{\_\!\_}%{\underline{\;\;}}
%\newcommand{\substack}[2]{{{#1} \atop {#2}}}

%% Category Names
\newcommand{\CASLsign}{\Cat{CASLsign}}
\newcommand{\enrCASLsign}{\Cat{enrCASLsign}}
\newcommand{\refCASLsign}{\Cat{refCASLsign}}
%\newcommand{\Mod}{\Cat{Mod}\,}
% the \, caused bad effect when copmbined with indices etc.
\newcommand{\EnrMod}{\Cat{Mod}_e}
\newcommand{\CAT}{\Cat{CAT}}
\newcommand{\Sen}{\Cat{Sen}}
\newcommand{\Sign}{\Cat{Sign}}
\newcommand{\SignVar}{\Cat{Var}}
\newcommand{\EnrSign}{\Cat{EnrSign}}
%\newcommand{\Set}{\Cat{Set}}
\newcommand{\CLS}{\Cat{CLS}}

\newcommand{\SmallSpecName}[1]{\textsc{\small #1}}
%% Institution Names
\newcommand{\EnrInst}{EnrPCFOL}
\newcommand{\CASLInst}{SubPCFOL}
\newcommand{\PCFOLInst}{PCFOL}

%% Words
\newcommand{\Word}{\mathbf}
\newcommand{\Ba}{{\Word{a}}}
\newcommand{\Bb}{{\Word{b}}}
\newcommand{\Bc}{{\Word{c}}}
\newcommand{\Bf}{{\Word{f}}}
\newcommand{\Bg}{{\Word{g}}}
\newcommand{\Bh}{{\Word{h}}}
\newcommand{\Bl}{{\Word{l}}}
\newcommand{\Br}{{\Word{r}}}



%% For HasCASL:

\newcommand{\HasCASL}{{\sc HasCasl}\xspace}
\newcommand{\ML}{{\textsf{ML}}\xspace}
\newcommand{\BaseTypes}{B}
\newcommand{\SimpleTypes}{T}
\newcommand{\TotalFunType}[2]{#1\to #2}
\newcommand{\sbullet}{\mbox{ $\scriptstyle\bullet$ }}
\newcommand{\lambdaTotal}[2]{\lambda\, #1\sbullet\!\! !\, #2}
\newcommand{\PartialFunType}[2]{#1\to ?#2}
\newcommand{\lambdaPartial}[2]{\lambda\, #1\sbullet #2}
\newcommand{\lambdaType}[2]{\lambda\, #1\sbullet #2}
\newcommand{\NonStrictFunType}[2]{\PartialFunType{\Maybe{#1}}{#2}}
\newcommand{\lambdaNonStrict}[2]{\lambda\, #1\sbullet #2}
\newcommand{\substTerm}[3]{{#1[#3/#2]}}
\newcommand{\PredType}[1]{{\textsf{pred}(#1)}}
\newcommand{\ProdType}[2]{#1 * #2}
\newcommand{\pMonType}[2]{#1 -\!\!\mu\!\!\to\hspace{-1pt}? #2}
\newcommand{\tMonType}[2]{#1 -\!\!\mu\!\!\to #2}
\newcommand{\pContType}{Pcont}
\newcommand{\tContType}{Tcont}
\newcommand{\projOp}[1]{{pr_#1}}
\newcommand{\fst}{{\textsf{fst}}}
\newcommand{\snd}{{\textsf{snd}}}
\newcommand{\UnitType}{{\textsf{unit}}}
\newcommand{\UnitOp}{{()}}
\newcommand{\Maybe}[1]{?#1}
\newcommand{\TypeSchemes}{{TS}}
\newcommand{\TypeContext}{\mathcal{C}}
\newcommand{\UniversalType}[2]{{\forall #1\sbullet #2}}
\newcommand{\applyPoly}[2]{{#1[#2]}}
%\newcommand{\lambdaPoly}[2]{\Lambda #1:\TypeKind.\, #2}
\newcommand{\ConstrainedType}[2]{{#1 \Rightarrow #2}}
\newcommand{\Constraint}{C}
\newcommand{\TypeKind}{\textsf{type}}
\newcommand{\TypeVars}{{TV}}
\newcommand{\TypeOps}{{TO}}
\newcommand{\PseudoTypes}{{PT}}
\newcommand{\SubstType}[3]{{#1[#3/#2]}}
\newcommand{\varco}[1]{{v^+(#1)}}
\newcommand{\varcontra}[1]{{v^-(#1)}}
\newcommand{\Covar}{\textsf{covariant}}
\newcommand{\Contravar}{\textsf{contravariant}}
\newcommand{\sigSorts}{S}
\newcommand{\sigClasses}{C}
\newcommand{\sigConstrs}{T}
\newcommand{\sigAlias}{{A}}
\newcommand{\sigExpand}{{\varepsilon}}
\newcommand{\unalias}{{\bar\sigExpand}}
\newcommand{\sigSub}{\leq}
\newcommand{\extSub}{\leq_*}
\newcommand{\elPred}[2]{{#1\in #2}}
\newcommand{\downCast}[2]{{#1\textsf{ as }#2}}
\newcommand{\upCast}[2]{{#1:#2}}
\newcommand{\sigOps}{O}
\newcommand{\IndVars}{V}
\newcommand{\ifthenelse}[3]{{\textsf{if }#1\textsf{ then }#2
                        \textsf{  else }#3}}
\newcommand{\BT}{B}
\newcommand{\emb}{{emb}}

\newcommand{\Context}{\Gamma}
\newcommand{\ContextTerm}[3]{{#1 \rhd #2: #3}}
\newcommand{\ContextEq}[2]{{#1 \rhd #2}}
\newcommand{\ContextEntails}[3]{{#2 \entails_{#1} #3}}
\newcommand{\ContextImpl}[3]{{#2 \impl_{#1} #3}}
\newcommand{\exeq}{\stackrel{e}{=}}
\newcommand{\IsDef}{\operatorname{def}}

\newcommand{\isFormula}[1]{{#1\in\textsf{Prop}}}

\newcommand{\Model}{M}
\newcommand{\Sem}[1]{{[\![#1]\!]}}
\newcommand{\extendsTo}{\preceq}
\newcommand{\True}{\top}
\newcommand{\False}{\bot}
\newcommand{\uexists}{\exists !}
\newcommand{\iotaTerm}[2]{\iota #1\sbullet #2}
%\newcommand{\conj}{\wedge}
\newcommand{\disj}{\vee}
\newcommand{\modimpl}{\to}
\newcommand{\modiff}{\leftrightarrow}
\newcommand{\intEq}{eq}
\newcommand{\dOrder}[3]{#2 \le\![#1]\,#3}
\newcommand{\resTerm}[2]{#1\ res\ #2}
\newcommand{\mono}[2]{\operatorname{mono}_{#1}#2}
%\newcommand{\entails}{\vdash}

%% CASL constructs

%% HasCASL constructs
\newcommand{\Class}{\textbf{class}}
\newcommand{\Instance}{\textbf{instance}}
\newcommand{\Program}{\textbf{program}}
\newcommand{\Internal}{\textbf{internal}}
\providecommand{\pfun}{\mathrel{\rightarrow?}}
\newcommand{\iimpl}{\Rightarrow}
\newcommand{\iconj}{\wedge}

%% For Monads:
\newcommand{\BBT}{\mathbb{T}}
\newcommand{\DO}{\operatorname{do}}
\newcommand{\Let}{\operatorname{let}}
\newcommand{\retOp}{\operatorname{ret}}
\newcommand{\retOpHC}{\mathit{ret}}
\newcommand{\ret}[1]{\retOp #1}
\newcommand{\letTerm}[2]{\DO\, #1; \ #2}
\newcommand{\leteq}{\leftarrow}
\newcommand{\se}[2]{\mathsf{se}(#1,#2)}
\newcommand{\Pfin}{\mathcal P_{\mathit{fin}}}

%% Program logic
\newcommand{\nec}{\square\,}
\newcommand{\gbox}{\square\hspace{-6.5pt}\raisebox{2pt}{\tiny{G}}\;}
\newcommand{\pbox}[1]{[#1]\,}
\newcommand{\pdiamond}[1]{<\hspace{-3pt}#1\hspace{-3pt}>\!}
%\newcommand{\lbox}{\square\,}
\newcommand{\lbox}{\Box}
\newcommand{\ldiamond}{\Diamond}
\newcommand{\pmodal}[2]{[#1]_G\,#2}
\newcommand{\Timpl}{\Rightarrow_T}
\newcommand{\Tconj}{\wedge}
\newcommand{\HTriple}[3]{\{#1\}~#2~\{#3\}}
\newcommand{\ContextHTriple}[4]{#1\rhd\HTriple{#2}{#3}{#4}}
\newcommand{\DummyCHT}[4]{\HTriple{#2}{#3}{#4}}
\newcommand{\sef}{\mathit{sef}}
\newcommand{\dsef}{\mathit{dsef}}

%% For polymorphism
\newcommand{\extInst}[1]{\mathrm{Ext}(#1)}
%\newcommand{\gextInst}[1]{\mathrm{gExt}(#1)}
\newcommand{\Poly}[1]{\mathrm{Poly}(#1)}
%% For institutions


\newcommand{\ModFunctor}{\Cat{Mod}\,}


\newcommand{\la}{\langle}
\newcommand{\ra}{\rangle}


\newcommand{\si}{\sigma}

\newcommand{\ISig}{\mathit{Sig}}
\newcommand{\IMod}{\mathit{Mod}}
\newcommand{\IAx}{\mathit{Ax}}

\newcounter{blubber}

\newenvironment{algenumerate}
{\begin{list}
  {\arabic{blubber}.}
  {\usecounter{blubber}
   \setlength{\leftmargin}{3ex}
    \setlength{\parsep}{0pt}
    \setlength{\itemindent}{1ex}
    \setlength{\itemsep}{2pt}   
    \setlength{\listparindent}{0ex}
  }
}
{\end{list}}

\newenvironment{specsem}%
        {\smallskip\par\noindent\qquad$\begin{array}{@{}l@{{}={}}l}}%
        {\end{array}$\par\noindent}


%% Width of parboxes for figures
\newlength{\myboxwidth}
\setlength{\myboxwidth}{\textwidth}
\addtolength{\myboxwidth}{-20pt}
\newenvironment{myfigure}{\begin{figure}\begin{center}
	\setlength{\fboxsep}{10pt}}% 
	{\end{center}\end{figure}}

%% Coalgebraic Modal Logic
\newcommand{\Lang}{\mathcal{L}}	
\newcommand{\FLang}{\mathcal{F}}	
\newcommand{\pls}{\Lambda}
\newcommand{\pl}[3]{#1\in #2(#3)}
\newcommand{\polypl}[3]{#1\in #2 #3}
\newcommand{\negpl}[3]{#1\notin #2(#3)}
\newcommand{\plbox}[1]{[#1]}
%\newcommand{\pldiamond}[1]{<\hspace{-3pt}#1\hspace{-3pt}>\!}
\newcommand{\pldiamond}[1]{\langle#1\rangle}
\newcommand{\gldiamond}[1]{\Diamond_{#1}}
\newcommand{\glbox}[1]{\square_{#1}}

\newcommand{\HMLBox}[1]{\square\hspace{-5.5pt}\raisebox{2pt}{$\scriptscriptstyle{#1}$}\;}
\newcommand{\HMLDiamondb}[1]{\Diamond\hspace{-5pt}\raisebox{1.5pt}{$\scriptscriptstyle{#1}$}\;}
\newcommand{\HMLDiamonda}[1]{\Diamond\hspace{-5.3pt}\raisebox{1.9pt}{$\scriptscriptstyle{#1}$}\;}
\newcommand{\pldbox}[2]{[#1]_{#2}}
\newcommand{\plempty}{\plbox{\emptyset}}
\newcommand{\plcan}{\square}
\newcommand{\FC}{{\mathfrak C}}
\newcommand{\negcl}[1]{\mathit{cl}(#1)}
\newcommand{\Nat}{{\mathbb{N}}}
\newcommand{\Int}{{\mathbb{Z}}}
\newcommand{\Rat}{{\mathbb{Q}}}
\newcommand{\Real}{{\mathbb{R}}}
\newcommand{\List}{\mathsf{list}}
\newcommand{\PDist}{D_\omega}%% Provisionary!!
%\newcommand{\Id}{\mathit{Id}}
\newcommand{\SFun}{\mathcal{SP}}
\newcommand{\plcomp}{\circledast}
\newcommand{\contrapower}{2^{\argument}}
\newcommand{\trans}[1]{#1^\flat}
\newcommand{\boolcl}[2]{\mathrm{bcl}_{#1}(#2)}
\newcommand{\Bag}{\mathcal{B}}
\newcommand{\Baginfty}{\mathcal{B}_\infty}
%\newcommand{\Ab}{\mathcal{B}_{\mathbb{Z}}}
\newcommand{\Prop}{\mathsf{Prop}}
\newcommand{\Up}{\mathsf{Up}}
\newcommand{\Lit}{\mathsf{Lit}}
\newcommand{\nneg}{\sim}
\newcommand{\satisfies}{\vDash}
\newcommand{\gsatisfies}{\satisfies_g}
\newcommand{\nsatisfies}{\nvDash}
\newcommand{\PSPACE}{\mathit{PSPACE}}
\newcommand{\Rules}{\mathcal{R}}
\newcommand{\RulesC}{\Rules_C}
\newcommand{\Frame}[1]{2^{2^{#1}}}
\newcommand{\UpP}{\mathsf{Up}\mathcal{P}}
\newcommand{\one}{\mathbb{1}}
%\newcommand{\smaller}[2]{#1_{<#2}}
\newcommand{\total}{\nu}
\newcommand{\sgn}{\mathit{sgn}}
%\newcommand{\RedClause}{\Clause^r}
\newcommand{\size}{\mathit{size}}
\newcommand{\arity}{\mathit{ar}}

\newcommand{\Sorts}{\mathcal{S}}
\newcommand{\MSorts}{\Sorts_0}
\newcommand{\ModOp}{L}
\newcommand{\contrapow}{\mathcal{Q}}
\newcommand{\profto}{\stackrel{\bullet}{\to}}
\newcommand{\Struct}{\mathcal{M}}
\newcommand{\Id}{\mathit{Id}}
\newcommand{\Det}{\mathsf{Det}}
\newcommand{\hDet}{\mathsf{hDet}}
\newcommand{\FF}{\mathfrak{F}}
\newcommand{\idOp}{\plbox{\iota}}
\newcommand{\Sig}{\mathit{Sig}}

\newcommand{\NEXP}{\mi{NEXPTIME}}
\newcommand{\EXP}{\mi{EXPTIME}}
\newcommand{\NP}{\mi{NP}}
\newcommand{\at}{\mi{at}}
\newcommand{\Max}{\mi{max}}
\newcommand{\muprod}{\textstyle\prod^\mu}
\newcommand{\tsum}{\textstyle\sum}
\DeclareMathOperator{\mge}{\ge}
\DeclareMathOperator{\meq}{=}
\DeclareMathOperator{\mle}{\le}
\DeclareMathOperator{\fixarrow}{\uparrow}
\newcommand{\Stable}{S}
\newcommand{\Zero}{Z}
\newcommand{\Regular}{U}
\newcommand{\last}{\lambda}
\newcommand{\fix}[2]{#1\!\fixarrow #2}
\newcommand{\msf}{\mathsf}

\newcommand{\CKCMi}{\CK\!+\!\CMi\xspace}
\newcommand{\CK}{\mathit{CK}}
\newcommand{\CMi}{\mathit{CMi}}
\newcommand{\Ax}{\mathcal{A}}

%\newcommand{\PLentails}{\entails_{\mi{PL}}}
%\newcommand{\ModSig}{\Lambda}

\newcommand{\modelsCA}{\models}
\newcommand{\Th}{\mathsf{Th}}
\newcommand{\rank}{\operatorname{rank}}
\newcommand{\Space}{\mathcal{S}}
\newcommand{\osder}[2]{\entails_{#1}^{#2}}
\newcommand{\gentails}[1]{\entails^{#1}_g}

\sloppy
% The following is enclosed to allow easy detection of differences in
% ascii coding.
% Upper-case    A B C D E F G H I J K L M N O P Q R S T U V W X Y Z
% Lower-case    a b c d e f g h i j k l m n o p q r s t u v w x y z
% Digits        0 1 2 3 4 5 6 7 8 9
% Exclamation   !           Double quote "          Hash (number) #
% Dollar        $           Percent      %          Ampersand     &
% Acute accent  '           Left paren   (          Right paren   )
% Asterisk      *           Plus         +          Comma         ,
% Minus         -           Point        .          Solidus       /
% Colon         :           Semicolon    ;          Less than     <
% Equals        =3D           Greater than >          Question mark ?
% At            @           Left bracket [          Backslash     \
% Right bracket ]           Circumflex   ^          Underscore    _
% Grave accent  `           Left brace   {          Vertical bar  |
% Right brace   }           Tilde        ~

% A couple of exemplary definitions:

%\newcommand{\Nat}{{\mathbb N}}
%\newcommand{\Real}{{\mathbb R}}
\newcommand{\COLOSS}{{\textrm CoLoSS}}
\def\lastname{Hausmann and Schr\"oder}
\begin{document}
\begin{frontmatter}
  \title{Optimizing Conditional Logic Reasoning within \COLOSS}
  \author[DFKI]{Daniel Hausmann\thanksref{myemail}}
  \author[DFKI,UBremen]{Lutz Schr\"oder\thanksref{coemail}}
  \address[DFKI]{DFKI Bremen, SKS}
  \address[UBremen]{Department of Mathematics and Computer Science, Universit\"at Bremen, Germany}
%  \thanks[ALL]{Work forms part of DFG-project \emph{Generic Algorithms and Complexity
%      Bounds in Coalgebraic Modal Logic} (SCHR 1118/5-1)}
  \thanks[myemail]{Email: \href{mailto:Daniel.Hausmann@dfki.de} {\texttt{\normalshape Daniel.Hausmann@dfki.de}}}
  \thanks[coemail]{Email: \href{mailto:Lutz.Schroeder@dfki.de} {\texttt{\normalshape Lutz.Schroeder@dfki.de}}}
\begin{abstract} 
  The generic modal reasoner CoLoSS covers a wide variety of logics
  ranging from graded and probabilistic modal logic to coalition logic
  and conditional logics, being based on a broadly applicable
  coalgebraic semantics and an ensuing general treatment of modal
  sequent and tableau calculi. Here, we present research into
  optimisation of the reasoning strategies employed in
  CoLoSS. Specifically, we discuss strategies of memoisation and
  dynamic programming that are based on the observation that short
  sequents play a central role in many of the logics under
  study. These optimisations seem to be particularly useful for the
  case of conditional logics, for some of which dynamic programming
  even improves the theoretical complexity of the algorithm. These
  strategies have been implemented in CoLoSS; we give a detailed
  comparison of the different heuristics, observing that in the
  targeted domain of conditional logics, a substantial speed-up can be
  achieved.
\end{abstract}
\begin{keyword}
  Coalgebraic modal logic, conditional logic, automated reasoning,
  optimisation, heuristics, memoizing, dynamic programming
\end{keyword}
\end{frontmatter}
\section{Introduction}\label{intro}

In recent decades, modal logic has seen a development towards semantic
heterogeneity, witnessed by an emergence of numerous logics that,
while still of manifestly modal character, are not amenable to
standard Kripke semantics. Examples include probabilistic modal
logic~\cite{FaginHalpern94}, coalition logic~\cite{Pauly02}, and
conditional logic~\cite{Chellas80}, to name just a few. The move
beyond Kripke semantics, mirrored on the syntactical side by the
failure of normality, entails additional challenges for tableau and
sequent systems, as the correspondence between tableaux and models
becomes looser --- tableaux are still graphs, but models are generally
more complex structures.

This problem is tackled on the theoretical side by introducing the
semantic framework of coalgebraic modal
logic~\cite{Pattinson03,Schroder05}, which covers all logics mentioned
above and many more. It turns out that coalgebraic modal logic does
allow the design of generic reasoning algorithms, including a generic
tableau method originating from~\cite{SchroderPattinson09}; this
generic method may in fact be separated from the semantics and
developed purely syntactically, as carried out
in~\cite{PattinsonSchroder08b,PattinsonSchroder09a}. 

Generic tableau algorithms for coalgebraic modal logics, in particular
the algorithm described in~\cite{SchroderPattinson09}, have been
implemented in the reasoning tool
\COLOSS~\cite{CalinEA09}\footnote{available under
  \url{http://www.informatik.uni-bremen.de/cofi/CoLoSS/}}. As
indicated above, it is a necessary feature of the generic tableau
systems that they potentially generate multiple successor nodes for a
given modal demand, so that in addition to the typical depth problem,
proof search faces a rather noticeable problem of breadth. The search
for optimisation strategies to increase the efficiency of reasoning
thus becomes all the more urgent. Here we present one such strategy,
which is generally applicable, but particularly efficient in reducing
both depth and branching for the class of conditional logics. We
exploit a notable feature of this class, namely that many of the
relevant rules rely rather heavily on premises stating equivalence
between formulas; thus, conditional logics are a good candidate for
memoising strategies, applied judiciously to short sequents. We
describe the implementation of memoising and dynamic programming
strategies within \COLOSS, and discuss the outcome of various
comparative experiments.
 
\section{Generic Sequent Calculi for Coalgebraic Modal Logic}


Coalgebraic modal logic, originally introduced as a specification
language for coalgebras, seen as generic reactive
systems~\cite{Pattinson03}, has since evolved into a generic framework
for modal logic beyond Kripke semantics~\cite{CirsteaEA09}. The basic
idea is to encapsulate the branching type of the systems relevant for
the semantics of a particular modal logic, say probabilistic or
game-theoretic branching, in the choice of a set functor, the
signature functor (e.g.\ the distribution functor and the games
functor in the mentioned examples), and to capture the semantics of
modal operators in terms of so-called predicate liftings. For the
purposes of the present work, details of the semantics are less
relevant than proof-theoretic aspects, which we shall recall
presently. The range of logics covered by the coalgebraic approach is
extremely broad, including, besides standard Kripke and neighbourhood
semantics, e.g.\ graded modal logic~\cite{Fine72}, probabilistic modal
logic~\cite{FaginHalpern94}, coalition logic~\cite{Pauly02}, various
conditional logics equipped with selection function
semantics~\cite{Chellas80}, and many more.

Syntactically, logics are parametrised by the choice of a \emph{modal
  similarity type} $\Lambda$, i.e.\ a set of modal operators with
associated finite arities. This choice determines the set of formulas
$\phi,\psi$ via the grammar
\begin{equation*}
  \phi, \psi ::=  p \mid
  \phi \land \psi \mid \lnot \phi \mid \hearts(\phi_1, \dots, \phi_n)
\end{equation*}
where $\hearts$ is an $n$-ary operator in $\Lambda$. Examples are
$\Lambda=\{L_p\mid p\in[0,1]\cap\mathbb{Q}\}$, the unary operators
$L_p$ of probabilistic modal logic read `with probability at least
$p$'; $\Lambda=\{\gldiamond{k}\mid k\in\Nat\}$, the operators
$\gldiamond{k}$ of graded modal logic read `in more than $k$
successors'; $\Lambda=\{[C]\mid C\subseteq N\}$, the operators $[C]$
of coalition logic read `coalition $C$ (a subset of the set $N$ of
agents) can jointly enforce that'; and, for our main example here,
$\Lambda=\{\Rightarrow\}$, the \emph{binary} modal operator
$\Rightarrow$ of conditional logic read e.g.\ `if \dots then normally
\dots'.

Coalgebraic modal logic was originally limited to so-called
\emph{rank-$1$ logics} axiomatised by formulas with nesting depth of
modal operators uniformly equal to $1$~\cite{Schroder05}. It has since
been extended to the more general non-iterative
logics~\cite{SchroderPattinson08d} and to some degree to iterative
logics, axiomatised by formulas with nested
modalities~\cite{SchroderPattinson08PHQ}. The examples considered here
all happen to be rank-$1$, so we focus on this case. In the rank-$1$
setting, it has been shown~\cite{Schroder05} that all logics can be
axiomatised by \emph{one-step rules} $\phi/\psi$, where $\phi$ is
purely propositional and $\psi$ is a clause over formulas of the form
$\hearts(a_1,\dots,a_n)$, where the $a_i$ are propositional
variables. In the context of a sequent calculus, this takes the
following form~\cite{PattinsonSchroder08b}.
\begin{definition}
  If $S$ is a set (of formulas or variables) then $\Lambda(S)$ denotes
  the set $\lbrace \hearts(s_1, \dots, s_n) \mid \hearts \in \Lambda
  \mbox{ is $n$-ary, } s_1, \dots, s_n \in S \rbrace$ of formulas
  comprising exactly one application of a modality to elements of
  $S$. An \emph{$S$-sequent}, or just a \emph{sequent} in case
  that $S$ is the set of all formulas, is a finite subset of
  $S \cup \lbrace \neg A \mid A
  \in S \rbrace$. Then, a \emph{one-step rule}
  $\Gamma_1,\dots,\Gamma_n/\Gamma_0$ over a set $V$ of variables
  consists of $V$-sequents $\Gamma_1,\dots,\Gamma_n$, the
  \emph{premises}, and a $\Lambda(S)$-sequent $\Gamma_0$, the
  \emph{conclusion}. A \emph{goal} is a set of sequents, typically
  arising as the set of instantiated premises of a rule application.
\end{definition}
\noindent A given set of one-step rules then induces an instantiation
of the \emph{generic sequent calculus}~\cite{PattinsonSchroder08b}
which is given by a set of rules $\mathcal{R}_{sc}$ consisting of the
finishing and the branching rules $\mathcal{R}^b_{sc}$ (i.e.\ rules
with no premise or more than one premise), the linear rules
$\mathcal{R}^l_{sc}$ (i.e.\ rules with exactly one premise) and the
modal rules $\mathcal R^m_{sc}$, i.e.\ the given one-step rules. The
finishing and the branching rules are presented in
Figure~\ref{fig:branching} (where $\top=\neg\bot$ and $p$ is an atom),
the linear rules are shown in Figure~\ref{fig:linear}. So far, all
these rules are purely propositional. As an example for a set of modal
one-step rules, consider the modal rules $\mathcal R^m_{sc}$ of the
standard modal logic \textbf{K} as given by Figure~\ref{fig:modalK}.
\begin{figure}[!h]
  \begin{center}
    \begin{tabular}{| c c c |}
    \hline
      & & \\[-5pt]
      (\textsc {$\neg$F}) \inferrule{ }{\Gamma, \neg\bot} &
      (\textsc {Ax}) \inferrule{ }{\Gamma, p, \neg p} &
      (\textsc {$\wedge$}) \inferrule{\Gamma, A \\ \Gamma, B}{\Gamma, A\wedge B} \\[-5pt]
      & & \\
    \hline
    \end{tabular}
  \end{center}
  \caption{The finishing and the branching sequent rules $\mathcal{R}^b_{sc}$}
  \label{fig:branching}
\end{figure}
\begin{figure}[!h]
  \begin{center}
    \begin{tabular}{| c c |}
    \hline
      & \\[-5pt]
      (\textsc {$\neg\neg$})\inferrule{\Gamma, A}{\Gamma, \neg\neg A} &
      (\textsc {$\neg\wedge$}) \inferrule{\Gamma, \neg A, \neg B}{\Gamma, \neg(A\wedge B)} \\[-5pt]
      & \\
    \hline
    \end{tabular}
  \end{center}
  \caption{The linear sequent rules $\mathcal{R}^l_{sc}$}
  \label{fig:linear}
\end{figure}

\begin{figure}[!h]
  \begin{center}
    \begin{tabular}{| c |}
    \hline
      \\[-5pt]
      (\textsc {\textbf{K}})\inferrule{\neg A_1, \ldots , \neg A_n, A_0}
                      {\Gamma, \neg \Box A_1,\ldots,\neg \Box A_n, \Box A_0 } \\[-5pt]
      \\
    \hline
    \end{tabular}
  \end{center}
  \caption{The modal rule of \textbf{K}}
  \label{fig:modalK}
\end{figure}
\noindent This calculus has been shown to be complete under suitable
coherence assumptions on the rule set and the coalgebraic semantics,
provided that the set of rules \emph{absorbs cut and contraction} in a
precise sense~\cite{PattinsonSchroder08b}. We say that a formula is
\emph{provable} if it can be derived in the relevant instance of the
sequent calculus; the algorithms presented below are concerned with
the \emph{provability problem}, i.e.\ to decide whether a given
sequent is provable. This is made possible by the fact that the
calculus does not include a cut rule, and hence enables automatic
proof search. For rank-1 logics, proof search can be performed in
$\mathit{PSPACE}$ under suitable assumptions on the representation of
the rule set~\cite{SchroderPattinson09,PattinsonSchroder08b}. The
corresponding algorithm has been implemented in the system
CoLoSS~\cite{CalinEA09} which remains under continuous
development. Particular attention is being paid to finding heuristic
optimisations to enable practically efficient proof search, as
described here.

Our running example used in the presentation of our optimisation
strategies below is conditional logic as mentioned above. The most
basic conditional logic is $\textbf{CK}$~\cite{Chellas80} (we shall
consider a slightly extended logic below), which is characterised by
assuming normality of the right-hand argument of the non-monotonic
conditional $\Rightarrow$, but only replacement of equivalents in the
left-hand arguments. Absorption of cut and contraction requires a
unified rule set consisting of the rules depicted in Fig.~\ref{fig:modalCK}
with $A=C$ abbreviating the pair of sequents $\neg A,C$ and $\neg
C,A$.
\begin{figure}[!h]
  \begin{center}
    \begin{tabular}{| c |}
    \hline
      \\[-5pt]
      (\textsc {\textbf{CK}})\inferrule{A_0=A_1;\dots;A_n=A_0\\ \neg B_1,\dots,\neg B_n,B_0}
  {\Gamma, \neg(A_1\Rightarrow B_1),\dots,\neg(A_n\Rightarrow B_n),(A_0\Rightarrow B_0)}\\[-5pt]
      \\
    \hline
    \end{tabular}
  \end{center}
  \caption{The modal rule of \textbf{CK}}
  \label{fig:modalCK}
\end{figure}
%\begin{equation*}
%  \infrule{A_0=A_1;\;\dots;\; A_n=A_0\quad \neg B_1,\dots,\neg B_1,B_0}
%  {\neg(A_1\Rightarrow B_1),\dots,\neg(A_n\Rightarrow B_n),(A_0\Rightarrow B_0)}
%\end{equation*}
This illustrates two important points that play a role in the
optimisation strategies described below. First, the rule has a large
branching degree both existentially and universally -- we have to
check that there \emph{exists} a rule proving some sequent such that
\emph{all} its premises are provable, and hence we have to guess one
of exponentially many subsequents to match the rule and then attempt
to prove linearly many premises; compare this to linearly many rules
and a constant number of premises per rule (namely, $1$) in the case
of $\textbf{K}$ shown above. Secondly, many of the premises are short
sequents. This will be exploited in our memoisation strategy. We note
that \emph{labelled} sequent calculi for conditional logics have been
designed previously~\cite{OlivettiEA07} and have been implemented in
the CondLean prover~\cite{OlivettiPozzato03}; contrastingly, our
calculus is unlabelled, and turns out to be conceptually simpler. The
comparison of the relative efficiency of labelled and unlabelled
calculi remains an open issue for the time being.

\section{The Algorithm}

According to the framework introduced above, we now devise a generic
algorithm to decide the provability of formulas, which is easily
instantiated to a specific logic by just implementing the relevant modal
rules.


Algorithm~\ref{alg:seq} makes use of the sequent rules in the
following manner: In order to show the \emph{provability} of a formula
$\phi$, the algorithm starts with the sequent $\{\phi\}$ and just
keeps trying to apply all of the sequent rules in $\mathcal{R}_{sc}$ to it,
giving precedence to the linear rules.  Below, we refer to a sequent
as \emph{open} if it has not yet been checked for provability. Under
suitable tractability assumptions on the rule set as
in~\cite{SchroderPattinson09,PattinsonSchroder08b}, this algorithm
realises the theoretical upper bound $\mathit{PSPACE}$. It is the
starting point of the proof search algorithms employed in CoLoSS,
being essentially a sequent reformulation of the algorithm described
in~\cite{CalinEA09}, where it is easily verified that correctness and termination
of the algorithm are preserved by this reformulation; optimisations of this
algorithm are the subject of the present work. 

%\begin{algorithm}[h]
%\fbox{\parbox{\myboxwidth}{
\begin{alg}(Decide provability of a sequent $\Gamma$)
\begin{upshape}
 \begin{algenumerate}
   \item Try all possible applications of rules from $\mathcal{R}_{sc}$ to $\Gamma$, 
     giving precedence to linear rules. For every such rule application,
     perform the following steps, and answer `provable' in case these steps
     succeed for one of the rule applications.
   \item Let $\Lambda$ denote the  set of premises  arising from the rule 
     application.
   \item Check recursively that every sequent in $\Lambda$ is provable.
  \end{algenumerate}
\end{upshape}
\label{alg:seq}
\end{alg}
%}}
%\end{algorithm}

%\begin{proposition}
%\begin{upshape}
%Let $\mathcal{R}_{sc}$ be strictly one-step complete, closed under contraction,
%and PSPACE-tractable. Then Algorithm~\ref{alg:seq} is sound and complete w.r.t. provability
%and it is in PSPACE.
%\end{upshape}
%\end{proposition}%

%\begin{proof}
%We just note, that Algorithm~\ref{alg:seq} is an equivalent implementation of the algorithm proposed
%in~\cite{SchroderPattinson09}. For more details, refer to ~\cite{SchroderPattinson09}, Theorem 6.13.
%\end{proof}

\subsection{The conditional logic instance}

The genericity of the introduced sequent calculus allows us to easiliy
create instantiations of Algorithm~\ref{alg:seq} for a large variety
of modal logics: 

\noindent For instance it has been shown in~\cite{PattinsonSchroder09a} that the
complexity of \textbf{CKCEM} is $\mathit{coNP}$, using a dynamic
programming approach in the spirit of~\cite{Vardi89}; in fact, this
was the original motivation for exploring the optimisation strategies
pursued here.
Due to this reason, we restrict ourselves to the examplary
conditional logic \textbf{CKCEM} for the remainder of this section;
slightly adapted versions of the optimisation will work for other
conditional logics. \textbf{CKCEM} is characterised by the additional
axioms of \emph{conditional excluded middle} $(A\Rightarrow
B)\lor(A\Rightarrow\neg B)$, which to absorb cut and contraction is
integrated in the rule for \textbf{CK} as shown in Figure~\ref{fig:modalCKCEM}.

%By setting $\mathcal R^m_{sc}$ for to the
%rule shown in Figure~\ref{fig:modalCKCEM},
% (where $A_a = A_b$ is
%shorthand for $A_a\rightarrow A_b\wedge A_b\rightarrow A_a$), 
%we obtain an algorithm for deciding provability (and satisfiability)
%of conditional logic. 

\begin{figure}[h!]
  \begin{center}
    \begin{tabular}{| c |}
    \hline
      \\[-5pt]
      (\textsc {\textbf{CKCEM}})\inferrule{A_0 = A_1;\ldots;A_n = A_0 \\ B_0,\ldots, B_j,\neg B_{j+1},\ldots,\neg B_n}
                      {\Gamma, (A_0\Rightarrow B_0),\ldots,(A_j\Rightarrow B_j),
                      \neg(A_{j+1}\Rightarrow B_{j+1}),\ldots,\neg(A_n\Rightarrow B_n) } \\[-5pt]
      \\
    \hline
    \end{tabular}
  \end{center}
  \caption{The modal rule $\textbf{CKCEM}$ of conditional logic}
  \label{fig:modalCKCEM}
\end{figure}

In the following, we use the notions of \emph{conditional antecedent} and
\emph{conditional consequent} to refer to the parameters of the modal operator of
conditional logic. 

In order to decide using Algorithm~\ref{alg:seq} whether there is an
instance of the modal rule of conditional logic which can be applied
to the actual current sequent, it is necessary to create a preliminary
premise for each possible combination of equalities of all the
premises of the modal operators in this sequent. This results in
$2^n-1$ new premises for a sequent with $n$ top-level modal
operators. 
\begin{example}\label{ex:cond}
For the sequent $\Gamma=\{(A_0\Rightarrow B_0),
(A_1\Rightarrow B_1), (A_2\Rightarrow B_2)\}$, there are $2^3=8$
possible instances of the rule to be tried, corresponding to the
non-empty subsequents of the goal; the premise to be checked for
$I\subseteq\{1,2,3\}$ consists of $A_i=A_j$ for $i,j\in I$, and
$\{B_i\mid i\in I\}$.
\end{example}
\noindent It seems to be a more intelligent approach to first
partition the set of all antecedents of the top-level modal operators
in the current sequent into equivalence classes with respect to
logical equality.  This partition allows for a significant reduction
both of the number of rules to be tried and of the number of premises
to be actually proved for each rule.

\begin{example}
Consider again the sequent from Example~\ref{ex:cond}. By using the examplary
knowledge that $A_0=A_1$, $A_1\neq A_2$ and $A_0\neq A_2$, it is immediate
that there are just two reasonable instations of the modal rule, leading
to the two premises $\{\{B_0,B_1\}\}$ and $\{\{B_2\}\}$. For the
first of these two premises, note that it is not necessary to show the
equivalence of $A_0$ and $A_1$ again.
\end{example}


\noindent In the case of conditional logic, observe the following:
Since the modal antecedents that appear in a formula are not being
changed by any rule of the sequent calculus, it is possible to extract
all possibly relevant antecedents of a formula even \emph{before} the
actual sequent calculus is applied. This allows us to first compute
the equivalence classes of all the relevant antecedents and then feed
this knowledge into the actual sequent calculus, as illustrated next.

\section{The Optimisation}

\begin{definition}
A conditional antecedent of \emph{modal nesting depth} $i$ is a
conditional antecedent which contains at least one antecedent of
modal nesting depth $i-1$ and which contains no antecedent 
of modal nesting depth greater than $i-1$. A
conditional antecedent of nesting depth 0 is an antecedent
that does not contain any further modal operators.
Let $a_i$ denote the set of all conditional antecedents of modal
nesting depth $i$. Further, let $\prems(n)$ denote the set of all
conditional antecedents of modal nesting depth at most $n$ (i.e.
$\prems(n)=\bigcup_{j=1..n}^{} a_j$).
Finally, let $depth(\phi)$ denote the maximal modal nesting in
the formula $\phi$.
\end{definition}

%\begin{example}
%In the formula $\phi=((p_0\Rightarrow p_1) \wedge ((p_2\Rightarrow p_3)\Rightarrow p_4))
%\Rightarrow (p_5\Rightarrow p_6)$,
%the premise $((p_0\Rightarrow p_1) \wedge ((p_2\Rightarrow p_3)\Rightarrow p_4))$ has a
%nesting depth of 2 (and not 1), $(p_0\Rightarrow p_1)$ and $(p_2\Rightarrow p_3)$ both
%have a nesting depth of 1, and finally $p_0$, $p_2$ and $p_4$ all have a nesting depth of 0.
%Furthermore, $\prems(1)=\{p_0,p_2,p_4,(p_0\Rightarrow p_1),(p_2\Rightarrow p_3)\}$
%and $depth(\phi)=3$.
%\end{example}

\begin{definition}
  A set $\mathcal{K}$ of sequents together with a function
  $\eval:\mathcal{K}\rightarrow \{\top,\bot\}$ is called
  a \emph{knowledge base}.
\end{definition}

\noindent We may now construct an optimized algorithm which allows us
to decide provability (and satisfiability) of formulas more
efficiently in some cases. The optimized algorithm is constructed from
two functions (namely from the actual proving function and from the
so-called \emph{pre-proving} function):

%\begin{algorithm}[h] 
\begin{alg}\label{alg:opt}
  (Decide provability of $\phi$ using the knowledge
  base $(\mathcal{K},\eval)$)
{\upshape
 \begin{algenumerate}
 \item\label{step:look-up} If $\Gamma\in\mathcal{K}$, answer `provable' if
   $\eval(\Gamma)=\top$, else `unprovable'. Otherwise:
 \item\label{step:rule} Try all possible applications of rules from
   $\mathcal{RO}_{sc}$ to $\Gamma$, giving precedence to linear rules,
   where $\mathcal{RO}_{sc}$ is an optimised set of rules taking into
   account the knowledge base, as explained below. For every such rule
   application, perform the following steps, and answer `provable' in
   case these steps succeed for one of the rule applications.
   \item Let $\Lambda$ denote the  set of premises  arising from the rule 
     application.
   \item Check recursively that every sequent in $\Lambda$ is provable.
  \end{algenumerate}
}
\label{alg:optSeq}
\end{alg}
%\end{algorithm}

\noindent Algorithm~\ref{alg:optSeq} is very similar to
Algorithm~\ref{alg:seq} but relies on the knowledge base passed to it
and moreover uses a modified set of rules $\mathcal{RO}_{sc}$. The set
of rules $\mathcal{RO}_{sc}$ makes appropriate use of the knowledge
base. It is obtained from $\mathcal{R}_{sc}$ by replacing the modal
rule from Figure~\ref{fig:modalCKCEM} with the modified modal rule
from Figure~\ref{fig:modalCKCEMm}. The point is that the premises
$A_0=\dots=A_n$ are replaced by side conditions representing lookup in
the knowledge base. This improves on standard memoising as embodied by
the lookup operation in step~\ref{step:look-up} of the above algorithm
in that existential branching over potentially applicable rules is
reduced: the rule does not even match the target sequent unless the
equivalence premises are already in the knowledge base. This is still
a complete system due to the way memoising is organised, as explained
below.

 \begin{figure}[h!]
  \begin{center}
    \begin{tabular}{| c |}
    \hline
      \\[-5pt]
     (\textsc {\textbf{CKCEM}$^m$})\inferrule{B_0,\ldots, B_j,\neg B_{j+1},\ldots,\neg B_n}
                     {\Gamma, (A_0\Rightarrow B_0),\ldots,(A_j\Rightarrow B_j),
                     \neg(A_{j+1}\Rightarrow B_{j+1}),\ldots,\neg(A_n\Rightarrow B_n) } \\[-5pt]
     \\
     \hfill ($\bigwedge{}_{i,j\in\{1..n\}}{\eval(A_i=A_j)=\top}$)\\
   \hline
   \end{tabular}
 \end{center}
 \caption{The modified modal rule \textbf{CKCEM}$^m$ of conditional logic}
 \label{fig:modalCKCEMm}
\end{figure}

%\begin{algorithm}[h]

\noindent The knowledge base used in Algorithm~\ref{alg:opt} is
computed in Algorithm~\ref{alg:preprove}. The algorithm proceeds by
dynamic programming with stages corresponding to modal nesting depth,
in the spirit of~\cite{Vardi89}.  Thus, in order to show the
equivalence of two conditional antecedents of nesting depth at most
$i$, we assume that the equivalences $\mathcal{K}_{i}$ between modal
antecedents of nesting depth less than $i$ have already been computed
and the result is stored in $\eval_i$; hence, two antecedents are
equal, if their equivalence is provable by Algorithm~\ref{alg:optSeq}
using only the knowledge base $(\mathcal{K}_{i},\eval_i)$.
\begin{alg}
\begin{upshape}
  Step 1: Take a formula $\phi$ as input. Set $i=0$, $\mathcal{K}_0=\emptyset$, $\eval_0=\emptyset$.\\
  Step 2: Generate the set $\prems_i$ of all conditional antecedents of $\phi$
  of nesting depth at most $i$. If $i<depth(\phi)$ continue
  with Step 3, else set $\mathcal{K}=\mathcal{K}_{i-1}, \eval=\eval_{i-1}$ and continue with Step 4.\\
  Step 3: Let $eq_i$ denote the set of all equalities $A_a = A_b$ for different
  formulas $A_a,A_b\in \prems_i$. Compute
  Algorithm~\ref{alg:optSeq} ($\psi$, $(\mathcal{K}_i,\eval_i)$) for all $\psi\in eq_i$.
  Set $\mathcal{K}_{i+1} = eq_i$, set $i = i + 1$. For each equality $\psi\in eq_i$,
  set $\eval_{i+1}(\psi)=\top$ if the result of Algorithm~\ref{alg:optSeq} was `provable'
  and $\eval_{i+1}(\psi)=\bot$ otherwise. Continue with Step 2.\\
  Step 4: Call Algorithm~\ref{alg:optSeq} ($\phi$, $(\mathcal{K},\eval)$) and return its return
  value as result.
\label{alg:preprove}
\end{upshape}
\end{alg}
%\end{algorithm}


\subsection{Treatment of Requisite Equivalences Only}

Since Algorithm~\ref{alg:preprove} tries to show the logical equivalence of any combination
of two conditional antecedents that appear in $\phi$, it will have worse completion
time than Algorithm~\ref{alg:seq} on many formulas:

\begin{example}
Consider the formula 
\begin{quote}
$\phi=(((p_0\Rightarrow p_1)\Rightarrow p_2)\Rightarrow p_4)\vee
(((p_5\Rightarrow p_6)\Rightarrow p_7)\Rightarrow p_8)$.
\end{quote}
Algorithm~\ref{alg:preprove} will not only
try to show the necessary equivalences between the pairs
$(((p_0\Rightarrow p_1)\Rightarrow p_2), ((p_5\Rightarrow p_6)\Rightarrow p_7))$,
$((p_0\Rightarrow p_1), (p_5\Rightarrow p_6))$ and $(p_0,p_5)$, but it will
also try to show equivalences between any two conditional antecedents (e.g. $(p_0,
(p_5\Rightarrow p_6))$), even though these equivalences will not be needed
during the execution of Algorithm~\ref{alg:optSeq}.
\label{ex:requis}
\end{example}

Based on this observation it is possible to define a set $C$ of classes
of so-called \emph{connected} subformulas of a given formula $\phi$. We require that
each class $c_i\in C$ contains only formulas that are pairwise connected in $\phi$. Given
a formula $\phi$, two subformulas of it are said to be connected in $\phi$ if they \emph{may}
both appear in the conclusion of the application of a modal rule during the course of a proof
of $\phi$. Two subformulas of $\phi$ are said to be \emph{independent} in $\phi$ if they are
not connected.


In the case of conditional logic with conditional excluded middle, two formulas are connected
if they appear at identical positions (modulo different choice of modal operator on each level)
within $\phi$. We use the notions of \emph{classes of connected antecedents} (\emph{classes of
connected consequents}) to refer to the set of those classes which only contain antecedents (consequents,
respectively). 

\begin{example}
Considering again the formula from Example~\ref{ex:requis}, $\phi$ has the following classes
of connected antecedents:
\begin{quote}
$c_1=\{((p_0\Rightarrow p_1)\Rightarrow p_2, (p_5\Rightarrow p_6)\Rightarrow p_7)\}$,
$c_2=\{(p_0\Rightarrow p_1, p_5\Rightarrow p_6)\}$ and $c_3=\{(p_0,p_5)\}$.
\end{quote}
The classes of connected consequents in $\phi$ are as follows:
\begin{quote}
$c_4=\{(p_4, p_8)\}$, $c_5=\{(p_2, p_7)\}$ and $c_6=\{(p_1, p_6)\}$.
\end{quote}
\end{example}


Since two independent conditional antecedents will never appear in the scope of the
same application of the modal rule, it is in no case necessary to show (or
refute) the logical equivalence of independent conditional antecedents. Hence it suffices to focus
our attention to the connected conditional antecedents. It is then obvious that
any possibly requisite equivalence and its truth-value are allready included in $(K,\eval)$
when the main proving is induced. On the other hand, we have to be aware that it
may be the case, that we show equivalences of antecedents which are in fact not needed
(since antecedents may indeed be connected and still it is possible that they never appear together in an application
of the modal rule - this is the case whenever two preceding antecedents are not logically equivalent).

As result of these considerations, we devise Algorithm~\ref{alg:optPreprove},
an improved version of Algorithm~\ref{alg:preprove}. The crucial difference is
that before proving any equivalences, Algorithm~\ref{alg:optPreprove} partitions the
set of all relevant modal antecedents into the set of classes of connected antecedents.
Then the algorithm treats equivalences between any two pairs in each of the classes.
Hence independent pairs of antecedents remain untreated.
%\begin{algorithm}[h]
\begin{alg}
\begin{upshape}
  Step 1: Take a formula $\phi$ as input. Set $i=0$, $\mathcal{K}_0=\emptyset$, $\eval_0=\emptyset$.\\
  Step 2: Generate the set $C_i$ of all \emph{classes of connected} conditional antecedents of $\phi$
  of nesting depth at most $i$. If $i<depth(\phi)$ continue
  with Step 3, else set $\mathcal{K}=\mathcal{K}_{i-1}, \eval=\eval_{i-1}$ and continue with Step 4.\\
  Step 3: For each $c\in C_i$, let $eq_{c}$ denote the set of all equalities $A_a = A_b$ for different
  pairs of formulas $A_a,A_b\in c$. For each $eq_{c}$, compute
  Algorithm~\ref{alg:optSeq} ($\psi$, $(\mathcal{K}_i,\eval_i)$) for all $\psi\in eq_{c}$.
  Set $\mathcal{K}_{i+1} = \bigcup_{c\in C_i} eq_{c}$, set $i = i + 1$. For each equality $\psi\in\mathcal{K}_{i+1}$,
  set $\eval_{i+1}(\psi)=\top$ if the result of Algorithm~\ref{alg:optSeq} was `provable'
  and $\eval_{i+1}(\psi)=\bot$ otherwise. Continue with Step 2.\\
  Step 4: Call Algorithm~\ref{alg:optSeq} ($\phi$, $(\mathcal{K},\eval)$) and return its 
  return value as result.
\label{alg:optPreprove}
\end{upshape}
\end{alg}
%\end{algorithm}

\section{Implementation}

The proposed optimized algorithms have been implemented (using the programming
language Haskell) as part of the generic coalgebraic modal logic satisfiability
solver (\COLOSS\footnote{As already mentioned above, more information about \COLOSS,
a web-interface to the tool and 
the tested benchmarking formulas can be found at \url{http://www.informatik.uni-bremen.de/cofi/CoLoSS/}}).
\COLOSS~provides the general coalgebraic framework in which the generic
sequent calculus is embedded. It is easily possible to instantiate this generic sequent
calculus to specific modal logics, one particular example being conditional logic.
The matching function for conditional logic in \COLOSS~was hence adapted in order to realize
the different optimisations (closely following Algorithms~\ref{alg:seq},~\ref{alg:preprove} and
~\ref{alg:optPreprove}), so that \COLOSS~now provides an efficient algorithm for
deciding the provability (and satisfiability) of conditional logic formulas.

\subsection{Comparing the Proposed Algorithms}
\label{sec:bench}

In order to show the relevance of the proposed optimisations, we devise several classes
of conditional formulas. Each class has a characteristic general shape, defining its
complexity w.r.t. different parts of the algorithms and thus exhibiting specific
advantages or disadvantages of each algorithm:

\begin{itemize}
\item The formula \verb|bloat(|$i$\verb|)| is a full binary tree of depth $i$ (containing $2^i$ different atoms
and $2^i-1$ independent modal antecedents):
\begin{quote}
\verb|bloat(|$i$\verb|)| = $($\verb|bloat(|$i-1$\verb|)|$)\Rightarrow($\verb|bloat(|$i-1$\verb|)|)\\
\verb|bloat(|$0$\verb|)| = $p_{rand}$
\end{quote}
Formulas from this class should show the problematic performance of Algorithm~\ref{alg:preprove} whenever
a formula contains many modal antecedents which appear at different depths. A comparison of the different
algorithms w.r.t. formulas \verb|bloat(|$i$\verb|)| is depicted in Figure~\ref{fig:benchBloat}.
Since Algorithm~\ref{alg:preprove} does not check whether pairs of modal antecedents are independent or connected,
it performs considerably worse than Algorithm~\ref{alg:optPreprove} which only attempts to prove the logical
equivalence of formulas which are not independent. Algorithm~\ref{alg:seq} has the best performance in this
extreme case, as it only has to consider pairs of modal antecedents which actually appear during the course
of a proof. This is the (small) price to pay for the optimisation by dynamic programming.
\end{itemize}

\begin{figure}[!h]
  \begin{center}
\begin{tabular}{| l | r | r | r |}
\hline
$i$ & Algorithm~\ref{alg:seq} & Algorithm~\ref{alg:preprove} & Algorithm~\ref{alg:optPreprove}  \\
\hline
 1 & 0.004s & 0.004s & 0.004s\\
 2 & 0.004s & 0.004s & 0.004s\\
 3 & 0.004s & 0.006s & 0.004s\\
 4 & 0.005s & 0.014s & 0.005s\\
 5 & 0.005s & 0.108s & 0.006s\\
 6 & 0.007s & 1.149s & 0.008s\\
 7 & 0.009s & 11.993s & 0.015s\\
 \hline
 \end{tabular}
  \end{center}
  \caption{Results for bloat($i$)}
  \label{fig:benchBloat}
\end{figure}

\begin{itemize}
\item The formula \verb|conjunct(|$i$\verb|)| is just an $i$-fold conjunction of a specific formula $A$:
\begin{quote}
\verb|conjunct(|$i$\verb|)| = $A_1\wedge\ldots\wedge A_i$\\
$A=(((p_1\vee p_0)\Rightarrow p_2)\vee((p_0\vee p_1)\Rightarrow p_2))\vee\neg(((p_0\vee p_1)\Rightarrow p_2)\vee((p_1\vee p_0)\Rightarrow p_2))$)
\end{quote}
This class consists of formulas which contain logically (but not sytactically) equivalent antecedents.
As $i$ increases, so does the amount of appearances of identical modal antecedents in different positions
of the considered formula. A comparison of the different algorithms w.r.t. formulas \verb|conjunct(|$i$\verb|)| is depicted in
Figure~\ref{fig:benchConjunct}. It is obvious that the optimized algorithms perform considerably better than the unoptimized
Algorithm~\ref{alg:seq}. The reason for this is that Algorithm~\ref{alg:seq} repeatedly proves equivalences between the same
pairs of modal antecedents. The optimized algorithms on the other hand are equipped with knowledge about the modal antecedents,
so that these equivalences have to be proved only once. However, even the runtime of the optimized algorithms is exponential in $i$,
due to the exponentially increasing complexity of the underlying propositional formula. Note that the use of propositional tautologies (such as
$A \leftrightarrow (A\wedge A) $ in this case) would help to greatly reduce the computing time for \verb|conjunct(|$i$\verb|)|.
Optimisation of propositional reasoning is not the scope of this paper though, thus we devise the following examplary class of formulas
(for which propositional tautologies would not help):
\end{itemize}

\begin{figure}[!h]
  \begin{center}
\begin{tabular}{| l | r | r | r |}
\hline
$i$ & Algorithm~\ref{alg:seq} & Algorithm~\ref{alg:preprove} & Algorithm~\ref{alg:optPreprove}  \\
\hline
 1 & 0.005s & 0.005s & 0.005s\\
 2 & 0.006s & 0.006s & 0.005s\\
 3 & 0.009s & 0.006s & 0.006s\\
 4 & 0.056s & 0.007s & 0.007s\\
 5 & 1.334s & 0.012s & 0.009s\\
 6 & 42.885s & 0.035s & 0.019s\\
 7 & $>$600.000s & 0.217s & 0.092s\\
 8 & $\gg$600.000s & 1.944s & 0.761s\\
 9 & $\gg$600.000s & 17.826s & 6.929s\\
 \hline
 \end{tabular}
  \end{center}
  \caption{Results for conjunct($i$)}
  \label{fig:benchConjunct}
\end{figure}

\begin{itemize}
\item The formula \verb|explode(|$i$\verb|)| contains equivalent but not syntactically
equal and interchangingly nested modal antecedents of depth at most $i$:
\begin{quote}
\verb|explode(|$i$\verb|)| = $X^i_1\vee\ldots\vee X^i_i$\\
$X^i_1=(A^i_1\Rightarrow(\ldots(A^i_i\Rightarrow (c_1\wedge\ldots\wedge c_i))\ldots))$\\
$X^i_j=(A^i_{j\bmod i}\Rightarrow(\ldots(A^i_{(j+(i-1))\bmod i}\Rightarrow \neg c_j)\ldots))$\\
$A^i_j=p_{j \bmod i}\wedge\ldots\wedge p_{(j+(i-1)) \bmod i}$
\end{quote}
This class contains complex formulas for which the unoptimized
algorithm should not be efficient any more: Only the combined
knowledge about all appearing modal antecedents $A^i_j$ allows the
proving algorithm to reach all modal consequents $c_n$, and only the
combined sequent $\{(c_1\wedge\ldots\wedge c_i),\neg c_1,\ldots,\neg
c_i\}$ (containing every appearing consequent) is provable. For
formulas from this class (specifically designed to show the advantages
of optimization by dynamic programming), the optimized algorithms
vastly outperform the unoptimized algorithm (see
Figure~\ref{fig:benchExplode}).
\end{itemize}

\begin{figure}[!h]
  \begin{center}
\begin{tabular}{| l | r | r | r |}
\hline
$i$ & Algorithm~\ref{alg:seq} & Algorithm~\ref{alg:preprove} & Algorithm~\ref{alg:optPreprove}  \\
\hline
 1 & 0.004s & 0.004s & 0.004s\\
 2 & 0.005s & 0.005s & 0.005s\\
 3 & 0.006s & 0.006s & 0.006s\\
 4 & 0.014s & 0.007s & 0.007s\\
 5 & 0.130s & 0.009s & 0.008s\\
 6 & 0.302s & 0.011s & 0.012s\\
 7 & 430.684s& 0.015s & 0.016s\\
 8 & $\gg$600.000s& 0.021s & 0.019s\\
 9 & $\gg$600.000s& 0.029s & 0.022s\\
 \hline
 \end{tabular}
  \end{center}
  \caption{Results for explode($i$)}
  \label{fig:benchExplode}
\end{figure}

The tests were conducted on a Linux PC (Dual Core AMD Opteron 2220S
(2800MHZ), 16GB RAM).  It is obvious that a significant increase of
performance may be obtained through the proposed optimisations. In
general, the performance of the implementation of the proposed
algorithms in the generic reasoner CoLoSS is comparable to dedicated
conditional logic provers such as CondLean; a direct comparison is
presently made difficult by the fact that the benchmarking formulas
used to evaluate CondLean are not listed explicitly
in~\cite{OlivettiPozzato03}.

\section{Generalized Optimisation}

As previously mentioned, the demonstrated optimisation is not restricted to the
case of conditional
modal logics. 

\begin{definition}
If $\Gamma$ is a sequent, we denote the set of all arguments of
top-level modalities from $\Gamma$ by $arg(\Gamma)$.
A \emph{short sequent} is a sequent which consists of just one formula which
itself is a propositional formula over a fixed maximal number of modal arguments
from $arg(\Gamma)$. In the following, we fix the maximal number of modal arguments
in short sequents to be 2.
\end{definition}

The general method of the optimisation then takes the following form:
Let $S_1,\ldots, S_n$ be short sequents and assume that there is a
sound instance (w.r.t the logic at hand) of the generic rule depicted
in Figure~\ref{fig:modalOpt} (where $\mathcal{S}$ is any set of
sequents).

\begin{figure}[!h]
  \begin{center}
    \begin{tabular}{| c |}
    \hline
      \\[-5pt]
(\textsc {\textbf{Opt}}) \inferrule{ S_1 \\ \ldots \\ S_n \\ \mathcal{S} }
                      { \Gamma } \\[-5pt]
      \\
    \hline
    \end{tabular}
  \end{center}
  \caption{The general rule-scheme to which the optimisation may be applied}
  \label{fig:modalOpt}
\end{figure}

Then we devise a final version (Algorithm~\ref{alg:genOptPreprove}) of the
optimized algorithm: Instead of considering only equivalences of conditional antecedents
for pre-proving, we now extend our attention to any short sequents over any modal arguments.

\begin{algorithm}[h]
\begin{alg}
\begin{upshape}
  Step 1: Take a formula $\phi$ as input. Set $i=0$, $\mathcal{K}_0=\emptyset$, $\eval_0=\emptyset$.\\
  Step 2: Generate the set $args_i$ of all modal arguments of $\phi$
  which have nesting depth at most $i$. If $i<depth(\phi)$ continue
  with Step 3, else set $\mathcal{K}=\mathcal{K}_{i-1}, \eval=\eval_{i-1}$ and continue with Step 4.\\
  Step 3: Let $seq_i$ denote the set of all short sequents of form $S_i$ (where $S_i$ is a sequent
  from the premise of rule (\textbf{Opt})) over at most two formulas
  $A_a,A_b\in args_i$. Compute Algorithm~\ref{alg:optSeq} ($\psi$, $(\mathcal{K}_i,\eval_i)$) for all
  $\psi\in seq_i$. Set $\mathcal{K}_{i+1} = seq_i$, set $i = i + 1$. For each short sequent
  $\psi\in seq_i$, set $\eval_{i+1}(\psi)=\top$ if the result of Algorithm~\ref{alg:optSeq} was
  `provable' and $\eval_{i+1}(\psi)=\bot$ otherwise. Continue with Step 2.\\
  Step 4: Call Algorithm~\ref{alg:optSeq} ($\phi$, $(\mathcal{K},\eval)$) and return its return value
  as result.
\end{upshape}
\label{alg:genOptPreprove}
\end{alg}
\end{algorithm}

This new Algorithm~\ref{alg:genOptPreprove} may then be used to decide
provability of formulas, where the employed rule set has to be extended
by the generic modified rule given by Figure~\ref{fig:modModalOpt}.

\begin{figure}[!h]
  \begin{center}
    \begin{tabular}{| c |}
    \hline
      \\[-5pt]
      \hspace{75pt}(\textsc {\textbf{Opt}$^m$}) \inferrule{ \mathcal{S} }
                      { \Gamma }\hspace{75pt}\vspace{3pt}\\[-5pt]
     \hfill ($\bigwedge{}_{i\in\{1..n\}}{\eval(S_i)=\top}$) \\
    \hline
    \end{tabular}
  \end{center}
  \caption{The general optimized rule}
  \label{fig:modModalOpt}
\end{figure}

\begin{example}
The following two cases are instances of the generic optimisation:
\begin{enumerate}
\item (Classical modal Logics / Neighbourhood Semantics) Let $\Gamma =
  \{\Box A = \Box B\}$, $n=1$, $S_1=\{A=B\}$ and
  $\mathcal{S}=\emptyset$. Algorithm~\ref{alg:genOptPreprove} may be
  then applied whenever the following congruence rule is sound in the
  logic at hand:
\begin{quote}
\begin{center}
      (\textsc {\textbf{Opt}$_{Cong}$}) \inferrule{ A=B }
                      { \Box A = \Box B }
  \end{center}
\end{quote}
The according modified version of this rule is as follows:
\begin{quote}
\begin{center}
      (\textsc {\textbf{Opt}$^m_{Cong}$}) \inferrule{ {} }
                      { \Box A = \Box B }
  \end{center}
\end{quote}
with the side-condition $\eval(A=B)=\top$.
\item (Monotone modal logics) By setting $\Gamma = \{\Box A
  \rightarrow \Box B\}$, $n=1$, $S_1=\{A\rightarrow B\}$ and
  $\mathcal{S}=\emptyset$, we may instantiate the generic algorithm to
  the case of modal logics which are monotone w.r.t. their modal
  operator. So assume the following rule to be sound in the considered
  modal logic:
\begin{quote}
\begin{center}
      (\textsc {\textbf{Opt}$_{Mon}$}) \inferrule{ A\rightarrow B }
                      { \Box A \rightarrow \Box B }
  \end{center}
\end{quote}
The according modified version of this rule is as follows:
\begin{quote}
\begin{center}
      (\textsc {\textbf{Opt}$^m_{Mon}$}) \inferrule{ {} }
                      { \Box A \rightarrow \Box B }
  \end{center}
\end{quote}
with the side-condition $\eval(A\rightarrow B)=\top$.

In the case that (\textbf{Opt}$_{Mon}$) is the only modal rule in the
considered logic (i.e. the case of plain monotone modal logic), all
the prove-work which is connected to the modal operator is shifted to
the pre-proving process. In particular, matching with the modal rules
$\mathcal{RO}^m_{sc}$ becomes a mere lookup of the value of $\eval$.
This means that all calls of the Algorithm~\ref{alg:optSeq} correspond
in complexity just to SAT-solving in propositional logic.
Furthermore, Algorithm~\ref{alg:optSeq} will be called $|\phi|$
times. This observation may be generalized:
\end{enumerate}
\label{ex:neighMon}
\end{example}

\begin{remark}
  In the case that all modal rules of the considered logic are
  instances of the generic rule (\textbf{Opt}) with $P=\emptyset$ (as
  seen in Example~\ref{ex:neighMon}), the optimisation does not only
  allow for a reduction of computing time, but it also allows us to
  effectively reduce the sequent calculus to SAT-solving.
  Furthermore, the optimized algorithm will always be as efficient as
  the original one in this case (since every occurence of short
  sequents over $arg(\Gamma)$ which accord to the current
  instantiation of the rule (\textbf{Opt}) will have to be shown or
  refuted even during the course of the original algorithm).
\end{remark}

\section{Conclusion}  

We presented (from a practical point of view) two optimisations for reasoning
in conditional logic:
\begin{itemize}
\item The first optimisation makes use of the concept of dynamic programming
in order to separate the two tasks that showing validity of formulas in conditional
logic consists of: The first task of proving equivalences of antecedents and the second task
of ordinary sequent proving. The use of dynamic programming substantially decreases the
branching breadth of the resulting sequent calculus.
\item The second proposed optimisation introduces a strategy to reduce the amount
of pairs of antecedents whose equivalence has to be considered. This is achieved
by distinguishing between connected and independent pairs of modal arguments.
\end{itemize}
When both optimisations are applied at the same time, a significant
increase in performance of the sequent algorithm for conditional logic
can be observed.  This was shown in Section~\ref{sec:bench} by
considering the results of benchmarking a Haskell implementation (in
the framework of the generic reasoner \COLOSS) of the optimised
algorithms.

It remains as an open question whether the gain in perfomance which is obtained
by optimising the algorithm for conditional logic may be transferred to other
logics by making use of the generic optimisation strategy as described in the
last section.

\bibliographystyle{myabbrv}
\bibliography{coalgml}


\end{document}


\section{Implementation}\label{sec:implemen}

We describe in this section how the integration described in the
previous sections has been implemented. We have used Maude to parse
Maude modules, taking advantage of the reflective capabilities of rewriting
logic \cite{ClavelMeseguerPalomino07}, while the rest of the system
has been implemented in Haskell, the implementation language of \Hets.
%
Section \ref{subsec:abs-syntax}
shows the abstract syntax used to represent Maude modules in Haskell, while
Section \ref{subsec:maude-parser} presents how these data structures
are generated in Maude. Section \ref{sec:logic_imp} shows how
Maude signatures, sentences, and morphisms are obtained, and
Section \ref{sec:imp_dg} explains how
they are introduced into a development graph. Finally, Section \ref{sec:imp_free}
outlines how the freeness constraints are implemented.

\subsection{Abstract syntax}\label{subsec:abs-syntax}
%!TEX root = main.tex

In this section we show how the abstract syntax for Maude specifications
is defined in Haskell. This abstract syntax is based in the Maude grammar presented
in \cite[Chapter 24]{maude-book}.

The main datatype of this abstract syntax is \verb"Spec", that
distinguishes between the different specifications available in Maude:
modules, theories, and views. Although both modules and theories contain
the same information, their semantics are different and need different
constructors:

{\codesize
\begin{verbatim}
data Spec = SpecMod Module
          | SpecTh Module
          | SpecView View
          deriving (Show, Read, Ord, Eq)
\end{verbatim}
}

A \verb"Module" is composed of the identifier of the module, a list
of parameters, and a list of statements:

{\codesize
\begin{verbatim}
data Module = Module ModId [Parameter] [Statement]
            deriving (Show, Read, Ord, Eq)
\end{verbatim}
}

\noindent while a \verb"View" is composed of a module identifier, the
source and target module expressions, and a list of renamings:

{\codesize
\begin{verbatim}
data View = View ModId ModExp ModExp [Renaming]
            deriving (Show, Read, Ord, Eq)
\end{verbatim}
}

The \verb"Parameter" type contains the identifier of the parameter,
a sort (used as the parameter identifier), and its type (which is a
module expression):

{\codesize
\begin{verbatim}
data Parameter = Parameter Sort ModExp
               deriving (Show, Read, Ord, Eq)
\end{verbatim}
}

A \verb"Statement" can be any of the Maude statements:
importation, sort, subsort, and operator declarations, and
equation, membership axiom, and rule statements:

{\codesize
\begin{verbatim}
data Statement = ImportStmnt Import
               | SortStmnt Sort
               | SubsortStmnt SubsortDecl
               | OpStmnt Operator
               | EqStmnt Equation
               | MbStmnt Membership
               | RlStmnt Rule
               deriving (Show, Read, Ord, Eq)
\end{verbatim}
}

Importations consist of a module expression qualified by the type
of import:

{\codesize
\begin{verbatim}
data Import = Including ModExp
            | Extending ModExp
            | Protecting ModExp
            deriving (Show, Read, Ord, Eq)
\end{verbatim}
}

A subsort declaration keeps single relations between sorts, being
the first one the subsort and the second one the supersort:

{\codesize
\begin{verbatim}
data SubsortDecl = Subsort Sort Sort
                 deriving (Show, Read, Ord, Eq)
\end{verbatim}
}

Operator declarations are composed of the identifier of the
operator, a list of types giving the arity of the operator,
a type for its coarity, and a list of attributes:

{\codesize
\begin{verbatim}
data Operator = Op OpId [Type] Type [Attr]
              deriving (Show, Read, Ord, Eq)
\end{verbatim}
}

Membership statements consist of a term, its sort, a list
of conditions, and a list of statement attributes:

{\codesize
\begin{verbatim}
data Membership = Mb Term Sort [Condition] [StmntAttr]
                deriving (Show, Read, Ord, Eq)
\end{verbatim}
}

Equations and rules share the same elements: the lefthand
and righthand terms of the statement, a list of conditions,
and a list of statement attributes:

{\codesize
\begin{verbatim}
data Equation = Eq Term Term [Condition] [StmntAttr]
              deriving (Show, Read, Ord, Eq)

data Rule = Rl Term Term [Condition] [StmntAttr]
          deriving (Show, Read, Ord, Eq)
\end{verbatim}
}

We distinguish between the following module expressions:

\begin{itemize}
\item A single identifier:

{\codesize
\begin{verbatim}
data ModExp = ModExp ModId
\end{verbatim}
}

\item A summation, that keeps the two module
expressions involved:

{\codesize
\begin{verbatim}
            | SummationModExp ModExp ModExp
\end{verbatim}
}

\item A renaming, that contains the module expression renamed
and the list of renamings:

{\codesize
\begin{verbatim}
            | RenamingModExp ModExp [Renaming]
\end{verbatim}
}

\item An instantiation, composed of the module instantiated
and the list of view identifiers applied:

{\codesize
\begin{verbatim}
            | InstantiationModExp ModExp [ViewId]
            deriving (Show, Read, Ord, Eq)
\end{verbatim}
}

The \verb"Renaming" type distinguishes the different renamings
available in Maude:

\end{itemize}

\begin{itemize}

\item Renaming of sorts, that indicates that the first sort identifier
is changed to the second one:

{\codesize
\begin{verbatim}
data Renaming = SortRenaming Sort Sort
\end{verbatim}
}

\item Renaming of labels, where the first label is renamed to the
second one:

{\codesize
\begin{verbatim}
              | LabelRenaming LabelId LabelId
\end{verbatim}
}

\item Renaming of operators, that can be of three kinds: renaming
of operators without profile, with profile, or a map between terms,
as explained in Section \ref{subsec:views}:

{\codesize
\begin{verbatim}
              | OpRenaming1 OpId ToPartRenaming
              | OpRenaming2 OpId [Type] Type ToPartRenaming
              | TermMap Term Term
              deriving (Show, Read, Ord, Eq)
\end{verbatim}
}

\noindent where \verb"ToPartRenaming" specifies the new operator identifier
and the new attributes:

{\codesize
\begin{verbatim}
data ToPartRenaming = To OpId [Attr]
                    deriving (Show, Read, Ord, Eq)
\end{verbatim}
}

The \verb"Condition" type distinguishes between the different conditions
available in Maude, namely equational conditions, membership conditions,
matching conditions, and rewriting conditions:

{\codesize
\begin{verbatim}
data Condition = EqCond Term Term
               | MbCond Term Sort
               | MatchCond Term Term
               | RwCond Term Term
               deriving (Show, Read, Ord, Eq)
\end{verbatim}
}

We define the type \verb"Qid", a synonym of \verb"Token" that
will be used for identifiers:

{\codesize
\begin{verbatim}
type Qid = Token
\end{verbatim}
}

Terms are always represented in prefix notation. Notice that
the case of an operator applied to a list of terms is slightly different
to the Maude grammar because it also includes the type of the term.
It will be used later in the implementation to rename operators whose
profile has been specified:

{\codesize
\begin{verbatim}
data Term = Const Qid Type
          | Var Qid Type
          | Apply Qid [Term] Type
          deriving (Show, Read, Ord, Eq)
\end{verbatim}
}

Finally, the \verb"Type" distinguishes between sorts and kinds:

{\codesize
\begin{verbatim}
data Type = TypeSort Sort
          | TypeKind Kind
          deriving (Show, Read, Ord, Eq)
\end{verbatim}
}

\end{itemize}


























\subsection{Maude parsing}\label{subsec:maude-parser}
%!TEX root = main.tex

In this section we explain how the Maude specifications introduced in
\Hets are parsed in order to obtain a term following the abstract syntax
described in
the previous section. We are able to implement this parsing in Maude
itself thanks to Maude metalevel \cite[Chapter 14]{maude-book}, a
module that allows the programmer to use Maude entities such as modules, equations,
or rules as usual data by efficiently implementing the \emph{reflective}
capabilities of rewriting logic \cite{ClavelMeseguerPalomino07}.

The function \verb"haskellify" receives a module (the first parameter
stands for the original module, while the second one contains the
flattened one) and returns a list of quoted identifiers creating an
object of type \verb"Spec", that can be read by Haskell since this data
type derives the class \verb"Read":

{\codesize
\begin{verbatim}
  op haskellify : Module Module -> QidList .
  ceq haskellify(M, M') = 
      'SpecMod '`( 'Module haskellifyHeader(H) ' ' 
      '`[ haskellifyImports(IL) comma(IL, SS)
          haskellifySorts(SS) comma(IL, SS, SSDS)
          haskellifySubsorts(SSDS) comma(IL, SS, SSDS, ODS)
          haskellifyOpDeclSet(M', ODS) comma(IL, SS, SSDS, ODS, MAS)
          haskellifyMembAxSet(M', MAS) comma(IL, SS, SSDS, ODS, MAS, EqS)
          haskellifyEqSet(M', EqS) '`] '`) '\n '@#$endHetsSpec$#@ '\n
    if fmod H is IL sorts SS . SSDS ODS MAS EqS endfm := M .
\end{verbatim}
}

This function prints the keyword \verb"SpecMod" and uses the \verb"haskellify"
auxiliary functions to print the different parts of the module.
The functions \verb"comma" introduce a comma whenever it is necessary.
Since all the ``haskellify'' functions are very similar, we describe
them by using \verb"haskellifyImports" as example. This function traverses
all the imports in the list and applies the auxiliary function
\verb"haskellifyImport" to each of them:

{\codesize
\begin{verbatim}
  op haskellifyImports : ImportList -> QidList .
  eq haskellifyImports(nil) = nil .
  eq haskellifyImports(I IL) = 'ImportStmnt ' '`( haskellifyImport(I) '`)
                               comma(IL) haskellifyImports(IL) .
\end{verbatim}
}

This auxiliary function distinguishes between the importation modes,
using the appropriate keyword for each of them:

{\codesize
\begin{verbatim}
  op haskellifyImport : Import -> QidList .
  eq haskellifyImport(protecting ME .) = 'Protecting haskellifyME(ME) .
  eq haskellifyImport(including ME .) = 'Including haskellifyME(ME) .
  eq haskellifyImport(extending ME .) = 'Extending haskellifyME(ME) .
\end{verbatim}
}

\noindent where \verb"haskellifyME" is in charge of printing the 
module expression. When it is just an identifier, it prints it 
preceded by the word \verb"ModId":

{\codesize
\begin{verbatim}
  op haskellifyME : ModuleExpression -> QidList .
  eq haskellifyME(Q) = ' '`( 'ModExp ' '`( 'ModId qid2token(Q) '`) '`) ' .
\end{verbatim}
}

The summation module expression recursively prints the summands, and uses
the keyword \verb"SummationModExp" to indicate the type of module expression:

{\codesize
\begin{verbatim}
  eq haskellifyME(ME + ME') = ' '`( 'SummationModExp haskellifyME(ME) 
                              haskellifyME(ME')  '`) ' .
\end{verbatim}
}

To print a renaming we recursively apply \verb"haskellifyME" for the inner
module expression and then we use the auxiliary function \verb"haskellifyMaps" to 
print the renamings. In this case we use the constant \verb"no-module" as argument
because it will only be used when parsing mappings from views, since it may
be needed to parse terms:

{\codesize
\begin{verbatim}
  eq haskellifyME(ME * (RNMS)) = ' '`( 'RenamingModExp haskellifyME(ME) 
                                 '`[ haskellifyMaps(no-module, no-module, RNMS) '`]  '`) ' .
\end{verbatim}
}

Finally, an instantiation is printed by using an auxiliary function \verb"haskellifyPL"
in charge of the parameters:

{\codesize
\begin{verbatim}
  eq haskellifyME(ME {PL}) = ' '`( 'InstantiationModExp haskellifyME(ME) 
                             '`[ haskellifyPL(PL)  '`]  '`) ' .
\end{verbatim}
}


\subsection{Logic}\label{sec:logic_imp}
\input{logic-imp}

\subsection{Development graph}\label{sec:imp_dg}
%!TEX root = main.tex

We describe in this section the main functions used to draw the
development graph for Maude specifications. The most important function is
\verb"anaMaudeFile", that receives a record of all the options received
from the command line (of type \verb"HetcatsOpts") and the path of
the Maude file to be parsed and returns a pair with the library
name and its environment. This environment contains two development
graphs, the first one containing the modules used in the Maude prelude
and another one with the user specification:

{\codesize
\begin{verbatim}
anaMaudeFile :: HetcatsOpts -> FilePath -> IO (Maybe (LibName, LibEnv))
anaMaudeFile _ file = do
    (dg1, dg2) <- directMaudeParsing file
    let ln = emptyLibName file
        lib1 = Map.singleton preludeLib $
                 computeDGraphTheories Map.empty $ markFree Map.empty $
                 markHiding Map.empty dg1
        lib2 = Map.insert ln
                 (computeDGraphTheories lib1 $ markFree lib1 $
                 markHiding lib1 dg2) lib1
    return $ Just (ln, lib2)
\end{verbatim}
}

This environment is computed with the function \verb"directMaudeParsing", that
receives the path introduced by the user and returns a pair of development graphs.
These graphs are obtained with the function \verb"maude2DG", that receives
the predefined specifications (obtained with \verb"predefinedSpecs")
and the user defined specifications (obtained with \verb"traverseSpecs"):

{\codesize
\begin{verbatim}
directMaudeParsing :: FilePath -> IO (DGraph, DGraph)
directMaudeParsing fp = do
  ml <- getEnvDef "MAUDE_LIB" ""
  if null ml then error "environment variable MAUDE_LIB is not set" else do
    ns <- parse fp
    let ns' = either (const []) id ns
    (hIn, hOut, hErr, procH) <- runMaude
    exitCode <- getProcessExitCode procH
    case exitCode of
      Nothing -> do
              hPutStrLn hIn $ "load " ++ fp
              hFlush hIn
              hPutStrLn hIn "."
              hFlush hIn
              hPutStrLn hIn "in Maude/hets.prj"
              psps <- predefinedSpecs hIn hOut
              sps <- traverseSpecs hIn hOut ns'
              (ok, errs) <- getErrors hErr
              if ok
                  then do
                        hClose hIn
                        hClose hOut
                        hClose hErr
                        return $ maude2DG psps sps
                  else do
                        hClose hIn
                        hClose hOut
                        hClose hErr
                        error errs
      Just ExitSuccess -> error "maude terminated immediately"
      Just (ExitFailure i) -> error $ "calling maude failed with exitCode: " ++ show i
\end{verbatim}
}

The function \verb"maude2DG" first computes the data structures associated to the
predefined specifications and then uses them to compute the development
graph related to the specifications introduced by the user. These
data structures are computed with \verb"insertSpecs":

{\codesize
\begin{verbatim}
maude2DG :: [Spec] -> [Spec] -> (DGraph, DGraph)
maude2DG psps sps = (dg1, dg2)
   where (_, tim, vm, tks, dg1) = insertSpecs psps emptyDG Map.empty 
                                              Map.empty Map.empty [] emptyDG
         (_,_, _, _, dg2) = insertSpecs sps dg1 tim Map.empty vm tks emptyDG
\end{verbatim}
}


Before describing this function,
we briefly explain the data structures used during the generation of
the development graph:

\begin{itemize}

\item The type \verb"ParamSort" defines a pair with a symbol representing
a sort and a list of tokens indicating the parameters present in the sort,
so for example the sort \verb"List{X, Y}" generates the pair
\verb"(List{X, Y}, [X,Y])":

{\codesize
\begin{verbatim}
type ParamSort = (Symbol, [Token])
\end{verbatim}
}

\item The information of each node introduced
in the development graph is stored in the tuple \verb"ProcInfo", that
contains the following information:

\begin{itemize}
\item The identifier of the node.
\item The signature of the node.
\item A list of symbols standing for the sorts that are not instantiated.
\item A list of triples with information about the parameters of the
specification, namely the name of the parameter, the name of the theory,
and the list of not instantiated sorts from this theory.
\item A list with information about the parameterized sorts.
\end{itemize}

{\codesize
\begin{verbatim}
type ProcInfo = (Node, Sign, Symbols, [(Token, Token, Symbols)], [ParamSort])
\end{verbatim}
}

\item Each \verb"ProcInfo" tuple is associated to its corresponding module
expression in the \verb"TokenInfoMap" map:

{\codesize
\begin{verbatim}
type TokenInfoMap = Map.Map Token ProcInfo
\end{verbatim}
}

\item When a module expression is parsed a \verb"ModExpProc" tuple is
returned, containing the following information:

\begin{itemize}
\item The identifier of the module expression.
\item The \verb"TokenInfoMap" structure updated with the data
in the module expression.
\item The morphism associated to the module expression.
\item The list of sorts parameterized in this module expression.
\item The development graph thus far.
\end{itemize}

{\codesize
\begin{verbatim}
type ModExpProc = (Token, TokenInfoMap, Morphism, [ParamSort], DGraph)
\end{verbatim}
}

\item When parsing a list of importation statements we return a
\verb"ParamInfo" tuple, containing:

\begin{itemize}
\item The list of parameter information: the name of the parameter,
the name of the theory, and the sorts that are not instantiated.
\item The updated \verb"TokenInfoMap" map.
\item The list of morphisms associated with each parameter.
\item The updated development graph.
\end{itemize}

{\codesize
\begin{verbatim}
type ParamInfo = ([(Token, Token, Symbols)], TokenInfoMap, [Morphism], DGraph)
\end{verbatim}
}

\item Data about views is kept in a separated way from data about theories
and modules. The \verb"ViewMap" map associates to each view identifier a
tuple with:

\begin{itemize}
\item The identifier of the target node of the view.
\item The morphism generated by the view.
\item The list of renamings that generated the morphism.
\item A Boolean value indicating whether the target is a theory
(\verb"True") or a module (\verb"False").
\end{itemize}

{\codesize
\begin{verbatim}
type ViewMap = Map.Map Token (Node, Token, Morphism, [Renaming], Bool)
\end{verbatim}
}

\item Finally, we describe the tuple \verb"InsSpecRes",
used to return the data structures
updated when a specification or a view is introduced in the development
graph. It contains:

\begin{itemize}
\item Two values of type \verb"TokenInfoMap". The first one includes all
the information related to the specification, including the one from the
predefined modules, while the
second one only contains information related to the current development
graph.
\item The updated \verb"ViewMap".
\item A list of tokens indicating the theories introduced thus far.
\item The new development graph.
\end{itemize}

{\codesize
\begin{verbatim}
type InsSpecRes = (TokenInfoMap, TokenInfoMap, ViewMap, [Token], DGraph)
\end{verbatim}
}

\end{itemize}

The function \verb"insertSpecs" traverses the specifications updating the
data structures and the development graph with \verb"insertSpec":

{\codesize
\begin{verbatim}
insertSpecs :: [Spec] -> DGraph -> TokenInfoMap -> TokenInfoMap -> ViewMap -> [Token] -> DGraph
               -> InsSpecRes
insertSpecs [] _ ptim tim vm tks dg = (ptim, tim, vm, tks, dg)
insertSpecs (s : ss) pdg ptim tim vm ths dg = insertSpecs ss pdg ptim' tim' vm' ths' dg'
              where (ptim', tim', vm', ths', dg') = insertSpec s pdg ptim tim vm ths dg
\end{verbatim}
}

The behavior of \verb"insertSpec" is different for each type of Maude
specification. When the introduced specification is a module, the
following actions are performed:

\begin{itemize}
\item The parameters are parsed:

\begin{itemize}
\item The list of parameter declarations is obtained with the auxiliary
function \verb"getParams".
\item These declarations are processed with \verb"processParameters",
that returns a tuple of type \verb"ParamInfo" shown above.
\item Given the parameters names, we traverse the list of sorts to check
whether the module defines parameterized sorts with \verb"getSortsParameterizedBy".
\item The links between the theories in the parameters and the current module
are created with \verb"createEdgesParams".
\end{itemize}

\item The importations are handled:

\begin{itemize}
\item The importation statements are obtained with \verb"getImportsSorts".
Although this function also returns the sorts declared in the module, in
this case they are not needed and its value is ignored.
\item These importations are handled by \verb"processImports", that
returns a list containing the information of each parameter.
\item The definition links generated by the imports are created with
\verb"createEdgesImports".
\end{itemize}

\item The final signature is obtained with \verb"sign_union_morphs"
by merging the signature in the current module with the ones obtained
from the morphisms from the parameters and the imports.

\end{itemize}

{\codesize
\begin{verbatim}
insertSpec :: Spec -> DGraph -> TokenInfoMap -> TokenInfoMap -> ViewMap -> [Token] -> DGraph
              -> InsSpecRes
insertSpec (SpecMod sp_mod) pdg ptim tim vm ths dg = (ptimUp, tim5, vm, ths, dg6)
      where ps = getParams sp_mod
            (il, _) = getImportsSorts sp_mod
            up = incPredImps il pdg (ptim, tim, dg)
            (ptimUp, timUp, dgUp) = incPredParams ps pdg up
            (pl, tim1, morphs, dg1) = processParameters ps timUp dgUp
            top_sg = Maude.Sign.fromSpec sp_mod
            paramSorts = getSortsParameterizedBy (paramNames ps) (Set.toList $ sorts top_sg
            ips = processImports tim1 vm dg1 il
            (tim2, dg2) = last_da ips (tim1, dg1)
            sg = sign_union_morphs morphs $ sign_union top_sg ips
            ext_sg = makeExtSign Maude sg
            nm_sns = map (makeNamed "") $ Maude.Sentence.fromSpec sp_mod
            sens = toThSens nm_sns
            gt = G_theory Maude ext_sg startSigId sens startThId
            tok = HasName.getName sp_mod
            name = makeName tok
            (ns, dg3) = insGTheory dg2 name DGBasic gt
            tim3 = Map.insert tok (getNode ns, sg, [], pl, paramSorts) tim2
            (tim4, dg4) = createEdgesImports tok ips sg tim3 dg3
            dg5 = createEdgesParams tok pl morphs sg tim4 dg4
            (_, tim5, dg6) = insertFreeNode tok tim4 morphs dg5
\end{verbatim}
}

When the specification inserted is a theory the process varies slightly:

\begin{itemize}
\item Theories cannot be parameterized in Core Maude, so the parameter
handling is not required.
\item The specified sorts have to be qualified with the parameter
name when used in a parameterized module. These sorts are extracted
with \verb"getImportsSorts" and kept in the corresponding field of
\verb"TokenInfoMap".
\end{itemize}

{\codesize
\begin{verbatim}
insertSpec (SpecTh sp_th) pdg ptim tim vm ths dg = (ptimUp, tim3, vm, tok : ths, dg3)
      where (il, ss1) = getImportsSorts sp_th
            (ptimUp, timUp, dgUp) = incPredImps il pdg (ptim, tim, dg)
            ips = processImports timUp vm dgUp il
            ss2 = getThSorts ips
            (tim1, dg1) = last_da ips (tim, dg)
            sg = sign_union (Maude.Sign.fromSpec sp_th) ips
            ext_sg = makeExtSign Maude sg
            nm_sns = map (makeNamed "") $ Maude.Sentence.fromSpec sp_th
            sens = toThSens nm_sns
            gt = G_theory Maude ext_sg startSigId sens startThId
            tok = HasName.getName sp_th
            name = makeName tok
            (ns, dg2) = insGTheory dg1 name DGBasic gt
            tim2 = Map.insert tok (getNode ns, sg, ss1 ++ ss2, [], []) tim1
            (tim3, dg3) = createEdgesImports tok ips sg tim2 dg2
\end{verbatim}
}

The introduction of views into the development graph follows these steps:

\begin{itemize}
\item The function \verb"isInstantiated" checks whether the target of the
view is a theory or a module. This value will be used to decide whether the
sorts have to be qualified when this is view is used.
\item A morphism is generated between the signatures of the source and
target specifications.
\item If there is a renaming between terms the function \verb"sign4renamings"
generates the extra signature and sentences needed. These values, kept in
\verb"new_sign" and \verb"new_sens" are used to create an inner node with
the function \verb"insertInnerNode".
\item Finally, a theorem link stating the proof obligations generated by
the view is introduced between the source and the target of the view with
\verb"insertThmEdgeMorphism".
\end{itemize}

{\codesize
\begin{verbatim}
insertSpec (SpecView sp_v) pdg ptim tim vm ths dg = (ptimUp, tim3, vm', ths, dg4)
      where View name from to rnms = sp_v
            (ptimUp, timUp, dgUp) = incPredView from to pdg (ptim, tim, dg)
            inst = isInstantiated ths to
            tok_name = HasName.getName name
            (tok1, tim1, morph1, _, dg1) = processModExp timUp vm dgUp from
            (tok2, tim2, morph2, _, dg2) = processModExp tim1 vm dg1 to
            (n1, _, _, _, _) = fromJust $ Map.lookup tok1 tim2
            (n2, _, _, _, _) = fromJust $ Map.lookup tok2 tim2
            morph = fromSignsRenamings (target morph1) (target morph2) rnms
            morph' = fromJust $ maybeResult $ compose morph1 morph
            (new_sign, new_sens) = sign4renamings (target morph1) (sortMap morph) rnms
            (n3, tim3, dg3) = insertInnerNode n2 tim2 tok2 morph2 new_sign new_sens dg2
            vm' = Map.insert (HasName.getName name) (n3, tok2, morph', rnms, inst) vm
            dg4 = insertThmEdgeMorphism tok_name n3 n1 morph' dg3
\end{verbatim}
}

We describe now the main auxiliary functions used above.
Module expressions are parsed following the guidelines outlined in
Section \ref{subsec:me}:

\begin{itemize}
\item When the module expression is a simple identifier its signature
and its parameterized sorts are extracted from the \verb"TokenInfoMap"
and returned, while the generated morphism is an inclusion:

{\codesize
\begin{verbatim}
processModExp :: TokenInfoMap -> ViewMap -> DGraph -> ModExp -> ModExpProc
processModExp tim _ dg (ModExp modId) = (tok, tim, morph, ps, dg)
                     where tok = HasName.getName modId
                           (_, sg, _, _, ps) = fromJust $ Map.lookup tok tim
                           morph = Maude.Morphism.inclusion sg sg
\end{verbatim}
}

\item The parsing of the summation expression performs the following
steps:

\begin{itemize}
\item The information about the module expressions is recursively
computed with \verb"processModExp".
\item The signature of the resulting module expression is obtained
with the \verb"union" of signatures.
\item The morphism generated by the summation is just an inclusion.
\item A new node for the summation is introduced with \verb"insertNode".
\item The target signature of the obtained morphisms is substituted
by this new signature with \verb"setTarget".
\item These new morphisms are used to generate the links between the
summation and its summands in \verb"insertDefEdgeMorphism".
\end{itemize}

{\codesize
\begin{verbatim}
processModExp tim vm dg (SummationModExp modExp1 modExp2) = (tok, tim3, morph, ps', dg5)
          where (tok1, tim1, morph1, ps1, dg1) = processModExp tim vm dg modExp1
                (tok2, tim2, morph2, ps2, dg2) = processModExp tim1 vm dg1 modExp2
                ps' = deleteRepeated $ ps1 ++ ps2
                tok = mkSimpleId $ concat ["{", show tok1, " + ", show tok2, "}"]
                (n1, _, ss1, _, _) = fromJust $ Map.lookup tok1 tim2
                (n2, _, ss2, _, _) = fromJust $ Map.lookup tok2 tim2
                ss1' = translateSorts morph1 ss1
                ss2' = translateSorts morph1 ss2
                sg1 = target morph1
                sg2 = target morph2
                sg = Maude.Sign.union sg1 sg2
                morph = Maude.Morphism.inclusion sg sg
                morph1' = setTarget sg morph1
                morph2' = setTarget sg morph2
                (tim3, dg3) = insertNode tok sg tim2 (ss1' ++ ss2') [] dg2
                (n3, _, _, _, _) = fromJust $ Map.lookup tok tim3
                dg4 = insertDefEdgeMorphism n3 n1 morph1' dg3
                dg5 = insertDefEdgeMorphism n3 n2 morph2' dg4
\end{verbatim}
}

\item The renaming module expression recursively parses the inner expression, computes the morphism from the given renamings with \verb"fromSignRenamings",
taking special care of the renaming of the parameterized sorts with
\verb"applyRenamingParamSorts". Once the values are computed, the final morphism
is obtained from the composition of the morphisms computed for the inner
expression and the one computed from the renamings:


{\codesize
\begin{verbatim}
processModExp tim vm dg (RenamingModExp modExp rnms) = (tok, tim', comp_morph, ps', dg')
              where (tok, tim', morph, ps, dg') = processModExp tim vm dg modExp
                    morph' = fromSignRenamings (target morph) rnms
                    ps' = applyRenamingParamSorts (sortMap morph') ps
                    comp_morph = fromJust $ maybeResult $ compose morph morph'
\end{verbatim}
}

\item The parsing of the instantiation module expression works as follows:

\begin{itemize}
\item The information of the instantiated parameterized module is obtained
with \verb"processModExp".
\item The parameter names are obtained by applying \verb"fstTpl", that
extracts the first component of a triple, to the information about the
parameters of the parameterized module.
\item Parameterized sorts are instantiated with \verb"instantiateSorts",
that returns the new parameterized sorts, in case the target of the view
is a theory, and the morphism associated.
\item The view identifiers are processed with \verb"processViews". This
function returns the token identifying the list of views, the morphism
to be applied from the parameterized module, a list of pairs of nodes
and morphisms, indicating the morphism that has to be used in the link
from each view, and a list with the updated information about the
parameters due to the views with theories as target.
\item The morphism returned is the inclusion morphism.
\item The links between the targets of the views and the expression
are created with \verb"updateGraphViews". 
\end{itemize}

{\codesize
\begin{verbatim}
processModExp tim vm dg (InstantiationModExp modExp views) = 
                                        (tok'', tim'', final_morph, new_param_sorts, dg'')
       where (tok, tim', morph, paramSorts, dg') = processModExp tim vm dg modExp
             (_, _, _, ps, _) = fromJust $ Map.lookup tok tim'
             param_names = map fstTpl ps
             view_names = map HasName.getName views
             (new_param_sorts, ps_morph) = instantiateSorts param_names 
                                                            view_names vm morph paramSorts
             (tok', morph1, ns, deps) = processViews views (mkSimpleId "") tim' 
                                                     vm ps (ps_morph, [], [])
             tok'' = mkSimpleId $ concat [show tok, "{", show tok', "}"]
             sg2 = target morph1
             final_morph = Maude.Morphism.inclusion sg2 sg2
             (tim'', dg'') = if Map.member tok'' tim
                             then (tim', dg')
                             else updateGraphViews tok tok'' sg2 morph1 ns tim' deps dg'
\end{verbatim}
}

\end{itemize}

We present the function \verb"insertNode" to describe how the nodes are
introduced into the development graph. This function receives the
identifier of the node, its signature,\footnote{Note that when the
function \texttt{insertNode} is used there are not sentences.}
the \verb"TokenInfoMap" map, a list of sorts, and information
about the parameters and returns the updated map and the new development
graph. First, it checks whether
the node is already in the development graph. If it is in the graph,
the current map and graph are returned. In other case, the extended
signature is computed with \verb"makeExtSign" and used to create a graph
theory that will be inserted with \verb"insGTheory", obtaining the new
node information and the new development graph. Finally, the map is
updated with the information received as parameter and the node identifier
obtained when the node was introduced:

{\codesize
\begin{verbatim}
insertNode :: Token -> Sign -> TokenInfoMap -> Symbols -> [(Token, Token, Symbols)]
              -> DGraph -> (TokenInfoMap, DGraph)
insertNode tok sg tim ss deps dg = if Map.member tok tim
                     then (tim, dg)
                     else let
                            ext_sg = makeExtSign Maude sg
                            gt = G_theory Maude ext_sg startSigId noSens startThId
                            name = makeName tok
                            (ns, dg') = insGTheory dg name DGBasic gt
                            tim' = Map.insert tok (getNode ns, sg, ss, deps, []) tim
                          in (tim', dg')
\end{verbatim}
}

The function \verb"insertDefEdgeMorphism" describes how the definition links
are introduced into the
development graph. It receives the identifier of the source and target
nodes, the morphism to be used in the link, and the current development
graph. The morphism is transformed into a development graph morphism
indicating the current logic (\verb"Maude") and the type (\verb"globalDef")
and is introduced in the development graph with \verb"insLEdgeDG":

{\codesize
\begin{verbatim}
insertDefEdgeMorphism :: Node -> Node -> Morphism -> DGraph -> DGraph
insertDefEdgeMorphism n1 n2 morph dg = snd $ insLEdgeDG (n2, n1, edg) dg
                     where mor = G_morphism Maude morph startMorId
                           edg = globDefLink (gEmbed mor) SeeTarget
\end{verbatim}
}

Theorem links are introduced with \verb"insertThmEdgeMorphism"
in the same way, but specifying with
\verb"globalThm" that the link is a theorem link. This function receives
as extra argument the name of the view generating the proof obligations,
that is used to name the link:

{\codesize
\begin{verbatim}
insertThmEdgeMorphism :: Token -> Node -> Node -> Morphism -> DGraph -> DGraph
insertThmEdgeMorphism name n1 n2 morph dg = snd $ insLEdgeDG (n2, n1, edg) dg
                     where mor = G_morphism Maude morph startMorId
                           edg = defDGLink (gEmbed mor) globalThm
                                 (DGLinkView name $ Fitted [])
\end{verbatim}
}

The function \verb"insertFreeEdge" receives the names of the nodes and
the \verb"TokenInfoMap" and builds an inclusion morphism to use it in
the \verb"FreeOrCofreeDefLink" link:

{\codesize
\begin{verbatim}
insertFreeEdge :: Token -> Token -> TokenInfoMap -> DGraph -> DGraph
insertFreeEdge tok1 tok2 tim dg = snd $ insLEdgeDG (n2, n1, edg) dg
          where (n1, _, _, _, _) = fromJust $ Map.lookup tok1 tim
                (n2, sg2, _, _, _) = fromJust $ Map.lookup tok2 tim
                mor = G_morphism Maude (Maude.Morphism.inclusion Maude.Sign.empty sg2) startMorId
                dgt = FreeOrCofreeDefLink NPFree $ EmptyNode (Logic Maude)
                edg = defDGLink (gEmbed mor) dgt SeeTarget
\end{verbatim}
}

























\subsection{Comorphism}\label{sec:comorphism}
%!TEX root = main.tex

We show in this section how the comorphism
from Maude to \CASL described in Section \ref{sec:comoprh} is implemented.
The function in charge of computing the comorphism is
\verb"maude2casl", that returns the \CASL signature and sentences
given the Maude signature and sentences:

{\codesize
\begin{verbatim}
maude2casl :: MSign.Sign -> [Named MSentence.Sentence] -> (CSign.CASLSign, 
                                                           [Named CAS.CASLFORMULA])
\end{verbatim}
}

This function splits the work into different stages:

\begin{itemize}

\item
The function \verb"rewPredicates" generates the \verb"rew" predicates for
each sort to simulate the rewrite rules in the Maude specification.

\item
The function \verb"rewPredicatesSens" creates the formulas associated to
the \verb"rew" predicates created above, stating that they are reflexive
and transitive.

\item
The \CASL operators are obtained from the Maude operators:

\begin{itemize}
\item
The function \verb"translateOps" splits the Maude operator map
into a tuple of \CASL operators and \CASL associative operators,
(which are required for parsing purposes).

\item
Since \CASL does not allow the definition
of polymorphic operators, these operators are removed from the map
with \verb"deleteUniversal" and for each one of these Maude operators we
create a set of \CASL operators with all the possible profiles with
\verb"universalOps".

\end{itemize}

\item \CASL sentences are obtained from the Maude sentences and from
predefined \CASL libraries:

\begin{itemize}
\item In the computation of the \CASL formulas we split Maude sentences in
equations defined without the \verb"owise" attribute, equations defined
with \verb"owise", and the rest of statements with the function
\verb"splitOwiseEqs".
\item The equations defined without the \verb"owise" attribute are
translated as universally quantified equations, as shown in Section
\ref{sec:comoprh}, with \verb"noOwiseSen2Formula".
\item Equations with the \verb"owise" attribute are translated using
a negative existential quantification, as we will show later, with
the function \verb"owiseSen2Formula". This function requires as additional
parameter the definition of the formulas defined without the \verb"owise"
attribute, in order to state that the equations defined with \verb"owise"
are applied when the rest of possible equations cannot.
\item The rest of statements, namely memberships and rules, are translated
with the function \verb"mb_rl2formula".
\item There are some built-in operators in Maude that are not defined by
means of equations. To allow the user to reason about them we provide
some libraries with the definitions of these operators as \CASL formulas,
obtained with \verb"loadLibraries".
\end{itemize}

\item Finally, the \CASL symbols are created:

\begin{itemize}
\item The kinds are translated to symbols with \verb"kinds2syms".
\item The operators are translated with \verb"ops2symbols".
\item The symbol predicates are obtained with \verb"preds2syms".
\end{itemize}

\end{itemize}

{\codesize
\begin{verbatim}
maude2casl msign nsens = (csign { CSign.sortSet = cs,
                            CSign.sortRel = sbs',
                            CSign.opMap = cops',
                            CSign.assocOps = assoc_ops,
                            CSign.predMap = preds,
                            CSign.declaredSymbols = syms }, new_sens)
   where csign = CSign.emptySign ()
         ss = MSign.sorts msign
         ss' = Set.map sym2id ss
         mk = kindMapId $ MSign.kindRel msign
         sbs = MSign.subsorts msign
         sbs' = maudeSbs2caslSbs sbs mk
         cs = Set.union ss' $ kindsFromMap mk
         preds = rewPredicates cs
         rs = rewPredicatesSens cs
         ops = deleteUniversal $ MSign.ops msign
         ksyms = kinds2syms cs
         (cops, assoc_ops, _) = translateOps mk ops
         cops' = universalOps cs cops $ booleanImported ops
         rs' = rewPredicatesCongSens cops'
         pred_forms = loadLibraries (MSign.sorts msign) ops
         ops_syms = ops2symbols cops'
         (no_owise_sens, owise_sens, mbs_rls_sens) = splitOwiseEqs nsens
         no_owise_forms = map (noOwiseSen2Formula mk) no_owise_sens
         owise_forms = map (owiseSen2Formula mk no_owise_forms) owise_sens
         mb_rl_forms = map (mb_rl2formula mk) mbs_rls_sens
         preds_syms = preds2syms preds
         syms = Set.union ksyms $ Set.union ops_syms preds_syms
         new_sens = concat [rs, rs', no_owise_forms, owise_forms,
                            mb_rl_forms, pred_forms]
\end{verbatim}
}

The \verb"rew" predicates are declared with the function
\verb"rewPredicates", that traverses the set of sorts applying
the function \verb"rewPredicate":

{\codesize
\begin{verbatim}
rewPredicates :: Set.Set Id -> Map.Map Id (Set.Set CSign.PredType)
rewPredicates = Set.fold rewPredicate Map.empty
\end{verbatim}
}

This function defines a binary predicate using as identifier the constant
\verb"rewID" and the sort as type of the arguments:

{\codesize
\begin{verbatim}
rewPredicate :: Id -> Map.Map Id (Set.Set CSign.PredType)
                -> Map.Map Id (Set.Set CSign.PredType)
rewPredicate sort m = Map.insertWith (Set.union) rewID ar m
   where ar = Set.singleton $ CSign.PredType [sort, sort]
\end{verbatim}
}

Once these predicates have been declared, we have to introduce
formulas to state their properties. The function \verb"rewPredicatesSens"
accomplishes this task by traversing the set of sorts and applying
\verb"rewPredicateSens":

{\codesize
\begin{verbatim}
rewPredicatesSens :: Set.Set Id -> [Named CAS.CASLFORMULA]
rewPredicatesSens = Set.fold rewPredicateSens []
\end{verbatim}
}

This function generates the formulas for each sort:

{\codesize
\begin{verbatim}
rewPredicateSens :: Id -> [Named CAS.CASLFORMULA] -> [Named CAS.CASLFORMULA]
rewPredicateSens sort acc = ref : trans : acc
        where ref = reflSen sort
              trans = transSen sort
\end{verbatim}
}

We describe the formula for the reflexivity, being the formula for the
transitivity analogous. A new variable of the required sort is created with
the auxiliary function \verb"newVar", then the qualified predicate
name is created with the \verb"rewID" constant and applied to the
variable. Finally, the formula is named with the prefix \verb"rew_refl_"
followed by the name of the sort:

{\codesize
\begin{verbatim}
reflSen :: Id -> Named CAS.CASLFORMULA
reflSen sort = makeNamed name $ quantifyUniversally form
        where v = newVar sort
              pred_type = CAS.Pred_type [sort, sort] nullRange
              pn = CAS.Qual_pred_name rewID pred_type nullRange
              form = CAS.Predication pn [v, v] nullRange
              name = "rew_refl_" ++ show sort
\end{verbatim}
}

The function \verb"translateOps" traverses the map of Maude operators,
applying to each of them the function \verb"translateOpDeclSet":

{\codesize
\begin{verbatim}
translateOps :: IdMap -> MSign.OpMap -> OpTransTuple
translateOps im = Map.fold (translateOpDeclSet im) (Map.empty, Map.empty, Set.empty)
\end{verbatim}
}

Since the values in the Maude operator map are sets of operator declarations
the auxiliary function \verb"translateOpDeclSet" has to traverse these sets, applying
\verb"translateOpDecl" to each operator declaration:

{\codesize
\begin{verbatim}
translateOpDeclSet :: IdMap -> MSign.OpDeclSet -> OpTransTuple -> OpTransTuple
translateOpDeclSet im ods tpl = Set.fold (translateOpDecl im) tpl ods
\end{verbatim}
}

The function \verb"translateOpDecl" receives an operator declaration,
that consists of all the operators declared with the same profile at
the kind level. The function traverses these operators, transforming
them into \CASL operators with the function \verb"ops2pred" and returning
a tuple containing the operators, the associative operators, and the
constructors:

{\codesize
\begin{verbatim}
translateOpDecl :: IdMap -> MSign.OpDecl -> OpTransTuple -> OpTransTuple
translateOpDecl im (syms, ats) (ops, assoc_ops, cs) = case tl of
                      [] -> (ops', assoc_ops', cs')
                      _ -> translateOpDecl im (syms', ats) (ops', assoc_ops', cs')
      where sym = head $ Set.toList syms
            tl = tail $ Set.toList syms
            syms' = Set.fromList tl
            (cop_id, ot, _) = fromJust $ maudeSym2CASLOp im sym
            cop_type = Set.singleton ot
            ops' = Map.insertWith (Set.union) cop_id cop_type ops
            assoc_ops' = if any MAS.assoc ats
                         then Map.insertWith (Set.union) cop_id cop_type assoc_ops
                         else assoc_ops
            cs' = if any MAS.ctor ats
                  then Set.insert (Component cop_id ot) cs
                  else cs
\end{verbatim}
}

As said above, Maude equations that are not defined with the \verb"owise"
attribute are translated to \CASL with \verb"noOwiseSen2Formula". This
function extracts the current equation from the named sentence, translates
it with \verb"noOwiseEq2Formula" and creates a new named sentence
with the resulting formula:

{\codesize
\begin{verbatim}
noOwiseSen2Formula ::  IdMap -> Named MSentence.Sentence -> Named CAS.CASLFORMULA
noOwiseSen2Formula im s = s'
       where MSentence.Equation eq = sentence s
             sen' = noOwiseEq2Formula im eq
             s' = s { sentence = sen' }
\end{verbatim}
}

The function \verb"noOwiseEq2Formula" distinguishes whether the equation
is conditional or not. In both cases, the Maude terms in the equation
are translated into \CASL terms with \verb"maudeTerm2caslTerm", and a
strong equation is used to create a formula. If the equation has no
conditions this formula is universally quantified and returned as result,
while if it has conditions each of them generates a formula and their
conjunction, computed with \verb"conds2formula", will be used as premise
of the equality formula:

{\codesize
\begin{verbatim}
noOwiseEq2Formula :: IdMap -> MAS.Equation -> CAS.CASLFORMULA
noOwiseEq2Formula im (MAS.Eq t t' [] _) = quantifyUniversally form
      where ct = maudeTerm2caslTerm im t
            ct' = maudeTerm2caslTerm im t'
            form = CAS.Strong_equation ct ct' nullRange
noOwiseEq2Formula im (MAS.Eq t t' conds@(_:_) _) = quantifyUniversally form
      where ct = maudeTerm2caslTerm im t
            ct' = maudeTerm2caslTerm im t'
            conds_form = conds2formula im conds
            concl_form = CAS.Strong_equation ct ct' nullRange
            form = createImpForm conds_form concl_form
\end{verbatim}
}

\verb"maudeTerm2caslTerm" is defined for each Maude term:

\begin{itemize}

\item Variables are translated into qualified \CASL variables, and their
type is translated to the corresponding type in \CASL:

{\codesize
\begin{verbatim}
maudeTerm2caslTerm :: IdMap -> MAS.Term -> CAS.CASLTERM
maudeTerm2caslTerm im (MAS.Var q ty) = CAS.Qual_var q ty' nullRange
        where ty' = maudeType2caslSort ty im
\end{verbatim}
}

\item Constants are translated as functions applied
to the empty list of arguments:

{\codesize
\begin{verbatim}
maudeTerm2caslTerm im (MAS.Const q ty) = CAS.Application op [] nullRange
        where name = token2id q
              ty' = maudeType2caslSort ty im
              op_type = CAS.Op_type CAS.Total [] ty' nullRange
              op = CAS.Qual_op_name name op_type nullRange
\end{verbatim}
}

\item The application of an operator to a list of terms is translated
into another application, translating recursively the arguments into
valid \CASL terms:

{\codesize
\begin{verbatim}
maudeTerm2caslTerm im (MAS.Apply q ts ty) = CAS.Application op tts nullRange
        where name = token2id q
              tts = map (maudeTerm2caslTerm im) ts
              ty' = maudeType2caslSort ty im
              types_tts = getTypes tts
              op_type = CAS.Op_type CAS.Total types_tts ty' nullRange
              op = CAS.Qual_op_name name op_type nullRange
\end{verbatim}
}

\end{itemize}

The conditions are translated into a conjunction with \verb"conds2formula",
that traverses the conditions applying \verb"cond2formula" to each of them,
and then creates the conjunction of the obtained formulas:

{\codesize
\begin{verbatim}
conds2formula :: IdMap -> [MAS.Condition] -> CAS.CASLFORMULA
conds2formula im conds = CAS.Conjunction forms nullRange
        where forms = map (cond2formula im) conds
\end{verbatim}
}

\begin{itemize}

\item Both equality and matching conditions are translated into
strong equations:

{\codesize
\begin{verbatim}
cond2formula :: IdMap -> MAS.Condition -> CAS.CASLFORMULA
cond2formula im (MAS.EqCond t t') = CAS.Strong_equation ct ct' nullRange
       where ct = maudeTerm2caslTerm im t
             ct' = maudeTerm2caslTerm im t'
cond2formula im (MAS.MatchCond t t') = CAS.Strong_equation ct ct' nullRange
       where ct = maudeTerm2caslTerm im t
             ct' = maudeTerm2caslTerm im t'
\end{verbatim}
}

\item Membership conditions are translated into \CASL memberships by translating
the term and the sort:

{\codesize
\begin{verbatim}
cond2formula im (MAS.MbCond t s) = CAS.Membership ct s' nullRange
      where ct = maudeTerm2caslTerm im t
            s' = token2id $ getName s
\end{verbatim}
}

\item Rewrite conditions are translated into formulas by using both terms
as arguments of the corresponding \verb"rew" predicate: 

{\codesize
\begin{verbatim}
cond2formula im (MAS.RwCond t t') = CAS.Predication pred_name [ct, ct'] nullRange
       where ct = maudeTerm2caslTerm im t
             ct' = maudeTerm2caslTerm im t'
             ty = token2id $ getName $ MAS.getTermType t
             kind = Map.findWithDefault (errorId "rw cond to formula") ty im
             pred_type = CAS.Pred_type [kind, kind] nullRange
             pred_name = CAS.Qual_pred_name rewID pred_type nullRange
\end{verbatim}
}

\end{itemize}

The equations defined with the \verb"owise" attribute are translated
with \verb"owiseSen2Formula", that traverses them and applies
\verb"owiseEq2Formula" to the inner equation:

{\codesize
\begin{verbatim}
owiseSen2Formula ::  IdMap -> [Named CAS.CASLFORMULA] -> Named MSentence.Sentence 
                     -> Named CAS.CASLFORMULA
owiseSen2Formula im owise_forms s = s'
       where MSentence.Equation eq = sentence s
             sen' = owiseEq2Formula im owise_forms eq
             s' = s { sentence = sen' }
\end{verbatim}
}

This function receives all the formulas defined without the \verb"owise"
attribute and, for each formula with the same operator in the lefthand
side as the current equation (obtained with \verb"getLeftApp"), it
generates with \verb"existencialNegationOtherEqs" a negative existential
quantification stating that the arguments do not match or the condition
does not hold that is used as premise of the equation:

{\codesize
\begin{verbatim}
owiseEq2Formula :: IdMap -> [Named CAS.CASLFORMULA] -> MAS.Equation -> CAS.CASLFORMULA
owiseEq2Formula im no_owise_form eq = form
      where (eq_form, vars) = noQuantification $ noOwiseEq2Formula im eq
            (op, ts, _) = fromJust $ getLeftApp eq_form
            ex_form = existencialNegationOtherEqs op ts no_owise_form
            imp_form = createImpForm ex_form eq_form
            form = CAS.Quantification CAS.Universal vars imp_form nullRange
\end{verbatim}
}

%The rest of Maude sentences are translated in a similar way to the one
%shown for the conditions above.
%The rest of the sentences generated in the comorphism are obtained
%from external libraries with the function \verb"readLib". We describe
%below how sentences defining the behavior of the natural numbers are
%loaded: once the library is obtained, we transform the theory sentences
%into named sentences with \verb"toNamedList" and then we ``coerce''
%them with \verb"coerceSens" to indicate that they are \CASL sentences.
%Finally, the sentence about the generators is filtered and the result
%returned:

%{\codesize
%\begin{verbatim}
%loadNaturalNatSens :: [Named CAS.CASLFORMULA]
%loadNaturalNatSens = 
%       let lib = head $ unsafePerformIO $ readLib "Maude/MaudeNumbers.casl"
%       in case lib of
%           G_theory lid _ _ thSens _ -> let sens = toNamedList thSens
%                                        in do
%                                            sens' <- coerceSens lid CASL "" sens
%                                            filter (not . ctorCons) sens'
%\end{verbatim}
%}

The translation from sorts, operators, and predicates to symbols
works in a similar way to the transformations shown above, so we only
describe  how the predicate symbols are obtained. The function
\verb"preds2syms" traverses the map of predicates and inserts each
obtained symbol into the set with \verb"pred2sym":

{\codesize
\begin{verbatim}
preds2syms :: Map.Map Id (Set.Set CSign.PredType) -> Set.Set CSign.Symbol
preds2syms = Map.foldWithKey pred2sym Set.empty
\end{verbatim}
}

This function traverses the set of predicate types and creates the
symbol corresponding to each one with \verb"createSym4id":

{\codesize
\begin{verbatim}
pred2sym :: Id -> Set.Set CSign.PredType -> Set.Set CSign.Symbol -> Set.Set CSign.Symbol
pred2sym pn spt acc = Set.fold (createSym4id pn) acc spt
\end{verbatim}
}

\verb"createSym4id" generates the symbol and inserts it into the
accumulated set:

{\codesize
\begin{verbatim}
createSym4id :: Id -> CSign.PredType -> Set.Set CSign.Symbol -> Set.Set CSign.Symbol
createSym4id pn pt acc = Set.insert sym acc
      where sym = CSign.Symbol pn $ CSign.PredAsItemType pt
\end{verbatim}
}























\subsection{Freeness constraints}\label{sec:imp_free}
\input{freeness}

\section{Concluding remarks and future work}\label{sec:conclusions}
%!TEX root = main.tex

We have presented how Maude has been integrated into
\Hets, a parsing, static analysis, and proof management tool that
combines various tools for different specification languages. To
achieve this integration, we consider preordered algebra semantics for
Maude and define an institution comorphism from Maude to \CASL.  This
integration allows to prove properties of Maude specifications like
those expressed in Maude views. We have also implemented a
normalization of the development graphs that allows us to prove
freeness constraints. We have used this transformation to connect
Maude to Isabelle \cite{Isabelle02}, a Higher Order Logic prover, and
have demonstrated a small example proof about reversal of lists.
Moreover, this encoding is suited for proofs of e.g.\ extensionality
of sets, which require first-order logic, going beyond the abilities
of existing Maude provers like ITP.

Since interactive proofs are often not easy to conduct, future work
will make proving more efficient by adopting automated induction
strategies like rippling~\cite{DBLP:conf/tphol/DixonF04}.  We also
have the idea to use the automatic first-order prover SPASS for
induction proofs by integrating special induction strategies directly
into \Hets.

We have also studied the possible comorphisms from \CASL to Maude. We
distinguish whether the formulas in the source theory are confluent and
terminating or not. In the first case, that we plan to check with the
Maude termination~\cite{MTT08} and confluence checker \cite{ChurchRoss10},
we map formulas to equations,
whose execution in Maude is more efficient, while in the second case
we map formulas to rules.

Finally, we also plan to relate \Hets' Modal Logic and Maude models in order to use
the Maude model checker \cite[Chapter 13]{maude-book} for linear temporal
logic.

\vspace{2ex}

\textbf{Acknowledgments} We wish to thank Francisco Dur\'an for discussions 
regarding freeness in Maude, Martin K\"uhl for cooperation
on the implementation of the system presented here,
and Maksym Bortin for help with the Isabelle proofs.
This work has been
supported by the German Federal Ministry of Education and Research
(Project 01 IW 07002 FormalSafe), the German Research Council (DFG)
under grant KO-2428/9-1, the Comunidad de Madrid project \emph{PROMETIDOS}
(S2009/TIC--1465), the MICINN Spanish project
\emph{DESAFIOS10} (TIN2009-14599-C03-01), and the Ministerio de Educaci\'on
y Ciencia under grant AP2005-007.


%strat and frozen attributes. Co-algebraic constructions.

{\small
\bibliographystyle{abbrv}
\bibliography{alberto}
}

\end{document}
